\chapter{Input Data}

In some configurations, \glc\ requires additional input data to run. For example, if asked to process galaxy formation through a set of externally derived merger trees, then a file describing those trees must be given. In the remainder of this section we describe the structure of external datasets which can be inputs to \glc.

\section{Broadband Filters}\index{filters!broadband}

To compute luminosities through a given filter, \glc\ requires the response function, $R(\lambda)$, of that filter to be defined. \glc\ follows the convention of \cite{hogg_k_2002} in defining the filter response to be the fraction of incident photons received by the detector at a given wavelength, multiplied by the relative photon response (which will be 1 for a photon-counting detector such as a CCD, or proportional to the photon energy for a bolometer/calorimeter type detector. Filter response files are stored in {\normalfont \ttfamily data/filters/}. Their structure is shown below, with the {\normalfont \ttfamily SDSS\_g.xml} filter reponse file used as an example:
\begin{verbatim}
 <filter>
  <description>SDSS g vacuum (filter+CCD +0 air mass)</description>
  <name>SDSS g</name>
  <origin>Michael Blanton</origin>
  <response>
    <datum>   3630.000      0.0000000E+00</datum>
    <datum>   3680.000      2.2690000E-03</datum>
    <datum>   3730.000      5.4120002E-03</datum>
    <datum>   3780.000      9.8719997E-03</datum>
    <datum>   3830.000      2.9449999E-02</datum>
    .
    .
    . 
  </response>
  <effectiveWavelength>4727.02994472695</effectiveWavelength>
  <vegaOffset>0.107430167298754</vegaOffset>
</filter>
\end{verbatim}
The {\normalfont \ttfamily description} tag should provide a description of the filter, while the {\normalfont \ttfamily name} tag provides a shorter name. The {\normalfont \ttfamily origin} tag should describe from where/whom this filter originated. The {\normalfont \ttfamily response} element contains a list of {\normalfont \ttfamily datum} tags each giving a wavelength (in Angstroms) and response pair. The normalization of the response is arbitrary. The {\normalfont \ttfamily effectiveWavelength} tag gives the mean, response-weighted wavelength of the filter and is used, for example, in dust attenuation calculations. The {\normalfont \ttfamily vegaOffset} tag gives the value (in magnitudes) which must be added to an AB-system magnitude in this system to place it into the Vega system. Both {\normalfont \ttfamily effectiveWavelength} and {\normalfont \ttfamily vegaOffset} can be computed by running
\begin{verbatim}
 scripts/filters/vega_offset_effective_lambda.pl data/filters
\end{verbatim}
which will compute these values for any filter files that do not already contain them and append them to the files.
