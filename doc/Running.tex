\chapter{Running Galacticus}

Before running \glc\ you will need to set two environment variables which specify where the \glc\ source code and datasets can be found. First, the environment variable {\normalfont \ttfamily GALACTICUS\_EXEC\_PATH} should be set to the full path to the build directory\index{path!galacticus root@{\glc\ root}}. Second, the environment variable {\normalfont \ttfamily GALACTICUS\_DATA\_PATH} should be set to the full path to the {\normalfont \ttfamily datasets} directory which was created when you installed \glc.

\glc\ is running using
\begin{verbatim}
 Galacticus.exe [<parameterFile>]
\end{verbatim}
where {\normalfont \ttfamily parameterFile} is the name of the file containing parameter settings for \glc. \glc\ will display messages indicating its progress as it runs (the verbosity can be controlled with the {\normalfont \ttfamily verbosityLevel} parameter).

\section{Parameter Files}\label{sec:ParameterFiles}

As described above, \glc\ requires a file of parameters to be given as a command line argument. The parameter file is an \gls{xml} file (which makes it easy to manipulate and construct these files from within many languages, e.g. Python) with the following structure:
\begin{verbatim}
 <parameters>
   <version>0.9.4</version>
   <formatVersion>2</formatVersion>
   <parameter1Name value= "parameter1Value" />
   <parameter2Name>
     <value>parameter2Value</value>
   </parameter2Name>
   <parameter3Name value= "parameter3Value" >
      <subParameter1Name value= "subParameter1Value" />
      <subParameter2Name value= "subParameter2Value" />
      .
      .
      .
   </parameter3Name>
   .
   .
   .
   <parameter4Name value= "parameter4Value" id="myRefParam" >
      <subParameter1Name value= "subParameter1Value" />
      <subParameter2Name value= "subParameter2Value" />
      .
      .
      .
   </parameter3Name>
   .
   .
   .
   <parameter4Name idRef="myRefParam"/>
   .
   .
   .
 </parameters>
\end{verbatim}
Each named element must have a {\normalfont \ttfamily value} attribute (preferred), or else contains a value element, which contains the desired value. The value can be a number, word or an array of space-separated numbers or words. Parameters are used to control the values of numerical parameters and also to select methods and other options. In many cases, if a parameter is not specified in the file a default value (hard coded into \glc) will be used instead. The default values have been chosen to produce a realistic model of galaxy formation, but may change as \glc\ evolves. Parameters may have sub-parameters embedded within them, as in the example above.

Sub-parameters are used in object composition within \glc. For example, the following would specify that linear growth of cosmological large scale structure should be modeled using the {\normalfont \ttfamily collisionlessMatter} method:
\begin{verbatim}
  <linearGrowthMethod value="collisionlessMatter">
    <cosmologyParametersMethod value="simple">
      <HubbleConstant   value="70.0" />
      <OmegaMatter      value="0.31"  />
      <OmegaDarkEnergy  value="0.69"  />
      <OmegaBaryon      value="0.045"/>
      <temperatureCMB   value="2.725"/>
    </cosmologyParametersMethod>
    <cosmologyFunctionsMethod value="matterLambda"/>
  </linearGrowthMethod>
\end{verbatim}
The linear growth function object requires knowledge about the cosmological parameters and model. In the above, we specify this explicitly by including a definition of the cosmological parameter object and cosmological functions object that our linear growth function object should use. Note that the cosmological functions object also requires knowledge of the cosmological parameters. When the {\normalfont \ttfamily cosmologyFunctionsMethod} object is built from the above definition it will first check for a cosmological parameters object defined in its own subparameters. Since it does not find one in this instance it will check for a cosmological parameters definition in its parent object (the {\normalfont \ttfamily linearGrowthMethod} element) and, in this case, will use that definition. If no definition were to be found in any parent element, a default set of cosmological parameters would be used instead\footnote{This approach allows a direct connection to be made between the structure of the input parameter XML file and the internal object hierarchy used by \glc, allowing very fine-grained control over the composition of \glc\ functionality. In particular it permits easy construction of objects which work by modifying results from other objects, such as the {\normalfont \ttfamily schneider2015} model for dark matter halo concentrations (\S\ref{phys:darkMatterProfileConcentration:darkMatterProfileConcentrationSchneider2015}).}.

Parameters can also be defined with an ``{\normalfont \ttfamily id}'' attribute. Such parameters are targets for pointers elsewhere in the file, and as such are inactive (i.e. they will be ignored by \glc\ except as targets for pointers). A pointer to a target is created by specifying an element with the same parameter name and an ``{\normalfont \ttfamily idRef}'' element with a value equal to that of the {\normalfont \ttfamily id} element of the target. The pointer then acts as a regular parameter, adopting the value and subparameters of the target. Targets can be targetted by multiple pointers. This allows for a single object to be shared between multiple other objects.

The optional {\normalfont \ttfamily version} element specifies which version of \glc\ this parameter file is intended for. The optional {\normalfont \ttfamily formatVersion} element specifies the parameter file version number (the current standard for parameter files is version 2). While optional, these elements can be useful when migrating parameter files between versions of \glc.

All parameter values (both those specified in this file and those set to default) used during a \glc\ run are output to the {\normalfont \ttfamily Parameters} group within the \glc\ output file. If parameters are present in the parameter file which do not match any known parameter in \glc\ then a warning message, listing all unknown parameters, will be given when \glc\ is run. Note that this will \emph{not} prevent \glc\ from running---sometimes it is convenient to include parameters which are not used by \glc, but which might be used by some other code.

\subsection{Validating Parameter Files}\index{parameters!validating}

A script, {\normalfont \ttfamily scripts/aux/validateParameters.pl}, is provided to validate parameter files and thereby ensure that they are consistent with \glc's expectations and requirements. To use simply execute:
\begin{verbatim}
 scripts/aux/validateParameters.pl myParameters.xml
\end{verbatim}
No output (and an exit value of 0) indicates a valid parameter file. Invalid parameter files will result in an exit value other than 0 and will produce error messages that should help to track down the problem with the file.

\subsection{Generating Parameter Files}\index{parameters!generating}

Some scripts are provided which assist in the generation of parameter files. These are located in the {\normalfont \ttfamily scripts/parameters/} folder and are detailed below:
\begin{description}
\item [{\normalfont \ttfamily cosmologicalParametersMonteCarlo.pl}] This script will generate a set of cosmological parameters drawn at random from the WMAP-9 constraints \citep{hinshaw_nine-year_2012}. It uses the covariance matrix (currently defined in {\normalfont \ttfamily data/Cosmological\_Parameters\_WMAP-9.xml}) to produce correlated random variables\footnote{Note that this does not capture the full details of the correlations between parameters, since it uses just the covariance matrix. For a more accurate calculation the full Monte Carlo Markov Chains used in the WMAP-9 parameter fitting should be used instead.}. The generated parameters are printed to standard output as \glc-compatible XML.
\end{description}

\subsection{Writing Data To a Temporary File}\index{output!temporary}

When running \glc\ on a compute cluster it is often advantageous to have output written to a local scratch disk during run time and only moved to networked storage after the run is complete. (Otherwise, \glc\ will perform many small writes to networked storage which can result in extremely slow run times.) To do this, simply set the parameter {\normalfont \ttfamily [galacticusOutputScratchFileName]} to the full path of a file to write to on local scratch space. During the run, data will be written to this file. After the run is finished, \glc\ will move this file to its permanent location as specified by the parameter {\normalfont \ttfamily [galacticusOutputFileName]}.

\subsection{Restarting A Crashed Run}\label{sec:Restarting}

If \glc\ crashes, it can be useful to restart the calculation from just prior to the crash to speed the debugging process. \glc\ has functionality to store and retrieve the internal state of any modules and to recover this to permit such restarting. Currently, this is implemented with the {\normalfont \ttfamily build} and {\normalfont \ttfamily read} methods of merger tree construction, such that the internal state is stored prior to commencing the building or reading of each tree, thereby allowing a calculation to be restarted with the tree that crashed. More general store/retrieve behavior is planned for future releases.

To cause \glc\ to periodically store its internal state include the following input parameter:
\begin{verbatim}
  <stateFileRoot value="galacticusState" />
\end{verbatim}
This will cause the internal state to be stored to files {\normalfont \ttfamily galacticusState.state} prior to commencing building each merger tree. Should a tree crash then replace this input parameter with:
\begin{verbatim}
  <stateRetrieveFileRoot       value="galacticusState"/>
  <mergerTreeConstructorMethod value="build"           >
   <treeBeginAt value="N"/>
  </mergerTreeConstructorMethod>
\end{verbatim}
where {\normalfont \ttfamily N} is the number of the tree that crashed. This will cause calculations to begin with tree {\normalfont \ttfamily N} and for the internal state to be recovered from the above mentioned files. The resulting tree and all galaxy formation calculations should therefore proceed just as in the original run (and so create the same crash condition).

\subsubsection{OpenMP}\index{debugging!OpenMP}\index{OpenMP!debugging}

When running a model in parallel using OpenMP, a separate state file will be written for each thread, with the thread number appended to the end of each state file name. For debugging purposes, it is suggested that a crashed OpenMP run be restarted using just a single thread. To do this, change the appended thread number on the state files corresponding to the thread which crashed to 0 such that they will be used by the single thread when the run is restarted.

\subsection{Running Grids of Models}\label{sec:RunningGrids}

You can easily write your own scripts to generate parameter files and run \glc\ on these files. An example of such a script is {\normalfont \ttfamily scripts/aux/launch.pl}. This script will loop over a sequence of parameter values, generate appropriate parameter files, run \glc\ using those parameters and analyze the results. This script currently supports running of \glc\ on a local machine, via a PBS queue (as multiple jobs or a single job), or on a \href{http://www.cs.wisc.edu/condor/}{{\normalfont \scshape Condor}}\index{Condor} cluster. To run the script simply enter:
\begin{verbatim}
 ./scripts/aux/launch.pl <runFile>
\end{verbatim}
This will launch a single instance of the script. Multiple instances can be launched and will share the work load (i.e. they will not attempt to run a model which another instance is already running or has finished). If multiple instances are to be launched on multiple machines a command line option to {\normalfont \ttfamily launch.pl} can be used to ensure that they do not duplicate work. Adding {\normalfont \ttfamily -{}-instance 2:4} for example will tell the script to run only the second model from each block of four models it finds. Launching for {\normalfont \ttfamily launch.pl} scripts on four different machines with {\normalfont \ttfamily -{}-instance 1:4}, {\normalfont \ttfamily -{}-instance 2:4}, {\normalfont \ttfamily -{}-instance 3:4} and {\normalfont \ttfamily -{}-instance 4:4} will then divide the models between those machines.

The {\normalfont \ttfamily runFile} is an XML file with the following structure:
\begin{verbatim}
<parameterGrid>
 <modelRootDirectory>models.new</modelRootDirectory>
 <baseParameters>newBestParametersQuick.xml</baseParameters>
 <compressModels>no</compressModels>
 <splitModels>4</splitModels>

 <launchMethod>pbs</launchMethod>

  <local>
   <threadCount>3</threadCount>
   <ompThreads>4</ompThreads>
  </local>

 <condor>
  <galacticusDirectory>/home/condor/Galacticus/v0.9.3</galacticusDirectory>
  <universe>vanilla</universe>
  <environment>LD_LIBRARY_PATH=/usr/lib:/usr/lib64:/usr/local/lib</environment>
  <requirement>Memory &gt;= 1000 &amp;&amp; Memory &lt; 2000</requirement>
  <transferFile>{PWD}/myFile.data</transferFile>
  <wholeMachine>true</wholeMachine>
  <postSubmitSleepDuration>5</postSubmitSleepDuration>
  <jobWaitSleepDuration>10</jobWaitSleepDuration>
 </condor>

 <pbs>
  <scratchPath>/scratch/me</scratchPath>
  <wallTime>48:00:00</wallTime>
  <memory>3gb</memory>
  <ompThreads>8</ompThreads>
  <queue>standard</queue>
  <maxJobsInQueue>10</maxJobsInQueue>
  <mpiLaunch>yes</mpiLaunch>
  <mpiRun>/opt/openmpi/bin/mpirun</mpiRun>
  <environment>LD_LIBRARY_PATH=/home/me/software/Galacticus/Tools/lib64:$LD_LIBRARY_PATH</environment>
  <postSubmitSleepDuration>10</postSubmitSleepDuration>
  <jobWaitSleepDuration>60</jobWaitSleepDuration>
 </pbs>

 <monolithicPBS>
  <mpiLaunch>yes</mpiLaunch>
  <nodes>1</nodes>
  <threadsPerNode>12</threadsPerNode>
  <ompThreads>6</ompThreads>
  <jobWaitSleepDuration>60</jobWaitSleepDuration>
  <analyze>no</analyze>
  <environment>LD_LIBRARY_PATH=/home/me/software/Galacticus/Tools/lib64:$LD_LIBRARY_PATH</environment>
  <includePath>/my/include/path</includePath>
  <libraryPath>/opt/sgi/mpt/mpt-2.04/lib</libraryPath>
  <shell>csh</shell>
  <pbsCommand>source /usr/share/modules/init/csh</pbsCommand>
  <pbsCommand>module load mpi-sgi/2.04_64</pbsCommand>
 </monolithicPBS>
                  
 <parameters>
  <label>modelLabel</label>
  <stabilityThresholdStellar value="1.1"/>
  <stabilityThresholdStellar value="0.9"/>
 </parameters>

 <parameters>
  <starFormationFeedbackDisksMethod value="powerLaw">
   <exponent value="2.5"/>
   <exponent value="3.0"/>
  </starFormationFeedbackDisksMethod>
  <starFormationFeedbackDisksMethod value="creasey2012"/>
 </parameters>

 <parameters>
  <imfSelectionMethod value="fixed">
    <imfSelectionFixed value="Chabrier" parameterLevel="top"/>
    <imfSelectionFixed value="Salpeter" parameterLevel="top"/>
  </imfSelectionMethod>
  <imfSelectionMethod value="diskSpheroid">
    <imfSelectionDisk     value="Chabrier"  parameterLevel="top"/>
    <imfSelectionSpheroid value="Kennicutt" parameterLevel="top"/>
  </imfSelectionMethod>
 </parameters>

 <parameters>
   <coolingFunctionMethod value="summation">
     <coolingFunctionMethod value="atomicCIECloudy"             iterable="no"/>
     <coolingFunctionMethod value="CMBCompton"                  iterable="no"/>
     <coolingFunctionMethod value="molecularHydrogenGalliPalla" iterable="no"/>
   </coolingFunctionMethod>
 </parameters>

</parameterGrid>
\end{verbatim}
Each {\normalfont \ttfamily parameters} block contains a list of parameters following the format used in standard \glc\ parameter files, with the difference that each parameter can appear multiple times, each time with a different {\normalfont \ttfamily value} attribute, as is the case for {\normalfont \ttfamily stabilityThresholdStellar} in the first {\normalfont \ttfamily parameters} element in the above. A model will be run for all possible combinations of these values. For nested parameters with multiple values, all possible values of these parameters will be looped over when, and only when, the appropriate value of the containing parameter is being used. For example, in the second {\normalfont \ttfamily parameters} element in the above example, models will be run with subparameter {\normalfont \ttfamily [exponent]}$=${\normalfont \ttfamily 2.5} and {\normalfont \ttfamily 3.5} for the {\normalfont \ttfamily starFormationFeedbackDisksMethod} element only when {\normalfont \ttfamily [starFormationFeedbackDisksMethod]}$=${\normalfont \ttfamily powerLaw} and not when {\normalfont \ttfamily [starFormationFeedbackDisksMethod]}$=${\normalfont \ttfamily creasey2012}. It is also possible to specify that subparameter should be promoted to the top-level of the parameter file. In the third {\normalfont \ttfamily parameters} element in the above example, {\normalfont \ttfamily imfSelectionFixed} will take on values of {\normalfont \ttfamily Chabrier} and {\normalfont \ttfamily Salpeter} only when {\normalfont \ttfamily imfSelectionMethod}$=${\normalfont \ttfamily fixed}, and the {\normalfont \ttfamily imfSelectionFixed} element will be promoted from a sub-parameter of {\normalfont \ttfamily imfSelectionMethod} to the top-level of the parameter file due to the presence of the \verb|parameterLevel="top"| attribute. Finally in some cases a parameter which appears multiple times is not to be iterated over. In the fourth {\normalfont \ttfamily parameters} element in the above example, this is the case for the {\normalfont \ttfamily coolingFunctionMethod} subparameters. The addition of an \verb|iterable="no"| attribute specifies that these parameters are not to be iterated over, but simply left as they are.

Some variables, which are expanded at run time, are available. These include:
\begin{description}
\item [{\normalfont \ttfamily \%\%galacticusOutputPath\%\%}] This will be expanded to the output path of a model. Useful for specifying paths for any additional output.
\end{description}

By default, each model is output into a sequentially numbered directory within the {\normalfont \ttfamily ./models} directory. By default, these directories have the prefix {\normalfont \ttfamily galacticus}. This can be changed by including a {\normalfont \ttfamily label} element inside a {\normalfont \ttfamily parameters} block, in which case the content of the {\normalfont \ttfamily label} element will be used as the prefix. This root directory can be modfified by the optional {\normalfont \ttfamily modelRootDirectory} element. Additionally, a set of base parameters can be read from a file specified by the {\normalfont \ttfamily baseParameters} file---these will be read before each model is run and before any variations in parameters for the specific model are applied. As such, it defines the default model around which parameter variations occur. Additional options that may be present in the file (as elements within the {\normalfont \ttfamily parameterGrid} element) are:
\begin{description}
\item[{\normalfont \ttfamily doAnalysis}]If set to ``no'' then no analysis scripts will be run on completed models, otherwise, they will be. Optionally, the analysis script to run can be specified via the {\normalfont \ttfamily analysisScript} element (see \S\ref{sec:AnalysisScripts});
\item[{\normalfont \ttfamily emailReport}] If set to ``yes'' a report will be e-mailed to the address specified in {\normalfont \ttfamily galacticusConfig.xml} when a model fails. Otherwise, the report will be written to standard output instead.
\item[{\normalfont \ttfamily compressModels}] If ``no'' then models are not compressed after being run. Otherwise, the contents of the model output directory will be compressed using {\normalfont \ttfamily bzip2}.
\item[{\normalfont \ttfamily splitModels}] If set to an integer larger than $1$, each \glc\ model will be split into that number of jobs, and those jobs will be launched (using the selected method) independently. Once finished, the outputs from these split models will be merged back into a single model. This allows, for example, effectively distributing a single \glc\ model over multiple nodes of a PBS cluster.
\end{description}

The method by which to launch jobs must be specified in the {\normalfont \ttfamily launchMethod} element. Currently available options are:
\begin{description}
\item[{\normalfont \ttfamily local}] The models will be run on the local machine. Two additional options can be specified within a {\normalfont \ttfamily local} XML block:
\begin{description}
\item[{\normalfont \ttfamily threadCount}] The number of individual model threads to be launched.
\item[{\normalfont \ttfamily ompThreads}] The number of OpenMP threads to be used by each model.
\end{description}

\item[{\normalfont \ttfamily pbs}] Jobs will be submitted to a {\normalfont \ttfamily PBS} batch queue system. The following options are available and can be specified within a {\normalfont \ttfamily pbs} XML block:
\begin{description}
\item[{\normalfont \ttfamily scratchPath}] An optional path to which the model output will be written at run time. At the completion of each run, the data will be transferred to the usual output location. This is useful to avoid network I/O during run time;
\item[{\normalfont \ttfamily wallTime}] A limit on the wall time allowed for each model (optional);
\item[{\normalfont \ttfamily memory}] A limit on the memory allowed for each model (optional);
\item[{\normalfont \ttfamily ompThreads}] The number of OpenMP threads to use for each model (optional). This is used to request an appropriate number of processors per node;
\item[{\normalfont \ttfamily queue}] The name of the queue to submit the jobs to (optional);
\item[{\normalfont \ttfamily maxJobsInQueue}] The maximum number of jobs to place in the queue. Additional jobs will be held and submitted once the number of jobs in the queue drops below this value (optional);
\item[{\normalfont \ttfamily mpiLaunch}] If set to ``{\normalfont \ttfamily yes}'' then the {\normalfont \ttfamily mpirun} command will be used to launch a single copy of \glc\ (which may then spawn multiple OpenMP threads). If instead set to ``{\normalfont \ttfamily no}'' then \glc\ is launch without the use of the {\normalfont \ttfamily mpirun} command. Some systems will limit a code launched with {\normalfont \ttfamily mpirun} to using just a single CPU (even if multiple OpenMP threads are spawned). In such cases, setting this option to ``{\normalfont \ttfamily no}'' should permit multiple CPUs to be utilized.
\item[{\normalfont \ttfamily mpiRun}] The path to the {\normalfont \ttfamily mpirun} executable (optional---if not present, {\normalfont \ttfamily mpirun} must be in {\normalfont \ttfamily PATH});
\item[{\normalfont \ttfamily environment}] Any settings here are set in each {\normalfont \scshape PBS} job in order to set appropriate environment variables on the machine where a job is executed;
\item[{\normalfont \ttfamily analyze}] If set to ``{\normalfont \ttfamily yes}'' then analysis (if any) will be performed as part of the PBS job. Otherwise, analysis is performed by the submitting machine.
\item[{\normalfont \ttfamily postSubmitSleepDuration}] The time (in seconds) to wait after submitting each job. This prevents flooding the PBS queue manager with a large number of jobs in rapid succession.
\item[{\normalfont \ttfamily jobWaitSleepDuration}] The time (in seconds) to sleep between successive checks of the PBS queue to see if any of the submitted jobs have finished.
\end{description}

\item[{\normalfont \ttfamily monolithicPBS}] A single job will be submitted to a {\normalfont \ttfamily PBS} batch queue system. This job will internally run multiple copies of \glc\ each with a different set of parameters. The following options are available and can be specified within a {\normalfont \ttfamily monolithicPBS} XML block:
\begin{description}
\item[{\normalfont \ttfamily nodes}] The total number of nodes to use for the PBS job.
\item[{\normalfont \ttfamily threadsPerNode}] The number of threads per node to use for the PBS job.
\item[{\normalfont \ttfamily ompThreads}] The number of OpenMP threads to use for each model (optional). This is used to request an appropriate number of processors per node, and must be an factor of {\normalfont \ttfamily threadsPerNode};
\item[{\normalfont \ttfamily scratchPath}] An optional path to which the model output will be written at run time. At the completion of each run, the data will be transferred to the usual output location. This is useful to avoid network I/O during run time;
\item[{\normalfont \ttfamily wallTime}] A limit on the wall time allowed for each model (optional);
\item[{\normalfont \ttfamily memory}] A limit on the memory allowed for each model (optional);
\item[{\normalfont \ttfamily queue}] The name of the queue to submit the jobs to (optional);
\item[{\normalfont \ttfamily mpiRun}] The path to the {\normalfont \ttfamily mpirun} executable (optional---if not present, {\normalfont \ttfamily mpirun} must be in {\normalfont \ttfamily PATH});
\item[{\normalfont \ttfamily environment}] Any settings here are set in each {\normalfont \scshape PBS} job in order to set appropriate environment variables on the machine where a job is executed;
\item[{\normalfont \ttfamily analyze}] If set to ``{\normalfont \ttfamily yes}'' then analysis (if any) will be performed as part of the PBS job. Otherwise, analysis is performed by the submitting machine.
\item[{\normalfont \ttfamily postSubmitSleepDuration}] The time (in seconds) to wait after submitting each job. This prevents flooding the PBS queue manager with a large number of jobs in rapid succession.
\item[{\normalfont \ttfamily jobWaitSleepDuration}] The time (in seconds) to sleep between successive checks of the PBS queue to see if any of the submitted jobs have finished.
\end{description}

\item[{\normalfont \ttfamily condor}] Jobs will be submitted to a {\normalfont \ttfamily Condor} cluster. The following options are available and can be specified within a {\normalfont \ttfamily condor} XML block:
\begin{description}
\item[{\normalfont \ttfamily galacticusDirectory}] When a \glc\ job is submitted to a {\normalfont \scshape Condor} cluster the \glc\ executable and the input parameter file are transferred to the machine where the job runs. Other files, such as data files, are not transferred. Therefore, they must be already present on any remote machine on which the job can run. This option specifies where a complete \glc\ installation can be found on the remote machine. If not present, it defaults to {\normalfont \ttfamily /home/condor/Galacticus/v0.9.0};
\item[{\normalfont \ttfamily universe}] Specifies to which {\normalfont \scshape Condor} universe jobs should be submitted. Allowed options are ``vanilla'' and ``standard''. If the standard universe is to be used then \glc\ must have been linked with {\normalfont \ttfamily condor\_compile}---the {\normalfont \ttfamily Makefile} allows this if the relevant lines are uncommented;
\item[{\normalfont \ttfamily environment}] Any settings here are passed to {\normalfont \scshape Condor}'s {\normalfont \ttfamily environment} option in order to set appropriate environment variables on the machine where a job is executed;
\item[{\normalfont \ttfamily requirement}] Any setting here is passed to {\normalfont \scshape Condor}'s {\normalfont \ttfamily requirements} option to specify requirements for each job. Multiple {\normalfont \ttfamily requirement} entries will be combined (using logical and).
\item[{\normalfont \ttfamily transferFile}] Any files listed here will be transferred the Condor worker (and so will be accessible from the path in which \glc\ is running). The macro {\normalfont \ttfamily \{PWD\}} will be automatically expanded to the present working directory. Multiple {\normalfont \ttfamily transferFile} entries can be given.
\item [{\normalfont \ttfamily wholeFile}] Setting this option to {\normalfont \ttfamily true} will add {\normalfont \ttfamily +RequiresWholeMachine = True} to the Condor submit file. If Condor has been configured to allow jobs to take over a whole machine\footnote{As described \protect\href{https://www-auth.cs.wisc.edu/lists/condor-users/2009-January/msg00086.shtml}{here} for example.}, this will cause jobs to do so. This is useful if you want to run OpenMP \glc\ on a Condor cluster.
\item[{\normalfont \ttfamily postSubmitSleepDuration}] The time (in seconds) to wait after submitting each job. This prevents flooding the Condor queue manager with a large number of jobs in rapid succession.
\item[{\normalfont \ttfamily jobWaitSleepDuration}] The time (in seconds) to sleep between successive checks of the Condor queue to see if any of the submitted jobs have finished.
\end{description}

\end{description}

In addition to the {\normalfont \ttfamily galacticus.hdf5} output file, each model directory will contain a file {\normalfont \ttfamily parameters.xml} which contains the parameters used to run the model and {\normalfont \ttfamily galacticus.log} which contains any output from \glc\ during the run.

If present, the file {\normalfont \ttfamily galacticusConfig.xml}, described in \S\ref{sec:ConfigFile}, is parsed for configuration options. If the {\normalfont \ttfamily contact} element is present, the listed name and e-mail address will be used to determine who should receive error reports should a model crash. The error report will contain the host name of the computer running the model, the location of the model output and the log file (which may be incomplete if output is being buffered). Additionally, any core file produced will be stored in the model directory for later perusal, and the state files (see \S\ref{sec:Restarting}) for the run can also be found in the model directory.

\subsection{Processing Individual Merger Trees In Parallel}

By default, \glc\ utilizes the available parallel threads to process multiple merger trees simulataneously, with one tree processed by each thread. When the total number of trees to be processed is large, and there are not a small number of outlier trees with masses very much larger than the other trees, this approach generally results in good parallel efficiency.

However, in cases where a small number of trees are much more massive than any other (or are just slow to process for some other reason) it may be more efficient to have multiple parallel threads process each tree. To achieve this, set {\normalfont \ttfamily [treeEvolveSingleForest]}$=${\normalfont \ttfamily true}. In this case, trees are processed sequentially, with multiple threads assigned to each tree. To do this, a tree is broken up into a set of time slices, or ``sections''. The number of sections between each successive output (or between the earliest node in the tree and the first output) is specified by the {\normalfont \ttfamily [treeEvolveSingleForestSections]} parameter. Individual branches of the tree within each section are assigned to parallel threads. \emph{Note that this results in valid evolution only if the evolution of disjoint tree branches are independent of each other.}

\section{Tasks}

By default, \glc\ executes a single ``task'' when run---to evolve a set of merger tree forests and output the resulting halo and galaxy properties. Different tasks can be performed however, controlled by the {\normalfont \ttfamily taskMethod} parameter.

\subsection{{\normalfont \ttfamily evolveForests}}

This is the standard task performed by \glc. It takes a set of merger tree forests, evolves them, and outputs the resulting halo and galaxy properties.

\subsection{{\normalfont \ttfamily multi}}

This task allows multiple sub-tasks to be run. For example:
\begin{verbatim}
  <taskMethod value="multi">
    <taskMethod value="haloMassFunction"/>
    <taskMethod value="evolveForests"   />
  </taskMethod>
\end{verbatim}
would result in the {\normalfont \ttfamily haloMassFunction} task being performed, followed by the {\normalfont \ttfamily evolveForests} task.

\subsection{{\normalfont \ttfamily haloMassFunction}}

The {\normalfont \ttfamily haloMassFunction} task will tabulate the halo mass function and related quantities and output them to the main \glc\ output file. The tabulation is controlled by three sub- parameters:
\begin{description}
\item [{\normalfont \ttfamily [haloMassMinimum]}] The lowest mass halo (in units of $M_\odot$) at which to tabulate;
\item [{\normalfont \ttfamily [haloMassMaximum]}] The highest mass halo (in units of $M_\odot$) at which to tabulate;
\item [{\normalfont \ttfamily [pointsPerDecade]}] The number of points per decade of halo mass at which to tabulate.
\end{description}T
The halo mass function is computed at each output specified by the {\normalfont \ttfamily [outputRedshifts]} parameter, and written to the corresponding {\normalfont \ttfamily Outputs/OutputN} group with the following structure:
\begin{verbatim}
+-> Outputs
|   |
|   +-> OutputN
|       |
|       +-> massHaloCharacteristic         [attribute]
|       |
|       +-> criticalOverdensity            [attribute]
|       |
|       +-> outputExpansionFactor          [attribute]
|       |
|       +-> growthFactor                   [attribute]
|       |
|       +-> outputRedshift                 [attribute]
|       |
|       +-> outputTime                     [attribute]
|       |
|       +-> virialDensityContrast          [attribute]
|       |
|       +-> haloMass                       [dataset]
|       |
|       +-> haloSigma                      [dataset]
|       | 
|       +-> haloAlpha                      [dataset]
|       |
|       +-> haloPeakHeightNu               [dataset]
|       |
|       +-> haloMassFunctionM              [dataset]
|       |
|       +-> haloMassFunctionLnM            [dataset]
|       |
|       +-> haloMassFunctionLnMBinAveraged [dataset]
|       |
|       +-> haloMassFunctionNuFNu          [dataset]
|       |
|       +-> haloMassFunctionCumulative     [dataset]
|       |
|       +-> haloMassFractionCumulative     [dataset]
|       |
|       +-> subhaloMassFunctionCumulative  [dataset]
|       |
|       +-> haloBias                       [dataset]
|       |
|       +-> haloVirialRadius               [dataset]
|       |
|       +-> haloVirialTemperature          [dataset]
|       |
|       +-> haloVirialVelocity             [dataset]
|       |
|       +-> haloScaleRadius                [dataset]
|       |
|       +-> haloVelocityMaximum            [dataset]
|    
+-> cosmology
|   |
|   +-> densityCritical                    [attribute]
|    
+-> powerSpectrum
    |
    +-> massHalfMode                       [attribute]
\end{verbatim}
The {\normalfont \ttfamily Outputs/OutputN} groups contain atrributes and datasets which give properties at the corresponding output time as follows:
\begin{description}
\item [{\normalfont \ttfamily massHaloCharacteristic}] The characteristic mass scale (in units of $\mathrm{M}_\odot$), $M_*$, at which $\sigma(M)=\delta_\mathrm{c}(z)$;
\item [{\normalfont \ttfamily criticalOverdensity}] The critical overdensity for collapse of halos, $\delta_\mathrm{c}$;
\item [{\normalfont \ttfamily outputExpansionFactor}] The expansion factor;
\item [{\normalfont \ttfamily growthFactor}] The linear growth factor;
\item [{\normalfont \ttfamily outputRedshift}] The redshift;
\item [{\normalfont \ttfamily outputTime}] The cosmic time (in units of Gyr);
\item [{\normalfont \ttfamily virialDensityContrast}] The virial density contrast of halos.
\item [{\normalfont \ttfamily haloMass}] The mass of the halo, $M_\mathrm{halo}$ (in $\mathrm{M}_\odot$);
\item [{\normalfont \ttfamily haloSigma}] The root-variance of the mass field smoothed in top-hat spheres, $\sigma(M)$;
\item [{\normalfont \ttfamily haloAlpha}] The loagrithmic gradient of the root-variance of the mass field smoothed in top-hat spheres with mass, $\mathrm{d}\log\sigma(M)/\mathrm{d}\log M_\mathrm{halo}$;
\item [{\normalfont \ttfamily haloPeakHeightNu}] The peak height of the halo, $\nu = \delta_\mathrm{c}/\sigma(M)$;
\item [{\normalfont \ttfamily haloMassFunctionM}] The halo mass function per halo mass, $\mathrm{d}n/\mathrm{d}M_\mathrm{halo}$ (in units of Mpc$^{-3} \mathrm{M}_\odot^{-1}$);
\item [{\normalfont \ttfamily haloMassFunctionLnM}] The halo mass function per logarithmic halo mass, $\mathrm{d}n/\mathrm{d}\log M_\mathrm{halo}$ (in units of Mpc$^{-3}$);
\item [{\normalfont \ttfamily haloMassFunctionLnMBinAveraged}] The halo mass function per logarithmic halo mass averaged over the finite width of the bin (in units of Mpc$^{-3}$);
\item [{\normalfont \ttfamily haloMassFunctionNuFNu}] The halo mass function defined in terms of the peak-height parameter, $\nu F(\nu)$;
\item [{\normalfont \ttfamily haloMassFractionCumulative}] The mass fraction in halos above the current halo mass;
\item [{\normalfont \ttfamily haloMassFunctionCumulative}] The cumulative number of halos per unit volume above the current halo mass (in units of Mpc$^{-3}$);
\item [{\normalfont \ttfamily subhaloMassFunctionCumulative}] The cumulative number of sub-halos per unit volume above the current halo mass (in units of Mpc$^{-3}$);
\item [{\normalfont \ttfamily haloBias}] The large scale linear theory bias of the halo;
\item [{\normalfont \ttfamily haloVirialRadius}] The virial radius (in units of Mpc) of the current halo mass;
\item [{\normalfont \ttfamily haloVirialTemperature}] The virial temperature (in units of Kelvin) of the current halo mass;
\item [{\normalfont \ttfamily haloVirialVelocity}] The virial velocity (in units of km/s) of the current halo mass;
\item [{\normalfont \ttfamily haloScaleRadius}] The scale radius (in units of Mpc) of the current halo mass;
\item [{\normalfont \ttfamily haloVelocityMaximum}] The peak of the rotation curve (in units of km/s) of the current halo mass;
\end{description}
Dimensionful datasets have an {\normalfont \ttfamily unitsInSI} attribute that gives their units in the SI system.

Additionally, an attribute giving the critical density of the universe (in units of $\mathrm{M}_\odot \mathrm{Mpc}^{-3}$) is written to the {\normalfont \ttfamily cosmology} group and, if such a scale is well-defined, an attribute given the mass corresponding to the scale at which the power spectrum is reduce by half relative to a \gls{cdm} power spectrum in units of $\mathrm{M}_\odot$ is written as {\normalfont \ttfamily massHalfMode} to the {\normalfont \ttfamily powerSpectrum} group.

\subsection{{\normalfont \ttfamily excursionSets}}\index{excursion sets}

The {\normalfont \ttfamily excursionSets} task will generate output which contains a variety of measures related to excursion sets in the Press-Schechter formalism. The results are output to a group specified by the {\normalfont \ttfamily [outputGroup]} subparameter (which defaults to {\normalfont \ttfamily excursionSets}). The output group contains the following structure:
\begin{verbatim}
+-> barrier                  [dataset]
|
+-> firstCrossingProbability [dataset]
|
+-> firstCrossingRate        [dataset]
|
+-> mass                     [dataset]
|
+-> massFunction             [dataset]
|
+-> powerSpectrum            [dataset]
|
+-> time                     [dataset]
|
+-> variance                 [dataset]
|
+-> wavenumber               [dataset]
\end{verbatim}
These datasets contain the following information:
\begin{description}
 \item [{\normalfont \ttfamily mass}] Halo mass [$\mathrm{M}_\odot$];
 \item [{\normalfont \ttfamily time}] Cosmic time [Gyr];
 \item [{\normalfont \ttfamily wavenumber}] Wavenumber corresponding to this halo mass [Mpc$^{-1}$];
 \item [{\normalfont \ttfamily powerSpectrum}] Power spectrum at this wavenumber [Mpc$^3$];
 \item [{\normalfont \ttfamily variance}] The variance, $S(M)\equiv\sigma^2(M)$, at this halo mass;
 \item [{\normalfont \ttfamily barrier}] The excursion set barrier, $B(S)$;
 \item [{\normalfont \ttfamily firstCrossingProbability}] The probability of first crossing this barrier between $S$ and $S+\mathrm{d}S$;
 \item [{\normalfont \ttfamily firstCrossingRate}] The rate of first crossing of the barrier per unit time [Gyr$^{-1}$] for all pairs of halo mass;
 \item [{\normalfont \ttfamily massFunction}] The halo mass function [$\mathrm{M}^{-1}_\odot$~Mpc$^{-3}$].
\end{description}

\subsection{{\normalfont \ttfamily powerSpectra}}\index{power spectrum!outputting}

The {\normalfont \ttfamily powerSpectra} task will output a variety of measures of the matter power spectrum tabulated as a function of wavenumber. The output data has the following structure:
\begin{verbatim}
+-> powerSpectrum
    |
    +-> alpha         [dataset]
    |
    +-> mass          [dataset]
    |
    +-> powerSpectrum [dataset]
    |
    +-> sigma         [dataset]
    |
    +-> wavenumber    [dataset]
\end{verbatim}
The {\normalfont \ttfamily Parameters} group contains attributes giving the values of all used parameters (just as in a \glc\ output file). The {\normalfont \ttfamily powerSpectrum} group contains datasets which give the power spectrum and related properties as follows:
\begin{description}
\item [{\normalfont \ttfamily alpha}] The logarithmic slope of $\sigma(M)$: $\alpha = \mathrm{d} \ln \sigma / \mathrm{d} \ln M$;
\item [{\normalfont \ttfamily mass}] The mass scale, $M$, corresponding to the given wavenumber, $k$, defined such that $M=4 \pi \Omega_\mathrm{M} \rho_\mathrm{crit} / 3 k^3$ (in units of $\mathrm{M}_\odot$);
\item [{\normalfont \ttfamily powerSpectrum}] The linear theory power spectrum at $z=0$: $P(k)$ in units of Mpc$^3$;
\item [{\normalfont \ttfamily sigma}] The dimensionless linear theory mass fluctuation at $z=0$: $\sigma(M)$;
\item [{\normalfont \ttfamily wavenumber}] The wavenumber in units of Mpc$^{-1}$.
\end{description}
Dimensionful datasets have an {\normalfont \ttfamily unitsInSI} attribute that gives their units in the SI system.

\section{Configuration File}\label{sec:ConfigFile}\index{galacticusConfig.xml@{\normalfont \ttfamily galacticusConfig.xml}}\index{configuration}

The file {\normalfont \ttfamily galacticusConfig.xml}, is present, is used to configure \glc\ and provide useful information. It should have the following structure:
\begin{verbatim}
<config>
  <contact>
    <name>My Name</name>
    <email>me@ivory.towers.edu</email>
  </contact>
  <email>
    <host>
      <name>myComputerHostName</name>
      <method>smtp</method>
      <host>smtp-server.ivory.towers.edu</host>
      <user>myUserName</user>
      <passwordFrom>kdewallet</passwordFrom>
    </host>
    <host>
      <name>default</name>
      <method>sendmail</method>
    </host>
  </email>
</config>
\end{verbatim}
The name and e-mail address in the {\normalfont \ttfamily contact} section will be stored in any \glc\ models run---this helps track the provenance of the model. The {\normalfont \ttfamily email} section determines how e-mail will be sent. Within this section, you can place one or more {\normalfont \ttfamily host} elements, the {\normalfont \ttfamily name} element of which specifies the host name of the computer to which these rules apply (the {\normalfont \ttfamily default} host is used if no other match is found). For each host, the {\normalfont \ttfamily method} element specifies how e-mail should be sent, either by {\normalfont \ttfamily sendmail} or via {\normalfont \ttfamily smtp}. For SMTP transport (which currently supports SSL connections only), you must specify the {\normalfont \ttfamily host} SMTP server, {\normalfont \ttfamily user} name. The {\normalfont \ttfamily passwordFrom} element specifies how the password for the SMTP log in should be obtained. If set to {\normalfont \ttfamily input} then the user will be prompted for the password as needed. Alternatively, if you use the \href{http://www.kde.org/}{KDE} desktop and the \href{http://utils.kde.org/projects/kwalletmanager/}{KDEWallet} password manager, 
setting {\normalfont \ttfamily passwordFrom} to {\normalfont \ttfamily kdewallet} will cause the password to be stored in the KDE wallet and retrieved from there subsequently.
