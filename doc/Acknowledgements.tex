\chapter{Acknowledgements}

In addition to the tools and libraries required to compile and run \glc, development of \glc\ has benefitted from extensive use of the following: \href{http://www.gnu.org/software/octave/}{{\sc GNU Octave}}, \href{http://maxima.sourceforge.net/}{{\sc Maxima}}, \href{http://edu.kde.org/cantor/}{{\sc Cantor}}, \href{http://kile.sourceforge.net/}{{\sc Kile}}, \href{http://www.gnu.org/software/emacs/}{{\sc Emacs}} and \href{http://valgrind.org/}{{\sc Valgrind}}. We are grateful to the members of the {\sc GNU Fortran} mailing list for invaluable discussions and fixes for compiler problems. We thank John Burkardt for making available the {\sc Bivar} algorithm for performing interpolation on data irregularly spaced on a 2D plane and Dima Verner for making available codes to compute various atomic data for astrophysics. The community of \glc\ users\footnote{In particular, Jianling Gan, Markus Haider, Ting-Wen Lan, Luiz Felippe Rodrigues, Sergio Sanes, Martin White and Liyan Xu.} have provided invaluable feedback and bug reports. Gian Luigi Granato and Laura Silva kindly provided modifications to their \href{http://adlibitum.oat.ts.astro.it/silva/grasil/grasil.html}{\sc Grasil} code to allow it to read \glc\ outputs.
