% Acronyms.
\newacronym{cmb}{CMB}{cosmic microwave background}
\newacronym{igm}{IGM}{intergalactic medium}
\newacronym{imf}{IMF}{initial mass function}
\newacronym{isco}{ISCO}{innermost stable circular orbit}
\newacronym{ism}{ISM}{interstellar medium}
\newacronym{ode}{ODE}{ordinary differential equation}
\newacronym{nfw}{NFW}{Navarro-Frenk-White (dark matter halo profile)}
\newacronym{sed}{SED}{spectral energy distribution}
\newacronym{sne}{SNe}{supernovae}
\newglossaryentry{adaf}{type=\acronymtype, name={ADAF}, description=\glslink{adafg}{advection-dominated accretion flow}, first={advection-dominated accretion flow (ADAF)}, see=[Glossary:]{adafg}}
\newglossaryentry{pah}{type=\acronymtype, name={PAH}, description=\glslink{pahg}{polycyclic aromatic hydrocarbon}, first={polycyclic aromatic hydrocarbon (PAH)}, see=[Glossary:]{pahg}, firstplural={polycyclic aromatic hydrocarbons (PAH)}, see=[Glossary:]{pahg}}
\newglossaryentry{dsl}{type=\acronymtype, name={DSL}, description=\glslink{dslg}{domain-specific language}, first={domain specific language (DSL)}, see=[Glossary:]{dslg}, firstplural={domain-specific languages (DSLs)}, see=[Glossary:]{dslg}}
\newglossaryentry{wdm}{type=\acronymtype, name={WDM}, description=\glslink{wdmg}{warm dark matter}, first={warm dark matter (WDM)}, see=[Glossary:]{wdmg}, firstplural={warm dark matters (WDM)}, see=[Glossary:]{wdmg}}
\newacronym{sdss}{SDSS}{Sloan Digitial Sky Survey}
\newglossaryentry{ahf}{type=\acronymtype, name={AHF}, description=\glslink{ahfg}{Amiga's Halo Finder}, first={Amiga's Halo Finder (AHF)}, see=[Glossary:]{ahfg}}
\newglossaryentry{cdm}{type=\acronymtype, name={CDM}, description=\glslink{cdmg}{cold dark matter}, first={cold dark matter (CDM)}, see=[Glossary:]{cdmg}, firstplural={cold dark matters (CDM)}, see=[Glossary:]{cdmg}}
\newacronym{mcmc}{MCMC}{Markov Chain Monte Carlo}
\newglossaryentry{sam}{type=\acronymtype, name={SAM}, description=\glslink{samg}{semi-analytic model}, first={semi-analytic model (SAM)}, see=[Glossary:]{samg}, firstplural={semi-analytic models (SAMs)}, see=[Glossary:]{samg}}
\newglossaryentry{bie}{type=\acronymtype, name={BIE}, description=\glslink{bieg}{semi-analytic model}, first={Bayesian inference engine (BIE)}, see=[Glossary:]{bieg}, firstplural={Bayesian inference engines (BIEs)}, see=[Glossary:]{bieg}}
\newglossaryentry{hod}{type=\acronymtype, name={HOD}, description=\glslink{hodg}{halo occupation distribution}, first={halo occupation distribution (HOD)}, see=[Glossary:]{hodg}, firstplural={halo occupation distributions (HODs)}, see=[Glossary:]{hodg}}
\newglossaryentry{mpi}{type=\acronymtype, name={MPI}, description=\glslink{mpig}{message passing interface}, first={message passing interface (MPI)}, see=[Glossary:]{mpig}, firstplural={message passing interfaces (MPIs)}, see=[Glossary:]{mpig}}
\newglossaryentry{pbs}{type=\acronymtype, name={PBS}, description=\glslink{pbsg}{portable batch system}, first={portable batch system (PBS)}, see=[Glossary:]{pbsg}, firstplural={portable batch systems (PBSs)}, see=[Glossary:]{pbsg}}
\newglossaryentry{pdf}{type=\acronymtype, name={PDF}, description=\glslink{pdfg}{probability density function}, first={probability density function (PDF)}, see=[Glossary:]{pdfg}, firstplural={probability density functions (PDFs)}, see=[Glossary:]{pdfg}}
\newglossaryentry{gsl}{type=\acronymtype, name={GSL}, description=\glslink{gslg}{GNU Scientific Library}, first={GNU Scientific Library (GSL)}, see=[Glossary:]{gslg}}

% Glossary entries.
\newglossaryentry{FFTLog}{name={FFTLog}, description={\href{http://casa.colorado.edu/~ajsh/FFTLog/}{FFTLog} is a code to compute fast Fourier transforms of discrete periodic sequences of logarithmically spaced data}}

\newglossaryentry{component}{name={component}, description={An individual physical system within a \gls{node}, such as a dark matter halo, a galactic disk or a supermassive black hole}}

\newglossaryentry{forest}{name={forest}, description={A collection of merger trees that are linked together by virtue of nodes which jump between trees}}

\newglossaryentry{node}{name={node}, description={A single point in a merger tree, consisting of a dark matter halo and associated baryons}}

\newglossaryentry{mergee}{name={mergee}, description={For a given node in a merger tree, the set of mergee nodes consists of all nodes which will undergo a galaxy merger with the node at some point in the future}}

\newglossaryentry{primary progenitor}{name={primary progenitor}, description={The progenitor of a given \gls{node} which is regarding as the direct descendent of that \gls{node} (often, but not always, the most massive progenitor). Other progenitors are considered to merge into this primary progenitor}}

\newglossaryentry{parent}{name={parent}, description={In a merger tree, the parent node of any given node that exists at time $t_0$ is that node to which it is directly connected in the tree at time $t_1>t_0$}}

\newglossaryentry{Bernoulli distribution}{name={Bernoulli distribution}, description={A discrete probability distribution which takes value $1$ with success probability $p$ and value $0$ with failure probability $q = 1-p$. Read more on \href{http://en.wikipedia.org/wiki/Bernoulli_distribution}{Wikipedia}}}

\newglossaryentry{UUID}{name={UUID}, description={A \href{http://en.wikipedia.org/wiki/Universally_unique_identifier}{universally unique identifier}---this is a label which uniquely identifies some object (in this case, a \glc\ model)}}

\newglossaryentry{ABmagnitude}{name={AB magnitude}, description={An astronomical magnitude system in which the apparent magnitude is defined as $m=-2.5\log_{10}f-48.60$ for a flux density, $f$, measured in ergs per second per square centimeter per hertz}}

\newglossaryentry{maggie}{name={maggie}, description={A unit of luminosity defined to be equal to the luminosity of a zeroth magnitude object in the \gls{ABmagnitude} system}}

\newglossaryentry{forwardDescendent}{name={forward descendent}, description={The node with which the mass (or majority of the mass) of a node will become associated with at a later time. This type of descendent is usually relevant when considering how halos and galaxies evolve forward in time and should be distinguished from a \gls{backwardDescendent} which is relevant when building merger trees}}

\newglossaryentry{backwardDescendent}{name={backward descendent}, description={The \gls{primary progenitor} of a node. This type of descendent is usually relevant when building merger trees and should be distinguished from a \gls{forwardDescendent} which is relevant when considering how halos and galaxies evolve forward in time}}

\newglossaryentry{MD5hash}{name={MD5 hash}, description={The \href{http://en.wikipedia.org/wiki/MD5}{MD5 Message-Digest Algorithm} is a widely used cryptographic hash function that produces a 128-bit (16-byte) hash value. In \glc\ it is used to encode unique labels for modules which are incorporated into file names. \glc\ uses the \href{http://www.gnu.org/software/libc/}{\normalfont \ttfamily glibc} \href{http://en.wikipedia.org/wiki/Crypt_(Unix)}{\normalfont \ttfamily crypt()} function to compute MD5 hashes, but switches ``{\normalfont \ttfamily /}'' for ``{\normalfont \ttfamily @}'' in the hash (since ``{\normalfont \ttfamily /}'' is inconvenient for use in file names)}}

\newglossaryentry{LymanContinuum}{name={Lyman continuum}, description={The part of the electromagnetic spectrum which is capable of ionizing hydrogen, i.e. photons with wavelengths shorter than 91.1267~nanometres and with energy above 13.6~eV}}

\newglossaryentry{millenniumSimulation}{name={Millennium Simulation}, description={The \href{http://www.mpa-garching.mpg.de/galform/virgo/millennium/}{Millennium Simulation} is a high-resolution N-body simulation of structure formation in a cold dark matter universe.}}

\newglossaryentry{yeppp}{name={YEPPP!}, description={The \href{http://www.yeppp.info/}{Yeppp!} library provides high-performance SIMD-optimized mathematical functions optimized for a variety of architectures.}}

\newglossaryentry{graphviz}{name={GraphViz}, description={\href{http://www.graphviz.org/}{\sc GraphViz} is an open source graph visualization package. It is used by \glc\ in making diagrams of merger trees.}}

\newglossaryentry{latinhypercube}{name={Latin hypercube}, description={A \href{http://en.wikipedia.org/wiki/Latin_hypercube_sampling}{Latin hypercube} is a construct for generating a sample of plausible collections of parameter values from a multidimensional distribution}}

\newglossaryentry{maximin}{name={maximin}, description={In \gls{latinhypercube} design, the ``maximin'' approach generates a large number of \glspl{latinhypercube} and selects the one which has the greatest minimum distance between all pairs of points in the hypercube}}

\newglossaryentry{mangle}{name={{\normalfont \scshape mangle}}, description={\href{http://space.mit.edu/~molly/mangle/}{\normalfont \scshape mangle} is a software package used for defining angular masks on the surface of a sphere. It is used primarily for defining the geometry of galaxy surveys}}

\newglossaryentry{irate}{name={IRATE}, description={\href{http://www.physics.uci.edu/~etolleru/irate-docs/formatspec.html}{IRATE} is file format designed for N-body simulation particle, halo, merger tree, and galaxy data}}

\newglossaryentry{ann}{name={Approximate Nearest Neighbor}, description={The \href{http://www.cs.umd.edu/~mount/ANN/}{\normalfont \scshape Approximate Nearest Neighbor} library is a software package used finding the closest set of points to a given point, approximately}}

\newglossaryentry{enzo}{name={ENZO}, description={\href{http://enzo-project.org/}{Enzo} is an adaptive mesh refinement hydrodynamics code}}

\newglossaryentry{adafg}{name={ADAF},
    description={An advection-dominated accretion flow (ADAF) is a particular solution for an accretion flow around a black hole, star or compact object in which energy liberated by viscous forces is stored within the accretion flow and advected inward to the central object (see \citealt{narayan_advection-dominated_1998})}}

\newglossaryentry{pahg}{name={PAH},
    description={\href{http://en.wikipedia.org/wiki/Polycyclic_aromatic_hydrocarbon}{Polycyclic aromatic hydrocarbons} (PAH) are large organic molecules consisting of fused aromatic rings}}

\newglossaryentry{dslg}{name={DSL},
    description={\href{http://en.wikipedia.org/wiki/Domain-specific_language}{Domain-specific languages} (DSL) are a type of programming language dedicated to a particular problem. In \glc\ a DSL is used to specify the structure of \glspl{component}}}

\newglossaryentry{cdmg}{name={CDM},
    description={\href{http://en.wikipedia.org/wiki/Cold_dark_matter}{Cold dark matter} (CDM) is a hypothesized type of dark matter in which the particles move slowly compared to the speed of light}}

\newglossaryentry{wdmg}{name={WDM},
    description={\href{http://en.wikipedia.org/wiki/Warm_dark_matter}{Warm dark matter} (WDM) is a hypothesized type of dark matter in which the particle has non-negligible thermal velocity at decoupling}}

\newglossaryentry{samg}{name={SAM},
    description={Semi-analytic models (SAMs) are a type of galaxy formation model utilizing simple parameterizations of physical processes to follow the evolution of galaxies through a merging hierarchy of galaxies.}}

\newglossaryentry{ahfg}{name={AHF},
    description={\href{http://popia.ft.uam.es/AHF/files/AHF.pdf}{Amiga's Halo Finder} (AHF) is a software package which identifies dark matter halos in N-body simulations. Full details are given by \cite{gill_evolution_2004} and \cite{knollmann_ahf:_2009}}}

\newglossaryentry{hodg}{name={HOD},
    description={A halo occupation distribution (HOD) is a mathematical model describing the distribution of the number of galaxies (of some given physical properties) found in a dark matter halo of given mass.}}

\newglossaryentry{bieg}{name={BIE},
    description={The \href{http://www.astro.umass.edu/BIE/}{Bayesian Inference Engine} (BIE) is a software package designed to facilitate exploration of complexes parameter spaces using Bayesian techniques.}}

\newglossaryentry{mpig}{name={MPI},
    description={\href{http://en.wikipedia.org/wiki/Message_Passing_Interface}{Message Passing Interface} (MPI) is a standard for passing messages between processes on parallel computers.}}

\newglossaryentry{pbsg}{name={PBS},
    description={\href{http://en.wikipedia.org/wiki/Portable_Batch_System}{Portable Batch System} (PBS) is a job scheduler used on many compute cluster environments.}}

\newglossaryentry{pdfg}{name={PDF},
    description={\href{http://en.wikipedia.org/wiki/Probability_density_function}{Probability Density Function} (PDF) is a function which describes the probability for a random variable to take on a given value.}}

\newglossaryentry{gslg}{name={GSL},
    description={\href{http://www.gnu.org/software/gsl/}{GNU Scientific Library} (GSL) is a library providing a variety of numerical algorithms.}}
