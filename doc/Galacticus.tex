\documentclass[letterpaper,10pt,headsepline]{scrbook}
\usepackage[T1]{fontenc} 
\usepackage{natbib}
\usepackage{supertabular}
\usepackage{epsfig}
\usepackage{ifthen}
\usepackage{index}
\usepackage[backref,colorlinks]{hyperref}
\usepackage{listings}
\usepackage{verbatim}
\usepackage{hyphenat}
\usepackage{ragged2e}
\usepackage[acronym]{glossaries}
\usepackage{color}
\usepackage{tensor}
\usepackage{textcomp}
\usepackage{amssymb}
\usepackage{amsmath}
\usepackage{bm}

% Adaptive labelling.
\makeatletter
\newcommand{\iflabelexists}[3]{\@ifundefined{r@#1}{#3}{#2}}
\makeatother

% Table of contents
\setcounter{tocdepth}{5}

% Margins.
\setlength{\topmargin}{0mm}
\setlength{\textwidth}{160mm}
\setlength{\textheight}{210mm}
\setlength{\oddsidemargin}{0mm}
\setlength{\evensidemargin}{0mm}

% Names
\def\glc{{\normalfont \scshape Galacticus}}

% Physical constants.
\def\G{\mathrm{G}}
\def\clight{\mathrm{c}}
\def\d{\mathrm{d}}
\def\e{\mathrm{e}}

% Variable specifiers.
\def\void{\textcolor{red}{\textless void\textgreater}}
\def\logicalzero{\textcolor{red}{\textless logical\textgreater}}
\def\logicalone{\textcolor{red}{\textless logical(:)\textgreater}}
\def\logicaltwo{\textcolor{red}{\textless logical(:,:)\textgreater}}
\def\intzero{\textcolor{red}{\textless integer\textgreater}}
\def\intone{\textcolor{red}{\textless integer(:)\textgreater}}
\def\inttwo{\textcolor{red}{\textless integer(:,:)\textgreater}}
\def\doublezero{\textcolor{red}{\textless double\textgreater}}
\def\doubleone{\textcolor{red}{\textless double(:)\textgreater}}
\def\doubletwo{\textcolor{red}{\textless double(:,:)\textgreater}}
\def\doublethree{\textcolor{red}{\textless double(:,:,:)\textgreater}}
\def\enum{\textcolor{red}{\textless enumeration\textgreater}}
\def\argin{\textcolor{blue}{$\rightarrow$}}
\def\arginout{\textcolor{blue}{$\leftrightarrow$}}
\def\argout{\textcolor{blue}{$\leftarrow$}}
\include{autoEnumerationSpecifiers}
\include{dataEnumerationSpecifiers}

\makeglossary
\glstoctrue
\makeindex
\newindex{code}{cdx}{cnd}{Code Index}

% Acronyms.
\newacronym{cgm}{CGM}{circum-galactic medium}
\newacronym{cmb}{CMB}{cosmic microwave background}
\newacronym{igm}{IGM}{intergalactic medium}
\newacronym{imf}{IMF}{initial mass function}
\newacronym{isco}{ISCO}{innermost stable circular orbit}
\newacronym{ism}{ISM}{interstellar medium}
\newacronym{ode}{ODE}{ordinary differential equation}
\newacronym{nfw}{NFW}{Navarro-Frenk-White (dark matter halo profile)}
\newacronym{sed}{SED}{spectral energy distribution}
\newacronym{sne}{SNe}{supernovae}
\newglossaryentry{adaf}{type=\acronymtype, name={ADAF}, description=\glslink{adafg}{advection-dominated accretion flow}, first={advection-dominated accretion flow (ADAF)}}
\newglossaryentry{pah}{type=\acronymtype, name={PAH}, description=\glslink{pahg}{polycyclic aromatic hydrocarbon}, first={polycyclic aromatic hydrocarbon (PAH)}, firstplural={polycyclic aromatic hydrocarbons (PAH)}}
\newglossaryentry{dsl}{type=\acronymtype, name={DSL}, description=\glslink{dslg}{domain-specific language}, first={domain specific language (DSL)}, firstplural={domain-specific languages (DSLs)}}
\newglossaryentry{wdm}{type=\acronymtype, name={WDM}, description=\glslink{wdmg}{warm dark matter}, first={warm dark matter (WDM)}, firstplural={warm dark matters (WDM)}}
\newacronym{sdss}{SDSS}{Sloan Digital Sky Survey}
\newglossaryentry{ahf}{type=\acronymtype, name={AHF}, description=\glslink{ahfg}{Amiga's Halo Finder}, first={Amiga's Halo Finder (AHF)}}
\newglossaryentry{cdm}{type=\acronymtype, name={CDM}, description=\glslink{cdmg}{cold dark matter}, first={cold dark matter (CDM)}, firstplural={cold dark matters (CDM)}}
\newglossaryentry{fdm}{type=\acronymtype, name={FDM}, description=\glslink{fdmg}{fuzzy dark matter}, first={fuzzy dark matter (FDM)}, firstplural={fuzzy dark matters (FDM)}}
\newacronym{mcmc}{MCMC}{Markov Chain Monte Carlo}
\newglossaryentry{sam}{type=\acronymtype, name={SAM}, description=\glslink{samg}{semi-analytic model}, first={semi-analytic model (SAM)}, firstplural={semi-analytic models (SAMs)}}
\newglossaryentry{hod}{type=\acronymtype, name={HOD}, description=\glslink{hodg}{halo occupation distribution}, first={halo occupation distribution (HOD)}, firstplural={halo occupation distributions (HODs)}}
\newglossaryentry{mpi}{type=\acronymtype, name={MPI}, description=\glslink{mpig}{message passing interface}, first={message passing interface (MPI)}, firstplural={message passing interfaces (MPIs)}}
\newglossaryentry{hdf5}{type=\acronymtype, name={HDF5}, description=\glslink{hdf5g}{hierarchical data format}, first={hierarchical data format (HDF5)}, firstplural={hierarchical data formats (HDF5s)}}
\newglossaryentry{pbs}{type=\acronymtype, name={PBS}, description=\glslink{pbsg}{portable batch system}, first={portable batch system (PBS)}, firstplural={portable batch systems (PBSs)}}
\newglossaryentry{pdf}{type=\acronymtype, name={PDF}, description=\glslink{pdfg}{probability density function}, first={probability density function (PDF)}, firstplural={probability density functions (PDFs)}}
\newglossaryentry{cdf}{type=\acronymtype, name={CDF}, description=\glslink{cdfg}{cumulative distribution function}, first={cumulative distribution function (CDF)}, firstplural={cumulative distribution functions (CDFs)}}
\newglossaryentry{gsl}{type=\acronymtype, name={GSL}, description=\glslink{gslg}{GNU Scientific Library}, first={GNU Scientific Library (GSL)}}
\newglossaryentry{gama}{type=\acronymtype, name={GAMA}, description=\glslink{gamag}{Galaxy and Mass Assembly}, first={Galaxy and Mass Assembly (GAMA)}}
\newglossaryentry{nsc}{type=\acronymtype, name={NSC}, description=\glslink{nscg}{nuclear star cluster}, first={nuclear star cluster (NSC)}, firstplural={nuclear star clusters (NSC)}}

% Glossary entries.
\newglossaryentry{FFTLog}{name={FFTLog}, description={\href{https://jila.colorado.edu/~ajsh/FFTLog/index.html}{FFTLog} is a code to compute fast Fourier transforms of discrete periodic sequences of logarithmically spaced data}}

\newglossaryentry{component}{name={component}, description={An individual physical system within a \gls{node}, such as a dark matter halo, a galactic disk or a supermassive black hole}}

\newglossaryentry{forest}{name={forest}, description={A collection of merger trees that are linked together by virtue of nodes which jump between trees}}

\newglossaryentry{node}{name={node}, description={A single point in a merger tree, consisting of a dark matter halo and associated baryons}}

\newglossaryentry{mergee}{name={mergee}, description={For a given node in a merger tree, the set of mergee nodes consists of all nodes which will undergo a galaxy merger with the node at some point in the future}}

\newglossaryentry{primary progenitor}{name={primary progenitor}, description={The progenitor of a given \gls{node} which is regarding as the direct descendant of that \gls{node} (often, but not always, the most massive progenitor). Other progenitors are considered to merge into this primary progenitor}}

\newglossaryentry{parent}{name={parent}, description={In a merger tree, the parent node of any given node that exists at time $t_0$ is that node to which it is directly connected in the tree at time $t_1>t_0$}}

\newglossaryentry{Bernoulli distribution}{name={Bernoulli distribution}, description={A discrete probability distribution which takes value $1$ with success probability $p$ and value $0$ with failure probability $q = 1-p$. Read more on \href{http://en.wikipedia.org/wiki/Bernoulli_distribution}{Wikipedia}}}

\newglossaryentry{UUID}{name={UUID}, description={A \href{http://en.wikipedia.org/wiki/Universally_unique_identifier}{universally unique identifier}---this is a label which uniquely identifies some object (in this case, a \glc\ model)}}

\newglossaryentry{ABmagnitude}{name={AB magnitude}, description={An astronomical magnitude system in which the apparent magnitude is defined as $m=-2.5\log_{10}f-48.60$ for a flux density, $f$, measured in ergs per second per square centimeter per hertz}}

\newglossaryentry{maggie}{name={maggie}, description={A unit of luminosity defined to be equal to the luminosity of a zeroth magnitude object in the \gls{ABmagnitude} system}}

\newglossaryentry{forwardDescendant}{name={forward descendant}, description={The node with which the mass (or majority of the mass) of a node will become associated with at a later time. This type of descendant is usually relevant when considering how halos and galaxies evolve forward in time and should be distinguished from a \gls{backwardDescendant} which is relevant when building merger trees}}

\newglossaryentry{backwardDescendant}{name={backward descendant}, description={The \gls{primary progenitor} of a node. This type of descendant is usually relevant when building merger trees and should be distinguished from a \gls{forwardDescendant} which is relevant when considering how halos and galaxies evolve forward in time}}

\newglossaryentry{dmou}{name={dark matter-only universe}, description={Masses of nodes/halos are often defined as the mass which the node/halo would have in a dark matter-only universe in which the total matter density is the same as the actual model universe, but in which no baryons are present (i.e. all matter is dark matter)}}

\newglossaryentry{MD5hash}{name={MD5 hash}, description={The \href{http://en.wikipedia.org/wiki/MD5}{MD5 Message-Digest Algorithm} is a widely used cryptographic hash function that produces a 128-bit (16-byte) hash value. In \glc\ it is used to encode unique labels for modules which are incorporated into file names. \glc\ uses the \href{http://www.gnu.org/software/libc/}{\normalfont \ttfamily glibc} \href{http://en.wikipedia.org/wiki/Crypt_(Unix)}{\normalfont \ttfamily crypt()} function to compute MD5 hashes, but switches ``{\normalfont \ttfamily /}'' for ``{\normalfont \ttfamily @}'' in the hash (since ``{\normalfont \ttfamily /}'' is inconvenient for use in file names)}}

\newglossaryentry{LymanContinuum}{name={Lyman continuum}, description={The part of the electromagnetic spectrum which is capable of ionizing hydrogen, i.e. photons with wavelengths shorter than 91.1267~nanometers and with energy above 13.6~eV}}

\newglossaryentry{millenniumSimulation}{name={Millennium Simulation}, description={The \href{http://www.mpa-garching.mpg.de/galform/virgo/millennium/}{Millennium Simulation} is a high-resolution N-body simulation of structure formation in a cold dark matter universe.}}

\newglossaryentry{openmp}{name={OpenMP}, description={\href{https://www.openmp.org/}{OpenMP} is an API for shared memory parallel programming.}}

\newglossaryentry{graphviz}{name={{\normalfont \scshape GraphViz}}, description={\href{http://www.graphviz.org/}{\normalfont \scshape GraphViz} is an open source graph visualization package. It is used by \glc\ in making diagrams of merger trees.}}

\newglossaryentry{latinhypercube}{name={Latin hypercube}, description={A \href{http://en.wikipedia.org/wiki/Latin_hypercube_sampling}{Latin hypercube} is a construct for generating a sample of plausible collections of parameter values from a multidimensional distribution}}

\newglossaryentry{deadlock}{name={deadlock}, description={A deadlock describes a situation in which no node in a merger tree (or forest) can be evolved further forward in time due to the existence of circular dependencies between nodes. Deadlocks can occur due to incompatible parameter choices, or may indicate a bug in \glc.}}

\newglossaryentry{maximin}{name={maximin}, description={In \gls{latinhypercube} design, the ``maximin'' approach generates a large number of \glspl{latinhypercube} and selects the one which has the greatest minimum distance between all pairs of points in the hypercube}}

\newglossaryentry{mangle}{name={{\normalfont \scshape mangle}}, description={\href{http://space.mit.edu/~molly/mangle/}{\normalfont \scshape mangle} is a software package used for defining angular masks on the surface of a sphere. It is used primarily for defining the geometry of galaxy surveys}}

\newglossaryentry{irate}{name={IRATE}, description={\href{https://irate-format.readthedocs.io/en/latest/}{IRATE} is file format designed for N-body simulation particle, halo, merger tree, and galaxy data}}

\newglossaryentry{ann}{name={Approximate Nearest Neighbor}, description={The \href{http://www.cs.umd.edu/~mount/ANN/}{\normalfont \scshape Approximate Nearest Neighbor} library is a software package used finding the closest set of points to a given point, approximately}}

\newglossaryentry{enzo}{name={ENZO}, description={\href{http://enzo-project.org/}{Enzo} is an adaptive mesh refinement hydrodynamics code}}

\newglossaryentry{adafg}{name={ADAF},
    description={An advection-dominated accretion flow (ADAF) is a particular solution for an accretion flow around a black hole, star or compact object in which energy liberated by viscous forces is stored within the accretion flow and advected inward to the central object (see \citealt{narayan_advection-dominated_1998})}}

\newglossaryentry{pahg}{name={PAH},
    description={\href{http://en.wikipedia.org/wiki/Polycyclic_aromatic_hydrocarbon}{Polycyclic aromatic hydrocarbons} (PAH) are large organic molecules consisting of fused aromatic rings}}

\newglossaryentry{dslg}{name={DSL},
    description={\href{http://en.wikipedia.org/wiki/Domain-specific_language}{Domain-specific languages} (DSL) are a type of programming language dedicated to a particular problem. In \glc\ a DSL is used to specify the structure of \glspl{component}}}

\newglossaryentry{cdmg}{name={CDM},
    description={\href{http://en.wikipedia.org/wiki/Cold_dark_matter}{Cold dark matter} (CDM) is a hypothesized type of dark matter in which the particles move slowly compared to the speed of light}}

\newglossaryentry{fdmg}{name={FDM},
    description={\href{https://en.wikipedia.org/wiki/Fuzzy_cold_dark_matter}{Fuzzy dark matter} (FDM) is a hypothesized type of dark matter consisting of extremely light scalar particles with masses on the order of $10^{-22}$~eV}}

\newglossaryentry{nscg}{name={NSC},
    description={\href{https://en.wikipedia.org/wiki/Nuclear_star_cluster}{Nuclear star clusters} (NSC) are high-density star clusters found in the centers of some galaxies}}

\newglossaryentry{wdmg}{name={WDM},
    description={\href{http://en.wikipedia.org/wiki/Warm_dark_matter}{Warm dark matter} (WDM) is a hypothesized type of dark matter in which the particle has non-negligible thermal velocity at decoupling}}

\newglossaryentry{samg}{name={SAM},
    description={Semi-analytic models (SAMs) are a type of galaxy formation model utilizing simple parameterizations of physical processes to follow the evolution of galaxies through a merging hierarchy of galaxies.}}

\newglossaryentry{ahfg}{name={AHF},
    description={\href{http://popia.ft.uam.es/AHF/files/AHF.pdf}{Amiga's Halo Finder} (AHF) is a software package which identifies dark matter halos in N-body simulations. Full details are given by \cite{gill_evolution_2004} and \cite{knollmann_ahf:_2009}}}

\newglossaryentry{hodg}{name={HOD},
    description={A halo occupation distribution (HOD) is a mathematical model describing the distribution of the number of galaxies (of some given physical properties) found in a dark matter halo of given mass.}}

\newglossaryentry{hdf5g}{name={HDF5},
    description={The \href{https://www.hdfgroup.org/solutions/hdf5/}{hierarchical data format} (version 5; HDF5) is a file format designed for storing scientific data.}}

\newglossaryentry{mpig}{name={MPI},
    description={\href{http://en.wikipedia.org/wiki/Message_Passing_Interface}{Message Passing Interface} (MPI) is a standard for passing messages between processes on parallel computers.}}

\newglossaryentry{pbsg}{name={PBS},
    description={\href{http://en.wikipedia.org/wiki/Portable_Batch_System}{Portable Batch System} (PBS) is a job scheduler used on many compute cluster environments.}}

\newglossaryentry{pdfg}{name={PDF},
    description={\href{http://en.wikipedia.org/wiki/Probability_density_function}{Probability Density Function} (PDF) is a function which describes the probability for a random variable to take on a given value.}}

\newglossaryentry{cdfg}{name={CDF},
    description={\href{https://en.wikipedia.org/wiki/Cumulative_distribution_function}{Cumulative Distribution Function} (CDF) is a function which describes the cumulative probability for a random variable to be equal to or less than a given value.}}

\newglossaryentry{gslg}{name={GSL},
    description={\href{http://www.gnu.org/software/gsl/}{GNU Scientific Library} (GSL) is a library providing a variety of numerical algorithms.}}

\newglossaryentry{gamag}{name={GAMA},
    description={The \href{http://www.gama-survey.org/}{Galaxy and Mass Assembly} (GAMA) survey is a spectroscopic survey of $\approx 300,000$ galaxies down to r $<$ 19.8 mag over $\approx 286$ deg$^2$.}}

\newglossaryentry{cloudy}{name={Cloudy}, description={\href{http://www.nublado.org/}{Cloudy} is a code to compute the ionization structure of HII regions}}

\newglossaryentry{camb}{name={CAMB}, description={\href{https://camb.info/}{CAMB} is a code to compute anisotropies in the cosmic microwave background, and the linear theory power spectra of matter and radiation}}

\newglossaryentry{axioncamb}{name={AxionCAMB}, description={\href{https://github.com/dgrin1/axionCAMB}{AxionCAMB} is a code to compute anisotropies in the cosmic microwave background, and the linear theory power spectra of matter and radiation including an axion component in the dark matter}}

\newglossaryentry{class}{name={CLASS}, description={\href{https://lesgourg.github.io/class_public/class.html}{CLASS} is a code to compute anisotropies in the cosmic microwave background, and the linear theory power spectra of matter and radiation}}

\newglossaryentry{recfast}{name={RecFast}, description={\href{https://www.astro.ubc.ca/people/scott/recfast.html}{RecFast} is a code to calculate the recombination history of the universe}}

\newglossaryentry{xml}{name={XML}, description={\href{https://en.wikipedia.org/wiki/XML}{XML} is a markup language, used for \glc\ parameter files}}


\begin{document}

\lstset{language=[95]Fortran}

\frontmatter

\pagestyle{empty}
\begin{center}
\includegraphics[width=125mm]{GalacticusLogo.png}\\

\includegraphics{New_Logo_Galaxy_192_Transparent.png}\\
A semi-analytic galaxy formation code.\\

\copyright\ 2009, 2010 2011, 2012, 2013, 2014, 2015, 2016, 2017, 2018, 2019, 2020 Andrew Benson
\end{center}

\tableofcontents

\mainmatter
\pagestyle{headings}

\part{Installation and Basic Use}

\chapter{About Galacticus}

At its core, \glc\ is a semi-analytic model of galaxy formation. It solves equations describing how galaxies evolve in a merging hierarchy of dark matter halos in a cold dark matter universe. \glc\ has much in common with other semi-analytic models, such as the range of physical processes included and the type of quantities that it can predict. \glc\ also provides a wide variety of other functionality, such as computing halo mass functions, power spectra, analyzing particle simulations, and performing \gls{mcmc} simulations.

In designing \glc\ our main goal was to make the code flexible, modular and easily extensible. Much greater priority was placed on making the code easy to use and modify than on making it fast. We believe that a modular and extensible nature is crucial as galaxy formation is an evolving science. In particular, key design features are:
\begin{description}
 \item [Extensible implementations for all functions:] Essentially all functions within \glc\ are designed to be extensible following an Object Oriented methodology, meaning that you can write your own version and insert it into \glc\ easily. For example, suppose you want to use an improved functional form for the \gls{cdm} halo mass function. You would simply write a new {\normalfont \ttfamily haloMassFunction} class that computes this mass function, decorate it with a short directive (see \href{https://github.com/galacticusorg/galacticus/releases/download/masterRelease/Galacticus_Development.pdf#sec.CodeDirectives}{here}) which explains to the build system how to insert this class into \glc. A recompile of the code will then incorporate your new function.

 \item [Extensible components for tree nodes:] The basic structure in \glc\ is a merger tree, which consists of a linked tree of nodes (each corresponding a dark matter halo and its content) which have various properties. \glc\ works by evolving the nodes forwards in time subject to a collection of differential equations and other rules. Each node can contain an arbitrary number of \emph{components}. A component may be a dark matter halo, a galactic disk, a black hole etc. Each component may have an arbitrary number of \emph{properties} (some of which may be evolving, others of which can be fixed). \glc\ makes it easy to add additional components. For example, suppose you wanted to add a ``stellar halo'' components (consisting of stars stripped from satellite galaxies). To do this, you would write a module which specifies the following for this component:
 \begin{itemize}
  \item Properties (their names, types, and ranks);
  \item Functions describing the differential equations which govern the evolution of the properties;
  \item Functions describing how the component responds to various events (e.g. the node becoming a satellite, a galaxy-galaxy merger, etc.);
  \item ``Pipes'' which allow for flows of mass/energy/etc. from one component to another.
 \end{itemize}
 Short directives embedded in this module explain to the \glc\ build system how to incorporate the new component. A recompile will then build your new component into \glc. Typically, a new component can be created quickly by copying an existing one and modifying it as necessary. Furthermore, multiple implementations of a component are allowed. For example, \glc\ contains a component which tracks the scale length of the dark matter halo. You could add a new component which additionally tracks the axis ratios of the (now triaxial) halo. A simple input parameter then allows you to select which implementation will be used in a given run.

 \item [Centralized ODE solver:] \glc\ evolves nodes in merger trees by calling an ODE solver which integrates forwards in time to solve for the evolution of the properties of each component in a node. This means that you do not need to provide explicit solutions for ODEs (in many cases such solutions are not available anyway) and timestepping is automatically handled to achieve a specified level of precision. The ODE solver allows for the evolution to be interrupted. A component may trigger an interrupt at any time and may do so for a number of reasons. A typical use is to actually create a component within a given node---for example when gas first begins to cool and inflow in a node the disk component must be created. Other uses include interrupting evolution when a merging event occurs.
\end{description}

\section{Getting Galacticus}

Downloads and installation instructions can be found at \href{https://github.com/galacticusorg/galacticus/wiki#how-do-i-install-and-use-galacticus}{\normalfont \ttfamily https://github.com/galacticusorg/galacticus/wiki\#how-do-i-install-and-use-galacticus}.

\section{License}

Copyright 2009, 2010, 2011, 2012, 2013, 2014, 2015, 2016, 2017, 2018, 2019, 2020, Andrew Benson \href{mailto:abenson@carnegiescience.edu}{\normalfont \ttfamily <abenson@carnegiescience.edu>}\\

\glc\ is free software: you can redistribute it and/or modify
it under the terms of the GNU General Public License as published by
the Free Software Foundation, either version 3 of the License, or
(at your option) any later version.

\glc\ is distributed in the hope that it will be useful,
but WITHOUT ANY WARRANTY; without even the implied warranty of
MERCHANTABILITY or FITNESS FOR A PARTICULAR PURPOSE.  See the
GNU General Public License for more details.

You should have received a copy of the GNU General Public License
along with \glc.  If not, see \href{http://www.gnu.org/licenses/}{\normalfont \ttfamily <http://www.gnu.org/licenses/>}.


\chapter{Running Galacticus}

If you have not yet installed \glc\ you should follow the instructions \href{https://github.com/galacticusorg/galacticus/wiki#how-do-i-install-and-use-galacticus}{here} to do so.

\section{Setting the Environment}

Before running \glc\ you will need to set two environment variables which specify where the \glc\ source code and datasets can be found. First, the environment variable {\normalfont \ttfamily GALACTICUS\_EXEC\_PATH} should be set to the full path to the build directory\index{path!galacticus root@{\glc\ root}}. Second, the environment variable {\normalfont \ttfamily GALACTICUS\_DATA\_PATH} should be set to the full path to the {\normalfont \ttfamily datasets} directory which was created when you installed \glc.

\section{Running Galacticus}

\glc\ is run using
\begin{verbatim}
 Galacticus.exe <parameterFile>
\end{verbatim}
where {\normalfont \ttfamily parameterFile} is the name of the file containing parameter settings for \glc. \glc\ will display messages indicating its progress as it runs.

A simple example, useful to check that everything is working as expected for you, is to do:
\begin{verbatim}
 Galacticus.exe $GALACTICUS_EXEC_PATH/parameters/quickTest.xml
\end{verbatim}
This will run a small model. You should expect to see output which looks something like this:
\begin{verbatim}
              ##                                     
   ####        #                  #                  
  #   #        #             #                       
 #       ###   #  ###   ### ###  ##   ### ## ##   ## 
 #       #  #  #  #  # #  #  #    #  #  #  #  #  #   
 #   ###  ###  #   ### #     #    #  #     #  #   #  
  #   #  #  #  #  #  # #     #    #  #     #  #    # 
   ####  #### ### ####  ###   ## ###  ###   #### ##  

  2009, 2010, 2011, 2012, 2013, 2014, 2015, 2016,
  2017, 2018, 2019, 2020, 2021, 2022
   - Andrew Benson

MM: -> Begin task: merger tree evolution
 0:     -> Evolving tree number 1 {1}
 9:     -> Evolving tree number 3 {1}
 7:     -> Evolving tree number 4 {1}
 3:     -> Evolving tree number 2 {1}
 0:         Output tree data at t=  13.79 Gyr
 1:     -> Evolving tree number 5 {1}
11:     -> Evolving tree number 6 {1}
 0:     <- Finished tree
 9:         Output tree data at t=  13.79 Gyr
 9:     <- Finished tree
 3:         Output tree data at t=  13.79 Gyr
 3:     <- Finished tree
 7:         Output tree data at t=  13.79 Gyr
 7:     <- Finished tree
 1:         Output tree data at t=  13.79 Gyr
 1:     <- Finished tree
11:         Output tree data at t=  13.79 Gyr
11:     <- Finished tree
MM: <- Done task: merger tree evolution
\end{verbatim}
In this simple model, \glc\ is told to build six merger trees, and form galaxies within them, outputting the results at $z=0$. After the initial ``\glc'' banner is displayed there are several messages shown. Each line begins with either ``{\normalfont \ttfamily MM:}'', or a number. \glc\ runs in parallel using the available cores on your computer---these prefixes on each line tell you which parallel thread is sending the message. ``{\normalfont \ttfamily MM:}'' means that the message is from the main thread (and that \glc\ is currently running in a serial part of the code), while a numerical prefixes gives the number of that parallel thread reporting the message (starting from 0).

The first message, from the main thread, tells you that the ``merger tree evolution'' task has begun---in this task, \glc\ forms galaxies within a set of merger trees. After that initial message you'll see that \glc\ enters parallel calculation and messages start being reported from each parallel thread. The threads report when they begin evolving a merger tree (in this case merger trees are numbered consecutively, starting from 1), when they are outputting the data from that tree (at a time of $13.79$~Gyr, which corresponds to $z=0$ in this model), and when they have finished the tree. Once all trees are finished there is a final message from the main thread (parallel calculation is over at this point) indicating that the merger tree evolution task is finished. \glc\ then exits, since it has finished all of the work we asked it to do.

The results will have been written to an output file named {\normalfont \ttfamily galacticus.hdf5}, which you can see by doing:
\begin{verbatim}
ls -l galacticus.hdf5
\end{verbatim}
which will show something similar to:
\begin{verbatim}
-rw-r--r-- 1 abenson users 568601 Oct 15 20:58 galacticus.hdf5
\end{verbatim}
\glc\ outputs use the \href{https://www.hdfgroup.org/solutions/hdf5/}{HDF5} format---libraries for accessing HDF5 files exist in all major languages, including Python\footnote{\href{https://www.h5py.org/}{{\normalfont \scshape h5py}}}. We will explore the structure of the output file in more detail in \S\ref{sec:outputFile}, but for now we'll simply explore some of the key features using the command line.

The content of the output file can be explored using the ``{\normalfont \ttfamily h5ls}'' tool, for example:
\begin{verbatim}
$ h5ls galacticus.hdf5
Build                    Group
Outputs                  Group
Parameters               Group
Version                  Group
\end{verbatim}
The file contains several groups---the data corresponding to the galaxies that were formed can be found in the {\normalfont \ttfamily Outputs} group:
\begin{verbatim}
$ h5ls galacticus.hdf5/Outputs
Output1                  Group
\end{verbatim}
In this case there is only a single output, corresponding to $z=0$, we can explore that group using:
\begin{verbatim}
$ h5ls galacticus.hdf5/Outputs/Output1
mergerTreeCount          Dataset {6/Inf}
mergerTreeIndex          Dataset {6/Inf}
mergerTreeSeed           Dataset {6/Inf}
mergerTreeStartIndex     Dataset {6/Inf}
mergerTreeWeight         Dataset {6/Inf}
nodeData                 Group
\end{verbatim}
In this output group we find some datasets which give information on the merger trees that were used, plus another group, ``{\normalfont \ttfamily nodeData}'', inside of which the galaxy data is stored:
\begin{verbatim}
$ h5ls galacticus.hdf5/Outputs/Output1/nodeData
basicMass                Dataset {7/Inf}
basicTimeLastIsolated    Dataset {7/Inf}
blackHoleCount           Dataset {7/Inf}
blackHoleMass            Dataset {7/Inf}
blackHoleSpin            Dataset {7/Inf}
darkMatterProfileScale   Dataset {7/Inf}
diskAbundancesGasMetals  Dataset {7/Inf}
diskAbundancesStellarMetals Dataset {7/Inf}
diskAngularMomentum      Dataset {7/Inf}
diskMassGas              Dataset {7/Inf}
diskMassStellar          Dataset {7/Inf}
diskRadius               Dataset {7/Inf}
diskVelocity             Dataset {7/Inf}
hotHaloAbundancesMetals  Dataset {7/Inf}
hotHaloAngularMomentum   Dataset {7/Inf}
hotHaloMass              Dataset {7/Inf}
hotHaloOuterRadius       Dataset {7/Inf}
hotHaloOutflowedAbundancesMetals Dataset {7/Inf}
hotHaloOutflowedAngularMomentum Dataset {7/Inf}
hotHaloOutflowedMass     Dataset {7/Inf}
hotHaloUnaccretedMass    Dataset {7/Inf}
nodeIndex                Dataset {7/Inf}
nodeIsIsolated           Dataset {7/Inf}
parentIndex              Dataset {7/Inf}
satelliteBoundMass       Dataset {7/Inf}
satelliteIndex           Dataset {7/Inf}
satelliteMergeTime       Dataset {7/Inf}
siblingIndex             Dataset {7/Inf}
spheroidAbundancesGasMetals Dataset {7/Inf}
spheroidAbundancesStellarMetals Dataset {7/Inf}
spheroidAngularMomentum  Dataset {7/Inf}
spheroidMassGas          Dataset {7/Inf}
spheroidMassStellar      Dataset {7/Inf}
spheroidRadius           Dataset {7/Inf}
spheroidVelocity         Dataset {7/Inf}
spinSpin                 Dataset {7/Inf}
\end{verbatim}
Each dataset here is an array containing the named property of each galaxy formed in the model. In this tiny example model only 7 galaxies were formed. To see the values of each property we can do:
\begin{verbatim}
$ h5ls -d galacticus.hdf5/Outputs/Output1/nodeData/diskMassStellar
diskMassStellar          Dataset {7/Inf}
    Data:
        (0) 0, 719592.675733525, 0, 36664762.1643418, 75533178.4182319, 0, 764729109.215361
\end{verbatim}
which lists the mass of stars in each galaxy disk (in units of $\mathrm{M}_\odot$) (note that some of them are zero---these halos in the merger tree either formed no galaxy, or formed a galaxy with no disk component).

These data can be extracted and analyzed using any software or language that supports reading HDF5 files.

\subsection{Dry Runs}

You can tell Galacticus to parse your parameter file, report any warnings, and write the parameters to the output HDF5 file, but then no nothing else (i.e. don't actually run the model) by adding the {\normalfont \ttfamily --dry-run} option, for example:
\begin{verbatim}
 Galacticus.exe parameters.xml --dry-run
\end{verbatim}
This can be useful to check that your parameter file is valid, and allow you to explore the values of any parameters that were set to defaults (as these will have been output to the HDF5 file).


\chapter{Input Parameters}

The following is an alphnumerically sorted list of all input parameters defined in \glc. Each parameter is listed by name, along with a description, default value (if one is specified in \glc), the file and program unit with which it is associated. Where relevant, references for parameters and the default values are given.

\input{dataParameters}



\chapter{Extracting and Analyzing Results}

\glc\ stores its output in an \href{http://www.hdfgroup.org/HDF5/}{HDF5} file. The contents of this file can be viewed and manipulated using a variety of ways including:
\begin{description}
 \item[\href{http://www.hdfgroup.org/hdf-java-html/hdfview/}{{\normalfont \scshape HDFView}}] This is a graphical viewer for exploring the contents of HDF5 files;
 \item[\href{https://portal.hdfgroup.org/display/HDF5/HDF5+Command-line+Tools}{HDF5 Command Line Tools}] A set of tools which can be used to extract data from HDF5 files (\href{https://portal.hdfgroup.org/display/HDF5/h5dump}{{\normalfont \ttfamily h5dump}} and \href{https://portal.hdfgroup.org/display/HDF5/h5ls}{{\normalfont \ttfamily h5ls}} are particularly useful);
 \item[\href{https://portal.hdfgroup.org/pages/viewpage.action?pageId=50073943}{C Fortran 90 APIs}] Allow access to, and manipulation of data in HDF5 files;
 \item[\href{https://www.h5py.org/}{{\normalfont \scshape h5py}}] A Python interface to HDF5 files.
\end{description}

In the remainder of this section the structure of \glc\ HDF5 files is described.

\section{General Structure of Output File}

Figure~\ref{fig:glcOutputFileStructure} shows the structure of a typical \glc\ output file. The various groups and subgroups are described below.

\begin{figure}
\begin{center}
\begin{verbatim}
outputFile.hdf5
 |
 +-> UUID                                     Attribute {1}
 |
 +-> Build                                    Group
 |    |
 |    +-> FoX_library_version                 Attribute {1}
 |    +-> GSL_library_version                 Attribute {1}
 |    +-> HDF5_library_version                Attribute {1}
 |    +-> make_CCOMPILER                      Attribute {1}
 |    +-> make_CCOMPILER_VERSION              Attribute {1}
 |    +-> make_CFLAGS                         Attribute {1}
 |    +-> make_CPPCOMPILER                    Attribute {1}
 |    +-> make_CPPCOMPILER_VERSION            Attribute {1}
 |    +-> make_CPPFLAGS                       Attribute {1}
 |    +-> make_FCCOMPILER                     Attribute {1}
 |    +-> make_FCCOMPILER_VERSION             Attribute {1}
 |    +-> make_FCFLAGS                        Attribute {1}
 |    +-> make_FCFLAGS_NOOPT                  Attribute {1}
 |    +-> make_MODULETYPE                     Attribute {1}
 |    +-> make_PREPROCESSOR                   Attribute {1}
 |    +-> sourceChangeSetDiff                 Dataset   {1}
 |    +-> sourceChangeSetMerge                Dataset   {1}
 |
 +-> Outputs                                  Group
 |    |
 |    +-> Output1                             Group
 |    |    |
 |    |    +-> nodeData                       Group
 |    |    |     |
 |    |    |     +-> nodeProperty1            Dataset {<nodeCount>}
 |    |    |     +-> ...                      Dataset {<nodeCount>}
 |    |    |     +-> ...                      Dataset {<nodeCount>}
 |    |    |     +-> ...                      Dataset {<nodeCount>}
 |    |    |     +-> nodePropertyN            Dataset {<nodeCount>}
 |    |    |
 |    |    +-> mergerTreeCount                Dataset {<treeCount>}
 |    |    |
 |    |    +-> mergerTreeIndex                Dataset {<treeCount>}
 |    |    |
 |    |    +-> mergerTreeStartIndex           Dataset {<treeCount>}
 |    |    |
 |    |    +-> mergerTreeWeight               Dataset {<treeCount>}
 |    |    |
 |    |    +-> mergerTree1                    Group              [optional]
 |    |    |     |
 |    |    |     +-> nodeProperty1            Reference
 |    |    |     +-> ...                      Reference
 |    |    |     +-> ...                      Reference
 |    |    |     +-> ...                      Reference
 |    |    |     +-> nodePropertyN            Reference
 |    |    |
 |    |    x-> ...                            Group              [optional]
 |    |    x-> ...                            Group              [optional]
 |    |    x-> ...                            Group              [optional]
 |    |    x-> mergerTree<treeCount>          Group              [optional]
 |    |    |
 |    |    +-> outputExpansionFactor          Attribute {1}
 |    |    +-> outputTime                     Attribute {1}
 |    |
 |    x-> Output2                             Group
 |
 +-> Filters                                  Group
 |    |
 |    +-> name                                Dataset   {<filterCount>}
 |    +-> wavelengthEffective                 Dataset   {<filterCount>}
 |
 +-> Parameters                               Group
 |    |
 |    +-> inputParameter1                     Attribute {}
 |    +-> ...                                 Attribute {}
 |    +-> ...                                 Attribute {}
 |    +-> ...                                 Attribute {}
 |    +-> inputParameterN                     Attribute {}
 |    +-> inputParameter1                     Group
 |         |
 |         +-> subInputParameter1             Attribute {}
 |         +-> ...                            Attribute {}
 |         +-> subInputParameterN             Attribute {}
 |    x-> ...                                 Attribute {}
 |    x-> ...                                 Attribute {}
 |    x-> ...                                 Attribute {}
 |    x-> inputParameterN                     Group
 |
 +-> Version                                  Group
 |    |
 |    +-> runTime                             Attribute {1}
 |    +-> versionMajor                        Attribute {1}
 |    +-> versionMinor                        Attribute {1}
 |    +-> versionRevision                     Attribute {1}
 |    +-> hgRevision                          Attribute {1}
 |    +-> hgHash                              Attribute {1}
 |    +-> runByName                           Attribute {1}
 |    +-> runByEmail                          Attribute {1}
 |
 +-> globalHistory                            Group
      |
      +-> historyExpansion                    Dataset {<historyCount>}
      +-> historyStarFormationRate            Dataset {<historyCount>}
      +-> historyTime                         Dataset {<historyCount>}
\end{verbatim}
\end{center}
\caption{Structure of a \glc\ HDF5 output file. {\normalfont \ttfamily <treeCount>} is the total number of merger trees present in a given output, and {\normalfont \ttfamily <nodeCount} is the total number of nodes (in all trees) present in an output.}
\label{fig:glcOutputFileStructure}
\end{figure}

\subsection{UUID}\label{sec:UUID}

The UUID (\href{https://secure.wikimedia.org/wikipedia/en/wiki/Universally_unique_identifier}{Universally Unique Identifier}) is a unique identifier assigned to each \glc\ model that is run. It allows identification of a given model and can be referenced from, for example, an external database.

\subsection{Build Information}\label{sec:BuildInformation}

\glc\ automatically stores various information about how it was built in the {\normalfont \ttfamily Build} group attributes. Currently, included attributes consist of:
\begin{description}
\item[{\normalfont \ttfamily FoX\_library\_version}] The version number of the FoX library;
\item[{\normalfont \ttfamily GSL\_library\_version}] The version number of the GSL library;
\item[{\normalfont \ttfamily HDF5\_library\_version}] The version number of the HDF5 library;
\item[{\normalfont \ttfamily make\_CCOMPILER}] The C compiler command used;
\item[{\normalfont \ttfamily make\_CCOMPILER\_VERSION}] The C compiler version information;
\item[{\normalfont \ttfamily make\_CFLAGS}] The flags passed to the C compiler;
\item[{\normalfont \ttfamily make\_CPPCOMPILER}] The C++ compiler command used;
\item[{\normalfont \ttfamily make\_CPPCOMPILER\_VERSION}] The C++ compiler version information;
\item[{\normalfont \ttfamily make\_CPPFLAGS}] The flags passed to the C++ compiler;
\item[{\normalfont \ttfamily make\_FCCOMPILER}] The Fortran compiler command used;
\item[{\normalfont \ttfamily make\_FCCOMPILER\_VERSION}] The Fortran compiler version information;
\item[{\normalfont \ttfamily make\_FCFLAGS}] The flags passed to the Fortran compiler;
\item[{\normalfont \ttfamily make\_FCFLAGS\_NOOPT}] The flags passed to the Fortran compiler for unoptimized compiles;
\item[{\normalfont \ttfamily make\_MODULETYPE}] The Fortran module type identifier string;
\item[{\normalfont \ttfamily make\_PREPROCESSOR}] The preprocessor command used.
\end{description}

Additionally, two datasets are included which store details of the \glc\ source changeset. {\normalfont \ttfamily sourceChangeSetMerge} contains the output of ``{\normalfont \ttfamily hg bundle create HEAD \^origin}'', that is, it contains a Git archive that incorporates any changes made to the current branch relative to the main \glc\ branch. {\normalfont \ttfamily sourceChangeSetDiff} contains the output of ``{\normalfont \ttfamily git diff}'', that is, all differences between the source code in the working directory and that which has been committed to Git. Used together, these two datasets allow the precise source code used to run the model to be recovered from the main branch \glc\ source.

\subsection{Filters}

For each broadband filter used in the \glc\ model run an entry is added to the datasets in this group. Currently, two datasets are generated:
\begin{description}
\item[{\normalfont \ttfamily name}] The name of each filter used.
\item[{\normalfont \ttfamily wavelengthEffective}] The effective wavelength, $\lambda_\mathrm{eff}$ (defined as $\lambda_\mathrm{eff}=\left. \int_0^\infty \lambda R(\lambda) \mathrm{d}\lambda \right/ \int_0^\infty R(\lambda) \mathrm{d}\lambda$, where $R(\lambda)$ is the filter response) of the filter in \AA.
\end{description}

\subsection{Parameters}

The {\normalfont \ttfamily Parameters} group contains a record of all parameter values (either input or default) that were used for this \glc\ run. The group contains a long list of attributes, each attribute named for the corresponding parameter and with a single entry giving the value of that parameter. If a parameter has subparameters, a group is created having the same name as the parameter, which will contain attributes corresponding to each subparameter.

\subsection{Version}

The {\normalfont \ttfamily Version} group contains a record of the \glc\ version used for this model, storing the major and minor version numbers, the revision number and the {\normalfont \scshape Git} branch and hash (if the code is being maintained using {\normalfont \scshape Git}, otherwise a value of ``{\normalfont \ttfamily unknown}'' is entered). Additionally, the time at which the model was run is stored and, if the {\normalfont \ttfamily galacticusConfig.xml} file (see \S\ref{sec:ConfigFile}) is present and contains contact details, the name and e-mail address of the person who ran the model.

\subsection{globalHistory}\label{sec:globalHistory}\hyperdef{sec}{globalHistory}{}\index{history!global}\index{outputs!global history}

The {\normalfont \ttfamily globalHistory} group stores volume averaged properties of the model universe as a function of time. Currently, the properties stored are:
\begin{description}
 \item[{\normalfont \ttfamily historyTime}] Cosmic time (in Gyr);
 \item[{\normalfont \ttfamily historyExpansion}] Expansion factor;
 \item[{\normalfont \ttfamily historyStarFormationRate}] Volume averaged star formation rate (in $M_\odot/$Gyr/Mpc$^3$).
 \item[{\normalfont \ttfamily historyDiskStarFormationRate}] Volume averaged star formation rate in disks (in $M_\odot/$Gyr/Mpc$^3$).
 \item[{\normalfont \ttfamily historySpheroidStarFormationRate}] Volume averaged star formation rate in spheroids (in $M_\odot/$Gyr/Mpc$^3$).
 \item[{\normalfont \ttfamily historyStellarDensity}] Volume averaged stellar mass density (in $M_\odot/$Mpc$^3$).
 \item[{\normalfont \ttfamily historyDiskStellarDensity}] Volume averaged stellar mass density in disks (in $M_\odot/$Mpc$^3$).
 \item[{\normalfont \ttfamily historySpheroidStellarDensity}] Volume averaged stellar mass density in spheroids (in $M_\odot/$Mpc$^3$).
 \item[{\normalfont \ttfamily historyGasDensity}] Volume averaged cooled gas density (in $M_\odot/$Mpc$^3$).
 \item[{\normalfont \ttfamily historyNodeDensity}] Volume averaged resolved node density (in $M_\odot/$Mpc$^3$).
\end{description}
Dimensionful datasets have a {\normalfont \ttfamily unitsInSI} attribute which gives their units\index{units} in the SI system.

\subsection{Outputs}

The {\normalfont \ttfamily Outputs} group contains one or more sub-groups corresponding to the output times requested from \glc. Each sub-group contains the following information:
\begin{description}
 \item[{\normalfont \ttfamily outputTime} \emph{(attribute)}] The cosmic time (in Gyr) at this output;
 \item[{\normalfont \ttfamily outputExpansionFactor} \emph{(attribute)}] The expansion factor at this output;
 \item[{\normalfont \ttfamily nodeData}] A group of node properties as described below.
 \item[{\normalfont \ttfamily mergerTree} subgroups \emph{(optional)}] A set of {\normalfont \ttfamily mergerTree} groups as described below.
\end{description}

Output is controlled by parameters given within the {\normalfont \ttfamily mergerTreeOutput} section of the parameter file. Current options are:
\begin{description}
\item[{\normalfont \ttfamily outputMergerTrees}] If {\normalfont \ttfamily true} then each merger tree is output to the relevant sub-group at each output time (see \S\ref{sec:nodeDataGroup}). Otherwise merger trees are not output. [Default: {\normalfont \ttfamily true}.]
\item[{\normalfont \ttfamily outputReferences}] If {\normalfont \ttfamily true} then an HDF5 reference dataset is written for each merger tree subgroup (see \S\ref{sec:mergerTreeSubgroups}). [Default: {\normalfont \ttfamily false}.]
\item[{\normalfont \ttfamily galacticFilterMethod}] A \href{https://github.com/galacticusorg/galacticus/releases/download/masterRelease/Galacticus_Development.pdf#methods.galacticFilter}{\normalfont \ttfamily galacticFilter} which is applied to each node in the tree to determine whether or not it should be output. By combining multiple filters it is possible to construct arbitrarily complex criteria for output. [Default: {\normalfont \ttfamily always}.]
\end{description}

\subsubsection{nodeData group}\label{sec:nodeDataGroup}\hyperdef{sec}{nodeDataGroup}{}

The {\normalfont \ttfamily nodeData} group contains all data from nodes in all merger trees. The group consists of a collection of datasets each of which lists a property of all nodes in the trees which exist at the output time. Where relevant, each dataset contains an attribute, {\normalfont \ttfamily unitsInSI}, which gives the units\index{units} of the dataset in the SI system.

\subsubsection{mergerTree datasets}\label{sec:mergerTreeDatasets}

To allow locating of nodes belonging to a given merger tree in the datasets in the {\normalfont \ttfamily nodeData} group, the {\normalfont \ttfamily mergerTreeStartIndex} and {\normalfont \ttfamily mergerTreeCount} datasets list the starting index of each tree's nodes in the {\normalfont \ttfamily nodeData} datasets, and the number of nodes belonging to each tree respectively. Additionally, the {\normalfont \ttfamily mergerTreeWeight} dataset lists the {\normalfont \ttfamily volumeWeight} property for each tree (see \S\ref{sec:mergerTreeSubgroups}) which gives the weight (in Mpc$^{-3}$) which should be assigned to this tree (and all nodes in it) to create a volume-averaged sample (see \S\ref{sec:volumeLimitedSamples}). Finally, the {\normalfont \ttfamily mergerTreeIndex} dataset gives the index of each tree stored in the {\normalfont \ttfamily nodeData} datasets.

\subsubsection{mergerTree subgroups}\label{sec:mergerTreeSubgroups}

These subgroups will be present if the {\normalfont \ttfamily [mergerTreeOutputReferences]} parameter is set to true. Each {\normalfont \ttfamily mergerTree} subgroup contains HDF5 references to all data on a single merger tree. The group consists of a collection of scalar references each of which points to the appropriate region of the corresponding dataset in the {\normalfont \ttfamily nodeData} group. Additionally, the {\normalfont \ttfamily volumeWeight} attribute of this group gives the weight (in Mpc$^{-3}$) which should be assigned to this tree (and all nodes in it) to create a volume-averaged sample. (A second attribute, {\normalfont \ttfamily volumeWeightUnitsInSI}, gives the units of {\normalfont \ttfamily volumeWeight} in the SI system.)

\section{Topics in Analysis of \glc\ Outputs}

\subsection{Building Volume Limited Samples}\label{sec:volumeLimitedSamples}\index{samples!volume limited}\index{galaxies!weighting}\index{{\normalfont \ttfamily mergerTreeWeight}@mergerTreeWeight}

The {\normalfont \ttfamily mergerTreeWeight} property (see \S\ref{sec:mergerTreeDatasets}) property specifies the weight to be assigned to each merger tree in a model to construct a representative (i.e. volume limited) sample of galaxies. \glc\ does not typically generate every merger tree in a fixed volume of the Universe (as an N-body simulation might for example) as it's generally a waste of time to simulate millions of low mass halos and only a small number of high mass halos. The {\normalfont \ttfamily mergerTreeWeight} factors correct for this sampling. If merger trees are being built, then the {\normalfont \ttfamily mergerTreeWeight}, $w_i$, for each tree of mass $M_i$ (where the trees are ranked in order of increasing mass) is given by
\begin{equation}
 w_i = \int_{M_\mathrm{min}}^{M_\mathrm{max}} n(M) \mathrm{d}M,
\end{equation}
where $n(M)$ is the dark matter halo mass function and
\begin{eqnarray}
 M_\mathrm{min} &=& \sqrt{M_{i-1}M_i}, \\
 M_\mathrm{min} &=& \sqrt{M_i M_{i+1}}.
\end{eqnarray}

Suppose, for example, that we wish to construct a luminosity function of galaxies. In particular, we consider a luminosity bin $k$ which extends from $L_k-\Delta k/2$ to $L_k+\Delta k/2$. If tree $i$ contains $N_i$ galaxies with luminosities $l_{i,j}$, where $j$ runs from $1$ to $N_i$, then the luminosity function in this bin is given by:
\begin{equation}
 \phi_k = \sum_i \sum_{j=1}^{N_i} \left\{ \begin{array}{ll} w_i & \hbox{ if  } L_k-\Delta k/2 < l_{i,j} \le L_k+\Delta k/2 \\ 0 & \hbox{ otherwise.} \end{array} \right.
\end{equation}


\chapter{Input Data}

In some configurations, \glc\ requires additional input data to run. For example, if asked to process galaxy formation through a set of externally derived merger trees, then a file describing those trees must be given. In the remainder of this section we describe the structure of external datasets which can be inputs to \glc.

\section{Broadband Filters}\index{filters!broadband}

To compute luminosities through a given filter, \glc\ requires the response function, $R(\lambda)$, of that filter to be defined. \glc\ follows the convention of \cite{hogg_k_2002} in defining the filter response to be the fraction of incident photons received by the detector at a given wavelength, multiplied by the relative photon response (which will be 1 for a photon-counting detector such as a CCD, or proportional to the photon energy for a bolometer/calorimeter type detector. Filter response files are stored in {\tt data/filters/}. Their structure is shown below, with the {\tt SDSS\_g.xml} filter reponse file used as an example:
\begin{verbatim}
 <filter>
  <description>SDSS g vacuum (filter+CCD +0 air mass)</description>
  <name>SDSS g</name>
  <origin>Michael Blanton</origin>
  <response>
    <datum>   3630.000      0.0000000E+00</datum>
    <datum>   3680.000      2.2690000E-03</datum>
    <datum>   3730.000      5.4120002E-03</datum>
    <datum>   3780.000      9.8719997E-03</datum>
    <datum>   3830.000      2.9449999E-02</datum>
    .
    .
    . 
  </response>
  <effectiveWavelength>4727.02994472695</effectiveWavelength>
  <vegaOffset>0.107430167298754</vegaOffset>
</filter>
\end{verbatim}
The {\tt description} tag should provide a description of the filter, while the {\tt name} tag provides a shorter name. The {\tt origin} tag should describe from where/whom this filter originated. The {\tt response} element contains a list of {\tt datum} tags each giving a wavelength (in Angstroms) and response pair. The normalization of the response is arbitrary. The {\tt effectiveWavelength} tag gives the mean, response-weighted wavelength of the filter and is used, for example, in dust attenuation calculations. The {\tt vegaOffset} tag gives the value (in magnitudes) which must be added to an AB-system magnitude in this system to place it into the Vega system. Both {\tt effectiveWavelength} and {\tt vegaOffset} can be computed by running
\begin{verbatim}
 scripts/filters/vega_offset_effective_lambda.pl data/filters
\end{verbatim}
which will compute these values for any filter files that do not already contain them and append them to the files.

\section{Merger Trees}\label{sec:MergerTreeFiles}

While \glc\ can build merger trees using analytic methods it is often usedful to be able to utilize merger trees from other sources (e.g. extracted from an N-body simulation). To facilitate this, \glc\ allows merger trees to be read from an HDF5 files. To do so, set the {\tt [mergerTreeConstructMethod]} input parameter to {\tt read} and specify the filename to read via their {\tt [mergerTreeReadFileName]} parameter.

The HDF5 file should have the following structure:
\begin{verbatim}
mergerTreeFile.hdf5 {
  [mergerTrees] {
    [mergerTree1] {
      volumeWeight => (1)
      treeIndex    => (1)
      nodeIndex    => (nodeCount1)
      parentNode   => (nodeCount1)
      childNode    => (nodeCount1)
      siblingNode  => (nodeCount1)
      nodeMass     => (nodeCount1)
      nodeRedshift => (nodeCount1)
    }
    [mergerTree2] {
      volumeWeight => (1)
      treeIndex    => (1)
      nodeIndex    => (nodeCount2)
      parentNode   => (nodeCount2)
      childNode    => (nodeCount2)
      siblingNode  => (nodeCount2)
      nodeMass     => (nodeCount2)
      nodeRedshift => (nodeCount2)
    }
    .
    .
    .
  }
}
\end{verbatim}
where {\tt volumeWeight} is the number of such trees per unit volume, {\tt treeIndex} is an index for this tree, {\tt nodeIndex} is a unique integer identifier for their node, {\tt parentNode} is the index of the parent node (or $-1$ if none exists), {\tt childNode} is the index of the most massive child node (or $-1$ if none exists), {\tt siblingNode} is the index of the sibling node (or $-1$ if none exists), {\tt nodeMass} is the mass of the nodeCount1 (in $M_\odot$) and {\tt nodeRedshift} is the redshift at which the node exists. Here, {\tt nodeCount1} is the number of nodes in tree number 1 for example. The trees must be self-consistent (i.e. parent and child indices must make sense, sibling should all have the same parent, parents must exist at lower redshift than their children, there must be a unique root to the tree).

An example of how to construct such a file is given by the {\tt scripts/aux/Millennium\_Trees\_Grab.pl} script which pulls merger tree data from the Millennium Simulation database and outputs it as an HDF5 file in the format required by \glc. This script is used as follows:
\begin{verbatim}
 scripts/aux/Millennium_Trees_Grab.pl --output <outputFile> --user <userName> --password <password> --select <sqlSelection>
\end{verbatim}
All arguments are optional. If present the {\tt user} and {\tt password} will be used to log in to the database server. Merger trees are output to the file specified by {\tt output} if present, or to {\tt Millennium\_Trees.hdf5} otherwise. If the {\tt select} keyword is present, then its contents are added to the SQL query sent to the database and permits you to select subsamples of merger trees (e.g. those in a given mass range). The script {\tt examples/Millennium\_Trees\_Grab.csh} shows an example of how this script should be used.



\chapter{Tutorials}

This chapter contains step-by-step guides to performing common tasks with \glc.

\section{Running \glc\ on N-body Merger Trees}\label{sec:nBodyRun}\index{merger trees!N-body}\index{N-body!merger trees}

See \S\ref{sec:MergerTreeBuilder} for details of how to build merger tree files suitable for input into \glc. There are many options which control precisely how merger trees read from file should be handled. The following section provides guidance on the best choice of parameters.

\subsection{Setting Input Parameters}

To utilize merger trees from the file\footnote{The following assumes that merger trees will be read from a file following \protect\glc's standard HDF5 format which is described \href{https://github.com/galacticusorg/galacticus/wiki/Merger-Tree-File-Format}{here}.} that you created in a \glc\ run it's necessary to set two parameters in the input parameter file that you will use for the run:
\begin{verbatim}
  <!-- Specify that merger trees are to be read from file, and give the name of the file to read -->
  <mergerTreeConstructMethod value="read"             />
  <mergerTreeReadFileName    value="myNBodyTrees.hdf5"/>
\end{verbatim}
The first of these {\normalfont \ttfamily [mergerTreeConstructMethod]}$=${\normalfont \ttfamily read} tells \glc\ that merger trees will be constructed by reading them from a file. The second, {\normalfont \ttfamily [mergerTreeReadFileName]}, gives the name of the file from which to read the trees.

In order to choose sensible settings for the various parameters that control merger trees read from file, it is recommended that you read through each of the items below and follow the guidance given.

{\normalfont \bfseries Cosmology:} In addition to specifying that trees should be read from a file, it's also important to ensure that the values of cosmological parameters in \glc\ match those in the merger tree file. (If they don't match, \glc\ will stop with an error message unless you set {\normalfont \ttfamily [mergerTreeReadMismatchIsFatal]}$=${\normalfont \ttfamily false} in which case you'll just be warned about any mismatch.) In our case of using merger trees from the Millennium Simulation, the correct cosmological parameter values can be set as follows:
\begin{verbatim}
  <!-- Use Millennium Simulation cosmology. -->
  <cosmologyParametersMethod value="simple"/>
   <HubbleConstant  value="73.0"  />
   <OmegaMatter     value="0.25"  />
   <OmegaDarkEnergy value="0.75"  />
   <OmegaBaryon     value="0.0455"/>
  </cosmologyParametersMethod>
  <cosmologicalMassVarianceMethod value="filteredPower">
    <sigma_8 value="0.900"/>
  </cosmologicalMassVarianceMethod>
  <powerSpectrumPrimordialMethod value="powerLaw">
    <index               value="1.000"/>
    <wavenumberReference value="1.000"/>
    <running              value="0.000"/>
  </powerSpectrumPrimordialMethod>
\end{verbatim}

{\normalfont \bfseries Existance at Final Time:} Normally, \glc\ assumes that all merger trees will exist (i.e. have at least one node present) at the final output time. This may not be true of trees extracted from an N-body simulation---in this case \glc\ can be informed of this fact by setting:
\begin{verbatim}
  <allTreesExistAtFinalTime value="false"/>
\end{verbatim}

{\normalfont \bfseries Snapping Nodes to Snapshots:} N-body merger trees are often built from ``snapshots'' of the simulation, i.e. all of the nodes exist at a set of discrete times. Often we want to output nodes at precisely these output times. In such cases it is useful to set:
\begin{verbatim}
  <mergerTreeReadOutputTimeSnapTolerance value="1.0d-3"/>
\end{verbatim}
which ensures that the times of nodes are adjusted to lie at precisely the output time if that time is within the specified relative tolerance (this avoids any small differences between node times and output times that can arises due to rounding errors when converting from redshifts to times and vice-versa).

{\normalfont \bfseries Missing Hosts:} \glc\ expects to find each \gls{node}'s host \gls{node} present in a merger tree \gls{forest}. If a \gls{node}'s host is not found this is cause for a fatal error to be issued, since it is impossible to correctly construct and evolve the corresponding \gls{forest}. If you absolutely want to run a \gls{forest} for which one or more host \glspl{node} are missing, you can allow this by setting {\normalfont \ttfamily [mergerTreeReadMissingHostsAreFatal]}$=${\normalfont \ttfamily false}---in this case missing host \glspl{node} trigger a warning only and \glspl{node} without a host are forced to become isolated \glspl{node}. This will lead to incorrect tree evolution however, so the recommended setting is:
\begin{verbatim}
<mergerTreeReadMissingHostsAreFatal value="true"/>
\end{verbatim}

{\normalfont \bfseries Branch Jumps and Subhalo Promotions:} If your merger trees contain subhalos they will most likely exhibit two specific behaviors\footnote{These two behaviors are called out as they specifically \emph{do not} occur in merger trees created using Press-Schechter-based algorithms for example.}: i) \glspl{node} which are subhalos in one timestep may become non-subhalos (isolated halos) in a subsequent timestep (``subhalo promotion''), and ii) \glspl{node} which are subhalos in one branch of the tree may ``jump'' to another branch\footnote{That is, the subhalo's descendented is hosted by a \gls{node} other than the descendent of the subhalo's host.} of the tree becoming a subhalo there (``branch jumping''). These behaviors are fully supported by \glc\ and so we recommend the following settings:
\begin{verbatim}
<mergerTreeReadAllowSubhaloPromotions value="true"/>
<mergerTreeReadAllowBranchJumps       value="true"/>
\end{verbatim}
You may choose to disallow these behaviors by setting either of the above parameters to {\normalfont \ttfamily false}---for example if you wish to explore how your results would differ if subhalos were forced to remain subhalos forever in their original branch. Note that allowing subhalo promotion while not allowing branch jumping can lead to \glspl{deadlock} in merger tree evolution, so change these settings with caution.

Note that for trees which do not contain subhalos these two parameters are irrelevant.

{\normalfont \bfseries Subhalo Masses:} If your trees contain subhalos, the mass evolution of those subhalos can be preset in the satellite component of each \gls{node}. In this way, the subhalo mass in \glc\ will track that specified by the merger tree file. This requires the use of a satellite component which allows presetting of subhalo masses. Recommended settings are therefore:
\begin{verbatim}
<treeNodeMethodSatellite           value="preset"/>
<mergerTreeReadPresetSubhaloMasses value="true"  />
\end{verbatim}
If your trees do not contain subhalos, recommended settings are instead:
\begin{verbatim}
<treeNodeMethodSatellite           value="standard"/>
<mergerTreeReadPresetSubhaloMasses value="false"   />
\end{verbatim}

{\normalfont \bfseries Halo Positions/Velocities:} If your trees contain position and velocity information for halos, those positions and velocities can be preset in the position component of each \gls{node}. This requires the use of a position component which allows presetting of positions and velocities. Recommended settings are therefore:
\begin{verbatim}
<treeNodeMethodPosition        value="preset"/>
<mergerTreeReadPresetPositions value="true"  />
\end{verbatim}
If your trees do not contain position information recommended settings are:
\begin{verbatim}
<treeNodeMethodPosition        value="null" />
<mergerTreeReadPresetPositions value="false"/>
\end{verbatim}

{\normalfont \bfseries Subhalo Orbits:} If your trees contain position and velocity information they can be used to preset initial orbit information for subhalos. Note that it is not required that your trees contain subhalos for this orbit presetting to be performed---\glc\ can follow subhalo orbits even if subhalos are not included in the trees themselves. The following settings are recommended:
\begin{verbatim}
<mergerTreeReadPresetOrbits             value="true"/>
<mergerTreeReadPresetOrbitsSetAll       value="true"/>
<mergerTreeReadPresetOrbitsAssertAllSet value="true"/>
<mergerTreeReadPresetOrbitsBoundOnly    value="true"/>
\end{verbatim}
These options will cause an orbit to be preset for each subhalo based on the relative position and velocity of merging halos and assuming that the orbital energy and angular momentum are conserved between the time immediately prior to the merger and the time of virial radius crossing. If the computed orbit does not cross the virial radius of the larger halo or if the computed orbit is unbound, the above options cause an orbit to be preset by drawing orbital parameters at random from the chosen cosmological distribution (see \S\ref{sec:SatelliteVirialOrbits}).

{\normalfont \bfseries Subhalo Merging:} If your merger trees contain subhalo information, that information can be used to specify when, and with which other node, each subhalo merges. Specifically, a subhalo is assumed to merge at the time at which it is not the primary progenitor of its descendent halo---possibly with some other delay to be described below. Recommended settings are:
\begin{verbatim}
<mergerTreeReadPresetMergerTimes          value="true"             />
<mergerTreeReadPresetMergerNodes          value="true"             />
<mergerTreeReadSubresolutionMergingMethod value="boylanKolchin2008"/>
\end{verbatim}
The first two options cause subhalos to merge at the time described above, and with their descendent node. The final option accounts for the possibility that the subhalo should not actually merge immediately at this time. For example, in N-body simulations, the subhalo may have simply been lost due to limitations of resolution or halo finder algorithms. The final option specifies that some additional time until merging be added based on the subhalo merging timescale algorithm of \citeauthor{boylan-kolchin_dynamical_2008}~[\citeyear{boylan-kolchin_dynamical_2008}; see \S\ref{phys:satelliteMergingTimescales:satelliteMergingTimescalesBoylanKolchin2008}], and computed using the last known orbital properties of the subhalo.

{\normalfont \bfseries Halo Scale Radii:} If your merger trees contain information on halo scale radii or half-mass radii, these can be used to preset the scale radius of each \gls{node}. This requires the use of a dark matter profile component which allows presetting of scale length. Recommended settings are therefore:
\begin{verbatim}
<mergerTreeReadPresetScaleRadii                     value="true"     />
<mergerTreeReadPresetScaleRadiiFailureIsFatal       value="true"     />
<mergerTreeReadPresetScaleRadiiConcentrationMinimum value="3"        />
<mergerTreeReadPresetScaleRadiiConcentrationMaximum value="60"       />
<mergerTreeReadPresetScaleRadiiMinimumMass          value="see below"/>
\end{verbatim}
Minimum and maximum concentrations are specified---these are used to restrict the range of scale radii that are allowed for a given halo. If scale radii are to be determined based on half-mass radii given in the merger tree file, and if the computed scale radius does not result in a concentration between these limits, then a fatal error is issued.

Finally, you can set a minimum halo mass via the {\normalfont \ttfamily [mergerTreeReadPresetScaleRadiiMinimumMass]} parameter below which the scale radii or half-mass radii in your file should be considered not reliable. For halos below this mass, scale radii will instead be assigned via the selected dark matter halo concentration method (see \S\ref{sec:DarkMatterProfileConcentration}).

{\normalfont \bfseries Halo Angular Momenta:} If your merger trees contain spin or angular momentum information these can be preset for each node. Recommended settings are:
\begin{verbatim}
<treeNodeMethodSpin                  value="preset"/>
<mergerTreeReadPresetSpins           value="true"  />
<mergerTreeReadPresetUnphysicalSpins value="true"  />
\end{verbatim}
The last of these options causes any halos for which the spin given in the merger tree file is non-positive to be assigned a spin at random instead, drawn from the specified cosmological distribution (see \S\ref{sec:SpinParameterDistribution}).

If subhalo masses are not included in their host halo masses in your merger tree file, you should specify how the angular momenta of subhalos should be accounted for when adding their mass to their host halo. If positions and 3D angular momenta are available in your merger tree file, the recommended setting is:
\begin{verbatim}
<mergerTreeReadSubhaloAngularMomentaMethod value="summation"/>
\end{verbatim}
If this information is not present 
\begin{verbatim}
<mergerTreeReadSubhaloAngularMomentaMethod value="scale"    />
\end{verbatim}
should be used instead.

If your merger tree file contains 3D spin or angular momentum information, you can choose to make that information available within \glc\ be using the settings:
\begin{verbatim}
<treeNodeMethodSpin          value="preset3D"/>
<mergerTreeReadPresetSpins3D value="true"    />
\end{verbatim}

{\normalfont \bfseries Subhalo Indices:} If your merger trees contain subhalos, you can tell \glc\ to keep track of the indices of subhalos by setting:
\begin{verbatim}
<treeNodeMethodSatellite            value="preset"/>
<mergerTreeReadPresetSubhaloIndices value="true"  />
\end{verbatim}
The \glc\ output file will then contain {\normalfont \ttfamily satelliteNodeIndex} datasets which list the index (as given in the merger tree file) for all subhalos and halos. Without specifying this presetting, the index of subhalos is frozen at the index of the halo immediately prior to it becomming a subhalo.

The remainder of this section gives more detail about many of the parameters described above and how they affect handling of merger trees read from file. Further parameters can be set to control what information from the stored trees will be used in \glc. Examples are given below.

\subsection{Further Details}

Further details of the effects of the many parameters controlling merger trees read from file are given below.

\subsubsection{Node Positions}

If position and velocity information for tree nodes is available within the merger tree file then \glc\ can be instructed to use this information by using the ``preset'' method for tree node positions and telling the merger tree construction method to preset node positions as follows:
\begin{verbatim}
  <!-- Use merger tree node positions -->
  <treeNodeMethodPosition        value="preset"/>
  <mergerTreeReadPresetPositions value="true"  />
\end{verbatim}
If position information is unavailable, the ``null'' position method can be selected and the merger tree construction method instructed not to preset positions as follows:
\begin{verbatim}
  <!-- Do not use merger tree node positions -->
  <treeNodeMethodPosition        value="null" />
  <mergerTreeReadPresetPositions value="false"/>
\end{verbatim}

\subsubsection{Virial Orbits}\index{orbits!virial}\index{orbits!setting}\index{orbits!N-body}

If position and velocity information for tree nodes is available within the merger tree file then \glc\ can be instructed to use this information to estimate the orbit of each subhalo at the point at which it crosses the virial radius of its host halo. This ``virial orbit'' may then be used by, for example, calculations of merging timescales.
\begin{verbatim}
  <!-- Use merger tree node positions to compute orbits at the virial radius -->
  <mergerTreeReadPresetOrbits             value="true"/>
  <mergerTreeReadPresetOrbitsBoundOnly    value="true"/>
  <mergerTreeReadPresetOrbitsSetAll       value="true"/>
  <mergerTreeReadPresetOrbitsAssertAllSet value="true"/>
\end{verbatim}
Typically, a merging halo is not seen at precisely the time at which it crosses the virial radius of its host (due to the fact that N-body simulations are output at discretely spaced timesteps). Therefore, \glc\ computes the orbit at the time just prior to merging and assumes that the orbital parameters (energy and angular momentum) remain fixed to propagate the orbit to the virial radius of the host. The second parameter in the above example, {\normalfont \ttfamily [mergerTreeReadPresetOrbitsBoundOnly]}, specifies whether or not only bound orbits should be set. Some calculations (e.g. of subhalo merging times) assume bound orbits and may fail if given an unbound orbit. Setting this option to {\normalfont \ttfamily true} causes only bound orbits to be preset---unbound orbits are ignored. Note that some orbits cannot be propagated to the virial radius (i.e. their pericenter is larger than the virial radius). The {\normalfont \ttfamily [mergerTreeReadPresetOrbitsSetAll]} option, if true, will cause such orbits to be assigned randomly using the selected {\normalfont \ttfamily [virialOrbitsMethod]}, such that all orbits are assigned. The {\normalfont \ttfamily [mergerTreeReadPresetOrbitsAssertAllSet]} option requires that all orbits be set---if {\normalfont \ttfamily [mergerTreeReadPresetOrbitsSetAll]}$=${\normalfont \ttfamily false} and {\normalfont \ttfamily [mergerTreeReadPresetOrbitsAssertAllSet]}$=${\normalfont \ttfamily true} then \glc\ will exit with an error message if any orbit cannot be set.

If the satellite component additionally permits setting of the satellite position and velocity, these properties will also be assigned based on the relative position and velocity of the satellite and host halos.

\subsubsection{Merging Times and Targets}

The times at which subhalos merge with their host halo can be determined directly from the merger tree file if subhalo information is included in that file. Merging is assumed to occur when the subhalo no longer has a distinct descendent (i.e. it descends into a non-subhalo). If merging times are to be computed in this way set
\begin{verbatim}
  <treeNodeMethodSatellite         value="preset"/>
  <mergerTreeReadPresetMergerTimes value="true"  />
\end{verbatim}
which select a satellite orbit method that allows merger times to be present and tell the merger tree construction method to preset those merger times respectively. If merger times are not to be computed in this way then instead set, for example,
\begin{verbatim}
  <treeNodeMethodSatellite         value="standard" />
  <mergerTreeReadPresetMergerNodes value="false"    />
  <satelliteMergingMethod          value="Jiang2008"/>
\end{verbatim}
which selects a standard satellite orbit method, prevents attempts to preset the merger times and selects the {\normalfont \ttfamily Jiang2008} method for computing merger times instead.

In addition to setting the times of merger events, it is possible to set the target node with which a merging node should merge. By default, \glc\ will assume that all merging occurs with the non-subhalo host node in which a subhalo is located. This may not be the desired behavior when using N-body merger trees. For example, such trees may indicate that a subhalo merges with another subhalo. Setting
\begin{verbatim}
  <mergerTreeReadPresetMergerNodes value="true"/>
\end{verbatim}
will cause the target node with which each merger should occur to be determined from the merger tree structure and preset for use in \glc.

It is possible to add a delay between the last time at which a subhalo was seen in a simulation and the time at which it is considered to merge. This functionality is motivated by the consideration that a subhalo vanishing from a simulation may be simply due to it dropping below resolution rather than it actually having undergone a merger. The parameter {\normalfont \ttfamily [mergerTreeReadSubresolutionMergingMethod]} can be used to select a satellite merging timescale method (see \S\ref{sec:SatelliteMergingTimescales}) to use in this case. (It is set by default to ``{\normalfont \ttfamily null}'' such that no delay before merging occurs.) The orbit of the subhalo around its parent at the last time it is present in the merger tree is passed to this method and used to estimate a time until merging. This delay is added to the time at which the subhalo merges and, if merge target nodes are being set, the target node is updated accordingly.

\subsubsection{Subhalo Indices}

The indices of subhalos are usually frozen at the index of the halo just prior to becoming a subhalo. The index of the corresponding halo in the original tree (as read from file) can be tracked as follows:
\begin{verbatim}
  <treeNodeMethodSatellite            value="preset"/>
  <mergerTreeReadPresetSubhaloIndices value="true"  />
\end{verbatim}
to first select the ``preset'' satellite orbit method (which allows subhalo indices to be preset) and, second, to instruct the merger tree construction algorithm to preset those indices. The index will then be available in output as {\normalfont \ttfamily satelliteNodeIndex}.

\subsubsection{Subhalo Masses}

The masses of subhalos (specifically their time evolution after they become subhalos) can be set using the values stored in the merger tree file (if available). To set subhalo masses in this way use
\begin{verbatim}
  <treeNodeMethodSatellite           value="preset"/>
  <mergerTreeReadPresetSubhaloMasses value="true"  />
\end{verbatim}
to first select the ``preset'' satellite orbit method (which allows subhalo masses to be preset) and, second, to instruct the merger tree construction algorithm to preset those masses.

\subsubsection{Node Spins}

If information on the angular momenta of nodes is available in the merger tree file, this can be used to preset the value of the spin parameter in each node\footnote{Before doing this, it is important to be sure that the angular momenta of the nodes are reliable. For example, in low mass nodes extracted from an N-body simulation resolution effect may limit the accuracy of the measured angular momentum.} by setting:

\begin{verbatim}
  <mergerTreeReadPresetSpins value="true"/>
\end{verbatim}

The spin parameter is set using the spin of each node if available, or otherwise using the angular momentum of each node stored in the merger tree file using:
\begin{equation}
 \lambda = {|\mathbf{J}| |E|^{1/2} \over \mathrm{G} M^{5/2}}
\end{equation}
where $|\mathbf{J}|$ is the magnitude of the node's angular momentum, $M$ is the node's mass and $E$ is its energy. Additionally, by setting:

\begin{verbatim}
  <mergerTreeReadPresetSpins3D value="true"/>
\end{verbatim}
the spin vector of each node will be set (assuming that the vector spin or angular momenta of nodes are available in the merger tree file) using:
\begin{equation}
 \mathbf{\lambda} = {\mathbf{J} |E|^{1/2} \over \mathrm{G} M^{5/2}}.
\end{equation}

If spins could not be determined for some halos the spin (or angular momentum) should be set to zero in the merger tree file, and the parameter {\normalfont \ttfamily [mergerTreeReadPresetUnphysicalSpins]} set to {\normalfont \ttfamily true}. \glc\ will then assign a spin to such halos by sampling from the selected spin distribution (see \S\ref{sec:SpinParameterDistribution}). 

\subsubsection{Node Scale Radii}

If information on the half-mass or scale radii of nodes is available in the merger tree file, it can be used to preset the value of the dark matter halo scale radius in each node by setting:

\begin{verbatim}
  <mergerTreeReadPresetScaleRadii value="true"/>
\end{verbatim}

Before doing this, it is important to be sure that the half-mass or scale radii of the nodes are reliable. For example, in low mass nodes extracted from an N-body simulation resolution effect may limit the accuracy of the measured half-mass or scale radius. In such cases, use the {\normalfont \ttfamily [mergerTreeReadPresetScaleRadiiMinimumMass]} parameter to specify the lowest mass halos for which the scale radii should be preset---lower mass halos will be assigned a scale radius using the method specified by the {\normalfont \ttfamily [mergerTreeReadConcentrationFallbackMethod]} parameter (which will default to the value of {\normalfont \ttfamily [darkMatterConcentrationMethod]}; see \S\ref{sec:DarkMatterProfileConcentration}). It is also possible to specify minimum and maximum allowed concentrations when computing the scale radius from the half mass radius using the {\normalfont \ttfamily [mergerTreeReadPresetScaleRadiiConcentrationMinimum]} and {\normalfont \ttfamily [mergerTreeReadPresetScaleRadiiConcentrationMaximum]} parameters. If matching the half mass radius would require a concentration outside of this range, \glc\ will abort unless {\normalfont \ttfamily [mergerTreeReadPresetScaleRadiiFailureIsFatal]}$=${\normalfont \ttfamily false}, in which case it will instead silently use the fallback concentration method described above.

If only half-mass radii are available, the scale radius is set by using a root finding algorithm to ensure that half of the total halo mass is enclosed within the specified half-mass radius.

\subsubsection{Miscellaneous N-body Properties}

Several miscellaneous properties often available from N-body merger trees can also be preset by setting the following parameters to {\normalfont \ttfamily true}:
\begin{description}
\item[{mergerTreeReadPresetParticleCounts}] Sets the number of particles in each halo (requires the {\normalfont \ttfamily particleCount} dataset to be present in the merger tree file);
\item[{mergerTreeReadPresetVelocityMaxima}] Sets the maxima of halo rotation curves (requires the {\normalfont \ttfamily velocityMaximum} dataset to be present in the merger tree file);
\item[{mergerTreeReadPresetVelocityDispersions}] Sets the velocity dispersion of halos (requires the {\normalfont \ttfamily velocityDispersion} dataset to be present in the merger tree file).
\end{description}

\subsubsection{Subhalo Promotion}\label{sec:Tutorial:NbodyTrees:SubhaloPromotion}

A subhalo may, at a later time, become an isolated halo once again. \glc\ allows you to control whether such behavior is allowed, or should be prohibited. To allow such ``subhalo promotion'', set:
\begin{verbatim}
<mergerTreeReadAllowSubhaloPromotions value="true"/>
\end{verbatim}
If you choose to inhibit this behavior by setting the above parameter to false, a halo that becomes a subhalo will remain a subhalo forever thereafter. Note that the isolated halo to which it would have been promoted will still exist, and may therefore form its own galaxy. This can result in double counting of mass, and so inhibiting subhalo promotion is not recommended.

\subsubsection{``Fly-by'' Halos}\label{sec:Tutorial:NbodyTrees:BranchJump}

In some cases, a halo that is part of one tree can later become part of another tree. This can happen in so-called ``fly-by'' encounters where a halo may briefly become a subhalo in a halo in tree A then leave that halo and become a subhalo in tree B.

The correct way to handle this issue is to combine trees A and B into a single tree (which will now have multiple base nodes). \glc\ will then process these two trees simultaneously, correctly handling the fly-by, and outputting the trees as two separate trees.

If for some reason this is not possible or desired, the fly-by problem will normally cause \glc\ to complain that the host halo of a node cannot be found (since it exists in a different tree). This problem can be avoided by setting:
\begin{verbatim}
  <mergerTreeReadMissingHostsAreFatal value="false"/>
\end{verbatim}
In this case, nodes with missing hosts are simply treated as being isolated halos. This will avoid an error condition, but is not a physically correct way to handle such cases, so use with caution.

\subsection{Using Particles to Track Unresolved Subhalos}

In N-body simulations it is possible that a subhalo can become ``lost'' from the simulation (i.e. can no longer be identified by a halo finder due to resolution issues) before it has actually merged with the central galaxy or been completely tidally destroyed. In such cases it is useful to be able to assign a position to the subhalo at later times. A common approach to assigning a position (and velocity) is to use that of the most bound particle in the subhalo at the last time it was identified. \glc\ allows for particle tracking in this way through the addition of particle information to the merger tree file.

To add particle tracking data to a merger tree file, follow these steps:
\begin{enumerate}
\item Identify all subhalos which are lost from the simulation prior to the final timestep;
\item Determine the index of the most bound particle in each such subhalo in the last timestep in which it was identified;
\item For each subhalo, extract the redshift, position, and velocity of that particle (which is usally trivial to do once its index is known) at each subsequent timestep in the simulation;
\item Write these data (along with the particle index) to the {\normalfont \ttfamily particles} group in the merger tree file as described \href{https://github.com/galacticusorg/galacticus/wiki/Merger-Tree-File-Format#particles-group}{here};
\item Add two datasets to the {\normalfont \ttfamily forestHalos} group:
  \begin{enumerate}
  \item {\normalfont \ttfamily particleIndexStart} which should indicate the index in the datasets in the {\normalfont \ttfamily particles} group at which the data for each halo begins (or $-1$ if no particle data is included for the halo);
  \item {\normalfont \ttfamily particleIndexCount} which should indicate the number of entries in the datasets in the {\normalfont \ttfamily particles} group for each halo (or $-0$ if no particle data is included for the halo).
  \end{enumerate}
\end{enumerate}

\subsection{Handling of Extremely Large Merger Tree Forests}\index{merger trees!large}\index{forests!large}

Halos can move between merger trees (see \S\ref{sec:Tutorial:NbodyTrees:SubhaloPromotion} and \S\ref{sec:Tutorial:NbodyTrees:BranchJump}), leading to the necessity of merger tree forests---interconnected groups of merger trees that \glc\ typically processes as a whole. These forests can become very large---in some cases so large that they do not fit within the available memory. \glc\ can handle such forests by splitting them into individual trees. Each tree is processed separately, and nodes are moved between trees as needed. If a tree needs a node from another tree before its evolution can continue, its state can be suspended to disk, and later re-read once the node it requires becomes available. In this way, very large forests can be processed without running out of memory (as trees are stored to disk while they are not being processed).

To cause forests to be split in this way, the following parameters should be set:
\begin{verbatim}
  <treeEvolveSuspendToRAM                   value="false"            />
  <treeEvolveSuspendPath                    value="/my/scratch/path/"/>
  <mergerTreeReadForestSizeMaximum          value="10000000"         />
  <mergerTreeReadSubresolutionMergingMethod value="infinite"         />
\end{verbatim}

Here, {\normalfont \ttfamily treeEvolveSuspendToRAM} specifies that merger trees should be suspended to disk (i.e. not to RAM which is the default), and {\normalfont \ttfamily treeEvolveSuspendPath} gives a path where the suspended trees can be written---typically scratch space local to the compute node where \glc\ is being run is a good option.
 
{\normalfont \ttfamily mergerTreeReadForestSizeMaximum} specifies the maximum number of nodes allowed in a forest before it will be split. A suitable number for this depends on the details of the available RAM, the number of threads sharing that RAM, and the characteristics of the \glc\ model being used (which will affect the memory required per node).
 
Finally, {\normalfont \ttfamily mergerTreeReadSubresolutionMergingMethod} is set to {\normalfont \ttfamily infinite} to prevent any merging (which is not supported for split forests at present, although it should be soon).

\subsection{Analyzing the Output}

\subsubsection{Positions and Velocities}

Components of the position of each node are output as {\normalfont \ttfamily positionX}, {\normalfont \ttfamily positionY} and {\normalfont \ttfamily positionZ} and can be accessed in the same way as other output properties from \glc\ (see \S\ref{sec:nodeDataGroup}).

\subsubsection{Subhalo Masses}

The current mass of subhalos is available via the {\normalfont \ttfamily nodeBoundMass} output dataset and can be accessed in the same way as other output properties from \glc\ (see \S\ref{sec:nodeDataGroup}). For non-subhalos this property is equal to the usual {\normalfont \ttfamily nodeMass} property.

\section{Generating Mock Catalogs with Lightcones from the Millennium Simulation}\index{lightcone}\index{Millennium Simulation}\index{mock catalog}

Suppose that you want to create a catalog of galaxies as would be found in a survey of an area of the sky out to some redshift. Such a ``mock catalog'' can be built by populating with galaxies all of the dark matter halos which happen to lie within the cone which that area makes as it is projected from the observer through the Universe.

Generating such a mock catalog using \glc\ involves first extracting the halos (and their merger trees) within this ``lightcone'' from a suitable N-body simulation, and then processing them through \glc. In this tutorial, we will specifically make use of the \href{http://gavo.mpa-garching.mpg.de/MyMillennium3/MyDB}{Millennium Simulation database} to provide the merger trees, but the same principles apply to any N-body simulation.

The script, {\normalfont \ttfamily scripts/aux/Millennium\_Lightcone\_Grab.pl} can be used to retrieve merger trees that intersect a given lightcone from the Millennium Database and to store them in \glc's format (see \href{https://github.com/galacticusorg/galacticus/wiki/Merger-Tree-File-Format}{here}). The script is used as follows:
\begin{verbatim}
scripts/aux/Millennium_Lightcone_Grab.pl <lightconeDirectory> <fieldSize> <maximumRedshift>
    --user <myUserName> --password <myPassword> --treesPerFile <treesPerFile>
\end{verbatim}
Here, {\normalfont \ttfamily \textless lightconeDirectory\textgreater} is the name of a (pre-existing) directory into which merger tree data will be stored, {\normalfont \ttfamily \textless fieldSize\textgreater} is the length (in degrees) of one side of the square field of view of the lightcone, {\normalfont \ttfamily \textless maximumRedshift\textgreater} is the highist redshift for which halos should be included in the catalog. The {\normalfont \ttfamily --user} and {\normalfont \ttfamily --password} options allow you to specify your username and password for accessing the Millennium Simulation database. Finally, the {\normalfont \ttfamily --treesPerFile} specifies how many merger trees should be stored in each file (the script will split the lightcone between many files---this is primarily so that each request sent to the Millennium Database server is not too large). If no value is specified a default of 200 trees per file will be used.

The script generates multiple SQL queries to the Millennium database in order to first find all halos which intersect the lightcone and second to retrieve the complete merger tree associated with each such halo. These merger trees are then stored in \glc's merger tree file format in files named {\normalfont \ttfamily Lightcone\_Trees\_AAA:BBB.hdf5} in the given {\normalfont \ttfamily \textless lightconeDirectory\textgreater}, where {\normalfont \ttfamily AAA} and {\normalfont \ttfamily BBB} are numbers giving the first and last trees in the file\footnote{Note that these are not the ID numbers of the trees, just a sequential count of all trees retrieved.}

Each of the merger tree files created can then be run through \glc\ in the usual way (see \S\ref{sec:nBodyRun}).  Outputs should be requested at every Millennium snapshot (up to the largest redshift to be considered), and the {\normalfont \ttfamily lightcone} filter should be used to cause only those galaxies which intersect the lightcone to be output---for example:
\begin{verbatim}
<!-- Set output redshifts to the snapshots in the milliMillennium. -->
<outputRedshifts value=
   "0.0000 0.0199 0.0414 0.0645 0.0893 0.1159 0.1444 0.1749 0.2075 0.2425
    0.2798 0.3197 0.3623 0.4079 0.4566 0.5086 0.5642 0.6236 0.6871 0.7550
    0.8277 0.9055 0.9887 1.0779 1.1734 1.2758 1.3857 1.5036 1.6303 1.7663
    1.9126 2.0700 2.2395 2.4220 2.6189 2.8312 3.0604 3.3081 3.5759 3.8657
    4.1795 4.5196 4.8884 5.2888 5.7239 6.1968"
/>

<!-- Add a lightcone filter with the required geometry -->
<mergerTreeOutput>
  <galacticFilterMethod value="lightcone"/>
</mergerTreeOutput>

<!-- Switch on output of lightcone data -->
<outputLightconeData value="true"/>

<!-- Prune away trees not appearing in the lightcone -->
<mergerTreeOperatorMethod value="pruneLightcone"/>

<!-- Specify lightcone geometry -->
<geometryLightconeMethod value="square">
  <origin value="0 0 0"/>
  <unitVector1 value=" 1 1  1"/>
  <unitVector2 value=" 0 1 -1"/>
  <unitVector3 value="-2 1  1"/>
  <lengthReplication value="500"/>
  <lengthHubbleExponent value="-1"/>
  <lengthUnitsInSI value="3.08567758135e22"/>
  <angularSize value="0.5"/>
  <timeEvolvesAlongLightcone value="true"/>
  <redshift value=
   "0.0000 0.0199 0.0414 0.0645 0.0893 0.1159 0.1444 0.1749 0.2075 0.2425
    0.2798 0.3197 0.3623 0.4079 0.4566 0.5086 0.5642 0.6236 0.6871 0.7550
    0.8277 0.9055 0.9887 1.0779 1.1734 1.2758 1.3857 1.5036 1.6303 1.7663
    1.9126 2.0700 2.2395 2.4220 2.6189 2.8312 3.0604 3.3081 3.5759 3.8657
    4.1795 4.5196 4.8884 5.2888 5.7239 6.1968"
  />
</geometryLightconeMethod>
\end{verbatim}
In the above {\normalfont \ttfamily [outputLightconeData]} causes lightcone coordinate information (i.e. the position and velocity of each galaxy in a coordinate system with axes aligned along the line of sight of the lightcone and parallel to the two edges of the square field of view, along with the redshift) to be output (see \S\ref{sec:OutputLightcone}), and {\normalfont \ttfamily [mergerTreeOperatorMethod]} is set to {\normalfont \ttfamily pruneLightcone} to cause any merger trees which have no nodes within the lightcone volume to be pruned away (as there is no need to process them). Finally, the {\normalfont \ttfamily geometryLightconeMethod} parameter describes the geometry of the lightcone to be used---see \S\ref{sec:methodsGeometryLightcone} for details.

\section{Using the Instantaneous Recycling Approximation}\index{recycling!instantaneous}\index{instantaneous recycling approximation}

Choosing {\normalfont \ttfamily [stellarPopulationPropertiesMethod]}$=${\normalfont \ttfamily instantaneous} will cause \glc\ to use the instantaneous recycling approximation for all calculations of stellar populations. The recyling rate and yield to use are set by the {\normalfont \ttfamily [imfNAMERecycledInstantaneous]} and {\normalfont \ttfamily [imfNAMEYieldInstantaneous]} parameters respectively, where {\normalfont \ttfamily NAME} is the name of the appropriate \gls{imf}.

Setting {\normalfont \ttfamily [stellarPopulationPropertiesMethod]}$=${\normalfont \ttfamily noninstantaneous} causes \glc\ to use a fully non-instantaneous, metal-depdendent calculation of recycling, metal production and \gls{sne} rates. However, it is possible to force this method to operate in the instantaneous recycling approximation limit (which can be useful for testing and comparison) by setting:
\begin{verbatim}
  <!-- Force the calculation of recycling, yields etc. to   -->
  <!-- be done assuming instantaneous recycling             -->
  <starFormationImfInstantaneousApproximation value="true"/>
  <!-- Set the mass of stars which should be used as the    -->
  <!-- dividing line between long-lived and instantaneously -->
  <!-- evolving in this approximation.                      -->
  <starFormationImfInstantaneousApproximationMassLongLived value="1.0"/>
  <!-- Set the effective age of populations to use in this -->
  <!-- approximation when computing SNe numbers.           -->
  <starFormationImfInstantaneousApproximationEffectiveAge value="13.8"/>
\end{verbatim}

\section{Postprocessing of Stellar Spectra}

Stellar luminosities are computed by convolving a library of simple stellar populations with the star formation history of each galaxy. \glc\ allows the spectra of those simple stellar populations to be postprocessed (after being read from file or internally generated for example) before they are utilized in the convolution integral. This postprocessing can modify the spectra in arbitrary ways that depend on wavelength, redshift, and age of stellar population. Furthermore, \glc\ allows you to chain together stellar spectra postprocessors into a set to allow multiple postprocessings to be applied. Furthermore again, you can define an arbitrary number of sets and apply different sets to different luminosities.

Typical uses of stellar spectra postprocessors include accounting for absorption of galaxy light by the intervening \gls{igm}, or capturing only the light from recent star formation\footnote{Perhaps so that additional dust extinction can be applied to the light of recently formed stars.}. A full list of the available postprocessors can be found in \S\ref{sec:StellarSpectraPostprocessing}.

If you don't specify a postprocessing set, the ``default'' set (consisting of the {\normalfont \ttfamily inoue2014} postprocessor; see \S\ref{phys:spectraPostprocessor:spectraPostprocessorInoue2014}) is applied to each luminosity calculation. To specify other postprocessing sets add the following to your parameter file:
\begin{verbatim}
 <luminosityPostprocessSet value="default recent unabsorbed recentUnabsorbed"/>
\end{verbatim}
where one set must be specified for each luminosity specified in the {\normalfont \ttfamily luminosityFilter} parameter. Note that set names can be reused in order to apply the same postprocessor set to multiple luminosities.

The chain of postprocessors to apply for each set is then specified as follows:
\begin{verbatim}
 <stellarPopulationSpectraPostprocessRecentMethods           value="inoue2014 recent"/>
 <stellarPopulationSpectraPostprocessUnabsorbedMethods       value="identity"        />
 <stellarPopulationSpectraPostprocessRecentUnabsorbedMethods value="recent"          />
\end{verbatim}
In this case we've constructed three new sets, in addition to the default set (which applies just the {\normalfont \ttfamily inoue2014} postprocessor). The {\normalfont \ttfamily recent} set applies both the {\normalfont \ttfamily inoue2014} \gls{igm} absorption postprocessor, followed by the {\normalfont \ttfamily recent} postprocessor to retain only recently emitted light. The {\normalfont \ttfamily unabsorbed} set ignores \gls{igm} absorption entirely---it does this by using the {\normalfont \ttfamily identity} postprocessor which leaves the spectrum unaffected. Finally, the {\normalfont \ttfamily recentUnabsorbed} set applies only the {\normalfont \ttfamily recent} filter while ignoring \gls{igm} absorption.

In this way it is relatively easy to extract multiple different measures of luminosity from a \glc\ model. For example, you could construct four postprocessor sets, each corresponding to one of the four different \gls{igm} absorption models ({\normalfont \ttfamily lycSuppress}, {\normalfont \ttfamily madau1995}, {\normalfont \ttfamily meiksin2006}, and {\normalfont \ttfamily inoue2014}) and apply these to the same luminosity filter to assess how luminosity depends on the \gls{igm} model used.

\section{Migrating Parameter Files to a New Version}\label{sec:MigrateParameters}\index{migration}\index{parameters}

The names and allowed values of parameters often change between versions of \glc. To permit easy and error-free migration between versions a script is provided to translate parameter files from earlier to later versions. To migrate a parameter file simply use:
\begin{verbatim}
scripts/aux/parametersMigrate.pl parameters.xml newParameters.xml
\end{verbatim}
By default, this script will translate from the previous to the current version of \glc. If your parameter file contains a {\normalfont \ttfamily version} element then this will be used to determine which version of \glc\ the parameter file was constructed for. The migration script will then migrate the parameter file through all intermediate versions to bring it into compliance with the current version. You can also specify input and output versions directly:
\begin{verbatim}
scripts/aux/parametersMigrate.pl parameters.xml newParameters.xml --inputVersion 0.9.0 --outputVersion 0.9.3
\end{verbatim}
will convert {\normalfont \ttfamily parameters.xml} from version 0.9.0 syntax to version 0.9.3 syntax.

\subsection{Reionization Calculations}\label{sed:ReionziationTutorial}\index{reionization}

\glc\ can self-consistently solve for the evolution of the \gls{igm} as it becomes photoionized by light emitted by stars and AGN. To activate this calculation, include the following in your parameters file:
\begin{verbatim}
  <!-- IGM evolver -->
  <intergalacticMediumStateMethod  value="internal"/>
  <igmPropertiesCompute            value="true"    />
  <igmPropertiesTimeCountPerDecade value="10"      />
  
  <!-- Background radiation -->
  <backgroundRadiationCompute                  value="true"    />
  <radiationIntergalacticBackgroundMethod      value="internal"/>
  <backgroundRadiationWavelengthCountPerDecade value="50"      />
  <backgroundRadiationTimeCountPerDecade       value="10"      />

  <!-- Halo accretion options -->
  <accretionHaloMethod value="naozBarkana2007"/>
\end{verbatim}
The first block of parameters switches \glc\ to using an internal calculation for the state of the \gls{igm}, instructs it to solve for \gls{igm} properties as a function of time, and specifies that \gls{igm} properties should be updated 10 times per decade of cosmic time. Specifically, at each of these time intervals, solving of galaxy evolution is halted and the \gls{igm} evolved up to this time using the currently computed photoionizing background spectrum.

The second block of parameters activates an internal calculation of cosmic background radiation, in which the background is computed from the emissivities of model galaxies and AGN. The number of points at which to tabulate the background per decade of wavelength and cosmic time are specified.

Finally, the third block of parameters tells \glc\ to use the \cite{naoz_formation_2007} prescription for computing gas accretion into halos from the \gls{igm}. This prescription uses the filtering mass to determine accretion rates, and will take the filtering mass from the internal \gls{igm} evolution calculation.

With these three sets of configurations, \glc\ will perform a self-consistent evolution of the \gls{igm}---in the sense that the \gls{igm} is ionized by photons emitted by model galaxies and AGN, while galaxy evolution is affected by the computed state of the model \gls{igm}. Note that, when run in this way, \glc\ needs to keep all merger trees in memory simultaneously (as they are run synchronously to allow the \gls{igm} properties to evolved alongside galaxy properties).

Once completed, data on the \gls{igm} and background radiation are written to the output file in the {\normalfont \ttfamily igmProperties} and {\normalfont \ttfamily backgroundRadiation} groups respectively.


\chapter{Numerical Implementation}

\section{Timestepping Criteria}\label{sec:TimesteppingCriteria}\index{timesteps!criteria}

\glc\ evolves each merger tree forest by repeatedly walking the trees and evolving each node forward in time by some timestep $\Delta t$. Nodes are evolved individually such that nodes in different branches of a tree may have reached different cosmic times at any given point in the execution of \glc. Each node is evolved over the interval $\Delta t$ using an adaptive \gls{ode} solver, which adjusts the smaller timesteps, $\delta t$, taken in evolving the system of \glspl{ode} to maintain a specified precision.

The choice of $\Delta t$ then depends on other considerations. For example, a node should not be evolved beyond the time at which it is due to merge with another galaxy. Also, we typically don't want satellite nodes to evolve too far ahead of their host node, such that any interactions between satellite and host occur (near) synchronously.

In the remainder of this section we list all criteria used to select $\Delta t$ for a node. All criteria are considered and the largest $\Delta t$ consistent with all criteria is selected.

\subsection{Tree Criteria}

The following \hyperlink{merger_trees.evolve.F90:merger_trees_evolve:evolve_to_time}{timestep criteria} ensure that tree evolution occurs in a way which correctly preserves tree structure and ordering of interactions between \glspl{node}.

\subsubsection{``Branch Segment'' Criteria}

For \glspl{node} which are the \gls{primary progenitor} of their \gls{parent}, the ``branch segment'' criterion asserts that
\begin{equation}
 \Delta t \le t_{\mathrm parent} - t
\end{equation}
where $t$ is current time in the \gls{node} and $t_{\mathrm parent}$ is the time of the \gls{parent} \gls{node}. This ensures that \gls{primary progenitor} \glspl{node} to not evolve beyond the time at which their \gls{parent} (which they will replace) exists.  If this criterion is the limiting criteria for $\Delta t$ then the \gls{node} will be promoted to replace its \gls{parent} at the end of the timestep. 

\subsubsection{``Parent'' Criteria}

For \glspl{node} which are satellites in a hosting \gls{node} the ``\gls{parent}'' timestep criterion asserts that
\begin{eqnarray}
\Delta t &\le& t_{\mathrm host}, \\
\Delta t &\le& \epsilon_{\mathrm host} (a/\dot{a}),
\end{eqnarray}
where $t_{\mathrm host}=${\normalfont \ttfamily [timestepHostAbsolute]}, $\epsilon_{\mathrm host}=${\normalfont \ttfamily [timestepHostRelative]}, and $a$ is expansion factor. These criteria are intended to prevent a satellite for evolving too far ahead of the host node before the host is allowed to ``catch up''.

\subsubsection{``Satellite'' Criteria}

For \glspl{node} which host satellite \glspl{node}, the ``satellite'' criterion asserts that
\begin{equation}
 \Delta t \le \hbox{min}(t_{\mathrm satellite}) - t,
\end{equation}
where $t$ is the time of the host \gls{node} and $t_{\mathrm satellite}$ are the times of all satellite \glspl{node} in the host. This criterion prevents a host from evolving ahead of any satellites.

\subsubsection{``Sibling'' Criteria}

For \glspl{node} which are \glspl{primary progenitor}, the ``sibling'' criterion asserts that
\begin{equation}
 \Delta t \le \hbox{min}(t_{\mathrm sibling}) - t,
\end{equation}
where $t$ is the time of the host \gls{node} and $t_{\mathrm sibling}$ are the times of all siblings of the \gls{node}. This criterion prevents a \gls{node} from reaching its \gls{parent} (and being promoted to replace it) before all of its siblings have reach the \gls{parent} and have become satellites within it.

\subsubsection{``Mergee'' Criteria}

For \glspl{node} with \gls{mergee} \glspl{node}, the ``\gls{mergee}'' criterion asserts that
\begin{equation}
 \Delta t \le \hbox{min}(t_{\mathrm merge}) - t,
\end{equation}
where $t$ is the time of the host \gls{node} and $t_{\mathrm merge}$ are the times at which the \glspl{mergee} will merge. This criterion prevents a \gls{node} from evolving past the time at which a merger event takes place.

\subsection{General Criteria}

\subsubsection{``Simple'' Criteria}

The \hyperlink{merger_trees.evolve.timesteps.simple.F90:merger_tree_timesteps_simple:merger_tree_timestep_simple}{``simple''} timestep criteria assert that
\begin{eqnarray}
\Delta t &\le& t_{\mathrm simple}, \\
\Delta t &\le& \epsilon_{\mathrm simple} (a/\dot{a}),
\end{eqnarray}
where $t_{\mathrm simple}=${\normalfont \ttfamily [timestepSimpleAbsolute]}, $\epsilon_{\mathrm simple}=${\normalfont \ttfamily [timestepSimpleRelative]}, and $a$ is expansion factor. These criteria are intended to prevent any one node evolving over an excessively large time in one step. In general, these criteria are not necessary, as nodes should be free to evolve as far as possible unless prevented by some physical requirement. These criteria are therefore present to provide a simple example of how timestep criteria work.

\subsubsection{``Satellite'' Criteria}

The \hyperlink{merger_trees.evolve.timesteps.satellite.F90:merger_tree_timesteps_satellite:merger_tree_timestep_satellite}{``satellite''} timestep criteria asserts the following for satellite \glspl{node}. If the satellite's merge target has been advanced to at least a time of $t_{\mathrm required} = t_{\mathrm satellite} + \Delta t_{\mathrm merge} - \delta t_{\mathrm merge,maximum}$ then 
\begin{equation}
\Delta t \le \Delta t_{\mathrm merge},
\end{equation}
where $t_{\mathrm satellite}$ is the current time for the satellite \gls{node}, $\Delta t_{\mathrm merge}$ is the time until the satellite is due to merge and $\delta t_{\mathrm merge,maximum}$ is the maximum allowed time difference between merging galaxies. This ensures that the satellite is not evolved past the time at which it is due to merge. If this criterion is the limiting criteria for $\Delta t$ then the merging of the satellite will be triggered at the end of the timestep.

If the merge target has not been advanced to at least $t_{\mathrm required}$ then instead
\begin{equation}
\Delta t \le \hbox{max}(\Delta t_{\mathrm merge}-\delta t_{\mathrm merge,maximum}/2,0),
\end{equation}
is asserted to ensure that the satellite does not reach the time of merging until its merge target is sufficiently close (within $\delta t_{\mathrm merge,maximum}$) of the time of merging.

\subsection{Output Criteria}

\subsubsection{``History'' Criteria}

The \hyperlink{merger_trees.evolve.timesteps.history.F90:merger_tree_timesteps_history:merger_tree_timestep_history}{``history''} timestep criterion asserts that
\begin{equation}
 \Delta t \le t_{{\mathrm history},i} - t
\end{equation}
where $t$ is the current time, $t_{{\mathrm history},i}$ is the $i^{\mathrm th}$ time at which the global history (see \S\ref{sec:globalHistory}) of galaxies is to be output and $i$ is chosen to be the smallest $i$ such that $t_{{\mathrm history},i} > t$. If there is no $i$ for which $t_{{\mathrm history},i} > t$ this criterion is not applied. If this criterion is the limiting criterion for $\Delta t$ then the properties of the galaxy will be accumulated to the global history arrays at the end of the timestep.

\subsubsection{``Main Branch Evolution'' Criteria}

If {\normalfont \ttfamily timestepRecordEvolution}$=${\normalfont \ttfamily true}, then the \hyperlink{merger_trees.evolve.timesteps.record_evolution.F90:merger_tree_timesteps_record_evolution:merger_tree_timestep_record_evolution}{``main branch evolution''} timestep criterion asserts that
\begin{equation}
 \Delta t \le t_{{\mathrm record},i} - t
\end{equation}
where $t$ is the current time, $t_{{\mathrm record},i}$ is the $i^{\mathrm th}$ time at which the evolution of main branch galaxies is to be output and $i$ is chosen to be the smallest $i$ such that $t_{{\mathrm record},i} > t$. If there is no $i$ for which $t_{{\mathrm record},i} > t$ this criterion is not applied. If this criterion is the limiting criterion for $\Delta t$ then the properties of the galaxy will be recorded at the end of the timestep.


\part{Advanced Use}

\chapter{Constraining {\normalfont \scshape Galacticus}}

\section{Constraint File}

\glc\ has a complete constraints infrastructure which implements various \gls{mcmc} algorithms to analyze the posterior probability distribution of the model given some compilation of constraints. The infrastructure is \gls{mpi} parallelized and ideal for running on large compute clusters.

To perform a constraint calculation simply build the constraint code:
\begin{verbatim}
 make Constrain_Galacticus.exe
\end{verbatim}
and run with a parameter file and configuration file. Typically, you will want to run this code under \gls{mpi}, for example:
\begin{verbatim}
 mpirun -n 4 Constrain_Galacticus.exe mcmcParameters.xml mcmcConfig.xml
\end{verbatim}
would run 4 processes (typically you will need to run many more than this). If running on a \gls{pbs} queue, embed this command in a suitable \gls{pbs} script and submit. 

The parameter file follows the same format as a standard \glc\ parameter file and specifies the values of parameters to be used. For example, the seed used fo pseudo-random number sequences can be specified in this file. 

The configuration file specifies the details of the constraint simulation to be performed. An example configuration file is:
\begin{verbatim}
<?xml version="1.0" encoding="UTF-8"?>
<simulationConfig>

  <likelihood>
    <type>Galacticus</type>
    <name>verySimplisticToStellarMassFunction</name>
    <compilation>stellarMassFunction_SDSS_z0.07.xml</compilation>
    <baseParameters>./mcmcWork/verySimplisticToStellarMassFunctionBase.xml</baseParameters>
    <workDirectory>./mcmcWork</workDirectory>
    <scratchDirectory>./mcmcScratch</scratchDirectory>
    <report>no</report>
    <randomize>no</randomize>
    <threads>4</threads>
    <saveState>no</saveState>
    <cpulimit>1200</cpulimit>
    <coredump>NO</coredump>
    <coredumpsize>0</coredumpsize>
    <sequentialModels>no</sequentialModels>
    <memoryLimit>2gb</memoryLimit>
    <environment>LD_LIBRARY_PATH=/opt/gcc-trunk/lib:/opt/gcc-trunk/lib64:/usr/local/upstream/lib:$LD_LIBRARY_PATH</environment>
    <environment>PATH=/opt/gcc-trunk/bin:$PATH</environment>
  </likelihood>

  <convergence>
    <type>GelmanRubin</type>
    <Rhat>1.2</Rhat>
    <burnCount>100</burnCount>
    <testCount>100</testCount>
    <outlierCountMaximum>0</outlierCountMaximum>
    <outlierSignificance>0.95</outlierSignificance>
    <outlierLogLikelihoodOffset>60</outlierLogLikelihoodOffset>
  </convergence>
  
  <state>
    <type>history</type>
    <acceptedStateCount>100</acceptedStateCount>
  </state>
  
  <proposalSize>
    <type>adaptive</type>
    <gammaInitial>1.77</gammaInitial>
    <gammaFactor>1.414</gammaFactor>
    <acceptanceRateMinimum>0.4</acceptanceRateMinimum>
    <acceptanceRateMaximum>0.6</acceptanceRateMaximum>
    <updateCount>10</updateCount>
  </proposalSize>
  
  <randomJump>
    <type>adaptive</type>
  </randomJump>
  
  <simulation>
    <type>temperedDifferentialEvolution</type>
    <stepsMaximum>1000000</stepsMaximum>
    <stepsPostConvergence>100000</stepsPostConvergence>
    <acceptanceAverageCount>100</acceptanceAverageCount>
    <logFileRoot>./mcmcWork/mcmc/chains</logFileRoot>
    <temperatureMaximum>64.0</temperatureMaximum>
    <untemperedStepCount>20</untemperedStepCount>
    <temperedLevels>10</temperedLevels>
    <stepsPerLevel>10</stepsPerLevel>
    <logFlushCount>10<logFlushCount>
  </simulation>

  <parameters>
    <parameter>
      <name>starFormationTimescaleDisksHaloScalingVirialVelocityExponent</name>
      <prior>
	<distribution>
	  <type>uniform</type>
	  <minimum>-6.0</minimum>
	  <maximum>+0.0</maximum>
	</distribution>
      </prior>
      <mapping>
        <type>linear</type>
      </mapping>
      <random>
	<type>Cauchy</type>
	<median>0.0</median>
	<scale>0.006</scale>
      </random>
    </parameter>
    <parameter>
      <name>starFormationTimescaleDisksHaloScalingRedshiftExponent</name>
      <prior>
	<distribution>
	  <type>uniform</type>
	  <minimum>-1.0</minimum>
	  <maximum>+4.0</maximum>
	</distribution> 
      </prior>
      <mapping>
        <type>linear</type>
      </mapping>
      <random>
	<type>Cauchy</type>
	<median>0.0</median>
	<scale>0.005</scale>
      </random>
    </parameter>
  </parameters>
  
</simulationConfig>
\end{verbatim}

\subsection{Parameters and Priors}\label{sec:ParametersPriors}

The {\normalfont \ttfamily parameters} section contains a list of all parameters to be varied in the analysis. Each parameter is described by one {\normalfont \ttfamily parameter} element. That element must contain a {\normalfont \ttfamily name} element, which gives the name of the parameter, a {\normalfont \ttfamily prior} element that contains a {\normalfont \ttfamily distribution} element defining the distribution for this prior, a {\normalfont \ttfamily mapping} element that describes the mapping of the parameter into the internal state used in \gls{mcmc} calculations, and (for differential evolution simulations) a {\normalfont \ttfamily random} element that defines the distribution to be used for the random perturbation to be added to this parameter in proposals.

The {\normalfont \ttfamily name} element can specify subparameters, elements in an array of parameters, and elements within a parameter's {\normalfont \ttfamily value} element. For example, consider the parameter file:
\begin{verbatim}
<parameter1 value="123"/>
<parameterA value="456"/>
<parameterA value="789"/>
<parameterA value="987"/>
<parameter2 value="abc">
 <parameter2a value="def"/>
</parameter2>
<parameterX value="1.0 2.0 3.0"/>
\end{verbatim}
\begin{itemize}
\item A name of {\normalfont \ttfamily parameter1} will change the value ({\normalfont \ttfamily 123}) of the {\normalfont \ttfamily parameter1} element;
\item A name of {\normalfont \ttfamily parameterA[1]} will change the value ({\normalfont \ttfamily 1789}) of the second {\normalfont \ttfamily parameterA} element (array indexing is 0-offset);
\item A name of {\normalfont \ttfamily parameter2->parameter2a} will change the value ({\normalfont \ttfamily def}) of the {\normalfont \ttfamily parameter2a} element;
\item A name of {\normalfont \ttfamily parameterX{1}} will change the value ({\normalfont \ttfamily 2.0}) of second entry in the value of the {\normalfont \ttfamily parameterX} element.
\end{itemize}

Currently allowed mappings are:
\begin{itemize}
\item[{\normalfont \ttfamily linear}] Effectively a null mapping, as the parameter is mapped into itself: $x \rightarrow x$.
\item[{\normalfont \ttfamily logarithmic}] The parameter is mapped logarithmically: $x \rightarrow \log(x)$. This mapping can be applied only to uniform priors.
\end{itemize}
Note that the mapping will map the values of the prior. For example, if you specify a uniform prior with a logarithmic mapping, the upper and lower limits of the prior should be specified on $x$, not on $\log(x)$. These limits will be mapped appropriately internally.

\subsubsection{Loading External Parameters/Priors}

It is also possible to load parameters and their priors from external files. This is useful to add common sets of parameters, such as cosmological parameter. To do so, add an element of the form:
\begin{verbatim}
<xi:include href="../../constraints/parameters/wmap9Cosmology.xml" 
   xmlns:xi="http://www.w3.org/2001/XInclude" />
\end{verbatim}
\emph{after} the {\normalfont \ttfamily parameters} section of the constraint file. The {\normalfont \ttfamily href} attribute must give the path (relative to the constraint file, or absolute) to the external parameter file. This file should contain its own {\normalfont \ttfamily parameters} block, describing all parameters to be varied along with their priors. 

\subsubsection{Derived Parameter Values}

It is possible to define parameters in terms of other parameters. Common uses for this include:
\begin{itemize}
 \item Setting $\Omega_\Lambda$ from the value of $\Omega_\mathrm{M}$ to enforce a flat Universe;
 \item Setting the values of parameters with correlated priors as linear combinations of dummy parameters for which the priors are independent.
\end{itemize}
To define a parameter in this way include a {\normalfont \ttfamily parameter} element of the form:
\begin{verbatim}
<parameter>
 <name>sigma_8</name>
 <define>0.8178+%cosmology0*0.003817+%cosmology1*0.007931+%cosmology2*0.01002
    +%cosmology3*0.001584+%cosmology4*0.002931+%cosmology5*0.001727</define>
</parameter>
\end{verbatim}
Here the {\normalfont \ttfamily define} element gives an equation for the parameter in terms of other parameters. All standard mathematical operators and functions (as recognized by Perl) can be used, and other parameters referenced by using their name prefixed with a ``\%''.

\subsubsection{Including External Parameters}

Predefined sets of parameters (along with their priors) can be included using the {\normalfont \ttfamily xi:include} element. For example,
\begin{verbatim}
 <xi:include href="../../constraints/parameters/wmap7Cosmology.xml"
    xmlns:xi="http://www.w3.org/2001/XInclude" />
\end{verbatim}
will include a set of parameters from the file {\normalfont \ttfamily ../../constraints/parameters/wmap7Cosmology.xml} which defines priors on cosmological parameters consistent with the covariance matrix of the WMAP-7 cosmological constraints \citep{komatsu_seven-year_2010}.


\part{Physical Implementation}

In this Part we describe the physical implementation of galaxy formation in \glc, including all components and their properties and additional output quantities from the code.

\chapter{Definitions and Conventions Used in \glc}

\glc\ adopts various definitions and conventions internally. These are explained below.

\section{Luminosity Units}

Galaxy luminosities are output in the \gls{ABmagnitude} system, such that a luminosity of $1$ corresponds to an object of $0^{\mathrm th}$ absolute magnitude in the \gls{ABmagnitude} system. This implies that the luminosities are in units of $4.4659\times 10^{13}$~W/Hz.

\section{Peculiar Velocities}\label{sec:GalacticusVelocityDefinitions}

Velocities in \glc\ are always \emph{physical} velocities. When reading merger tree properties (including velocities) from file it is often convenient to store velocities without the Hubble flow contribution, as ``peculiar velocities'', in the file---see \S\ref{sec:ForestHalosGroup} for how to specify whether or not  the velocities included in the file include the Hubble flow or not.

If peculiar velocities are stored it is important to use the same definition of pecular velocity as is used by \glc. Defininf $t$ to be physical time and ${\mathbf x}$ to be comoving position, \glc\ uses the conventional definition of peculiar velocity in a cosmological context, namely that it is the deviation of the physical velocity from the Hubble flow. Physical coordinates are given by ${\mathbf r} = a{\mathbf x}$, so the peculiar velocity is
\begin{equation}
{\mathbf v}_{\mathrm pec} \equiv {{\mathrm d} {\mathbf r} \over {\mathrm d} t} - H {\mathbf r} = a {{\mathrm d} {\mathbf x}\over{\mathrm d} t} = {{\mathrm d}{\mathbf x}\over{\mathrm d}\eta},
\end{equation}
where ${\mathrm d}\eta = {\mathrm d}t/a$ is conformal time. 


\chapter{Node Components}

\section{(Supermassive) Black Hole}

\subsection{``Null'' Implementation}

The null black hole implementation defines the same properties as all other black hole implementations, but sets the methods to point to dummy routines (for rate adjustment and derivative computation) or to {\tt null()} for get/set methods. It can be used to effectively switch off black holes. Of course, this is safe only if none of the other active components expect to get or set black hole properties (or if they rely on a sensible implementation of black hole evolution).

\subsection{``Standard'' Implementation}

\subsubsection{Properties}

The standard black hole implementation defines the following properties:
\begin{description}
 \item [{\tt Black\_Hole\_Mass}] The mass of the black hole: $M_\bullet$ {\tt [blackHoleMass]}.
 \item [{\tt Black\_Hole\_Spin}] The spin of the black hole, $j_\bullet$ {\tt [blackHoleSpin]}.
\end{description}

\subsubsection{Initialization}

Black holes are not initialized, they are created (with a seed mass given by {\tt blackHoleSeedMass} and zero spin) as needed.

\subsubsection{Differential Evolution}

In the standard black implementation the mass and spin evolve as:
\begin{eqnarray}
\dot{M}_\bullet &=& (1-\epsilon_{\rm radiation}) \dot{M}_0 \\
\dot{j}_\bullet &=& \dot{j}(M_\bullet,j_\bullet,\dot{M}_0),
\end{eqnarray}
where $\dot{M}_0$ is the rest mass accretion rate, $\epsilon_{\rm radiation}$ is the radiative efficiency of the accretion flow feeding the black hole and $\dot{j}(M_\bullet,j_\bullet,\dot{M}_0)$ is the spin-up function of that accretion flow (see \S\ref{sec:AccretionDisks}). The rest mass accretion rate is computed assuming Bondi-Hoyle-Lyttleton accretion from the spheroid gas reservoir (with an assumed temperature of {\tt [bondiHoyleAccretionTemperatureSpheroid]}) enhanced by a factor of {\tt [bondiHoyleAccretionEnhancementSpheroid]} and from the host halo (with whatever temperature that hot halo temperature profile specifies; see \S\ref{sec:HotHaloTemperature}) enhanced by a factor of {\tt [bondiHoyleAccretionEnhancementHotHalo]}. The rest mass accretion rate is removed (as a mass sink) from the spheroid component. The black hole is assumed to cause feedback in two ways:
\begin{description}
 \item [Radio-mode] If {\tt [blackHoleHeatsHotHalo]}$=${\tt true} then any jet power from the black hole-accretion disk system (see \S\ref{sec:CircumnuclearDisks}) is included in the hot halo heating rate providing that the halo is in the slow cooling regime\index{feedback!AGN}\index{active galactic nuclei (AGN)!feedback} (i.e. if the cooling radius is smaller than the virial radius; see, for example, \citealt{benson_cold_2010});
 \item [Quasar-mode] A mechanical wind luminosity of \citep{ostriker_momentum_2010}
\begin{equation}
 L_{\rm wind} = \epsilon_{\bullet, wind} \dot{M}_0 \clight^2,
\end{equation}
where $\epsilon_{\bullet wind}=${\tt [blackHoleWindEfficiency]} is the black hole wind efficiency, is added to the gas component of the spheroid (which, presumably, will respond with an outflow for example) if and only if the wind pressure (at the spheroid characteristic radius) is less than the typical thermal pressure in the spheroid gas \citep{ciotti_feedbackcentral_2009}, i.e.
\begin{eqnarray}
 P_{\rm wind} &<& P_{\rm ISM} \nonumber \\
 \frac{1}{2}\rho_{\rm wind} V_{\rm wind}^2 &<& {3 {\rm k_B} T_{\rm ISM} \langle \rho_{\rm ISM}\rangle \over 2 m_{\rm H}}.
\end{eqnarray}
Since $\Omega r^2 \rho_{\rm wind} V_{\rm wind}^3 = L_{\rm wind}$ where $\Omega$ is the solid angle of the wind flow, this can be rearranged to give $\langle\rho_{\rm ISM}\rangle > \rho_{\rm wind, critical}$ where
\begin{equation}
\rho_{\rm wind,critical} = {2 m_{\rm H} L_{\rm wind} \over 3 \Omega r^2 V_{\rm wind} {\rm k_B} T_{\rm ISM}}.
\end{equation}
This critical wind density is computed at the characteristic radius of the spheroid, $r_{\rm spheroid}$, assuming $V_{\rm wind}=10^4$km/s, $T_{\rm ISM}=10^4$K and $\Omega=\pi$, and the \ISM\ density is approximated by
\begin{equation}
 \langle\rho_{\rm ISM}\rangle = {3 M_{\rm gas, spheroid} \over 4 \pi} r_{\rm spheroid}^3.
\end{equation}
For numerical ease, the fraction, $f_{\rm wind}$, of the wind luminosity added to the spheroid is adjusted smoothly through the $\rho_{\rm ISM}\approx\rho_{\rm wind,critical}$ region according to
\begin{equation}
 f_{\rm wind} = \left\{ \begin{array}{ll} 0 & \hbox{ if } x < 0, \\ 3x^2-2x^3 & \hbox{ if } 0 \le x \le 1, \\ 1 & \hbox{ if } x > 1, \end{array} \right.
\end{equation}
where $x=\rho_{\rm ISM}/\rho_{\rm wind,critical}-1/2$.
\end{description}

\subsubsection{Event Evolution}

\noindent\emph{Node mergers:} None.\\

\noindent\emph{Satellite merging:} The black holes in the two merging galaxies are instantaneously merged. Properties are computed using the selected black hole binary merger method (see \S\ref{sec:BlackHoleBinaryMergers}).\\

\noindent\emph{Node promotion:} None.\\

\subsubsection{Additional Output}

If the {\tt [blackHoleOutputAccretion]}\index{black holes!accretion}\index{accretion!black holes} input parameter is set to true, then rest mass accretion rate (in $M_\odot$ Gyr$^{-1}$), jet power (in $M_\odot$ km$^2$ s$^{-1}$ Gyr$^{-1}$) and radiative efficiency of the black hole\footnote{Technically of the black hole plus accretion disk system.} are output as {\tt blackHoleAccretionRate}, {\tt blackHoleJetPower} and {\tt blackHoleRadiativeEfficiency} respectively.

\subsection{``Simple'' Implementation}

\subsubsection{Properties}

The simple black hole implementation defines the following property:
\begin{description}
 \item [{\tt Black\_Hole\_Mass}] The mass of the black hole: $M_\bullet$ {\tt [blackHoleMass]}.
\end{description}

\subsubsection{Initialization}

Black holes are not initialized, they are created (with a seed mass given by {\tt blackHoleSeedMass}) as needed.

\subsubsection{Differential Evolution}

In the simple black hole implementation the mass evolves as:
\begin{eqnarray}
\dot{M}_\bullet &=& (1-\epsilon_{\rm wind}-\epsilon_{\rm heat}) \epsilon_{\rm BH} \dot{M}_{\star,s{\rm pheroid}} \\
\end{eqnarray}
where $\epsilon_{\rm BH}$ is the ratio of rates at which the black hole and stellar spheroid grow. The black hole is assumed to cause feedback in two ways:
\begin{description}
 \item [Radio-mode] If {\tt [blackHoleHeatsHotHalo]}$=${\tt true} then a power $\epsilon_{\rm heat}=${\tt [blackHoleHeatingEfficiency]} is included in the hot halo heating rate providing that the halo is in the slow cooling regime\index{feedback!AGN}\index{active galactic nuclei (AGN)!feedback} (i.e. if the cooling radius is smaller than the virial radius; see, for example, \citealt{benson_cold_2010});
 \item [Quasar-mode] A mechanical wind luminosity of \citep{ostriker_momentum_2010}
\begin{equation}
 L_{\rm wind} = \epsilon_{\bullet, wind} \dot{M}_0 \clight^2,
\end{equation}
where $\epsilon_{\bullet wind}=${\tt [blackHoleWindEfficiency]} is the black hole wind efficiency, is added to the gas component of the spheroid (which, presumably, will respond with an outflow for example).
\end{description}

\subsubsection{Event Evolution}

\noindent\emph{Node mergers:} None.\\

\noindent\emph{Satellite merging:} The black holes in the two merging galaxies are instantaneously merged. Properties are computed using the selected black hole binary merger method (see \S\ref{sec:BlackHoleBinaryMergers}).\\

\noindent\emph{Node promotion:} None.\\

\subsubsection{Additional Output}

If the {\tt [blackHoleOutputAccretion]}\index{black holes!accretion}\index{accretion!black holes} input parameter is set to true, then rest mass accretion rate (in $M_\odot$ Gyr$^{-1}$) is output as {\tt blackHoleAccretionRate}.

\section{Hot Halo}

\subsection{``Null'' Implementation}

The null hot halo implementation leaves all methods to point to dummy routines (for rate adjustment and derivative computation) or to {\tt null()} for get/set methods. It can be used to effectively switch off hot halos. Of course, this is safe only if none of the other active components expect to get or set hot halo properties (or if they rely on a sensible implementation of hot halo evolution).

\subsection{``Standard'' Implementation}

\subsubsection{Properties}

The standard hot halo implementation defines the following properties:
\begin{description}
 \item [{\tt Hot\_Halo\_Unaccreted\_Mass}] The mass of gas in the hot halo: $M_{\rm failed}$.
 \item [{\tt Hot\_Halo\_Mass}] The mass of gas in the hot halo: $M_{\rm hot}$ {\tt [hotHaloMass]}.
 \item [{\tt Hot\_Halo\_Angular\_Momentum}] The angular momentum of the gas in the hot halo, $J_{\rm hot}$ {\tt [hotHaloAngularMomentum]}.
 \item [{\tt Hot\_Halo\_Abundances}] The mass(es) of heavy elements in gas in the hot halo, $M_{Z, {\rm hot}}$ {\tt [hotHalo\{abundanceName\}]}.
 \item [{\tt Hot\_Halo\_Outflowed\_Mass}] The mass of gas from outflows in the hot halo: $M_{\rm outflowed}$ {\tt [outflowedMass]}.
 \item [{\tt Hot\_Halo\_Outflowed\_Ang\_Mom}] The angular momentum of the outflowed gas in the hot halo, $J_{\rm outflowed}$ {\tt [outflowedAngularMomentum]}.
 \item [{\tt Hot\_Halo\_Outflowed\_Abundances}] The mass(es) of heavy elements in outflowed gas, $M_{Z, {\rm outflowed}}$ {\tt [outflowed\{abundanceName\}]}.
 \item [{\tt Hot\_Halo\_Molecules}] The mass(es) of molceules in the hot gas, $M_{\rm molecule}$ {\tt [hotHalo\{moleculeName\}]}.
\end{description}
and the following pipes:
\begin{description}
 \item [{\tt Hot\_Halo\_Heat\_Input}] Energy sent through this pipe is added to the hot halo and used to offset the cooling rate\index{hot halo!heating}\index{heating!hot halo} (see below; heat pushed should be in units if $M_\odot$ (km/s)$^2$ Gyr$^{-1}$).
 \item [{\tt Hot\_Halo\_Cooling\_$[$Mass|Angular\_Momentum|Abundances$]$\_To}] The net cooling rate of gas mass (and metal content and magnitude of angular momentum) is sent through this pipe. Any component may claim this pipe and connect to it, allowing it to receive the cooling gas.
 \item [{\tt Hot\_Halo\_Outflow\_$[$Mass|Angular\_Momentum|Abundances$]$\_To}] Galactic components that wish to expel gas due to an outflow can send that mass (plus metals and angular momentum) through this pipe, where it will be received into the hot halo component. 
 \item [{\tt Hot\_Halo\_Hot\_Gas\_Sink}] Removes gas (and proportionate amounts of angular momentum and elements) from the hot gas halo.
\end{description}

\subsubsection{Initialization}

At initialization, any nodes with no children are assigned a hot halo mass, and failed accreted mass as dictated by the baryonic accretion method (see \S\ref{sec:AccretionBaryonic}) and angular momentum based on the accreted mass and the halo spin parameter.

\subsubsection{Differential Evolution}

In the standard hot halo implementation the hot gas mass and heavy element mass(es) evolves as:
\begin{eqnarray}
 \dot{M}_{\rm failed} &=& \dot{M}_{\rm failed~accretion} \\
 \dot{M}_{\rm hot} &=& \dot{M}_{\rm accretion} - \dot{M}_{\rm cooling} + \dot{M}_{\rm outflow,return}, \\
 \dot{M}_{Z, {\rm hot}} &=& - \dot{M}_{\rm cooling} {M_{Z, {\rm hot}}\over M_{\rm hot}} + \dot{M}_{Z, {\rm outflow,return}}, \\
 \dot{M}_{\rm molecule} &=& - \dot{M}_{\rm cooling} {M_{\rm molecule}\over M_{\rm hot}} + f_{\rm molecule,outflow} \dot{M}_{\rm outflow,return} + \dot{M}_{\rm molecule,reactions} , \\
\end{eqnarray}
where $\dot{M}_{\rm accretion}$ is the rate of growth of the hot component due to accretion from the \IGM\ and $\dot{M}_{\rm failed~accretion}$ is the rate of failed accretion from the \IGM\ (these may include a component due to transfer of mass from the failed to accreted reservoirs) and $\dot{M}_{\rm cooling}$ is the rate of mass loss from the hot halo due to cooling (see \S\ref{sec:CoolingRate}---cooling rates are computed using the current node if {\tt [hotHaloCoolingFromNode]}$=${\tt current node} or from the formation node if that parameter is set to {\tt formation node}) minus any heating rate defined as
\begin{equation}
 \dot{M}_{\rm heating} = \dot{E}_{\rm input} / V_{\rm virial}^2,
\end{equation}
where $\dot{E}_{\rm input}$ is the rate at which energy is being sent through the ``energy input'' pipe and $V_{\rm virial}$ is the virial velocity of the halo.\footnote{The net cooling rate is never allowed to drop below zero. If the mass heating rate exceeds the mass cooling rate then the excess energy is not used.} In the above, $f_{\rm molecule,return}$ if the mass fraction of each molcular species in the outflowed gas and is assumed to be equal to that given by the atomic ionization state functions (see \S\ref{sec:IonizationStateMethod}) at the virial temperature and mean density of the halo. Finally, $\dot{M}_{\rm molecule,reactions}$ represents the rate of change of masses of molecular species due to chemical and atomic processes and is computed using the molecular rates functions (see \S\ref{sec:MolecularReactionRates}). The angular momentum of the hot gas evolves as:
\begin{equation}
 \dot{J}_{\rm hot} = \dot{M}_{\rm accretion} {\dot{J}_{\rm node} \over \dot{M}_{\rm node}} - \dot{M}_{\rm cooling} r_{\rm cool} V_{\rm rotate} + \dot{J}_{\rm outflow,return},
\end{equation}
where $\dot{M}_{\rm node}$ and $\dot{J}_{\rm node}$ are defined in \S\ref{sec:ComponentBasicProperties}. For the outflowed components:
\begin{eqnarray}
 \dot{M}_{\rm outflowed} &=& - \dot{M}_{\rm outflow,return} + \dot{M}_{\rm outflows}, \\
 \dot{M}_{Z, {\rm outflowed}} &=& - \dot{M}_{Z, {\rm outflow,return}} + \dot{M}_{Z, {\rm outflows}}, \\
\end{eqnarray}
and:
\begin{equation}
 \dot{J}_{\rm outflowed} = - \dot{J}_{\rm outflow,return} + \dot{J}_{\rm outflows}.
\end{equation}
In the above
\begin{equation}
 \dot{M}|\dot{M}_Z|\dot{J}_{\rm outflow,return} = \alpha_{\rm outflow~return~rate} {M|M_Z|J_{\rm outflowed}\over \tau_{\rm dynamical, halo}},
\end{equation}
where $\alpha_{\rm outflow~return~rate}=(${\tt hotHaloOutflowReturnRate}) is an input parameter controlling the rate at which gas flows from the outflowed to hot reservoirs, and $\dot{M}|\dot{M}_Z|\dot{J}_{\rm outflows}$ are the net rates of outflow from any components in the node. A fraction $1-${\tt [hotHaloAngularMomentumLossFraction]} of the cooling angular momentum rate, $\dot{M}_{\rm cooling} r_{\rm cool} V_{\rm rotate}$, is sent through the {\tt Hot\_Halo\_Cooling\_Angular\_Momentum} pipe.

\subsubsection{Event Evolution}

\noindent\emph{Node mergers:} If the {\tt starveSatellites} parameter is true, then any hot halo properties of the minor node are added to those of the major node and the hot halo component removed from the minor node. Additionally in this case, any material outflowed from the the satellite galaxy to its hot halo is transferred to the hot halo of the host dark matter halo after each timestep.\\

\noindent\emph{Satellite merging:} If the {\tt starveSatellites} parameter is false, then any hot halo properties of the satellite node are added to those of the host node and the hot halo component removed from the satellite node.\\

\noindent\emph{Node promotion:} Any hot halo properties of the parent node are added to those of the node prior to promotion.\\

\noindent\emph{Halo formation:} If {\tt [hotHaloOutflowReturnOnFormation]}$=${\tt true} then all outflowed gas is returned to the hot gas reservoir on halo formation events (see \S\ref{sec:HaloFormationEvents}).\\

\section{Galactic Disk}

\subsection{``Null'' Implementation}

The null disk implementation leaves all methods to point to dummy routines (for rate adjustment and derivative computation) or to {\tt null()} for get/set methods. It can be used to effectively switch off disks. Of course, this is safe only if none of the other active components expect to get or set disk properties (or if they rely on a sensible implementation of disk evolution).

\subsection{``Exponential'' Implementation}\label{sec:DiskExponential}

This implementation assumes a disk with an exponential surface density profile in which stars trace gas.

\subsubsection{Properties}

The exponential galactic disk implementation defines the following properties:
\begin{description}
 \item [{\tt Disk\_Gas\_Mass}] The mass of gas in the disk: $M_{\rm disk, gas}$ [{\tt diskGasMass}].
 \item [{\tt Disk\_Gas\_Abundances}] The mass of elements in the gaseous disk: $M_{Z, {\rm disk, gas}}$ [{\tt diskGas\{abundanceName\}}].
 \item [{\tt Disk\_Stellar\_Mass}] The mass of stars in the disk: $M_{\rm disk, stars}$ [{\tt diskStellarMass}].
 \item [{\tt Disk\_Stellar\_Abundances}] The mass of elements in the stellar disk: $M_{Z, {\rm disk, stars}}$ [{\tt diskStellar\{abundanceName\}}].
 \item [{\tt Disk\_Stellar\_Luminosities}] The luminosities (in multiple bands) of the stellar disk: $L_{\rm disk, stars}$ [{\tt diskStellar\{luminosityName\}}].
 \item [{\tt Disk\_Angular\_Momentum}] The angular momentum of the disk, $J_{\rm disk}$ [{\tt diskAngularMomentum}].
 \item [{\tt Disk\_Radius}] The radial scale length of the disk, $R_{\rm disk}$ [{\tt diskScaleLength}].
 \item [{\tt Disk\_Velocity}] The circular velocity of the disk at $R_{\rm disk}$, $V_{\rm disk}$ [{\tt diskCircularVelocity}].
\end{description}

\subsubsection{Initialization}

No initialization is performed---disks are created as needed.

\subsubsection{Differential Evolution}

In the exponential galactic disk implementation the gas mass evolves as:
\begin{equation}
 \dot{M}_{\rm disk, gas} = \dot{M}_{\rm cooling} - \dot{M}_{\rm outflow, disk} - \dot{M}_{\rm stars, disk} - M_{\rm disk, gas}/\tau_{\rm bar},
\end{equation}
where the rate of change of stellar mass is
\begin{equation}
 \dot{M}_{\rm stars, disk} = \Psi - \dot{R} - M_{\rm stars,disk}/\tau_{\rm bar},
\end{equation}
with
\begin{equation}
 \Psi = {M_{\rm disk, gas} \over \tau_{\rm disk, star~formation}}
\end{equation}
with $\tau_{\rm disk, star~formation}$ being the star formation timescale and $\dot{R}$ is the rate of mass recycling from stars and $\tau_{\rm bar}$ is a bar instability timescale (see \S\ref{sec:DiskStability}). The mass removed from the disk by the bar instability mechanism is added to the active spheroid component.
Element abundances (including total metals) evolve according to:
\begin{equation}
  \dot{M}_{Z, {\rm disk, gas}} = \dot{M}_{Z {\rm cooling}} - \dot{M}_{Z, {\rm outflow, disk}} - \dot{M}_{Z, {\rm stars, disk}} + \dot{y},
\end{equation}
and
\begin{equation}
 \dot{M}_{Z, {\rm stars, disk}} = \Psi {M_{Z, {\rm disk, gas}} \over M_{\rm disk, gas}} - \dot{R}_Z
\end{equation}
where $\dot{y}$ is the rate of element yield from stars and $\dot{R}_Z$ is the rate of element recycling. The angular momentum evolves as:
\begin{equation}
 \dot{J}_{\rm disk} = \dot{J}_{\rm cooling} - \left[ \dot{M}_{\rm outflow, disk} + {M_{\rm disk, gas}  + M_{\rm disk, stars} \over \tau_{\rm bar}}\right] {J_{\rm disk} \over M_{\rm disk, gas} + M_{\rm disk, stars}}.
\end{equation}
The outflow rate, $\dot{M}_{\rm outflow, disk}$, is computed for the current star formation rate and gas properties by the stellar properties subsystem (see \S\ref{sec:StellarPopulationProperties}) and prescriptions for expulsive and non-expulsive supernova feedback (see \S\ref{sec:sneExpulsiveFeedback} and \S\ref{sec:sneFeedback} respectively), but is not allowed to exceed $M_{\rm gas, disk}/ \alpha_{\rm outflow minimum, disk} \tau_{\rm disk, dynamical}$, where $\tau_{\rm disk, dynamical}=R_{\rm disk}/V_{\rm disk}$ is the dynamical time of the disk and $\alpha_{\rm outflow minimum, disk}=${\tt [diskOutflowTimescaleMinimum]} is the shortest timescale (in units of the dynamical timescale) on which gas can be removed from the disk. This limit prevents the disk being depleted on arbitrarily short timescales. The non-expulsive component of the outflow is piped to the hot halo component.

\subsubsection{Event Evolution}

\noindent\emph{Node mergers:} None\\

\noindent\emph{Satellite merging:} Disks may be destroyed (or, potentially, created or otherwise modified) as the result of a satellite merging event, as dictated by the selected merger remnant mass movement method (see \S\ref{sec:MergingMassMovements}).\\

\noindent\emph{Node promotion:} None\\

\subsubsection{Additional Output}

If the {\tt [diskOutputStarFormationRate]}\index{disk!star formation rate!output}\index{star formation rate!output!disk} input parameter is set to true, then the instantaneous star formation rate in the disk will be included in the output, as {\tt diskStarFormationRate}.

\subsubsection{Structure}

The radial size of the disk is found solving for equilibrium (i.e. the radius is such that the angular momentum of material at that radius is sufficient to provide rotational support) at the specified {\tt [diskStructureSolverRadius]} which is given in units of the disk scale length. In converting from the mean specific angular momentum of the disk to the angular momentum at that radius, a flat rotation curve is assumed, i.e.:
\begin{eqnarray}
 j(r)/\langle j \rangle &=& r V \left/ {\int_0^\infty 2 \pi r^\prime \Sigma(r^\prime) r^\prime V {\rm d} r^\prime \over \int_0^\infty 2 \pi r^\prime \Sigma(r^\prime) {\rm d} r^\prime} \right. \nonumber \\
 j(r))/\langle j \rangle &=& r / 2 r_{\rm disk}.
\end{eqnarray}
The option {\tt [diskRadiusSolverCole2000Method]}, if set to {\tt true}, alters this behavior to match that of the structure solver used by \cite{cole_hierarchical_2000}, in which adiabatic contraction of the dark matter halo is solved for assuming that the disk has a spherical mass dsitribution. The specific angular momentum passed to the structure solver will be modified as follows in this case:
\begin{equation}
 j(r) \rightarrow \left[ j^2(r) - \left( V_{\rm disk}^2(r) r^2 - {\rm G} M_{\rm disk}(<r) r \right) \right]^{1/2},
\end{equation}
where $V_{\rm disk}$ is the rotation curve in the plane of an infinitely thin exponential disk. This adjustment accounts for the difference between a thin disk and spherical mass distribution. Note that in this case (as in \citealt{cole_hierarchical_2000}) the resulting disk will not precisely satisfy $j(r) = r V_{\rm c}(r)$ where $V_{\rm c}(r)$ is the net rotation curve.

\section{Galactic Spheroid}

\subsection{``Null'' Implementation}

The null spheroid implementation leaves all methods to point to dummy routines (for rate adjustment and derivative computation) or to {\tt null()} for get/set methods. It can be used to effectively switch off spheroids. Of course, this is safe only if none of the other active components expect to get or set spheroid properties (or if they rely on a sensible implementation of spheroid evolution).

\subsection{``Hernquist'' Implementation}

This implementation assumes a Hernquist profile \citep{hernquist_analytical_1990} for the spheroidal component of a galaxy in which stars trace gas.

\subsubsection{Properties}

The Hernquist galactic spheroid implementation defines the following properties:
\begin{description}
 \item [{\tt Spheroid\_Gas\_Mass}] The mass of gas in the spheroid: $M_{\rm spheroid, gas}$ [{\tt spheroidGasMass}].
 \item [{\tt Spheroid\_Gas\_Abundances}] The mass of elements in the gaseous spheroid: $M_{Z, {\rm spheroid, gas}}$ [{\tt spheroidGas\{abundanceName\}}].
 \item [{\tt Spheroid\_Stellar\_Mass}] The mass of stars in the spheroid: $M_{\rm spheroid, stars}$ [{\tt spheroidStellarMass}].
 \item [{\tt Spheroid\_Stellar\_Abundances}] The mass of elements in the stellar spheroid: $M_{Z, {\rm spheroid, stars}}$ [{\tt spheroidStellar\{abundanceName\}}].
 \item [{\tt Spheroid\_Stellar\_Luminosities}] The luminosities (in multiple bands) of the stellar spheroid: $L_{\rm spheroid, stars}$ [{\tt spheroidStellar\{luminosityName\}}].
 \item [{\tt Spheroid\_Angular\_Momentum}] The pseudo-angular momentum\footnote{Effectively the angular momentum that the spheroid would have, were it rotationally supported rather than pressure supported.} of the spheroid, $J_{\rm spheroid}$ [{\tt spheroidAngularMomentum}]. The parameter {\tt [spheroidAngularMomentumAtScaleRadius]} controls the ratio of the specific pseudo-angular momentum at the scale radius of the Hernquist spheroid to the mean specific pseudo-angular momentum\index{spheroid!Hernquist!pseudo-angular momentum}\index{spheroid!Hernquist!radius}. The Hernquist spheroid has infinite angular momentum if a flat rotation curve is assumed and the profile is assumed to extend to infinity. If a finite truncation radius is assumed, or a different rotation curve is assumed, this ratio can be finite. The {\tt [spheroidAngularMomentumAtScaleRadius]} parameter allows control over these assumptions.
 \item [{\tt Spheroid\_Radius}] The radial scale length of the spheroid, $r_{\rm spheroid}$ [{\tt spheroidScaleLength}].
 \item [{\tt Spheroid\_Velocity}] The circular velocity of the spheroid at $r_{\rm spheroid}$, $V_{\rm spheroid}$ [{\tt spheroidCircularVelocity}].
\end{description}
and the following pipes:
\begin{description}
 \item [{\tt Tree\_Node\_Spheroid\_Gas\_Energy\_Input}] Energy sent through this pipe is added to the gas of the spheroid and will result in an outflow (see below). Input energy should be in units of $M_\odot$ km$^2$ s$^{-2}$ Gyr$^{-1}$ and must be positive (energy cannot be removed from the gas via this pipe).
 \item [{\tt Tree\_Node\_Spheroid\_Gas\_Sink}] Removes gas (and proportionate amounts of angular momentum and elements) from the spheroid gas. Removed mass should be in units of $M_\odot$ and must be positive (a negative mass sink would add mass to the spheroid which is not allowed via this pipe).
\end{description}

\subsubsection{Initialization}

No initialization is performed---spheroids are created as needed.

\subsubsection{Differential Evolution}

In the Hernquist galactic spheroid implementation the gas mass evolves as\footnote{There may be an additional contribution to the mass and angular momentum rates of change in the spheroid due to material transferred from the disk component via the bar instability mechanism (see \S\protect\ref{sec:DiskExponential}). This is not included here as it is not intrinsic to this specific spheroid implementation---it is handled explicitly by the disk component and so applies equally to any spheroid component implementation.}:
\begin{equation}
 \dot{M}_{\rm spheroid, gas} = - \dot{M}_{\rm outflow, spheroid} - \dot{M}_{\rm stars, spheroid},
\end{equation}
where the rate of change of stellar mass is
\begin{equation}
 \dot{M}_{\rm stars, spheroid} = \Psi - \dot{R}
\end{equation}
with
\begin{equation}
 \Psi = {M_{\rm spheroid, gas} \over \tau_{\rm spheroid, star~formation}}
\end{equation}
with $\tau_{\rm spheroid, star~formation}$ being the star formation timescale and $\dot{R}$ is the rate of mass recycling from stars.
Element abundances (including total metals) evolve according to:
\begin{equation}
  \dot{M}_{Z, {\rm spheroid, gas}} = - \dot{M}_{Z, {\rm outflow, spheroid}} - \dot{M}_{Z, {\rm stars, spheroid}} + \dot{y},
\end{equation}
and
\begin{equation}
 \dot{M}_{Z, {\rm stars, spheroid}} = \Psi {M_{Z, {\rm spheroid, gas}} \over M_{\rm spheroid, gas}} - \dot{R}_Z
\end{equation}
where $\dot{y}$ is the rate of element yield from stars and $\dot{R}_Z$ is the rate of element recycling. The angular momentum evolves as:
\begin{equation}
 \dot{J}_{\rm spheroid} = \dot{M}_{\rm outflow, spheroid} {J_{\rm spheroid} \over M_{\rm spheroid, gas} + M_{\rm spheroid, stars}}.
\end{equation}
The outflow rate, $\dot{M}_{\rm outflow, spheroid}$, is computed for the current star formation rate and gas properties by the stellar properties subsystem (see \S\ref{sec:StellarPopulationProperties}) and prescriptions for expulsive and non-expulsive supernova feedback (see \S\ref{sec:sneExpulsiveFeedback} and \S\ref{sec:sneFeedback} respectively), with an additional contribution given by
\begin{equation}
 \dot{M}_{\rm outflow, spheroid} = \beta_{\rm spheroid, energy} {\dot{E}_{\rm gas, spheroid} \over V_{\rm spheroid}^2}
\end{equation}
where $\beta_{\rm spheroid, energy}=${\tt [spheroidEnergeticOutflowMassRate]} is an input parameter, and $\dot{E}_{\rm gas,spheroid}$ is any input energy sent through the {\tt Tree\_Node\_Spheroid\_Gas\_Energy\_Input} pipe, but is not allowed to exceed $M_{\rm gas, spheroid}/ \alpha_{\rm outflow minimum, spheroid} \tau_{\rm spheroid, dynamical}$, where $\tau_{\rm spheroid, dynamical}=R_{\rm spheroid}/V_{\rm spheroid}$ is the dynamical time of the spheroid and $\alpha_{\rm outflow minimum, spheroid}=${\tt [spheroidOutflowTimescaleMinimum]} is the shortest timescale (in units of the dynamical timescale) on which gas can be removed from the spheroid. This limit prevents the spheroid being depleted on arbitrarily short timescales. The non-expulsive component of the outflow is piped to the hot halo component.

\subsubsection{Event Evolution}

\noindent\emph{Node mergers:} None\\

\noindent\emph{Satellite merging:} Spheroids may be created as the result of a satellite merging event, as dictated by the selected merger remnant mass movement method (see \S\ref{sec:satelliteMergerMassMovementMethod}).\\

\noindent\emph{Node promotion:} None.\\

\subsubsection{Additional Output}

If the {\tt [spheroidOutputStarFormationRate]}\index{spheroid!star formation rate!output}\index{star formation rate!output!spheroid} input parameter is set to true, then the instantaneous star formation rate in the spheroid will be included in the output, as {\tt spheroidStarFormationRate}.

\subsection{``S\'ersic'' Implementation}

This implementation assumes a S\'ersic profile (\citealt{sersic_influence_1963}; see also \citealt{mazure_exact_2002}) for the spheroidal component of a galaxy in which stars trace gas. The projected density profile of the spheroid is given by:
\begin{equation}
 \Sigma(R) \propto \exp\left(-b_{\rm n} R^{1/n} \right),
\end{equation}
where the S\'ersic index, $n=${\tt [spheroidSersicIndex]} and the coefficient $b_{\rm n}=2.303(0.8689 n-0.1447)$ \cite{wadadekar_two-dimensional_1999}. The 3D density distribution for a given $n$ is inferred by solving the relevant inverse Abel integral.

\subsubsection{Properties}

The S\'ersic galactic spheroid implementation defines the following properties:
\begin{description}
 \item [{\tt Spheroid\_Gas\_Mass}] The mass of gas in the spheroid: $M_{\rm spheroid, gas}$ [{\tt spheroidGasMass}].
 \item [{\tt Spheroid\_Gas\_Abundances}] The mass of elements in the gaseous spheroid: $M_{Z, {\rm spheroid, gas}}$ [{\tt spheroidGas\{abundanceName\}}].
 \item [{\tt Spheroid\_Stellar\_Mass}] The mass of stars in the spheroid: $M_{\rm spheroid, stars}$ [{\tt spheroidStellarMass}].
 \item [{\tt Spheroid\_Stellar\_Abundances}] The mass of elements in the stellar spheroid: $M_{Z, {\rm spheroid, stars}}$ [{\tt spheroidStellar\{abundanceName\}}].
 \item [{\tt Spheroid\_Stellar\_Luminosities}] The luminosities (in multiple bands) of the stellar spheroid: $L_{\rm spheroid, stars}$ [{\tt spheroidStellar\{luminosityName\}}].
 \item [{\tt Spheroid\_Angular\_Momentum}] The pseudo-angular momentum\footnote{Effectively the angular momentum that the spheroid would have, were it rotationally supported rather than pressure supported.} of the spheroid, $J_{\rm spheroid}$ [{\tt spheroidAngularMomentum}].
 \item [{\tt Spheroid\_Radius}] The 3D half-mass radius of the spheroid, $r_{\rm spheroid}$ [{\tt spheroidScaleLength}].
 \item [{\tt Spheroid\_Velocity}] The circular velocity of the spheroid at $r_{\rm spheroid}$, $V_{\rm spheroid}$ [{\tt spheroidCircularVelocity}].
\end{description}
and the following pipes:
\begin{description}
 \item [{\tt Tree\_Node\_Spheroid\_Gas\_Energy\_Input}] Energy sent through this pipe is added to the gas of the spheroid and will result in an outflow (see below). Input energy should be in units of $M_\odot$ km$^2$ s$^{-2}$ Gyr$^{-1}$ and must be positive (energy cannot be removed from the gas via this pipe).
 \item [{\tt Tree\_Node\_Spheroid\_Gas\_Sink}] Removes gas (and proportionate amounts of angular momentum and elements) from the spheroid gas. Removed mass should be in units of $M_\odot$ and must be positive (a negative mass sink would add mass to the spheroid which is not allowed via this pipe).
\end{description}

\subsubsection{Initialization}

No initialization is performed---spheroids are created as needed.

\subsubsection{Differential Evolution}

In the S\'ersic galactic spheroid implementation the gas mass evolves as\footnote{There may be an additional contribution to the mass and angular momentum rates of change in the spheroid due to material transferred from the disk component via the bar instability mechanism (see \S\protect\ref{sec:DiskExponential}). This is not included here as it is not intrinsic to this specific spheroid implementation---it is handled explicitly by the disk component and so applies equally to any spheroid component implementation.}:
\begin{equation}
 \dot{M}_{\rm spheroid, gas} = - \dot{M}_{\rm outflow, spheroid} - \dot{M}_{\rm stars, spheroid},
\end{equation}
where the rate of change of stellar mass is
\begin{equation}
 \dot{M}_{\rm stars, spheroid} = \Psi - \dot{R}
\end{equation}
with
\begin{equation}
 \Psi = {M_{\rm spheroid, gas} \over \tau_{\rm spheroid, star~formation}}
\end{equation}
with $\tau_{\rm spheroid, star~formation}$ being the star formation timescale and $\dot{R}$ is the rate of mass recycling from stars.
Element abundances (including total metals) evolve according to:
\begin{equation}
  \dot{M}_{Z, {\rm spheroid, gas}} = - \dot{M}_{Z, {\rm outflow, spheroid}} - \dot{M}_{Z, {\rm stars, spheroid}} + \dot{y},
\end{equation}
and
\begin{equation}
 \dot{M}_{Z, {\rm stars, spheroid}} = \Psi {M_{Z, {\rm spheroid, gas}} \over M_{\rm spheroid, gas}} - \dot{R}_Z
\end{equation}
where $\dot{y}$ is the rate of element yield from stars and $\dot{R}_Z$ is the rate of element recycling. The angular momentum evolves as:
\begin{equation}
 \dot{J}_{\rm spheroid} = \dot{M}_{\rm outflow, spheroid} {J_{\rm spheroid} \over M_{\rm spheroid, gas} + M_{\rm spheroid, stars}}.
\end{equation}
The outflow rate, $\dot{M}_{\rm outflow, spheroid}$, is computed for the current star formation rate and gas properties by the stellar properties subsystem (see \S\ref{sec:StellarPopulationProperties}) and prescriptions for expulsive and non-expulsive supernova feedback (see \S\ref{sec:sneExpulsiveFeedback} and \S\ref{sec:sneFeedback} respectively), with an additional contribution given by
\begin{equation}
 \dot{M}_{\rm outflow, spheroid} = \beta_{\rm spheroid, energy} {\dot{E}_{\rm gas, spheroid} \over V_{\rm spheroid}^2}
\end{equation}
where $\beta_{\rm spheroid, energy}=${\tt [spheroidEnergeticOutflowMassRate]} is an input parameter, and $\dot{E}_{\rm gas,spheroid}$ is any input energy sent through the {\tt Tree\_Node\_Spheroid\_Gas\_Energy\_Input} pipe, but is not allowed to exceed $M_{\rm gas, spheroid}/ \alpha_{\rm outflow minimum, spheroid} \tau_{\rm spheroid, dynamical}$, where $\tau_{\rm spheroid, dynamical}=R_{\rm spheroid}/V_{\rm spheroid}$ is the dynamical time of the spheroid and $\alpha_{\rm outflow minimum, spheroid}=${\tt [spheroidOutflowTimescaleMinimum]} is the shortest timescale (in units of the dynamical timescale) on which gas can be removed from the spheroid. This limit prevents the spheroid being depleted on arbitrarily short timescales. The non-expulsive component of the outflow is piped to the hot halo component.

\subsubsection{Event Evolution}

\noindent\emph{Node mergers:} None\\

\noindent\emph{Satellite merging:} Spheroids may be created as the result of a satellite merging event, as dictated by the selected merger remnant mass movement method (see \S\ref{sec:satelliteMergerMassMovementMethod}).\\

\noindent\emph{Node promotion:} None.\\

\subsubsection{Additional Output}

If the {\tt [spheroidOutputStarFormationRate]}\index{spheroid!star formation rate!output}\index{star formation rate!output!spheroid} input parameter is set to true, then the instantaneous star formation rate in the spheroid will be included in the output, as {\tt spheroidStarFormationRate}.


\section{Basic Properties}\label{sec:ComponentBasicProperties}

Basic properties are the total mass of a node and the cosmic time at which it currently exists.

\subsection{``Simple'' Implemenation}

\subsubsection{Properties}

The simple basic properties implementation defines the following properties:
\begin{description}
 \item [{\tt Mass}] The total mass of the node: $M_{\rm node}$ [{\tt nodeMass}].
 \item [{\tt Time}] The time at which the node is defined: $t_{\rm node}$.
 \item [{\tt TimeLastIsolated}] The time at which the node was last an isolated halo (i.e. not a subhalo): [\tt nodeTimeLastIsolated].
\end{description}

\subsubsection{Initialization}

All basic properties are required to be initialized by the merger tree construction routine.

\subsubsection{Differential Evolution}

Properties are evolved according to:
\begin{eqnarray}
 \dot{M}_{\rm node} &=& \left\{\begin{array}{ll}{M_{\rm node, parent} - M_{\rm node} \over t_{\rm node, parent} - t_{\rm node}} & \hbox{ if primary progenitor} \\ 0 & \hbox{ otherwise}, \end{array} \right. \\
 \dot{t}_{\rm node} &=& 1,
\end{eqnarray}
where the ``parent'' subscript indicates a property of the parent node in the merger tree.

\subsubsection{Event Evolution}

\noindent\emph{Node mergers:} None.\\

\noindent\emph{Satellite merging:} None.\\

\noindent\emph{Node promotion:} $M_{\rm node}$ is updated to the node mass of the parent prior to promotion.\\

\section{Position}\label{sec:ComponentPosition}

The position component implements the position and velocity of each galaxy.

\subsection{``Null'' Implementation}

The null position implementation leaves all methods to point to dummy routines (for rate adjustment and derivative computation) or to {\tt null()} for get/set methods. It can be used to effectively switch off positions. Of course, this is safe only if none of the other active components, functions or tasks expect to get or set position properties (or if they rely on a sensible implementation of position evolution).

\subsection{``Preset'' Implemenation}

\subsubsection{Properties}

The preset position implementation defines the following properties:
\begin{description}
 \item [{\tt Position}] The 3-D position of the node: ${\bf x}$ [{\tt position[X|Y|Z]}].
 \item [{\tt Velocity}] The 3-D velocity of the node: ${\bf v}$ [{\tt velocity[X|Y|Z]}].
 \item [{\tt Position\_6D\_History}] The history of the node's position in 6-D phase space, usually used for satellite nodes.
\end{description}

\subsubsection{Initialization}

None---all properties are assumed to have been preset, usually by the merger tree construction routine.

\subsubsection{Differential Evolution}

None. Positions and velocities do not evolve for a given node. When output, if a 6-D position history is available than the position and velocity from the history entry closest to the output time will be used\footnote{While interpolation could be used this is usually a bad idea. For nodes that are satellites in a halo for example, no simple interpolation algorithm can correctly account for the complex orbital dynamics by which the position and velocity is actually evolving.}.

\subsubsection{Event Evolution}

\noindent\emph{Node mergers:} None.\\

\noindent\emph{Satellite merging:} None.\\

\noindent\emph{Node promotion:} The position and velocity are updated to those of the parent node.\\

\section{Satellite Node Orbit}

This component tracks the orbital properties of subhalos.

\subsection{``Preset'' Implementation}

\subsubsection{Properties}

The preset satellite orbit implementation defines the following properties:
\begin{description}
 \item [{\tt Satellite\_Merge\_Time}] The time until the satellite will merge with its host: $t_{\rm satellite, merge}$ [{\tt timeToMerge}].
 \item [{\tt Satellite\_Time\_Of\_Merging}] The cosmological time at which the satellite will merge with its host: $T_{\rm satellite, merge}$.
 \item [{\tt Bound\_Mass}] The remaining, total bound mass of the satellite (this property is read only---it is determined from the {\tt Bound\_Mass\_History} property).
 \item [{\tt Bound\_Mass\_History}] A history time-series of the total bound mass of the satellite.
\end{description}

Note that the {\tt Satellite\_Merge\_Time} and {\tt Satellite\_Time\_Of\_Merging} effectively provide the same information. For that reason, setting one of them will automatically set the other accordingly.

\subsubsection{Initialization}

None. This method assumes that merging times and bound mass histories will be set externally (usually when the merger tree is constructed).

\subsubsection{Differential Evolution}

None.

\subsubsection{Event Evolution}

\noindent\emph{Node mergers:} None.\\

\noindent\emph{Satellite merging:} None.\\

\noindent\emph{Node promotion:} None.\\

\subsection{``Simple'' Implementation}

\subsubsection{Properties}

The simple satellite orbit implementation defines the following properties:
\begin{description}
 \item [{\tt Satellite\_Merge\_Time}] The time until the satellite will merge with its host: $t_{\rm satellite, merge}$ [{\tt timeToMerge}].
 \item [{\tt Bound\_Mass}] The remaining, total bound mass of the satellite: $M_{\rm node,bound}$ [{\tt nodeBoundMass}].
 \item[{\tt Satellite\_Virial\_Orbit}] The orbit (returned as a {\tt keplerOrbit} object; see \S\ref{sec:KeplerOrbits}) of the satellite at the point of virial radius crossing.
\end{description}

\subsubsection{Initialization}

None.

\subsubsection{Differential Evolution}

Properties are evolved according to:
\begin{equation}
 \dot{t}_{\rm satellite, merge} = -1,
\end{equation}
with $\dot{M}_{\rm node,bound}$ set to the rate given by the {\tt darkMatterHaloMassLossRateMethod} method (see \S\ref{sec:HaloMassLossRates}). The virial orbit is a fixed quantity and does not evolve.

\subsubsection{Event Evolution}

\noindent\emph{Node mergers:} The component is created and the time to merging is assigned a value. The bound mass is set to the current total mass of the node. If {\tt satelliteOrbitStoreOrbitalParameters}$=${\tt true} then a virial orbit is selected using the {\tt virialOrbitsMethod} (see \S\ref{sec:SatelliteVirialOrbits}) and stored (otherwise, a new virial orbit will be computed---possibly at random---each time the virial orbit is requested). If {\tt [satelliteOrbitResetOnHaloFormation]}$=${\tt true} then satellite orbits will be reset on halo formation events (see \S\ref{sec:ComponentFormationTimes}).\\

\noindent\emph{Satellite merging:} None.\\

\noindent\emph{Node promotion:} Not applicable (component only exists for satellite nodes).\\

\section{Dark Matter Halo Spin}\label{sec:DarkMatterHaloSpinComponent}

\subsection{``Null'' Implementation}

The null spin implementation leaves all methods to point to dummy routines (for rate adjustment and derivative computation) or to {\tt null()} for get/set methods. It can be used to effectively switch off spins. Of course, this is safe only if none of the other active components expect to get or set spin properties (or if they rely on a sensible implementation of spin evolution).

\subsection{``Random'' Implementation}

\subsubsection{Properties}

The random dark matter halo spin implementation defines the following properties:
\begin{description}
 \item [{\tt Spin}] The spin parameter of the halo: $\lambda$ [{\tt nodeSpin}].
\end{description}

\subsubsection{Initialization}

The spin parameter of each node, if not already assigned, is selected at random from a distribution of spin parameters. This value is assigned to the earliest progenitor of the halo traced along its primary branch. The value is then propagated forward along the primary branch until the node mass exceeds that of the node for which the spin was selected by a factor of {\tt [randomSpinResetMassFactor]}, at which point a new spin is selected at random, and the process repeated until the end of the branch is reached. 

\subsubsection{Differential Evolution}

The spin parameter does not evolve.

\subsubsection{Event Evolution}

\noindent\emph{Node mergers:} None.\\

\noindent\emph{Satellite merging:} None.\\

\noindent\emph{Node promotion:} The spin is updated to equal that of the parent node. (The two will differ only if this is a case where the new halo node was sufficiently more massive than the node for which a spin was last selected that a new spin value was chosen.)\\

\subsection{``Preset'' Implementation}

\subsubsection{Properties}

The preset dark matter halo spin implementation defines the following properties:
\begin{description}
 \item [{\tt Spin}] The spin parameter of the halo: $\lambda$ [{\tt nodeSpin}].
 \item [{\tt Spin\_Growth\_Rate}] The growth rate spin parameter of the halo (in units of Gyr$^{-1}$).
\end{description}

\subsubsection{Initialization}

The spin parameter of each node is assumed to have been preset prior to merger tree initialization. The growth rate is comptued assuming linear growth with time along each branch.

\subsubsection{Differential Evolution}

The spin parameter evolves linearly with time between node and parent node.

\subsubsection{Event Evolution}

\noindent\emph{Node mergers:} None.\\

\noindent\emph{Satellite merging:} None.\\

\noindent\emph{Node promotion:} The spin and growth rate are updated to equal those of the parent node.\\

\section{Dark Matter Profile}\label{sec:DarkMatterProfileComponent}

This component stores dynamic properties associated with dark matter halo density profiles.

\subsection{``Null'' Implementation}

The null profile implementation leaves all methods to point to dummy routines (for rate adjustment and derivative computation) or to {\tt null()} for get/set methods. It can be used to effectively switch off profiles. Of course, this is safe only if none of the other active components expect to get or set profile properties (or if they rely on a sensible implementation of profile evolution).

\subsection{``Scale'' Implementation}

\subsubsection{Properties}

The scale dark matter profile implementation defines the following properties:
\begin{description}
 \item [{\tt Scale}] The scale length of the density profile [{\tt darkMatterScaleRadius}];
 \item [{\tt Scale\_Growth\_Rate}] The growth rate of the scale length of the density profile.
\end{description}

\subsubsection{Initialization}

The scale length of each node, if not already assigned, is assigned using the concentration parameter function (see \S\ref{sec:DarkMatterProfileConcentration}), but is not allowed to drop below {\tt [darkMatterProfileMinimumConcentration]}, such that the scale length is equal to the virial radius divided by that concentration.

\subsubsection{Differential Evolution}

The scale radius does not evolve.

\subsubsection{Event Evolution}

\noindent\emph{Node mergers:} None.\\

\noindent\emph{Satellite merging:} None.\\

\noindent\emph{Node promotion:} None.\\

\subsection{``Scale + Shape'' Implementation}

\subsubsection{Properties}

The scale$+$shape dark matter profile implementation defines the following properties:
\begin{description}
 \item [{\tt Scale}] The scale length of the density profile [{\tt darkMatterScaleRadius}];
 \item [{\tt Scale\_Growth\_Rate}] The growth rate of the scale length of the density profile.
 \item [{\tt Shape}] A shape parameter describing the density profile [{\tt darkMatterShapeParameter}];
 \item [{\tt Shape\_Growth\_Rate}] The growth rate of the shape parameter of the density profile.
\end{description}

\subsubsection{Initialization}

The scale length of each node, if not already assigned, is assigned using the concentration parameter function (see \S\ref{sec:DarkMatterProfileConcentration}), but is not allowed to drop below {\tt [darkMatterProfileMinimumConcentration]}, such that the scale length is equal to the virial radius divided by that concentration. The shape parameter of each node is assigned using the dark matter profile shape function (see \S\ref{sec:darkMatterProfileShape}).

\subsubsection{Differential Evolution}

The scale radius and shape parameters do not evolve.

\subsubsection{Event Evolution}

\noindent\emph{Node mergers:} None.\\

\noindent\emph{Satellite merging:} None.\\

\noindent\emph{Node promotion:} None.\\


\section{Merging Statistics}

This component records statistics associated with galaxy merging.

\subsection{``Standard Implementation}

\subsubsection{Properties}

The scale dark matter profile implementation defines the following properties:
\begin{description}
 \item [{\tt Galaxy\_Major\_Merger\_Time}] The time of the last major merger associated with this galaxy, output as the time \emph{since} the last major merger [{\tt majorMergerTimeLapse}];
 \item [{\tt Node\_Major\_Merger\_Time}] The time of the last major merger (as defined by the {\tt [nodeMajorMergerFraction]} parameter) between this node and another, output as the time \emph{since} the last major merger [{\tt nodeMajorMergerTimeLapse}];
 \item [{\tt Node\_Formation\_Time}] The time at which the node is judged to have ``formed'', defined as the time at which its main branch progenitor had a mass equal to a fraction {\tt [nodeFormationMassFraction]} of the node's initial mass in the merger tree;
\end{description}

\subsubsection{Initialization}

The times of the last mergers are stored each time a major merger occurrs and this component is created (if necessary) the first time such a merger occurs. Formation times are computed during merger tree initialization.

\subsubsection{Differential Evolution}

The times of the last mergers do not evolve.

\subsubsection{Event Evolution}

\noindent\emph{Node mergers:} The time of the last node merger in the parent node is reset to the current time if the merger is major.\\

\noindent\emph{Satellite merging:} The time of the last merger is reset to the current time if the merger is major.\\

\noindent\emph{Node promotion:} The time of the last major node merger is updated to that of the parent if that time is more recent.\\

\section{Formation Times}\label{sec:ComponentFormationTimes}

This component implements ``formation times'' of dark matter halos.

\subsection{``Cole2000'' Implementation}

\subsubsection{Properties}

The ``Cole2000' formation times implementation defines the following properties:
\begin{description}
 \item [{\tt Formation\_Time}] The time at which the halo last ``formed''. Formation is defined as an increase in the mass of the halo by a factor {\tt [haloReformationMassFactor]}.
\end{description}

\subsubsection{Initialization}

The formation time is set to the current time and this component is created the first time such a merger occurs.

\subsubsection{Differential Evolution}

The formation time does not evolve. When the node mass exceeds the mass at the formation time by a factor {\tt [haloReformationMassFactor]} evolution is interrupted and the formation time reset to the current time.

\subsubsection{Event Evolution}

\noindent\emph{Node mergers:} None.\\

\noindent\emph{Satellite merging:} None.\\

\noindent\emph{Node promotion:} None.\\

\subsection{``Null'' Implementation}

The null formation times implementation leaves all methods to point to dummy routines (for rate adjustment and derivative computation) or to {\tt null()} for get/set methods. It can be used to effectively switch off formation times. Of course, this is safe only if none of the other active components expect to get or set formation time properties (or if they rely on a sensible implementation of formation time).


\chapter{Physical Implementations}

\section{Accretion of Gas into Halos}\label{sec:AccretionBaryonic}\index{accretion!baryonic}

The accretion rate of gas from the \IGM\ into a dark matter halo is expected to depend on (at least) the rate at which that halo mass is growing, the depth of its potential well and the thermodynamical properties of the accreting gas. \glc\ implements the following calculations of gas accretion from the \IGM, which can be selected via the {\tt accretionHalosMethod} input parameter.

\subsection{Simple Method}

Currently the only option, and selected using {\tt accretionHalosMethod}$=${\tt simple}, this method sets the accretion rate of baryons into a halo to be:
\begin{equation}
 \dot{M}_{\rm accretion} = \left\{ \begin{array}{ll} (\Omega_{\rm b}/\Omega_0) \dot{M}_{\rm halo} & \hbox{ if } V_{\rm virial} > V_{\rm reionization} \hbox{ or } z > z_{\rm reionization} \\ 0 & \hbox{ otherwise,}\end{array} \right.
\end{equation}
where $z_{\rm reionization}=${\tt [reionizationSuppressionRedshift]} is the redshift at which the Universe is reionized and $V_{\rm reionization}=${\tt [reionizationSuppressionVelocity]} is the virial velocity below which accretion is suppressed after reionization. Setting $V_{\rm reionization}$ to zero will effectively switch off the effects of reionization on the accretion of baryonics. This algorithm attempts to offer a simple prescription for the effects of reionization and has been explored by multiple authors (e.g. \citealt{benson_effects_2002}). In particular, \cite{font_modelingmilky_2010} show that it produces results in good agreement with more elaborate treatments of reionization. For halos below the accretion threshold, any accretion rate that would have otherwise occurred is instead placed into the ``failed'' accretion rate. For halos which can accrete, and which have some mass in their ``failed'' reservoir, that mass will be added to the regular accretion rate at a rate equal to the mass of the ``failed'' reservoir times the specific growth rate of the halo. The gas accreted is assumed to be from a pristine \IGM\ and so has zero abundances. Molecular abundances are computed from the atomic ionization state functions (see \S\ref{sec:IonizationStateMethod}).

\section{Background Cosmology}\index{cosmology}

The background cosmology describes the evolution of an isotropic, homoegeneous Universe within which our calculations are carried out. For the purposes of \glc, the background cosmology is used to relate expansion factor/redshift to cosmic time and to compute the density of various components (e.g. dark matter, dark energy, etc.) at different epochs. Background cosmological models are specified via the {\tt cosmologyMethod}, and the physics that must be implemented for each cosmological model is describe in more detail in \S\ref{sec:CosmologyMethods}. Currently implemented cosmological models are as follows.

\subsection{Matter + Lambda}

Selected with {\tt cosmologyMethod}$=${\tt matter + lambda}, in this implementation cosmological relations are computed assuming a universe that contains only collisionless matter and a cosmological constant.


\section{Circumnuclear Accretion Disks}\label{sec:CircumnuclearDisks}\index{accretion disks}\index{accretion!disk}

Circumnuclear accretion disks surrounding supermassive black holes at the centers of galaxies influence the evolution of both the black hole (via accretion rates of mass and angular momentum and possibly by extracting rotational energy from the black hole) and the surrounding galaxy if they lead to energetic outflows (e.g. jets) from the nuclear region. Accretion disk type is specified via the {\tt accretionDisksMethod}, and the physics that must be implemented for each accretion disk type is describe in more detail in \S\ref{sec:AccretionDisks}. Current implementations of accretion disks are as follows.

\subsection{Shakura-Sunyaev Geometrically Thin, Radiatively Efficient Disks}

Selected with {\tt accretionDisksMethod}$=${\tt Shakura-Sunyaev}, this implementation assumes that accretion disks are always described by a radiatively efficient, geometrically thin accretion disk as described by \cite{shakura_black_1973}. The radiative efficiency of the flow is computed assuming that material falls into the black hole without further energy loss from the \ISCO, while the spin-up rate of the black hole is computed assuming that the material enters the black hole with the specific angular momentum of the \ISCO\ (i.e. there are no torques on the material once it begins to fall in from the \ISCO; \citealt{bardeen_kerr_1970}). For these thin disks, jet power is computed, using the expressions from \citeauthor{meier_association_2001}~(\citeyear{meier_association_2001}; his equations 4 and 5).

\subsection{Advection Dominated, Geometrically Thick, Radiatively Inefficient Flows (ADAFs)}

Selected with {\tt accretionDisksMethod}$=${\tt ADAF}, this implementation assumes that accretion is via an advection dominated accretion flow \citep{narayan_advection-dominated_1994} which is radiatively inefficient and geometrically thick. The radiative efficiency of the flow, which will be zero for a pure ADAF, can be set via the input parameter {\tt [adafRadiativeEfficiency]}. The spin up rate of the black hole and the jet power produced as material accretes into the black hole are computed using the method of \cite{benson_maximum_2009}. The energy of the accreted material can be set equal to the energy at infinity (as expected for a pure ADAF) or the energy at the \ISCO\ by use of the {\tt [adafEnergyOption]} parameter (set to {\tt pure ADAF} or {\tt ISCO} respectively). The ADAF structure is controlled by the adiabatic index, $\gamma$, and viscosity parameter, $\alpha$, which are specified via the {\tt [adafAdiabaticIndex]} and {\tt [adafViscosityOption]} input parameters respectively. The field-enhancing shear, $g$, is computed using $g=\exp(\omega \tau)$ if {\tt [adafFieldEnhanceType]} is set to ``exponential'' where $\omega$ is the frame-dragging frequency and $\tau$ is the smaller of the radial inflow and azimuthal velocity timescales. If  {\tt [adafFieldEnhanceType]} is set to ``linear'' then the alternative version, $g=1+\omega \tau$ is used instead. {\tt [adafViscosityOption]} may be set to ``{\tt fit}'', in which case the fitting function for $\alpha$ as a function of black hole spin is used:
\begin{eqnarray}
\alpha(j)=0.015+0.02 j^4 & \hbox{ if  }& g=\exp(\omega\tau) \hbox{ and } E=E_{\rm ISCO}, \\
\alpha(j)=0.025+0.08 j^4 & \hbox{ if } & g=1+\omega\tau \hbox{ and } E=E_{\rm ISCO}, \\
\alpha(j)=0.010+0.00 j^4 & \hbox{ if } & g=\exp(\omega\tau) \hbox{ and } E=1, \\
\alpha(j)=0.025+0.02 j^4 & \hbox{ if } & g=1+\omega\tau \hbox{ and } E=1.  
\end{eqnarray}

\subsection{``Switched'' Disks}

Selected with {\tt accretionDisksMethod}$=${\tt switched}, this method allows for accretion disks to switched between radiatively efficient (Shakura-Sunyaev) and inefficient (ADAF) modes. TWhich mode is used is determine by the accretion rate onto the disk:
\begin{itemize}
 \item Radiatively efficient accretion if $\dot{M}/\dot{M}_{\rm Eddington}>${\tt accretionRateThinDiskMinimum} and $\dot{M}/\dot{M}_{\rm Eddington}<${\tt accretionRateThinDiskMaximum};
 \item Radiatively inefficient accretion otherwise.
\end{itemize}
Both {\tt accretionRateThinDiskMinimum} and {\tt accretionRateThinDiskMaximum} are adjustable input parameters.

\section{Cold Dark Matter Structure Formation}\index{structure formation}\index{cold dark matter}

A variety of functions are used to describe structure formation in cold dark matter dominated universes. These are described below.

\subsection{Primordial Power Spectrum}\label{sec:PrimordialPowerSpectrum}\index{power spectrum!primordial}

The functional form of the primordial dark matter power spectrum is selected via the {\tt powerSpectrumMethod} parameter. The power spectrum is computed from the specified primordial power spectrum and the transfer function (see \S\ref{sec:TransferFunction}) and normalized to a value of $\sigma_8$ specified by {\tt [sigma\_8]}.

\subsubsection{(Running) Power Law Spectrum}

Selected via {\tt powerSpectrumMethod}$=${\tt power law}, this method implements a primordial power spectrum of the form:
\begin{equation}
 P(k) \propto k^{n_{\rm eff}(k)},
\end{equation}
where
\begin{equation}
 n_{\rm eff}(k) = n_{\rm s} + {1\over 2}{\d n \over \d \ln k} \ln \left( {k \over k_{\rm ref}} \right),
\end{equation}
where $n_{\rm s}=${\tt powerSpectrumIndex} is the power spectrum index at wavenumber $k_{\rm ref}=${\tt powerSpectrumReferenceWavenumber} and $\d n / \d \ln k=${\tt powerSpectrumRunning} describes the running of this index with wavenumber.

\subsection{Transfer Function}\label{sec:TransferFunction}\index{transfer function}

The functional form of the cold dark matter transfer function is selected via the {\tt transferFunctionMethod} parameter. The power spectrum is computed from the specified transfer function and the primordial power spectrum (see \S\ref{sec:PrimordialPowerSpectrum}) and normalized to a value of $\sigma_8$ specified by {\tt [sigma\_8]}.

\subsubsection{BBKS}

Selected with {\tt transferFunctionMethod}$=${\tt BBKS}, this method uses the fitting function of \cite{bardeen_statistics_1986} to compute the \CDM\ transfer function.

\subsubsection{Eisenstein \& Hu}

Selected with {\tt transferFunctionMethod}$=${\tt Eisenstein + Hu}, this method uses the fitting function of \cite{eisenstein_power_1999} to compute the \CDM\ transfer function. It requires that the effective number of neutrino species be specified via the {\tt effectiveNumberNeutrinos} parameter and summed mass of all neutrino species (in eV) be specified via the {\tt summedNeutrinoMasses} parameter.

\subsubsection{{\sc CMBFast}}

Selected with {\tt transferFunctionMethod}$=${\tt CMBFast}, this method uses the {\sc CMBFast} code to compute the \CDM\ transfer function. It requires that the mass fraction of helium in the early Universe be specified via the {\tt Y\_He} parameter. {\sc CMBFast} will be downloaded and run if the transfer function needs to be computed. It will then be stored in a file for future reference.

\subsubsection{File}

Selected with {\tt transferFunctionMethod}$=${\tt file}, this method reads a tabulated transfer function from an XML file (specified via the {\tt transferFunctionFile} parameter), interpolating between tabulated points. The structure of the transfer function file is described in \S\ref{sec:TransferFunctionMethod}.

\subsection{Linear Growth Function}\index{linear growth}

The function describing the amplitude of linear perturbations is selected via the {\tt linearGrowthMethod} parameter.

\subsubsection{Simple}

Selected with {\tt linearGrowthMethod}$=${\tt simple}, this method calculates the growth of linear perturbations using standard perturbation theory in a Universe consisting of matter and a cosmological constant. Perturbations in the baryons are treated just as for dark matter (i.e. pressure forces are ignored), while perturbations in the radiation are assumed not to grow.

\subsection{Critical Overdensity}\index{density!critical}

The method used to compute the critical linear overdensity at which overdense regions virialize is selected via the {\tt criticalOverdensityMethod} parameter.

\subsubsection{Spherical Collapse (Matter + Cosmological Constant)}

Selected with {\tt criticalOverdensityMethod}$=${\tt spherical top hat} this method calculates critical overdensity using a spherical top-hat collapse model assuming a Universe which contains matter and a cosmological constant (see, for example, \citealt{percival_cosmological_2005}).

\subsection{Virial Density Contrast}\label{sec:VirialDensityConstrast}\index{density!virial}

The method used to compute the mean density contrast of virialized dark matter halos is selected via the {\tt virialDensityContrastMethod} parameter.

\subsubsection{Bryan \& Norman Fitting Function}

Selected with {\tt virialDensityContrastMethod}$=${\tt Bryan + Norman} this method calculates virial density contrast using the fitting functions given by \cite{bryan_statistical_1998}. As such, it is valid only for $\Omega_\Lambda=0$ or $\Omega_M+\Omega_\Lambda=1$ cosmologies and will abort on other cosmologies.

\subsubsection{Spherical Collapse (Matter + Cosmological Constant)}

Selected with {\tt virialDensityContrastMethod}$=${\tt spherical top hat} this method calculates virial density contrast using a spherical top-hat collapse model assuming a Universe which contains matter and a cosmological constant (see, for example, \citealt{percival_cosmological_2005}).

\subsection{Halo Bias}\index{halo bias}\index{dark matter halos!bias}\index{bias!halo}

The dark matter halo linear bias method is selected via the {\tt darkMatterHaloBiasMethod} parameter.

\subsubsection{Press-Schechter}

Selected with {\tt darkMatterHaloBiasMethod}$=${\tt Press-Schechter} this method uses a bias consistent with the halo mass function of \cite{press_formation_1974} (see \citep{mo_analytic_1996}).

\subsubsection{Sheth-Tormen}

Selected with {\tt darkMatterHaloBiasMethod}$=${\tt SMT} this method uses a bias consistent with the halo mass function of \cite{sheth_ellipsoidal_2001}.

\subsubsection{Tinker}

Selected with {\tt darkMatterHaloBiasMethod}$=${\tt Tinker2010} this method uses the functional form proposed by \cite{tinker_large_2010} to compute the halo bias. The bias is computed at the appropriate virial overdensity (see \S\ref{sec:VirialDensityConstrast}).

\subsection{Halo Mass Function}\index{halo mass function}\index{dark matter halos!mass function}

The dark matter halo mass function (i.e. the number of halos per unit volume per unit mass interval) is selected via the {\tt haloMassFunctionMethod} parameter.

\subsubsection{Press-Schechter}

Selected with {\tt haloMassFunctionMethod}$=${\tt Press-Schechter} this method uses the functional form proposed by \cite{press_formation_1974} to compute the halo mass function.

\subsubsection{Sheth-Tormen}

Selected with {\tt haloMassFunctionMethod}$=${\tt Sheth-Tormen} this method uses the functional form proposed by \cite{sheth_ellipsoidal_2001} to compute the halo mass function.

\subsubsection{Tinker}

Selected with {\tt haloMassFunctionMethod}$=${\tt Tinker2008} this method uses the functional form proposed by \cite{tinker_towardhalo_2008} to compute the halo mass function. The mass function is computed at the appropriate virial overdensity (see \S\ref{sec:VirialDensityConstrast}).

\section{Cooling of Gas Inside Halos}\index{cooling}

The cooling of gas within dark matter halos is controlled by a number of different algorithms which will be decribed below.

\subsection{Cooling Function}\index{cooling function}\index{cooling!cooling function}

The cooling function of gas, $\Lambda(\rho,T,{\bf Z})$, is implemented by the algorithm(s) selected using the {\tt coolingFunctionMethods} parameter. If more than one cooling function is specified, then the net cooling function is a sum over all of those selected.

\subsubsection{Atomic Collisional Ionization Equilibrium Using {\sc Cloudy}}

Selected using {\tt coolingFunctionMethods}$=${\tt atomic\_CIE\_Cloudy}, this method computes the cooling function using the {\sc Cloudy} code and under the assumption of collisional ionization equilibrium with no molecular contribution. Abundances are Solar, except for zero metallicity calculations which use {\sc Cloudy}'s ``primordial'' metallicity. The helium abundance for non-zero metallicity is scaled between primordial and Solar values linearly with metallicity. The {\sc Cloudy} code will be downloaded and run to compute the cooling function as needed, which will then be stored for future use. As this process is slow, a precomputed table is provided with \glc. If metallicities outside the range tabulated in this file are required it will be regenerated with an appropriate range.

\subsubsection{Collisional Ionization Eqiulibrium From File}

Selected using {\tt coolingFunctionMethods}$=${\tt CIE\_from\_file}, in this method the cooling function is read from a file specified by the {\tt coolingFunctionFile} parameter. The format of this file is specified in \S\ref{sec:CoolingFunctionMethods}. The cooling function is assumed to be computed under conditions of collisional ionization equilibrium and therefore to scale as $\rho^2$.

\subsubsection{CMB Compton Cooling}

Selected using {\tt coolingFunctionMethods}$=${\tt CMB\_Compton}, this method computes the cooling function due to Compton scattering off of \CMB\ photons:
\begin{equation}
\Lambda = {4 \sigma_{\rm T} {\rm a} {\rm k}_{\rm B} n_{\rm e } \over m_{\rm e} \clight} T_{\rm CMB}^4 \left( T - T_{\rm CMB} \right),
\end{equation}
where $\sigma_{\rm T}$ is the Thompson cross-section, $a$ is the radiation constant, ${\rm k}_{\rm B}$ is Boltzmann's constant, $n_{\rm e}$ is the number density of electrons, $m_{\rm e}$ is the electron mass, $\clight$ is the speed of light, $T_{\rm CMB}$ is the \CMB\ temperature at the current cosmic epoch and $T$ is the temperature of the gas. The electron density is computed from the selected ionization state method (see \S\ref{sec:IonizationStateMethod}).

\subsubsection{Molecular Hydrogen (Galli-Palla)}

Selected using {\tt coolingFunctionMethods}$=${\tt molecularHydrogenGalliPalla}, this method computes the cooling function due to molecular hydrogen using the results of \cite{galli_chemistry_1998}. For the H--H$_2$ cooling function, the fitting functions from \cite{galli_chemistry_1998} are used. For the H$_2^+$--e$^-$ and H--H$_2^+$ cooling functions fitting functions to the results plotted in  \cite{suchkov_cooling_1978} are used:
\begin{equation}
\log_{10}\left({\Lambda(T) \over \hbox{erg s}^{-1} \hbox{cm}^3}\right) = C_0 + C_1 \log_{10} \left({T\over\hbox{K}}\right) + C_2 \left[\log_{10} \left({T\over\hbox{K}}\right)\right]^2,
\label{eq:H2CoolingFunction}
\end{equation}
where the coefficients $C_{0-2}$ are given in Table~\ref{tb:H2CoolingFunctionCoefficients}.

\begin{table}
 \begin{center}
  \caption{Coefficients of H$_2^+$ cooling functions as appearing in the fitting function, eq.~\protect\ref{eq:H2CoolingFunction}.}
  \label{tb:H2CoolingFunctionCoefficients}
  \begin{tabular}{lrrr}
   \hline
   & \multicolumn{3}{c}{{\bf Coefficient}} \\
   {\bf Interaction} & \boldmath{$C_0$} & \boldmath{$C_1$} & \boldmath{$C_2$} \\
   \hline
   H$_2^+$--e$^-$ & -33.33 & 5.565 & -0.4675 \\
   H--H$_2^+$ & -35.28 & 5.862 & -0.5124 \\
   \hline
  \end{tabular}
 \end{center}
\end{table}


\subsection{Cooling Rate}\label{sec:CoolingRate}\index{cooling!rate}

The algorithm used to compute the rate at which gas drops out of the hot halo due to cooling is selected with the {\tt coolingRateMethod} parameter.

\subsubsection{White \& Frenk}

Selected with {\tt coolingRateMethod}$=${\tt White + Frenk}, this method computes the cooling rate using the expression given by \cite{white_galaxy_1991}, namely
\begin{equation}
\dot{M}_{\rm cool} = 4 \pi r_{\rm cool}^2 \rho(r_{\rm cool}) \dot{r}_{\rm cool},
\end{equation}
where $r_{\rm cool}$ is the cooling radius in the hot halo and $\rho(r)$ is the density profile of the hot halo.

\subsection{Cooling Radius}\index{cooling radius}\index{cooling!radius}

The algorithm used to compute the cooling radius is selected via the {\tt coolingRadiusMethod} parameter.

\subsubsection{Simple}

Selcted with {\tt coolingRadiusMethod}$=${\tt simple}, this method computes the cooling radius by seeking the radius at which the time available for cooling (see \S\ref{sec:TimeAvailableCooling}) equals the cooling time (see \S\ref{sec:CoolingTime}). The growth rate is determined consistently based on the slope of the density profile, the density dependence of the cooling function and the rate at which the time available for cooling is increasing. This method assumes that the cooling time is a monotonic function of radius.

\subsection{Cooling Time}\label{sec:CoolingTime}\index{cooling time}\index{cooling!time}

The algorithm used to compute the time taken for gas to cool (i.e. the cooling time) is selected via the {\tt coolingTimeMethod} parameter.

\subsubsection{Simple}

Selcted with {\tt coolingTimeMethod}$=${\tt simple}, this method assumes that the cooling time is simply
\begin{equation}
 t_{\rm cool} = {N \over 2} {{\rm k}_{\rm B} T n_{\rm tot} \over \Lambda},
\end{equation}
where $N=${\tt coolingTimeSimpleDegreesOfFreedom} is the number of degrees of freedom in the cooling gas which has temperature $T$ and total particle number density $n_{\rm tot}$ and $\Lambda$ is the cooling function.

\subsection{Time Available for Cooling}\label{sec:TimeAvailableCooling}\index{cooling!time available}

The method used to determine the time available for cooling (i.e. the time for which gas in a halo has been able to cool) is selected by the {\tt coolingTimeAvailableMethod} parameter.

\subsubsection{White \& Frenk}

Selected with {\tt coolingTimeAvailableMethod}$=${\tt White-Frenk}, this methods assumes that the time available for cooling is equal to
\begin{equation}
 t_{\rm available} = \exp\left[ f \ln t_{\rm Universe} + (1-f)\ln t_{\rm dynamical} \right],
\end{equation}
where $f=${\tt coolingTimeAvailableAgeFactor} is an interpolating factor, $t_{\rm Universe}$ is the age of the Universe and $t_{\rm dynamical}$ is the dynamical time in the halo. The original \cite{white_galaxy_1991} algorithm corresponds to $f=1$.

\section{Cosmology}

The method used to compute cosmological relations (e.g. expansion factor as a function of time) is selected by the {\tt cosmologyMethod} parameter.

\subsection{Matter + Cosmological Constant Universes}

Selected with {\tt cosmologyMethod}$=${\tt Matter + Lambda}, this method assumes a universe which contains only matter and a cosmological constant

\section{Dark Matter Halos}

Several algorithms are used to implement dark matter halos.

\subsection{Mass Accretion History}\index{mass accretion history!dark matter halo}\index{dark matter halo!mass accretion history}

The method used to compute mass accretion histories of dark matter halos is selected via the {\tt darkMatterAccretionHistoryMethod} parameter.

\subsubsection{Wechsler et al. (2002)}

Selected with {\tt darkMatterAccretionHistoryMethod}$=${\tt Wechsler 2002}, under this method the mass accretion history is given by \citep{wechsler_concentrations_2002}:
\begin{equation}
M(t) = M(t_0) \exp \left( - 2 a_{\rm c} \left[ {a(t_0)\over a(t)}-1 \right] \right),
\end{equation}
where $t_0$ is some reference time and $a_{\rm c}$ is a characteristic expansion factor defined by \cite{wechsler_concentrations_2002} to correspond to the formation time of the halo (using the formation time definition of \citealt{bullock_profiles_2001}).

\subsubsection{Zhao et al. (2009)}

Selected with {\tt darkMatterAccretionHistoryMethod}$=${\tt Zhao 2009}, under this method the algorithm given by \cite{zhao_accurate_2009} to compute mass accretion histories. In particular, \cite{zhao_accurate_2009} give a fitting function for the quantity ${\rm d} \ln \sigma(M)/{\rm d} \ln  \delta_{\rm c}(t)$ for the dimensionless growth rate in a mass accretion history at time $t$ and halo mass $M$. This is converted to a dimensionful growth rate using
\begin{equation}
 {{\rm d} M \over {\rm d} t} = \left({{\rm d} \ln \sigma(M) \over {\rm d} \ln M}\right)^{-1} \left({{\rm d} \delta_c(t) \over {\rm d} t}\right) \left( {M \over \delta_{\rm c}(t)} \right) \left({{\rm d} \ln \sigma(M) \over {\rm d} \ln \delta_{\rm c}(t)}\right).
\end{equation}
This differential equation is then solved numerically to find the mass accretion history.

\subsection{Density Profile}

The method uses to compute density profiles of dark matter halos is selected via the {\tt darkMatterProfileMethod} parameter.

\subsubsection{Isothermal}

Selected with {\tt darkMatterProfileMethod}$=${\tt isothermal}, under this method the density profile is given by:
\begin{equation}
 \rho_{\rm dark matter}(r) \propto r^{-2},
\end{equation}
normalized such that the total mass of the node is enclosed with the virial radius.

\subsubsection{NFW}

Selected with {\tt darkMatterProfileMethod}$=${\tt NFW}, under this method the \NFW\ density profile \citep{navarro_universal_1997} is used
\begin{equation}
  \rho_{\rm dark matter}(r) \propto \left({r\over r_{\rm s}}\right)^{-1} \left[1 + \left({r\over r_{\rm s}}\right) \right]^{-2},
\end{equation}
normalized such that the total mass of the node is enclosed with the virial radius and with the scale length $r_{\rm s} = r_{\rm virial}/c$ where $c$ is the halo concentration (see \S\ref{sec:DarkMatterProfileConcentration}).

\subsection{Density Profile Concentration}\label{sec:DarkMatterProfileConcentration}\index{dark matter profile!concentration}

The method uses to compute the concentrations of dark matter profiles is selected via the {\tt darkMatterConcentrationMethod} parameter.

\subsubsection{Gao (2008)}

Selected with {\tt darkMatterConcentrationMethod}$=${\tt Gao 2008}, under this method the concentration is computed using a fitting function from \cite{gao_redshift_2008}:
\begin{equation}
\log_{10} c = A \log_{10} M_{\rm halo} + B.
\end{equation}
The parameters are a functon of expansion factor, $a$. We use the following fits to the \cite{gao_redshift_2008} results:
\begin{eqnarray}
A &=& -0.140 \exp\left[-\left(\left\{\log_{10}a+0.05\right\}/0.35\right)^2\right], \\
B &=&  2.646 \exp\left[-\left(\log_{10}a/0.50\right)^2\right].
\end{eqnarray}

\subsubsection{Zhao (2009)}

Selected with {\tt darkMatterConcentrationMethod}$=${\tt Zhao 2009}, under this method the concentration is computed using a fitting function from \cite{zhao_accurate_2009}:
\begin{equation}
 c = 4 \left(1 + \left[ {t  \over 3.75 t_{\rm form}}\right]^{8.4}\right)^{1/8},
\end{equation}
where $t$ is the time for the halo and $t_{\rm form}$ is a formation time defined by \cite{zhao_accurate_2009} as the time at which the main branch progenitor of the halo had a mass equal to $0.04$ of the current halo mass. This formation time is computed directly from the merger tree branch associated with each halo. If the no branch exists or does not extend to the formation time then the formation time is computed by extrapolating the mass of the earliest resolved main branch progenitor to earlier times using the selected mass accretion history method (see \S\ref{sec:HaloMassAccretionHistory}).

\subsubsection{Mu\~noz-Cuartas (2011)}

Selected with {\tt darkMatterConcentrationMethod}$=${\tt Munoz-Cuartas 2011}, under this method the concentration is computed using a fitting function from \cite{munoz-cuartas_redshift_2011}:
\begin{equation}
\log_{10} c = a \log_{10} \left( {M_{\rm halo} \over h^{-1}M_\odot} \right) + b.
\end{equation}
The parameters are a functon of redshift, $z$, given by
\begin{eqnarray}
a &=& wz-m, \\
b &=& {\alpha \over (z+\gamma)} + {\beta \over (z+\gamma)^2},
\end{eqnarray}
where $w=0.029$, $m=0.097$, $\alpha=-110.001$, $\beta=2469.720$, $\gamma=16.885$.

\subsection{Spin Parameter Distribution}\label{sec:SpinParameterDistribution}\index{dark matter halo!spin!distribution}\index{spin!dark matter halo}

The method used to compute the distribution of dark matter halo spin parameters is selected via the {\tt haloSpinDistributionMethod} parameter.

\subsubsection{Lognormal}

Selected with {\tt haloSpinDistributionMethod}$=${\tt lognormal}, under this method the spin is drawn from a lognormal distribution with median {\tt [lognormalSpinDistributionMedian]} and width {\tt [lognormalSpinDistributionSigma]}.

\subsubsection{Bett et al. (2007)}

Selected with {\tt haloSpinDistributionMethod}$=${\tt Bett2007}, under this method the spin is drawn from the distribution found by \cite{bett_spin_2007}. The $\lambda_0$ and $\alpha$ parameter of Bett et al.'s distribution are set by the {\tt [spinDistributionBett2007Lambda0]} and {\tt [spinDistributionBett2007Alpha]} input parameters.

\section{Disk Stability/Bar Formation}\label{sec:DiskStability}\index{disks!stability}\index{bar instability}

The method uses to compute the bar instability timescale for galactic disks is selected via the {\tt barInstabilityMethod} parameter.

\subsection{Efstathiou, Lake \& Negroponte}

Selected with {\tt barInstabilityMethod}$=${\tt ELN}, this method uses the stability criterion of \cite{efstathiou_stability_1982} to estimate when disks are unstable to bar formation:
\begin{equation}
 \epsilon \left( \equiv {V_{\rm peak} \over \sqrt{\G M_{\rm disk}/r_{\rm disk}}} \right) < \epsilon_{\rm c},
\end{equation}
for stability, where $V_{\rm peak}$ is the peak velocity in the rotation curve (computed here assuming an isolated exponential disk), $M_{\rm disk}$ is the mass of the disk and $r_{\rm disk}$ is its scale length (assuming an exponential disk). The value of $\epsilon_{\rm c}$ is linearly interpolated in the disk gas fraction between values for purely gaseous and stellar disks as specified by {\tt stabilityThresholdStellar} and {\tt stabilityThresholdGaseous} respectively. For disks which are judged to be unstable, the timescale for bar formation is estimated to be
\begin{equation}
 t_{\rm bar} = t_{\rm disk} {\epsilon_{\rm c} - \epsilon_{\rm iso} \over \epsilon_{\rm c} - \epsilon},
\end{equation}
where $\epsilon_{\rm iso}$ is the value of $\epsilon$ for an isolated disk and $t_{\rm disk}$ is the disk dynamical time, defined as $r/V$, at one scale length. This form gives an infinite timescale at the stability threshold, reducing to a dynamical time for highly unstable disks.

\section{Galactic Structure}\index{galactic structure}

The algorithm to be used when solving for galactic structure (specifically, finding radii of galactic components) is selected via the {\tt galacticStructureRadiusSolverMethod} parameter.

\subsection{Simple}

Selected with {\tt galacticStructureRadiusSolverMethod}$=${\tt simple} this method determines the sizes of galactic components by assuming that their self-gravity is negligible (i.e. that the gravitational potential well is dominated by dark matter) and that, therefore, baryons do not modify the dark matter density profile. The radius of a given component is then found by solving
\begin{equation}
 j = \sqrt{\G M_{\rm DM}(r) r},
\end{equation}
where $j$ is the specific angular momentum of the component (at whatever point in the profile is to be solved for), $r$ is radius and $M(r)$ is the mass of dark matter within radius $r$.

\subsection{Adiabatic}

Selected with {\tt galacticStructureRadiusSolverMethod}$=${\tt adiabatic}, this method takes into account the baryonic self-gravity of all galactic components when solving for structure and additionally accounts for backreaction of the baryons on the dark matter density profile using the adiabatic contraction algorithm of \cite{gnedin_response_2004}. The parameter $A$ and $\omega$ of that model are specified via input parameters {\tt adiabaticContractionGnedinA} and {\tt adiabaticContractionGnedinOmega} respectively. Solution proceeds via an iterative procedure to find equilibrium radii for all galaxies in a consistently contracted halo. The method used follows that described by \cite{benson_galaxy_2010}.

\section{Galaxy Merging}\index{galaxy!merging}\index{merging!galaxy}

The process of merging two galaxies currently involves two algorithms: one which decides how the merger causes mass components from both galaxies to move and one which determines the size of the remnant galaxy spheroid.

\subsection{Mass Movements}\label{sec:MergingMassMovements}

The movement of mass elements in the merging galaxies is determined by the {\tt satelliteMergingMassMovementsMethod} parameter. 

\subsubsection{Simple}

Selected with {\tt satelliteMergingMassMovementsMethod}$=${\tt simple}, this method implements mass movements according to:
\begin{itemize}
 \item If $M_{\rm satellite} > f_{\rm major} M_{\rm central}$ then all mass from both satellite and central galaxies moves to the spheroid component of the central galaxy;
 \item Otherwise: Gas from the satellite moves to the component of the central specified by the {\tt [minorMergerGasMovesTo]} parameter (either ``{\tt disk}'' or ``{\tt spheroid}''), stars from the satellite moves to the spheroid of the central and mass in the central does not move.
\end{itemize}
Here, $f_{\rm major}=${\tt [majorMergerMassRatio]} is the mass ratio above which a merger is considered to be ``major''.

\subsection{Remnant Sizes}\index{merging!remnant size}

The method used to calculate the sizes of merger remnant spheroids is selected by the {\tt satelliteMergingRemnantSizeMethod} parameter.

\subsubsection{Null}

Selected using {\tt satelliteMergingRemnantSizeMethod}$=${\tt null}, this is a null method which does nothing at all. It is useful, for example, when running \glc\ to study dark matter only (i.e. when no galaxy properties are computed).

\subsubsection{Cole et al. (2000)}\label{sec:MergerRemnantSizeCole2000}

Selected using {\tt satelliteMergingRemnantSizeMethod}$=${\tt Cole2000}, this method uses the algorithm of \cite{cole_hierarchical_2000} to compute merger remnant spheroid sizes. Specifically
\begin{equation}
\frac{(M_1+M_2)^2}{ r_{\rm new}} =
\frac{M_1^2}{r_1} + \frac{M_2^2}{r_2} + \frac{ f_{\rm orbit}}{c}
\frac{M_1 M_2}{r_1+r_2},
\end{equation}
where $M_1$ and $M_2$ are the baryonic masses of the components of the merging galaxies that will end up in the spheroid component of the remnant\footnote{Depending on the merging rules (see \S\protect\ref{sec:MergingMassMovements}) not all mass may be placed into the spheroid component of the remnant.} and $r_1$ and $r_2$ are the half mass radii of those same components of the merging galaxies\footnote{In practice, \glc\ computes a weighted average of the disk and spheroid half-mass radii of each galaxy, with weights equal to the masses of each component (disk and spheroid) which will become part of the spheroid component of the remnant.}, $r_{\rm new}$ is the half mass radius of the spheroidal component of the remnant galaxy and $c$ is a constant which depends on the distribution of the mass. For a Hernquist spheroid $c=0.40$ can be found by numerical integration while for a exponential disk $c=0.49$. For simplicity a value of $c=0.5$ is adopted for all components. The parameter $f_{\rm orbit}=${\tt mergerRemnantSizeOrbitalEnergy} depends on the orbital parameters of the galaxy pair. For example, a value of $f_{\rm orbit} = 1$ corresponds to point mass galaxies in circular orbits about their center of mass. 

A subtelty arises because the above expression accounts for only the baryonic mass of material which becomes part of the spheroid component of the remnant. In reality, there are additional terms in the energy equation due to the interaction of this material with any dark matter mass in each galaxy and any baryonic mass of each galaxy which does not become part of the spheroid component of the remnant. To account for this additional matter, an effective boost factor, $f_{\rm boost}$, to the specific angular momentum of each component of each merging galaxy is computed:
\begin{equation}
 f_{\rm boost} = {j \over \sqrt{{\rm G} M r_{1/2}}},
\end{equation}
where $j$ is the specific angular momentum of the component, $M$ is its total baryonic mass and $r_{\rm 1/2}$ is its half-mass radius. The mass-weighted mean boost factor is found by combining those of all components which will form part of the spheroid of the remnant. The final specific angular momentum of the remnant spheroid is then given by:
\begin{equation}
 j_{\rm new} = \langle f_{\rm boost} \rangle r_{\rm new} V_{\rm new},
\end{equation}
where
\begin{equation}
 V_{\rm new}^2 = {{\rm G} (M_1+M_2)\over r_{\rm new}}.
\end{equation}

\subsubsection{Covington et al. (2008)}

Selected using {\tt satelliteMergingRemnantSizeMethod}$=${\tt Covington2008}, this method uses the algorithm of \cite{covington_predicting_2008} to compute merger remnant spheroid sizes. Specifically
\begin{equation}
\frac{(M_1+M_2)^2}{ r_{\rm new}} =
\left[ \frac{M_1^2}{r_1} + \frac{M_2^2}{r_2} + \frac{ f_{\rm orbit}}{c}
\frac{M_1 M_2}{r_1+r_2}\right] \left( 1 + f_{\rm gas} C_{\rm rad} \right),
\label{eq:Covington2008Radius}
\end{equation}
where $M_1$ and $M_2$ are the baryonic masses of the merging galaxies and $r_1$
and $r_2$ are their half mass radii, $r_{\rm new}$ is the half mass radius of the spheroidal component of the remnant galaxy and $c$ is a constant which depends on the distribution of the mass. For a Hernquist spheroid $c=0.40$ can be found by numerical integration while for a exponential disk $c=0.49$. For simplicity a value of $c=0.5$ is adopted for all components. The parameter $f_{\rm orbit}=${\tt mergerRemnantSizeOrbitalEnergy} depends on the orbital parameters of the galaxy pair. For example, a value of $f_{\rm orbit} = 1$ corresponds to point mass galaxies in circular orbits about their center of mass. The final term on the right hand side of eqn.~(\ref{eq:Covington2008Radius}) gives a correction to the final energy of the remnant due to dissipational losses based on the results of \cite{covington_effects_2011}, with
\begin{equation}
 f_{\rm gas} = {M_{\rm 1,gas}+M_{\rm 2,gas} \over M_1+M_2}
\end{equation}
begin the gas fraction of the progenitor galaxies. By default, $C_{\rm rad}=2.75$ \citep{covington_effects_2011}. To account for the effects of dark matter and non-spheroid baryonic matter the same approach is used as in the \cite{cole_hierarchical_2000} algorithm (see \S\ref{sec:MergerRemnantSizeCole2000}). 

\section{Hot Halo Density Profile}\index{hot halo!density profile}\index{density profile!hot halo}

The algorithm to be used when determining the hot halo density profile is selected via the {\tt hotHaloDensityMethod} parameter.

\subsection{Cored Isothermal}

Selected with {\tt hotHaloDensityMethod}$=${\tt cored isothermal} this method adopts a spherically symmetric cored-isothermal density profile for the hot halo. Specifically,
\begin{equation}
 \rho_{\rm hot halo}(r) \propto \left[ r^2 + r_{\rm core}^2 \right]^2,
\end{equation}
where the core radius, $r_{\rm core}$, is set to be a fixed fraction of the virial radius, that fraction being given by the input parameter {\tt [isothermalCoreRadiusOverVirialRadius]}. The profile is normalized such that the current mass in the hot gas profile is contained within the virial radius.

\subsection{Null}

Selected with {\tt hotHaloDensityMethod}$=${\tt null} this method assumes no hot halo density profile. It is useful, for example, when performing dark matter-only calculations.

\section{Hot Halo Temperature Profile}\label{sec:HotHaloTemperature}\index{hot halo!temperature profile}\index{temperature profile!hot halo}

The algorithm to be used when determining the hot halo temperature profile is selected via the {\tt hotHaloTemperatureMethod} parameter.

\subsection{Virial Temperature}

Selected with {\tt hotHaloTemperatureMethod}$=${\tt virial} this method assumes an isothermal halo with a temperature equal to the virial temperature of the halo.

\section{Initial Mass Functions}\index{initial mass function}

The stellar \IMF\ subsystem supports multiple IMFs and extensible algorithms to select which \IMF\ to use based on the physical conditions of star formation.

\subsection{Initial Mass Function Selection}\index{initial mass function!selection}

The method to use for selecting which \IMF\ to use is specified by the {\tt imfSelectionMethod} parameter.

\subsubsection{Fixed}

Selected by {\tt imfSelectionMethod}$=${\tt fixed}, this method uses a fixed \IMF\ irrespective of physical conditions. The \IMF\ to use is specified by the {\tt imfSelectionFixed} parameter (e.g. setting this parameter to {\tt Salpeter} selects the Salpeter \IMF).

\subsection{Initial Mass Functions}\label{sec:physicsIMF}\index{initial mass function}

A variety of different \IMF s are available. Each \IMF\ registers itself with the \glc\ \IMF\ subsystem and can then be looked-up by name or internal index. All IMFs are assumed to be continuous in $M$, unless otherwise noted and normalized to unit mass. Each \IMF\ supplies a recycled fraction and metal yield for use in the instantaneous recycling approximation. These can be set via the parameters {\tt imf[imfName]RecycledInstantaneous} and {\tt imf[imfName]YieldInstantaneous} where {\tt [imfName]} is the name of the \IMF. Their default values were computed using \glc 's internal stellar astrophysics modules for a Solar metallicity population with age of $13.8$ Gyr.

\subsubsection{Chabrier}

The {\tt Chabrier} \IMF\ is defined by \citep{chabrier_galactic_2001}:
\begin{equation}
 \phi(M) \propto \left\{ \begin{array}{ll}
 M^{-1} \exp(-[\log_{10}(M/M_{\rm c})/\sigma_{\rm c}]^2/2) & \hbox{ for } 0.1M_\odot < M < 1M_\odot \\
 M^{-2.3} & \hbox{ for } 1M_\odot < M < 125M_\odot \\
 0 & \hbox {otherwise,} \end{array} \right.
\end{equation}
where $\sigma_{\rm c}=0.69$ and $M_{\rm c}=0.08M_\odot$.

\subsubsection{Kennicutt}

The {\tt Kennicutt} \IMF\ is defined by \citep{kennicutt_rate_1983}:
\begin{equation}
 \phi(M) \propto \left\{ \begin{array}{ll}
 M^{-1.25} & \hbox{ for } 0.10M_\odot < M < 1.00M_\odot \\
 M^{-2.00} & \hbox{ for } 1.00M_\odot < M < 2.00M_\odot \\
 M^{-2.30} & \hbox{ for } 2.00M_\odot < M < 125M_\odot \\
 0 & \hbox {otherwise.} \end{array} \right.
\end{equation}

\subsubsection{Kroupa}

The {\tt Kroupa} \IMF\ is defined by \citep{kroupa_variation_2001}:
\begin{equation}
 \phi(M) \propto \left\{ \begin{array}{ll}
 M^{-0.3} & \hbox{ for } 0.01M_\odot < M < 0.08M_\odot \\ 
 M^{-1.8} & \hbox{ for } 0.08M_\odot < M < 0.5M_\odot \\ 
 M^{-2.7} & \hbox{ for } 0.5M_\odot < M < 1M_\odot \\ 
 M^{-2.3} & \hbox{ for } 1M_\odot < M < 125M_\odot \\ 
0 & \hbox {otherwise.} \end{array} \right.
\end{equation}

\subsubsection{Miller-Scalo}

The {\tt Miller-Scalo} \IMF\ is defined by \citep{miller_initial_1979}:
\begin{equation}
 \phi(M) \propto \left\{ \begin{array}{ll}
 M^{-1.25} & \hbox{ for } 0.10M_\odot < M < 1.00M_\odot \\
 M^{-2.00} & \hbox{ for } 1.00M_\odot < M < 2.00M_\odot \\
 M^{-2.30} & \hbox{ for } 2.00M_\odot < M < 10.0M_\odot \\
 M^{-3.30} & \hbox{ for } 10.0M_\odot < M < 125M_\odot \\
 0 & \hbox {otherwise.} \end{array} \right.
\end{equation}

\subsubsection{Salpeter}

The {\tt Salpeter} \IMF\ is defined by \citep{salpeter_luminosity_1955}:
\begin{equation}
 \phi(M) \propto \left\{ \begin{array}{ll} M^{-2.35} & \hbox{ for } 0.1M_\odot < M < 125M_\odot \\ 0 & \hbox {otherwise.} \end{array} \right.
\end{equation}

\subsubsection{Scalo}

The {\tt Scalo} \IMF\ is defined by \citep{scalo_stellar_1986}:
\begin{equation}
 \phi(M) \propto \left\{ \begin{array}{ll}
 M^{+1.60} & \hbox{ for } 0.10M_\odot < M < 0.18M_\odot \\
 M^{-1.01} & \hbox{ for } 0.18M_\odot < M < 0.42M_\odot \\
 M^{-2.75} & \hbox{ for } 0.42M_\odot < M < 0.62M_\odot \\
 M^{-2.08} & \hbox{ for } 0.62M_\odot < M < 1.18M_\odot \\
 M^{-3.50} & \hbox{ for } 1.18M_\odot < M < 3.50M_\odot \\
 M^{-2.63} & \hbox{ for } 3.50M_\odot < M < 125M_\odot \\
 0 & \hbox {otherwise.} \end{array} \right.
\end{equation}







\section{Intergalactic Medium State}\label{sec:IntergalacticMediumStateMethod}\index{intergalactic medium}

The thermal and ionization state of the intergalactic medium is implemented by the algorithm selected using the {\tt intergalaticMediumStateMethod} parameter.

\subsection{{\sc RecFast}}

Selected using {\tt intergalaticMediumStateMethod}$=${\tt RecFast}, this method computes the state of the intergalactic medium using the \href{http://www.astro.ubc.ca/people/scott/recfast.html}{{\sc RecFast}} code \cite{seager_how_2000,wong_how_2008}. The {\sc RecFast} code will be downloaded and run to compute the intergalactic medium state as needed, which will then be stored for future use.

\subsection{File}

Selected using {\tt intergalaticMediumStateMethod}$=${\tt file}, this method reads the state of the intergalactic medium from a file and interpolates in the tabulated results. The format of the file is specified in \S\ref{sec:IntergalacticMediumStateMethods}.

\section{Ionization State}\label{sec:IonizationStateMethod}\index{ionization state}

The ionization state of gas is implemented by the algorithm selected using the {\tt ionizatonStateMethod} parameter.

\subsection{Atomic Collisional Ionization Equilibrium Using {\sc Cloudy}}

Selected using {\tt ionizatonStateMethod}$=${\tt atomic\_CIE\_Cloudy}, this method computes the ionization state using the {\sc Cloudy} code and under the assumption of collisional ionization equilibrium with no molecular contribution. Abundances are Solar, except for zero metallicity calculations which use {\sc Cloudy}'s ``primordial'' metallicity. The helium abundance for non-zero metallicity is scaled between primordial and Solar values linearly with metallicity. The {\sc Cloudy} code will be downloaded and run to compute the cooling function as needed, which will then be stored for future use. As this process is slow, a precomputed table is provided with \glc. If metallicities outside the range tabulated in this file are required it will be regenerated with an appropriate range.

\subsection{Collisional Ionization Eqiulibruim From File}

Selected using {\tt ionizatonStateMethod}$=${\tt CIE\_from\_file}, in this method the ionization state is read from a file specified by the {\tt ionizationStateFile} parameter. The format of this file is specified in \S\ref{sec:IonizationStateMethods}. The ionization state is assumed to be computed under conditions of collisional ionization equilibrium and therefore densities scale as $\rho$. Optional H{\sc i} and H{\sc ii} densities, if present in the file, will be read and used when returning the densities of ``molecular'' species.

\section{Merger Tree Construction}\index{merger trees}

Merger trees are ``constructed\footnote{By ``construct'' we mean any process of creating a representation of a merger tree within \protect\glc.}'' by the method specified by the {\tt mergerTreeConstructMethod} parameter.

\subsection{Read From File}

Selected with {\tt mergerTreeConstructMethod}$=${\tt read}, this method reads merger tree structures from an HDF5 file specified by the {\tt mergerTreeReadFileName} parameter. The structure of these HDF5 files is described in \S\ref{sec:MergerTreeFiles}.

\subsection{Build}

Selected with {\tt mergerTreeConstructMethod}$=${\tt build}, this method first creates a distribution of tree root halo masses and then builds a merger tree using the algorithm specified by the {\tt mergerTreeBuildMethod} parameter.

If {\tt [mergerTreeBuildTreesHaloMassDistribution]}$=${\tt read}, then these masses will be read from a file specified by {\tt [mergerTreeBuildTreeMassesFile]}. Otherwise, the root halo masses are selected to range between {\tt mergerTreeBuildHaloMassMinimum} and {\tt mergerTreeBuildHaloMassMaximum} with {\tt mergerTreeBuildTreesPerDecade} trees per decade of root halo mass on average. Trees are rooted at {\tt mergerTreeBuildTreesBaseRedshift} and tree building will begin with the {\tt mergerTreeBuildTreesBeginAtTree}$^{\rm th}$ tree\footnote{This will normally be set to 1 to begin with the first tree. Other values allow to begin on later trees for debugging purposes.}. The distribution of halo masses is such that the mass of the $i^{\rm th}$ halo is
\begin{equation}
 M_{\rm halo,i} = \exp\left[ \ln(M_{\rm halo,min}) + \ln\left({M_{\rm halo,max}/M_{\rm halo,min}}\right) x_i^{1+\alpha} \right].
\end{equation}
Here, $x_i$ is a number between 0 and 1 and $\alpha=${\tt mergerTreeBuildTreesHaloMassExponent} is an input parameter that controls the relative number of low and high mass tree produced. The distribution of $x$ is determined by the input parameter {\tt mergerTreeBuildTreesHaloMassDistribution} with options:
\begin{description}
 \item [{\tt uniform}] $x$ is distributed uniformly between 0 and 1;
 \item [{\tt quasi}] $x$ is distributed using a quasi-random sequence.
\end{description}

In the case of reading root halo masses from a file, the file should be an XML file with the following form:
\begin{verbatim}
 <mergerTrees>
  <treeRootMass>13522377303.5998</treeRootMass>
  <treeRootMass>19579530191.8709</treeRootMass>
  <treeRootMass>21061025282.9613</treeRootMass>
  .
  .
  .
 </mergerTrees>
\end{verbatim}
where each {\tt treeRootMass} element gives the mass (in Solar masses) of the root halo of a tree to generate.

\section{Merger Tree Branching}\index{merger trees!branching}

The method to be used for computing branching probabilities in merger trees is specified by the {\tt treeBranchingMethod} parameter.

\subsection{Modified Press-Schechter}

Selected with {\tt treeBranchingMethod}$=${\tt modified Press-Schechter}, this method uses the algorithm of \cite{parkinson_generating_2008} to compute branching ratios. The parameters $G_0$, $\gamma_1$ and $\gamma_2$ of their algorithm are specified by the input parameters {\tt modifiedPressSchechterG0}, {\tt modifiedPressSchechterGamma1} and {\tt modifiedPressSchechterGamma2} respectively. Additionally, the parameter {\tt modifiedPressSchechterFirstOrderAccuracy} limits the step in $\delta_{\rm crit}$ so that it never exceeds {\tt modifiedPressSchechterFirstOrderAccuracy}$\sqrt{2[\sigma^2(M_2/2)-\sigma^2(M_2)]}$, which ensures the the first order expansion of the merging rate that is assumed is accurate.

\section{Merger Tree Building}\index{merger trees!building}

The method to be used for building merger trees is specified by the {\tt mergerTreeBuildMethod} parameter.

\subsection{Cole et al. (2000) Algorithm}

Selected with {\tt mergerTreeBuildMethod}$=${\tt Cole2000}, this method uses the algorithm described by \cite{cole_hierarchical_2000}, with a branching probability method selected via the {\tt treeBranchingMethod} parameter. This action of this algorithm is controlled by the following parameters:
\begin{description}
 \item [{\tt mergerTreeBuildCole2000MergeProbability}] The maximum probability for a binary merger allowed in a single timestep. This allows the probability to be kept small, such the the probability for multiple mergers within a single timestep is small.
 \item [{\tt mergerTreeBuildCole2000AccretionLimit}] The maximum fractional change in mass due to sub-esolution accretion allowed in any given timestep when building the tree.
 \item [{\tt mergerTreeBuildCole2000MassResolution}] The minimum halo mass (in $M_\odot$) that the algorithm will follow. Mass accretion below this scale is treated as smooth accretion and branches are truncated once they fall below this mass.
\end{description}

\subsection{Smooth Accretion}\label{sec:SmoothAccretion}

Selected with {\tt mergerTreeConstructMethod}$=${\tt smoothAccretion}, this method builds a branchless merger tree with a smooth accretion history using the selected mass accretion history method (see \S\ref{sec:HaloMassAccretionHistory}). The tree has a final mass of {\tt mergerTreeHaloMass} (in units of $M_\odot$) at redshift {\tt mergerTreeBaseRedshift} and is continued back in time by decreasing the halo mass by a factor {\tt mergerTreeHaloMassDeclineFactor} at each new node until a specified {\tt mergerTreeHaloMassResolution} (in units of $M_\odot$) is reached.

\section{Merger Tree Pre-evolution Processing}

Arbitrary processing of merger trees prior to their evolution can be carried out using the {\tt mergerTreePreEvolveTask} directive (see \S\ref{sec:MergerTreePreEvolveTask}). Currently defined tasks are defined below.

\subsection{Mass Accretion History Output}\index{merger tree!mass accretion history}

Output of the mass accretion history (i.e. the mass of the node on the primary branch as a function of time) for each merger tree can be requested by setting {\tt [massAccretionHistoryOutput]}$=${\tt true}. If requested, an additional group, {\tt massAccretionHistories}, is made in the \glc\ output file. This group will contain a subgroup for each merger tree ({\tt mergerTreeN} where {\tt N} is the merger tree index) within which three datasets, {\tt nodeIndex}, {\tt nodeTime} and {\tt nodeMass}, can be found. These give the index, time and mass of the node on the primary branch of the tree at all times for which the tree is defined.

\subsection{Tree Pruning}\index{merger tree!pruning}

This task allows for branches of merger trees to be pruned---i.e. nodes below a specified mass limit are removed from the tree prior to any evolution. This can be useful for convergence studies for example. To prune branches set {\tt [mergerTreePruneBranches]}$=${\tt true} and set {\tt [mergerTreePruningMassThreshold]} to the desired mass threshold below which nodes will be pruned.

\section{Molecular Reaction Rates}\index{molecules!reaction rates}\index{reaction rates!molecular}\label{sec:MolecularReactionRates}

Methods for computing molecular reaction rates are selected via the {\tt [molecularReactionRatesMethods]} parameter. Multiple methods can be selected---their rates are cumulated.

\subsection{Null}

Selected with {\tt molecularReactionRatesMethods}$=${\tt null} this is a null method which does not set any rates.

\subsection{Hydrogen Network}\index{hydrogen!molecular}

Selected with {\tt molecularReactionRatesMethods}$=${\tt hydrogenNetwork} this method computes rates using the network of reactions and fitting functions from \cite{abel_modeling_1997} and \cite{tegmark_small_1997}. The parameter {\tt [hydrogenNetworkFast]} controls the approximations made. If set {\tt true} then H$^-$ is assumed to be at equilibrium abundance, H$_2^+$ reactions are ignored and other slow reactions are ignored (see \citealt{abel_modeling_1997}).

\section{Star Formation Timescales}\index{star formation!timescale}

The methods for computing star formation timescales in disks and spheroids are selected via the {\tt starFormationTimescaleDisksMethod} and {\tt starFormationTimescaleSpheroidsMethod} respectively.

\subsection{Dynamical Time}

Selected with {\tt starFormationTimescale[Disks|Spheroids]Method}$=${\tt dynamical time} this method computes the star formation timescale to be:
\begin{equation}
 \tau_\star = \epsilon_\star^{-1} \tau_{\rm dynamical} \left( {V \over 200\hbox{km/s}} \right)^{\alpha_\star},
\end{equation}
where $\epsilon_\star=${\tt starFormation[Disks|Spheroids]Efficiency} and $\alpha_\star=${\tt starFormation[Disks|Spheroids]VelocityExponent} are input parameters, $\tau_{\rm dynamical}\equiv r/V$ is the dynamical timescale of the component and $r$ and $V$ are the characteristic radius and velocity respectively of the component. The timescale is not allowed to fall below a minimum value specified by {\tt starFormation[Disks|Spheroids]MinimumTimescale} (in Gyr).

\subsection{Kennicutt-Schmidt}\label{sec:StarFormationKennicuttSchmidt}\index{star formation!Kennicutt-Schmidt law}

Selected with {\tt starFormationTimescaleDisksMethod}$=${\tt Kennicutt-Schmidt} this method assumes that the Kennicutt-Schmidt law holds \citep{schmidt_rate_1959,kennicutt_global_1998}:
\begin{equation}
\dot{\Sigma}_\star = A \left({\Sigma_{\rm H} \over M_\odot \hbox{pc}^{-2}} \right)^N,
\end{equation}
where $A=${\tt [starFormationKennicuttSchmidtNormalization]} and $N=${\tt [starFormationKennicuttSchmidtExponent]} are parameters. Optionally, if the {\tt [starFormationKennicuttSchmidtTruncate]} parameter is set to true, then the star formation rate is truncated below a critical surface density such that
\begin{equation}
\dot{\Sigma}_\star = \left\{ \begin{array}{ll} A \left({\Sigma_{\rm H} \over M_\odot \hbox{pc}^{-2}} \right)^N & \hbox{ if } \Sigma_{\rm gas,disk} > \Sigma_{\rm crit} \\ A \left({\Sigma_{\rm H} \over M_\odot \hbox{pc}^{-2}} \right)^N \left(\Sigma_{\rm gas,disk}/\Sigma_{\rm crit}\right)^\alpha & \hbox{ otherwise.} \end{array} \right.
\end{equation}
Here, $\alpha=${\tt [starFormationKennicuttSchmidtExponentTruncated]} and $\Sigma_{\rm crit}$ is a critical surface density for star formation which we specify as
\begin{equation}
\Sigma_{\rm crit} = {q_{\rm crit} \kappa \sigma_{\rm gas} \over \pi \G},
\end{equation}
where $\kappa$ is the epicyclic frequency in the disk, $\sigma_{\rm gas}$ is the velocity dispersion of gas in the disk and $q_{\rm crit}=${\tt [toomreParameterCritical]} is a dimensionless constant of order unity which controls where the critical density occurs. We assume that $\sigma_{\rm gas}$ is a constant equal to {\tt [velocityDispersionDiskGas]} and that the disk has a flat rotation curve such that $\kappa = \sqrt{2} V/R$. The integrated star formation rate over the disk is then
\begin{equation}
\Psi = 2 \pi \int_0^\infty \dot{\Sigma}_\star(R) R \d R.
\end{equation}
From this the star formation timescale is found through the definition $\tau_\star \equiv M_{\rm gas, disk}/\Psi$.

\subsection{Blitz-Rosolowsky}\label{sec:StarFormationBlitzRosolowsky}\index{star formation!Blitz-Rosolowsky rule}

Selected with {\tt starFormationTimescaleDisksMethod}$=${\tt Blitz-Rosolowsky} this method assumes that the star formation rate is given by \citep{blitz_role_2006}:
\begin{equation}
 \dot{\Sigma}_\star(R) = \nu_{\rm SF}(R) \Sigma_{\rm H_2, disk}(R),
\end{equation}
where $\nu_{\rm SF}$ is a frequency given by
\begin{equation}
 \nu_{\rm SF}(R) = \nu_{\rm SF,0} \left[ 1 + \left({\Sigma_{\rm HI}\over \Sigma_0}\right)^q \right],
\end{equation}
where $q=${\tt [surfaceDensityExponentBlitzRosolowsky]} and $\Sigma_0=${\tt [surfaceDensityCriticalBlitzRosolowsky]} are parameters and the surface density of molecular gas $\Sigma_{\rm H_2} = (P_{\rm ext}/P_0)^\alpha \Sigma_{\rm HI}$, where $\alpha=${\tt [pressureExponentBlitzRosolowsky]} and $P_0=${\tt [pressureCharacteristicBlitzRosolowsky]} are parameters and the hydrostatic pressure in the disk plane assuming location isothermal gas and stellar components is given by
\begin{equation}
 P_{\rm ext} \approx {\pi\over 2} \G \Sigma_{\rm gas} \left[ \Sigma_{\rm gas} + \left({\sigma_{\rm gas}\over \sigma_\star}\right)\Sigma_\star\right]
\end{equation}
where we assume that the velocity dispersion in the gas is fixed at $\sigma_{\rm gas}=${\tt [velocityDispersionDiskGas]} and, assuming $\Sigma_\star \gg \Sigma_{\rm gas}$, we can write the stellar velocity dispersion in terms of the disk scale height, $h_\star$, as
\begin{equation}
 \sigma_\star = \sqrt{\pi \G h_\star \Sigma_\star}
\end{equation}
where we assume $h_\star/R_{\rm disk}=${\tt [heightToRadialScaleDisk]}. The star formation rate integated over the disk is
\begin{equation}
 \Psi = \int_0^\infty 2 \pi R \dot{\Sigma}_\star(R) \d R.
\end{equation}
From this the star formation timescale is found through the definition $\tau_\star \equiv M_{\rm gas, disk}/\Psi$.

\subsection{Krumholz-McKee-Tumlinson}\label{sec:StarFormationKMT09}\index{star formation!Krumholz-McKee-Tumlinson method}

Selected with {\tt starFormationTimescaleDisksMethod}$=${\tt KMT09} this method assumes that the star formation rate is given by \citep{krumholz_star_2009}:
\begin{equation}
 \dot{\Sigma}_\star(R) = \nu_{\rm SF} f_{\rm H_2}(R)\Sigma_{\rm HI, disk}(R) \left\{ \begin{array}{ll} (\Sigma_{\rm HI}/\Sigma_0)^{-1/3}, &  \hbox{ if } \Sigma_{\rm HI}/\Sigma_0 \le 1 \\ (\Sigma_{\rm HI}/\Sigma_0)^{1/3}, & \hbox{ if } \Sigma_{\rm HI}/\Sigma_0 > 1 \end{array} \right. ,
\end{equation}
where $\nu_{\rm SF}=${\tt [starFormationFrequencyKMT09]} is a frequency and $\Sigma_0=85 M_\odot \hbox{pc}^{-2}$. The molecular fraction is given by
\begin{equation}
 f_{\rm H_2} = 1 - \left( 1 + \left[ { 3 s \over 4 (1+\delta)} \right]^{-5} \right)^{-1/5},
\end{equation}
where
\begin{equation}
 \delta = 0.0712 \left[ 0.1 s^{-1} + 0.675 \right]^{-2.8},
\end{equation}
and
\begin{equation}
 s = {\ln(1+0.6\chi+0.01\chi^2) \over 0.04 \Sigma_{\rm comp,0} Z^\prime},
\end{equation}
with
\begin{equation}
 \chi = 0.77 \left[ 1 + 3.1 Z^{\prime 0.365} \right],
\end{equation}
and $\Sigma_{\rm comp,0}=c \Sigma_{\rm HI}/M_\odot \hbox{pc}^{-2}$ where $c=${\tt [molecularComplexClumpingFactorKMT09]} is a density enhancement factor relating the surface density of molecular complexes to the gas density on larger scales. Alternatively, if {\tt [molecularFractionFastKMT09]} is set to true, the molecular fraction will be computed using the faster (but less acccurate at low molecular fraction) formula
\begin{equation}
 f_{\rm H_2} = 1 - { 3s/4 \over (1 + s/4)}.
\end{equation}
The star formation rate integated over the disk is
\begin{equation}
 \Psi = \int_0^\infty 2 \pi R \dot{\Sigma}_\star(R) \d R.
\end{equation}
From this the star formation timescale is found through the definition $\tau_\star \equiv M_{\rm gas, disk}/\Psi$.

\subsection{Baugh et al. (2005)}

Selected with {\tt starFormationTimescaleDisksMethod}$=${\tt Baugh2005} this method assumes that the star formation rate is given by a modified version of the \cite{baugh_can_2005} prescription:
\begin{equation}
\tau_\star = \tau_0 (V_{\rm disk}/200\hbox{km/s})^\alpha a^\beta
\end{equation}
where $\tau_0=${\tt [starFormationDiskTimescale]}, $\alpha=${\tt [starFormationDiskVelocityExponent]} and $\beta=${\tt [starFormationExpansionExponent]}.

\section{Stellar Population Properties}\label{sec:StellarPopulationProperties}\index{stellar populations}

Algorithms for determining stellar population properties---essentially the rates of change of stellar and gas mass and abundances given a star formation rate and fuel abundances (and perhaps a historical record of star formation in the component)---are selected by the {\tt stellarPopulationPropertiesMethod} parameter.

\subsection{Instantaneous}

Selected with {\tt stellarPopulationPropertiesMethod}$=${\tt instantaneous} this method uses the instantaneous recycling approximation. Specifically, given a star formation rate $\phi$, this method assumes a rate of increase of stellar mass of $\dot{M}_\star=(1-R)\phi$, a corresponding rate of decrease in fuel mass. The rate of change of the metal content of stars follows from the fuel metallicity, while that of the fuel changes according to
\begin{equation}
 \dot{M}_{fuel,Z} = - (1-R) Z_{\rm fuel} \phi + p \phi.
\end{equation}
In the above $R$ is the instantaneous recycled fraction and $p$ is the yield, both of which are supplied by the \IMF\ subsystem. The rate of energy input from the stellar population is computed assuming that the canonical amount of energy from a single stellar population (as defined by the {\tt feedbackEnergyInputAtInfinityCanonical}) is input instantaneously.

\subsection{Noninstantaneous}

Selected with {\tt stellarPopulationPropertiesMethod}$=${\tt noninstantaneous} this method assumes fully non-instantaneous recycling and metal enrichment. Recycling and metal production rates from simple stellar populations are computed, for any given \IMF, from stellar evolution models. The rates of change are then:
\begin{eqnarray}
 \dot{M}_\star &=& \phi - \int_0^t \phi(t^\prime) \dot{R}(t-t^\prime;Z_{\rm fuel}[t^\prime]) \d t^\prime, \\
 \dot{M}_{\rm fuel} &=& -\phi + \int_0^t \phi(t^\prime) \dot{R}(t-t^\prime;Z_{\rm fuel}[t]) \d t^\prime, \\
 \dot{M}_{\star,Z} &=& Z_{\rm fuel} \phi - \int_0^t \phi(t^\prime) Z_{\rm fuel}(t^\prime)  \dot{R}(t-t^\prime;Z_{\rm fuel}[t^\prime]) \d t^\prime, \\
 \dot{M}_{{\rm fuel},Z} &=& -Z_{\rm fuel} \phi + \int_0^t  \phi(t^\prime) \{ Z_{\rm fuel}(t^\prime) \dot{R}(t-t^\prime;Z_{\rm fuel}[t^\prime]) + \dot{p}(t-t^\prime;Z_{\rm fuel}[t^\prime]) \} \d t^\prime, \\
\end{eqnarray}
where $\dot{R}(t;Z)$ and $\dot{p}(t;Z)$ are the recycling and metal yield rates respectively from a stellar population of age $t$ and metallicity $Z$. The energy input rate is computed self-consistently from the star formation history.

\section{Stellar Population Spectra}

Stellar population spectra are used to construct intgrated spectra of galaxies. The method used to compute such spectra is specified by the {\tt stellarPopulationSpectraMethod} parameter.

\subsection{Conroy, White \& Gunn}

Selected with {\tt stellarPopulationSpectraMethod}$=${\tt Conroy, White \& Gunn} this method uses v2.1 of the \href{http://www.cfa.harvard.edu/~cconroy/FSPS.html}{{\tt FSPS}} code of \cite{conroy_propagation_2009} to compute stellar spectra. If necessary, the {\tt FSPS} code will be downloaded, patched and compiled and run to generate spectra. These tabulations are then stored to file for later retrieval.

\subsection{File}

Selected with {\tt stellarPopulationSpectraMethod}$=${\tt file} this method reads stellar population spectra from an HDF5 file, with format described in \S\ref{sec:StellarPopulationSpectra}.

\section{Stellar Population Spectra Postprocessing}

Stellar population spectra are postprocessed (to handle, for example, absorption by the \IGM). The method used to postprocess spectra is specified by the {\tt stellarPopulationSpectraPostprocessMethod} parameter.

\subsection{Meiksin (2006) IGM Attenuation}

Selected with {\tt stellarPopulationSpectraPostprocessMethod}$=${\tt Meiksin2006} this method postprocesses spectra through absorption by the \IGM\ using the results of \cite{meiksin_colour_2006}.

\subsection{Madau (1995) IGM Attenuation}

Selected with {\tt stellarPopulationSpectraPostprocessMethod}$=${\tt Madau1995} this method postprocesses spectra through absorption by the \IGM\ using the results of \cite{madau_radiative_1995}.

\subsection{Null Method}

Selected with {\tt stellarPopulationSpectraPostprocessMethod}$=${\tt null} this method performs no postprocessing.

\section{Stellar Astrophysics}

Various properties related to stellar astrophysics are required by \glc. The following documents their implementation.

\subsection{Basics}

This subset of properties include recycled mass, metal yield and lifetime.  The method used to compute such properties is specified by the {\tt stellarAstrophysicsMethod} parameter.

\subsubsection{File}\label{sec:StellarAstrophysicsFile}

Selected with {\tt stellarAstrophysicsMethod}$=${\tt file} this method uses reads properties of individual stars of different initial mass and metallicity from an XML file and interpolates in them. The stars can be irregularly spaced in the plane of initial mass and metallicity. The XML file should have the following structure:
\begin{verbatim}
 <stars>
  <star>
    <initialMass>0.6</initialMass>
    <lifetime>28.19</lifetime>
    <metallicity>0.0000</metallicity>
    <ejectedMass>7.65</ejectedMass>
    <metalYieldMass>0.44435954</metalYieldMass>
    <elementYieldMassFe>2.2017e-13</elementYieldMassFe>
    <source>Table 2 of Tumlinson, Shull &amp; Venkatesan (2003, ApJ, 584, 608)</source>
    <url>http://adsabs.harvard.edu/abs/2003ApJ...584..608T</url>
  </star>
  <star>
    .
    .
    .
  </star>
  .
  .
  .
 </stars
\end{verbatim}
Each {\tt star} element must contain the {\tt initialMass} (given in $M_\odot$) and {\tt metallicity} tags. Other tags are optional. {\tt lifetime} gives the lifetime of such a star (in Gyr), {\tt ejectedMass} gives the total mass (in $M_\odot$) ejected by such a star during its lifetime, {\tt metalYieldMass} gives the total mass of metals yielded by the star during its lifetime while {\tt elementYieldMassX} gives the mass of element {\tt X} yielded by the star during its lifetime. The {\tt source} and {\tt url} tags are not used, but are strongly recommended to provide a reference to the origin of the stellar data.

\subsection{Stellar Winds}

Energy input to the \ISM\ from stellar winds is used in calculations of feedback efficiency. The method used to compute stellar wind properties is specified by the {\tt stellarWindsMethod} parameter.

\subsubsection{Leitherer et al. (1992)}

Selected with {\tt stellarWindsMethod}$=${\tt Leitherer1992} this method uses the fitting formulae of \cite{leitherer_deposition_1992} to compute stellar wind energy input from the luminosity and effective temperature of a star.

\subsection{Stellar Tracks}

The method used to compute stellar tracks is specified by the {\tt stellarTracksMethod} parameter.

\subsubsection{File}\label{sec:StellarTracksFile}

Selected with {\tt stellarTracksMethod}$=${\tt file} in this method luminosities and effective temperatures of stars are computed from a tabulated set of stellar tracks. The file containing the tracks to use is specified via the {\tt stellarTracksFile} parameter. The file specified must be an HDF5 file with the following structure:
\begin{verbatim}
 stellarTracksFile
  |
  +-> metallicity1
  |    |
  |    +-> metallicity
  |    |
  |    +-> mass1
  |    |    |
  |    |    +-> mass
  |    |    |
  |    |    +-> age
  |    |    |
  |    |    +-> luminosity
  |    |    |
  |    |    +-> effectiveTemperature
  |    |
  |    x-> massN
  |
  x-> metallicityN
\end{verbatim}
Each {\tt metallicityN} group tabulates tracks for a given metallicity (the value of which is stored in the {\tt metallicity} dataset within each group), and may contain an arbitrary number of {\tt massN} groups. Each {\tt massN} group should contain a track for a star of some mass (the value of which is given in the {\tt mass} dataset). Within each track three datasets specify the {\tt age} (in Gyr), {\tt luminosity} (in $L_\odot$) and {\tt effectiveTemperature} (in Kelvin) along the track.

\subsection{Supernovae Type Ia}\index{supernovae!Type Ia}

Properties of Type Ia supernovae, including the cumulative number occuring and metal yield, are handled by the method selected using the {\tt supernovaeIaMethod} parameter.

\subsubsection{Nagashima et al. (2005) Prescription}

Selected with {\tt supernovaeIaMethod}$=${\tt Nagashima} this method uses the prescriptions from \cite{nagashima_metal_2005} to compute the numbers and yields of Type Ia supernovae.

\subsection{Population III Supernovae}\index{supernovae!Population III}\index{Population III!supernovae}

Properties of Population III specific supernovae are handled by the method selected with the {\tt supernovaePopIIIMethod} parameter.

\subsubsection{Heger \& Woosley (2002)}

Selected with {\tt supernovaePopIIIMethod}$=${\tt Heger + Woosley} this method computes the energies of pair instability supernovae from the results of \cite{heger_nucleosynthetic_2002}.

\subsection{Stellar Feedback}

Aspects of stellar feedback are computed by the method selected with the {\tt stellarFeedbackMethod} parameter.

\subsubsection{Standard}

Selected with {\tt stellarFeedbackMethod}$=${\tt standard}, the method assumes that the cumulative energy input from a stellar population is equal to the total number of (Type II and Type Ia) supernovae multiplied by {\tt supernovaEnergy} (specified in ergs) plus any Population III-specific supernovae energy plus the integrated energy input from stellar winds. The minimum mass of a star required to form a Type II supernova is specified (in $M_\odot$) via the {\tt initialMassForSupernovaeTypeII} parameter.

\section{Substructure and Merging}\index{merging!substructure}\index{substructure}

Substructures and merging of nodes/substructures is controlled by several algorithms which are described below:

\subsection{Merging Timescales}\index{merging!dynamical friction}

The method used to compute merging timescales of substructures is specified by the {\tt satelliteMergingMethod} parameter.

\subsubsection{Dynamical Friction: Lacey \& Cole}

Selected with {\tt satelliteMergingMethod}$=${\tt Lacey-Cole}, this method computes merging timescales using the dynamical friction calculation of \cite{lacey_merger_1993}. Timescales are multiplied by the value of the {\tt mergingTimescaleMultiplier} input parameter.

\subsubsection{Dynamical Friction: Jiang (2008)}

Selected with {\tt satelliteMergingMethod}$=${\tt Jiang2008}, this method computes merging timescales using the dynamical friction calibration of \cite{jiang_fitting_2008}.

\subsubsection{Dynamical Friction: Boylan-Kolchin (2008)}

Selected with {\tt satelliteMergingMethod}$=${\tt BoylanKolchin2008}, this method computes merging timescales using the dynamical friction calibration of \cite{boylan-kolchin_dynamical_2008}.

\subsection{Virial Orbits}

The algorithm to be used to determine orbital parameters of substructures when they first enter the virial radius of their host is specified via the {\tt virialOrbitsMethod} parameter.

\subsubsection{Benson (2005)}

Selected with {\tt virialOrbitsMethod}$=${\tt Benson2005}, this method selects orbital parameters randomly from the distribution given by \cite{benson_orbital_2005}.

\subsubsection{Wetzel (2010)}

Selected with {\tt virialOrbitsMethod}$=${\tt Wetzel2010}, this method selects orbital parameters randomly from the distribution given by \cite{wetzel_orbits_2010}, including the redshift and mass dependence of the distributions. Note that the parameter $R_1$ can be come negative (which is unphysical) for certain regimes of mass and redshift according to the fitting function for $R_1$ given by \cite{wetzel_orbits_2010}. Therefore, we enforce $R_1>0.05$. Similarly, the parameter $C_1$ can become very large in some regimes which is probably an artifact of the fitting function used rather than physically meaningful (and which causes numerical difficulties in evaluating the distribution). We therefore prevent $C_1$ from exceeding $9.999999$\footnote{We use this value rather than $10$ since the GSL $_2F_1$ hypergeomtric function fails in some cases when $C_1\ge 10$.}

\subsection{Node Merging}

The algorithm to be used to process nodes when they become substructures is specified by the {\tt nodeMergersMethod} parameter.

\subsubsection{Single Level Hierarchy}

Selcted with {\tt nodeMergersMethod}$=${\tt single level hierarchy}, this method maintains a single level hierarchy of substructure, i.e. it tracks only substructures, not sub-substructures or deeper levels. When a node first becomes a satellite it is appended to the list of satellites associated with its host halo. If the node contains its own satellites they will be detached from the node and appended to the list of satellites of the new host (and assigned new merging times).

\section{Supernovae Feedback Models}\index{supernovae!feedback}\index{feedback}

The supernovae feedback driven outflow rate is computed using the method specified by the {\tt starFormationFeedback[Disks|Spheroids]Method} for disks and spheroids respectively.

\subsection{Power Law}

Selected with {\tt starFormationFeedback[Disks|Spheroids]Method}$=${\tt power law}, this method assumes an outflow rate of:
\begin{equation}
 \dot{M}_{\rm outflow} = \left({V_{\rm outflow} \over V}\right)^{\alpha_{\rm outflow}} {\dot{E} \over E_{\rm canonical}},
\end{equation}
where $V_{\rm outflow}=${\tt [disk|spheroid]OutflowVelocity} (in km/s) and $\alpha_{\rm outflow}=${\tt [disk|spheroid]OutflowVelocity} are input parameters, $V$ is the characteristic velocity of the component, $\dot{E}$ is the rate of energy input from stellar populations and $E_{\rm canonical}$ is the total energy input by a canonical stellar population normalized to $1 M_\odot$ after infinite time.

\section{Supermassive Black Holes Binary Mergers}\index{supermassive black holes!mergers}

The method to be used for computing the effects of binary mergers of supermassive black holes is specified by the {\tt blackHoleBinaryMergersMethod} parameter.

\subsection{Rezzolla et al. (2008)}

Selected with {\tt blackHoleBinaryMergersMethod}$=${\tt Rezzolla2008}, this method uses the fitting function of \cite{rezzolla_final_2008} to compute the spin of the black hole resulting from a binary merger. The mass of the resulting black hole is assumed to equal the sum of the mass of the initial black holes (i.e. there is negligible energy loss through gravitational waves).



\chapter{Additional Output Quantities}

\section{Merger Tree Links}\index{merger tree!links}

The following properties are output to permit the merger tree structure to be recovered:
\begin{description}
 \item [{\tt nodeIndex}] A unique (within a tree) integer index identifying the node;
 \item [{\tt childIndex}] The index of this node's primary child node (or $-1$ if it has no child);
 \item [{\tt parentIndex}] The index of this node's parent node (or $-1$ if it has no parent);
 \item [{\tt siblingIndex}] The index of this node's sibling node (or $-1$ if it has no sibling);
 \item [{\tt satelliteIndex}] The index of this node's first satellite node (or $-1$ if it has no satellites);
\end{description}

\section{Virial Quantities}\index{virial}

The following quantities related to the virialized region of each node are output if {\tt outputVirialData} is set to true:
\begin{description}
 \item [{\tt nodeVirialRadius}] The virial radius (following whatever definition of virial overdensity was selected in \glc) in units of Mpc;
 \item [{\tt nodeVirialVelocity}] The circular velocity at the virial radius (in km/s).
\end{description}




\part{Development}

In this Part we focus on how to modify \glc\ to meet your own needs. \glc\ is designed in a modular way to make it as simple as possible to introduce new implementations of physical processes or new galactic components without breaking the rest of the code. Nevertheless, some understanding of the structure of the code is necessary. In particular, \glc\ will happily compile and run calculations that make no physical sense whatsoever---it's up to you to ensure that the changes you make are physically reasonable and consistent with the behavior of the rest of the code.

\chapter{Developing \glc}

The following is a quickstart guide to making changes to the \glc\ source code and contributing them back to the project. Note that the preferred method to do this is through \href{https://bitbucket.org}{\sc BitBucket}.

\section{Getting Started}

It's easy to begin working with and changing the \glc\ source code. Assuming you have Mercurial (``{\tt hg}'') installed, just do:
\begin{verbatim}
 hg clone https://abensonca@bitbucket.org/abensonca/galacticus galacticus
\end{verbatim}
and you have a cloned copy of the \glc\ repository in the {\tt galacticus} directory.

\subsection{Using {\sc BitBucket}}

If you plan to contribute changes back to the \glc] project (please do!), you should consider using \href{https://bitbucket.org}{\sc BitBucket}. After you've created an account for yourself at {\sc BitBucket}, you can ``fork'' the \glc\ repository to have your own working copy. This can be done as follows:
\begin{itemize}
 \item visit the \glc\ repository on {\tt BitBucket} at \href{https://bitbucket.org/abensonca/galacticus/overview}{\tt https://bitbucket.org/abensonca/galacticus/overview};
 \item click on the ``Fork'' button---you'll be presented with a form;
 \item fill out the form (setting a name for your fork, a short description, etc.), then click the ``Fork repository'' button;
 \item after the fork completes, you'll be taken to the overview page for your forked repository.
\end{itemize}. 

You'll now want to clone this forked repository to your local system. Click on the ``Clone'' button and copy the {\tt hg clone} command presented there (you want the {\tt SSH} version so that you can push changes back to this repository). Run this command on your local system to get a cloned copy of your new repository. You can now work with this repository, make any changes, and commit then. We'll discuss how to send these changes back to {\sc BitBucket}, and back to the \glc\ project below.

\section{Making Simple Changes}

If you want to make some relatively minor changes to the \glc\ code, such as fixing a typo, adding a new filter, etc. you can just make changes directly on the {\tt default} branch (i.e. at the point of active development). To do this, make sure you're at {\tt default}:
\begin{verbatim}
 hg pull
 hg update default
\end{verbatim}
Then make your changes, add new files, etc. Once you're done, first check if there have been any changes to {\tt default} since you {\tt pull}ed:
\begin{verbatim}
 hg incoming
\end{verbatim}
If any new changesets are shown, us {\tt hg pull -u} to merge these in to your working copy. Then commit your changes:
\begin{verbatim}
 hg commit
\end{verbatim}
Your changes are now commited to your cloned repository.

\subsection{Contributing Your Changes Back To \glc}

Once you've committed your changes, you can contribute them back to the \glc\ project.

\subsubsection{Via E-mail}

If you just cloned the \glc\ repository directly you can send a patch containing your changes by e-mail to \href{mailto:abenson@obs.carnegiescience.edu}{abenson\@obs.carnegiescience.edu}. First, create the patch file using:
\begin{verbatim}
 hg export -r begin:end > changes.diff
\end{verbatim}
where {\tt begin} and {\tt end} are the first and last revisions that you want to include (you can specify more complicated sets of revisions of course). Then simply attached the {\tt changes.diff} file to an e-mail. It will be merged into the \glc\ project using
\begin{verbatim}
 hg import changes.diff
\end{verbatim}

\subsubsection{Using {\sc BitBucket}}

If you forked the \glc\ repository on {\sc BitBucket}, you can now push your changes back to {\tt BitBucket} using
\begin{verbatim}
 hg push
\end{verbatim}
If you want to contribute these changes back to the \glc\ project the best way to do so is to create a ``pull request''. Simply visit your forked respository on {\sc BitBucket} and click the ``Pull request'' button. The form you're presented with allows you to choose which branch in your respository you want to send changes from, and which branch in the \glc\ project you want them contributed to. Add a title and description of your changes (and, optionally, check the ``Close branch'' box if you're done with this branch of development) then click ``Create pull request''. Assuming your code looks godo and works, it can then be pulled into the \glc\ project.

\section{Making Bigger Changes}

For bigger changes, particularly those where you're adding a new feature, we recommend using Mercurial's ``feature branches''. These provide a permanent record of for which feature each changeset was added. Using feature branches is straightforward. Begin with {\tt default} and create a new branch:
\begin{verbatim}
 hg update default
 hg branch myNewFeature
\end{verbatim}
where {\tt myNewFeature} is a name for your feature branch. Then begin working, make changes, add new files etc. You can make commits when necessary (and it's good to make several small commits rather than one big one). You should merge {\tt default} into your feature branch as often as possible to avoid them getting out of sync (which makes for difficulty later when you want to merge your feature branch back into {\tt default}):
\begin{verbatim}
 hg update myFeatureBranch
 hg merge default
 hg commit -m "merged default into myFeatureBranch"
\end{verbatim}
Once the feature branch is stable, you can merge it back into {\tt default}:
\begin{verbatim}
 hg update default
 hg merge myFeatureBranch
 hg commit -m "merged myFeatureBranch"
\end{verbatim}
Once you're done developing this feature, you should close the feature branch:
\begin{verbatim}
 hg commit --close-branch -m "finished my feature"
\end{verbatim}
Note that you can always go back and work on a feature branch later, after you have closed it. Just do:
\begin{verbatim}
 hg up myFeatureBranch
 hg merge default
 hg commit -m "merged default into myFeatureBranch"
\end{verbatim}
then continue to work with your feature branch as normal (don't forget to close it again when you're finished working with it).

\section{Releases}

Each release of \glc\ exists as a separate branch within the main \glc\ repository. To work with a particular release use
\begin{verbatim}
 hg update v0.9.2
\end{verbatim}
replacing the version number with whichever version you want. To get back to the development tip use
\begin{verbatim}
 hg update default
\end{verbatim}

\subsection{Bug Fixes In Releases}

To make a bugfix in a release, simply {\tt hg update} to that release, fix the bug, and commit your changes. In many cases you'll want to fix the same bug in later releases and also in {\tt default}. To do that, just {\tt hg update} to each branch in turn, use {\tt hg merge fixedBranch} (where ``{\tt fixedBranch}'' is the name of the branch in which you fixed the branch, and then commit the merge. Once the bgu has been fixed you can contribute the fix back to the \glc\ project using the methods described above.


\chapter{Coding \glc}

\section{Node Class Hierarchy Builder}

The hierarchy of classes describing a merger tree node and its galaxy (i.e. {\normalfont \ttfamily treeNode}, {\normalfont \ttfamily nodeComponent}, {\normalfont \ttfamily nodeComponentBasic}, {\normalfont \ttfamily nodeComponentBasicStandard}, etc.) are built automatically to realize the set of component implementations and properties specified in source files.

\subsection{Variable Definitions}\label{sec:nodeBuilderVariableDefinitions}

Throughout the node objects builder code, Fortran variables are defined using a common specification. This is a simple hash containing the following entries:
\begin{description}
  \item[intrinsic] a string specifying the intrinsic Fortran type ({\normalfont \ttfamily integer}, {\normalfont \ttfamily logical}, {\normalfont \ttfamily real}, {\normalfont \ttfamily double precision}, {\normalfont \ttfamily complex}, {\normalfont \ttfamily double complex}, {\normalfont \ttfamily type}, {\normalfont \ttfamily class}, {\normalfont \ttfamily procedure});
  \item[type] \emph{[optional, except for {\normalfont \ttfamily type} and {\normalfont \ttfamily class} intrinsics]} a string specifying the type of the variables;
  \item[attributes] an array of strings specifying all attributes of the variables;
  \item[variables] an array of strings giving the names of the variables.
\end{description}
For example, rank-1, long integer arguments which will not be modified in their function would be specified as:
\begin{verbatim}
{
  intrinsic  => "integer", 
  type       => "kind_int8",
  attributes => [ "dimension(:)", "intent(in)" ],
  variables  => [ "argument1", "argument2" ]
}
\end{verbatim}

\subsection{The {\normalfont \ttfamily \$build} Data Structure}

The {\normalfont \ttfamily \$build} data structure is used to accumulate all objects, variables, interfaces, and functions needed to build the class hierarchy used to represent nodes and components in \glc. At the end of the node objects build process, this object is processed to generate the required Fortran code. The following subsections describe how to add information to this object.

\subsubsection{Component Classes and Implementations}

A reference to a hash of structures which define all component classes, keyed by class name, is provided as {\normalfont \ttfamily \$build->\{'componentClasses'\}}. Each component class structure has the form:
\begin{verbatim}
{
 name    => <name of the class>,
 members => <reference to a hash of structures which define all member implementation of the class>
}
\end{verbatim}
Each member implementation structure has the form:
\begin{verbatim}
{
 name       =>             <name of the implementation>,
 properties => property => <reference to a hash of structures which define all properties of this implementation>
}
\end{verbatim}

\subsubsection{Types}\label{sec:buildHierarchyTypes}

Definitions of derived types are accumulated to the hash {\normalfont \ttfamily \$build->\{'types'\}}, with the key of each entry being the derived type's name. Each derived type structure has the form:
\begin{verbatim}
{
 name           => <name of the derived type>
 comment        => <a description of the type>
 isPublic       => <1 if the class should have public visbility, 0 otherwise>
 dataContent    => <variable list containing all data for this class>
 boundFunctions => 
  [
   {
    type        => <procedure|generic>
    descriptor  => <a function descriptor data structure> [optional]
    name        => <name of the bound method>
    function    => <name of function to bind to> [optional; not required if descriptor is provided]
    description => <LaTeX-syntax description of the method> [optional; not required if descriptor is provided]
    returnType  => <LaTeX-syntax return type of the method> [optional; not required if descriptor is provided]
    arguments   => <LaTeX-syntax definition of all arguments to the method> [optional; not required if descriptor is provided]
   }
  ]
}
\end{verbatim}
The preferred approach is to provide a function descriptor (see \S\ref{sec:buildHierarchyFunctions}), and to omit the {\normalfont \ttfamily function}, {\normalfont \ttfamily description}, {\normalfont \ttfamily returnType}, and {\normalfont \ttfamily arguments} entries (which will be inferred from the function descriptor). The {\normalfont \ttfamily function}, {\normalfont \ttfamily description}, {\normalfont \ttfamily returnType}, and {\normalfont \ttfamily arguments} entries will eventually be deprecated in favor of the {\normalfont \ttfamily descriptor} entry.

\subsubsection{Interfaces}\label{sec:buildHierarchyInterfaces}

Definitions of interfaces are accumulated to the hash {\normalfont \ttfamily \$build->\{interfaces\}}, with the key of each entry being the interfaces name. Each interface structure has the form:
\begin{verbatim}
{
 name           => <name of the interface>
 comment        => <a description of the interface>
 intrinsic      => <the intrinsic type of the function (or "void" for a subroutine)> 
 data           => <variable list containing all arguments for this interface>
\end{verbatim}

\subsubsection{Functions}\label{sec:buildHierarchyFunctions}

Definitions of functions are accumulated either to the list {\normalfont \ttfamily \$build->\{'functions'\}} or are included as the {\normalfont \ttfamily descriptor} in a derived type structure (this is the preferred method for functions that \emph{are} bound to a derived type; see \S\ref{sec:buildHierarchyTypes}). Each function structure has the form:
\begin{verbatim}
{
 type        => <the type of function>
 name        => <function name>
 description => <LaTeX-syntax description of the function>
 modules     => <list of names of modules required by the function> [optional]
 variables   => <list of variables required by the function> [optional]
 content     => <the code of the function (excluding opener, closer, and variable definitions)>
}
\end{verbatim}

\subsubsection{Module-scope Variables}

Any module-scope variables can be pushed to the array {\normalfont \ttfamily \@\{\$build->\{'variables'\}\}}, using the usual definition format described in \S\ref{sec:nodeBuilderVariableDefinitions}.

\section{Galacticus Preprocessor Directives}\label{sec:sourceTreePreprocessor}

\glc\ has its own preprocessor for Fortran source files. This preprocesses parses each source file into an internal tree representation, performs various manipulations on that tree, and then outputs the preprocessed file for compilation. The preprocessor is used to automate and standardize many common tasks, through the inclusion of directives into the source code. Directives are specified in comment lines beginning {\normalfont \ttfamily !\#}, and are written in XML. The remainder of this section describes the various preprocessor functionalities, and gives examples of their usage.

\subsection{Source Code Introspection}

The source code introspection functionality allows automated generation of information about the source code. Specifically:
\begin{itemize}
\item The directive {\normalfont \ttfamily \{introspection:location\}} in source code will be replaced with a character string giving a backtrace of the current location in the code, including any function, module, and file (including line number).
\end{itemize}

\subsection{Function Attributes}

\glc\ allows the specification of attributes for functions which alter the way the compiler treats them. Function attributes are specified as:
\begin{verbatim}
!$GLC function attributes {attributes} :: {functionNames}
\end{verbatim}
where {\normalfont \ttfamily \{attributes\}} is a space-separated list of attributes, and {\normalfont \ttfamily \{functionNames\}} is a space-separated list of functions to apply these functions to. Function attribute directives may appear anywhere in a file containing the named function, but it is good practice to locate them immediately before the function.

Currently, the supported attributes are:
\begin{description}
\item[{\normalfont \ttfamily unused}] The function is marked as being \emph{possibly} unused. Compiler warnings about unused functions will be suppressed for this function.
\end{description}

\subsection{Source Digest}

The {\normalfont \ttfamily sourceDigest} directive will generate an \gls{MD5hash} hash of the source code of the file in which the directive is placed, along with the source code of any files upon which it depends. This can be useful in generating unique labels (e.g. to use as suffixes in file names) which automatically update if the source code is modified. To generate a source digest simply use:

\begin{lstlisting}
  !# <sourceDigest name="mySourceDigest"/>
\end{lstlisting}

A source digest will be generated and stored as a {\normalfont \ttfamily character(len=22)} variable called {\normalfont \ttfamily mySourceDigest}.

\subsection{Object Builder}

When constructing instances of a class from a provided parameter set, a common pattern is to need to construct other objects based on those parameters, which will be used by the instance. For example, a transfer function class might require a cosmological parameters object for its operation. In such cases, we often want to use the default instance of the required class unless a different instance is explicitly specified. The {\normalfont \ttfamily objectBuilder} directive automates this process.

As an example, the following constructor requires an instance of the {\normalfont \ttfamily cosmologyParameters} class, which is passes to an internal constructor:

\begin{lstlisting}  
  function myConstructorParameters(parameters)
    use Input_Parameters2
    implicit none
    type(myClass                  )                :: myConstructorParameters
    type(inputParameters          ), intent(in   ) :: parameters
    class(cosmologyParametersClass), pointer       :: cosmologyParameters_    

    !# <objectBuilder class="cosmologyParameters" name="cosmologyParameters_" source="parameters"/>
    myConstructorParameters=myConstructorInternal(cosmologyParameters_)
    return
  end function myConstructorParameters
\end{lstlisting}

The {\normalfont \ttfamily objectBuilder} directive will assign a member of the class specified by the {\normalfont \ttfamily class} attribute to the variable specified by the {\normalfont \ttfamily name} attribute. If the parameter set specified by the {\normalfont \ttfamily parameters} attribute contains an explicit definition of the relevant class, that definition will be used to construct the instance. Otherwise, the parent parameter set will be checked for a definition of the relevant class, and so on. If no definition is found when the root parameter set is found the default instance will be used. Note that objects created by an {\normalfont \ttfamily objectBuilder} directive are directly associated with the element in the input parameters XML document from which they were created. Therefore, if a later {\normalfont \ttfamily objectBuilder} requires the object from that same XML element, it will reuse the one previously created.

The {\normalfont \ttfamily objectBuilder} directive by default searches for a parameter named {\normalfont \ttfamily \{class\}} where {\normalfont \ttfamily \{class\}} is the value of the {\normalfont \ttfamily class} attribute in the directive. An alternative name may be specified via the addition of a {\normalfont \ttfamily parameterName} attribute.

In cases where an explicit {\normalfont \ttfamily parameterName} attribute is given, it is possible to specify an explicit default for the object if no such named parameter is found. (Where no explicit {\normalfont \ttfamily parameterName} attribute is given the global default of the class is used.) A default is specified by adding a {\normalfont \ttfamily default} element to the directive, which should contain default specification for the object (and any subobjects) in the usual format. For example:
\begin{verbatim}
 !# <objectBuilder class="massDistribution" parameterName="diskMassDistribution" name="diskMassDistribution" source="globalParameters">
 !#  <default>
 !#   <diskMassDistribution value="exponentialDisk">
 !#    <dimensionless value="true"/>
 !#   </diskMassDistribution>
 !#  </default>
 !# </objectBuilder>
\end{verbatim}

\subsection{Object Destructor}

This directive can (and should) be used to destroy objects built by the {\normalfont \ttfamily objectBuilder} directive. The {\normalfont \ttfamily objectDestructor} directive automates the process of deciding if these objects should be destroyed (or merely have pointers to them nullified). As an example, the following destructor destroys two associated objects:
\begin{lstlisting}  
  subroutine simpleDestructor(self)
    implicit none
    type(powerSpectrumPrimordialTransferredSimple), intent(inout) :: self

    !# <objectDestructor name="self%transferFunction_"       />
    !# <objectDestructor name="self%powerSpectrumPrimordial_"/>
   return
  end subroutine simpleDestructor
\end{lstlisting}

\subsection{Constructor Assignments}

A common requirement in object constructors is to assign the values of arguments to the constructor to corresponding entries in the object. The {\normalfont \ttfamily constructorAssign} directive performs this assignment for a list of comma-separated variables. For example:
\begin{lstlisting}  
 function stellarMassConstructorInternal(massThreshold)
    implicit none
    type            (galacticFilterStellarMass)                :: stellarMassConstructorInternal
    double precision                           , intent(in   ) :: massThreshold
    !# <constructorAssign variables="massThreshold"/>
    return
  end function stellarMassConstructorInternal
\end{lstlisting}
will cause the value of the {\normalfont \ttfamily massThreshold} argument to {\normalfont \ttfamily stellarMassConstructorInternal\%massThreshold}. If an argument name is prefixed with {\normalfont \ttfamily \textasteriskcentered} in the variables list, pointer assignment is used instead of standard assignment.

\subsection{State Storing}

\glc\ supports storing its internal state to file to allow \href{https://github.com/galacticusorg/galacticus/releases/download/bleeding-edge/Galacticus_Usage.pdf\#sec.Restarting}{restarts}. Code to store and restore the internal state of objects of a given class can be generated automatically through use of the {\normalfont \ttfamily stateStorable} directive. An example is:
\begin{verbatim}
 !# <stateStorable class="table">
 !#  <table1DGeneric>
 !#   <restoreTo variables="reset" state=".true."/>
 !#   <exclude variables="staticData"/>
 !#  </table1DGeneric>
 !#  <table2DLinLinLin>
 !#   <restoreTo variables="resetX, resetY" state=".true."/>
 !#  </table2DLinLinLin>  
 !# </stateStorable>
\end{verbatim}
This specifies that the {\normalfont \ttfamily table} class (and all child classes) can and should be stored to file as part of the representation of the internal state. Code to store and restore all data associated with any object of this class (as well as restoring polymorphic objects to the correct type) will be generated. If certain variables of the class or subclass should be restored to specific values this can be specified through a {\normalfont \ttfamily restoreTo} element placed within an element with the name of the class or subclass (e.g. the {\normalfont \ttfamily table1DGeneric} in the above example). The {\normalfont \ttfamily restoreTo} element should specify a comma-separated list of one or more variables to set in its {\normalfont \ttfamily variables} attribute, and the state to which they should be restored in its {\normalfont \ttfamily state} attribute. Any variables which should be excluded from state store/restore (e.g. if their values are known to be determined statically at construction) can be specified via a {\normalfont \ttfamily exclude} element---a list of variables to exclude should be given as a comma-separated list in its {\normalfont \ttfamily variables} attribute.

\subsection{Conditional Call}

In some instances it is useful to be able to call a function with different combinations of optional arguments depending on certain conditions. (For example, if some combinations of optional arguments are mutually exclusive.) This can be achieved using the {\normalfont \ttfamily conditionalCall} directive. An example is:
\begin{verbatim}
  !# <conditionalCall>
  !#  <call>self=massDistributionBetaProfile(beta{conditions})</call>
  !#  <argument name="densityNormalization" value="densityNormalization" parameterPresent="parameters"/>
  !#  <argument name="mass"                 value="mass"                 parameterPresent="parameters"/>
  !#  <argument name="outerRadius"          value="outerRadius"          parameterPresent="parameters"/>
  !#  <argument name="coreRadius"           value="coreRadius"           parameterPresent="parameters"/>
  !#  <argument name="dimensionless"        value="dimensionless"        parameterPresent="parameters"/>
  !# </conditionalCall>
  !# <inputParametersValidate source="parameters"/>
\end{verbatim}
The {\normalfont \ttfamily call} element specifies the function call, and contains the special sequence {\normalfont \ttfamily \{conditions\}} which will be replaced with the conditionally-present arguments. One or more {\normalfont \ttfamily argument} elements should specify the various arguments which should be included in the call. For each such element the {\normalfont \ttfamily name} attribute specifies the name of the dummy argument in the called function, the {\normalfont \ttfamily value} attribute specifies the value (or variables) to pass this this dummy argument. A condition for inclusion of the argument must also be specified. In the above, the special {\normalfont \ttfamily parameterPresent} condition is used. The argument will be included in the call if a parameter with a name matching the {\normalfont \ttfamily name} attribute exists in the parameter set named in the {\normalfont \ttfamily parameterPresent} attribute. Alternatively a {\normalfont \ttfamily condition} attribute can be given. An argument is included in the call if the expression given in the {\normalfont \ttfamily condition} attribute evaluates to true.

Code will be generated to call the function with all possible combinations of arguments.

\subsection{Optional Arguments}

Fortran supports optional arguments to functions, but does not provide for a default value if those arguments are not present. The {\normalfont \ttfamily optionalArgument} directive allows a default value to be specified. In the following example, a default value is defined for the {\normalfont \ttfamily units} argument:

\begin{lstlisting}
  double precision function simpleHubbleConstant(self,units)
    implicit none
    class  (cosmologyParametersSimple), intent(inout)           :: self
    integer                           , intent(in   ), optional :: units
    !# <optionalArgument name="units" defaultsTo="hubbleUnitsStandard" />

    select case (units_)
    case (hubbleUnitsStandard)
       ! Return the value using the default units.
    case ....
       ! Return the value using some other units.
    end select
    return
  end function simpleHubbleConstant
\end{lstlisting}
The {\normalfont \ttfamily optionalArgument} directive should appear after variable declarations and before any attempt to use the optional argument, and should have two attributes, {\normalfont \ttfamily name} and {\normalfont \ttfamily defaultsTo} which give the name of the argument variable and its default value respectively. The preprocessor will add a new variable with the same name plus an underscore suffix, and will ensure that it is initialized to the default value if the optional variable is not present, otherwise setting it to the value of the optional variable.

Note that for optional arguments that are {\normalfont \ttfamily intent(out)} or {\normalfont \ttfamily intent(inout)} the preprocessor currently \emph{does not} ensure that the value of the new variable is copied back to the argument prior to exit from the function.

\subsection{Enumerations}

The {\normalfont \ttfamily enumeration} directive allows specification of an enumeration (a set of labels), and (optionally) functions to decode such a label from user input. The following example illustrates this usage:

\begin{lstlisting}
module Cosmology_Parameters

  !# <enumeration>
  !#  <name>hubbleUnits</name>
  !#  <description>Specifies the units for the Hubble constant.</description>
  !#  <visibility>public</visbility>
  !#  <validator>yes</validator>
  !#  <encodeFunction>yes</encodeFunction>
  !#  <entry label="standard" />
  !#  <entry label="time"     />
  !#  <entry label="littleH"  />
  !# </enumeration>

contains

subroutine Test_Enumeration()
    use ISO_Varying_String
    implicit none
    class(cosmologyParametersClass), pointer :: cosmologyParameters_
    cosmologyParameters_ => cosmologyParameters()
    write (0,*) "Enumeration contains ",hubbleUnitsCount," entries from ",hubbleUnitsMin," to ",hubbleUnitsMax
    write (0,*) "Hubble constant in little-h units is: ",cosmologyParameters_%HubbleConstant(hubbleUnitsLittleH)
    if (enumerationHubbleUnitsEncode('hubbleUnitsStandard') == hubbleUnitsStandard) then
      write (0,*) "Enumeration decoding succeeded"
    else
      write (0,*) "Enumeration decoding failed"
    end if
    if (enumerationHubbleUnitsEncode('standard',includesPrefix=.false.) == hubbleUnitsStandard) then
      write (0,*) "Enumeration decoding succeeded"
    else
      write (0,*) "Enumeration decoding failed"
      end if
      write (0,*) "Name from time value is '",char(enumerationHubbleUnitsDecode(hubbleUnitsTime,includePrefix=.true.)),"'"
    return
  end subroutine Test_Enumeration

end module Cosmology_Parameters
\end{lstlisting}

Enumerations must be defined in the declaration section of a {\normalfont \ttfamily module}. The encoding and decoding functions will only be generated if the {\normalfont \ttfamily encode} element is present and has content {\normalfont \ttfamily yes}. The enumeration variables are given {\normalfont \ttfamily public} visibility by default---this can be overridden using the {\normalfont \ttfamily visibility} element. If the {\normalfont \ttfamily validator} element is present and set to {\normalfont \ttfamily yes} then a function is created which will return true if the given value is a valid one for the enumeration. Additionally, if the {\normalfont \ttfamily validator} element is present and set to {\normalfont \ttfamily yes} variables are created giving a count of the number of entries in the enumeration, along with the minimum and maximum values in the enumeration. Note that the {\normalfont \ttfamily description} element is used to generate an entry for the enumeration in the document, and so should be written in \LaTeX\ syntax.

\subsection{Input Parameters}

The {\normalfont \ttfamily inputParameter} directive reads an input parameter and assigns the appropriate value to the given variable. {\normalfont \ttfamily inputParameter} directives must occur within the main body of a function, subroutine, or program. A default value can be specified if desired. The following example illustrates this usage:

\begin{lstlisting}
  subroutine simpleParametersRead()
    implicit none
    double precision :: hubbleConstant
    !# <inputParameter>
    !#   <name>HubbleConstant</name>
    !#   <source>myParameters</source>
    !#   <variable>hubbleConstant</variable>
    !#   <defaultValue>69.7d0</defaultValue>
    !#   <defaultSource>(\citealt{hinshaw_nine-year_2012}; CMB$+H_0+$BAO)</defaultSource>
    !#   <description>The present day value of the Hubble parameter in units of km/s/Mpc.</description>
    !#   <type>real</type>
    !#   <cardinality>0..1</cardinality>
    !# </inputParameter>
    if (hubbleConstant < 0.0d0) write (0,*) "The universe is collapsing!"
    return
  end subroutine simpleParametersRead
\end{lstlisting}

In this case, the value of the {\normalfont \ttfamily HubbleConstant} parameter is assigned to the {\normalfont \ttfamily hubbleConstant} variable, with a default of $69.7$ if no value was specified in the input parameter file. If the {\normalfont \ttfamily source} element is present, the parameter will be read from the named {\normalfont \ttfamily inputParameters} set, otherwise the parameter will be read from the top-level of the parameters file. The {\normalfont \ttfamily defaultSource}, {\normalfont \ttfamily <description>The}, {\normalfont \ttfamily type}, and {\normalfont \ttfamily cardinality} elements are used only for adding an entry for the input parameter to the documentation, and so should be written in \LaTeX\ syntax.

It is also possible to specify a set of parameter which iterate over names defined by other directives. The following example would read one parameter named ``{\normalfont \ttfamily fileNameForXXXXXXIMF}'' where ``{\normalfont \ttfamily XXXXXX}'' equals the {\normalfont \ttfamily name} element each {\normalfont \ttfamily imfRegisterName} directive:
\begin{lstlisting}
    !# <inputParameter>
    !#   <iterator>fileNameFor(#imfRegisterName->name)IMF</iterator>
    !#   <source>parameters</source>
    !#   <variable>fileNames(IMF_Index("$1"))</variable>
    !#   <defaultValue>Galacticus_Input_Path()//"data/SSP_Spectra_imf$1.hdf5"</defaultValue>
    !#   <description>The name of the file of stellar populations to use for the named \gls{imf}.</description>
    !#   <type>string</type>
    !#   <cardinality>0..1</cardinality>
  !# </inputParameter>
\end{lstlisting}

\subsection{Input Parameter Lists}

The {\normalfont \ttfamily inputParameterList} directive will construct a {\normalfont \ttfamily varying\_string} array containing the names of all input parameters which are defined in the unit in which the directive appears. The name of the array is specified by the {\normalfont \ttfamily label} attribute of the {\normalfont \ttfamily inputParameterList} directive. If a {\normalfont \ttfamily source} attribute is specified in the directive then only parameters being read from the named variable will be included in the list, otherwise any parameters read will be included. Such a list can be used to validate the names of parameters passed to a function for example.

\subsection{Function Classes}

Most\footnote{At this time the \protect\glc\ code base is being transitioned to use this approach.} of the internal functionality within \glc\ is provided by ``function classes''. These are classes (in the object oriented sense) which model some particular physical entity or concept (e.g. the underlying cosmological model) and provide one or more functions associated with that entity or concept. A default implementation of each function class can be selected at run-time, allowing for simple user-defined control of model behavior. A function class is specified by a {\normalfont \ttfamily functionClass} directive, together with one or more implementations of the function class specified by their own directives.

An example of a function class directive, which defines a class for cosmological parameters (which in this case, for simplicity, consists of just the Hubble constant) is given below:

\begin{lstlisting}
  !# <functionClass>
  !#  <name>cosmologyParameters</name>
  !#  <descriptiveName>Cosmological Parameters</descriptiveName>
  !#  <description>Object providing various cosmological parameters.</description>
  !#  <default>simple</default>
  !#  <defaultThreadPrivate>no</defaultThreadPrivate>
  !#  <stateful>no</stateful>
  !#  <calculationReset>no</calculationReset>
  !#  <method name="HubbleConstant" >
  !#   <description>Return the Hubble constant at the present day. The optional {\normalfont \ttfamily units} argument specifies if the return value should be in units of km/s/Mpc (hubbleUnitsStandard), Gyr$^{-1}$ (hubbleUnitsTime), or 100 km/s/Mpc (hubbleUnitsLittleH).</description>
  !#   <type>double precision</type>
  !#   <pass>yes</pass>
  !#   <argument>integer, intent(in   ), optional :: units</argument>
  !#  </method>
  !# </functionClass>
\end{lstlisting}

The directive should contain the following elements:
\begin{description}
\item[{\normalfont \ttfamily name}] The name of this function class.
\item[{\normalfont \ttfamily descriptiveName}] A descriptive name for the function class, suitable for inclusion in the documentation.
\item[{\normalfont \ttfamily description}] A description of the purpose of this function class. This description will be included into the documentation so should be written in \LaTeX\ syntax.
\item[{\normalfont \ttfamily default}] The default implementation to use for this function class if no choice is made in the input parameter file.
\item[{\normalfont \ttfamily defaultThreadPrivate}] \emph{(optional)} If present and set to {\normalfont \ttfamily yes} then the default implementation of this function class will be made OpenMP thread private. Otherwise, the default implementation is shared between threads. A thread private default implementation can be useful if the function class may need to generate look-up tables unique to each thread on the fly for example.
\item[{\normalfont \ttfamily calculationReset}] \emph{(optional)} If present and set to {\normalfont \ttfamily yes} then the default implementation of the function class is assumed to possibly want to reset its calculations when the active \gls{node} changes (see \S\ref{sec:CalculationResetTask}). In this case, an additional method is generated for the function class: {\normalfont \ttfamily calculationReset} with interface:
  \begin{lstlisting}
    subroutine calculationReset(self,thisNode)
      class  (functionClassBaseName), intent(inout)          :: self
      type   (treeNode             ), intent(inout), pointer :: thisNode
    end subroutine calculationReset
  \end{lstlisting}
  This method has a null implementation for the base class of the function class, but can be overridden to reset calculations of any given implementation.
\item[{\normalfont \ttfamily method}] Each {\normalfont \ttfamily method} element defines a method which the function class will support, the name of which is given by a {\normalfont \ttfamily name} attribute. The method definition must contain the following elements:
  \begin{description}
  \item[{\normalfont \ttfamily description}] A description of this method (in \LaTeX\ syntax) suitable for inclusion into the documentation.
  \item[{\normalfont \ttfamily type}] The type of the function (e.g. {\normalfont \ttfamily double precision}; use {\normalfont \ttfamily void} for a subroutine).
  \item[{\normalfont \ttfamily pass}] If {\normalfont \ttfamily yes} pass the object that the method was called on as the first argument.
  \item[{\normalfont \ttfamily argument}] \emph{(optional)} Zero or more declarations (in standard Fortran syntax) for the method arguments (there is no need to specify the declaration for the object upon which the method was called in the case where this object is passed to the method function).
  \end{description}
\end{description}

Implementations of this function class must be declared with a directive having the same name as the function class (i.e. {\normalfont \ttfamily cosmologyParameters} in the example above). The directive must give the name of the implementation as an attribute, and a description of the implementation suitable for inclusion into the documentation. Each implementation should be placed in a separate file---the preprocessor will find these files and merge the implementations and function class definition into a single file for compilation. Each implementation should declare a class which extends the basic function class, constructor interfaces, and any module-scope data required by the class. The implementation should also define all necessary functions required by the class (separated from the declarations by a {\normalfont \ttfamily contains} keyword). An example is given below:

\begin{lstlisting}
  !# <cosmologyParameters name="cosmologyParametersSimple">
  !#  <description>Provides the Hubble constant: $H_0$.</description>
  !# </cosmologyParameters>
  type, extends(cosmologyParametersClass) :: cosmologyParametersSimple
     private
     double precision :: HubbleConstantValue
   contains
     final     ::                    simpleDestructor
     procedure :: HubbleConstant  => simpleHubbleConstant
  end type cosmologyParametersSimple

  interface cosmologyParametersSimple
     module procedure simpleDefaultConstructor
     module procedure simpleConstructor
  end interface cosmologyParametersSimple

contains

  function simpleDefaultConstructor()
    implicit none
    type(cosmologyParametersSimple) :: simpleDefaultConstructor

    ! Construct an instance of this class using a value of the Hubble constant read from the input parameter file.
    !# <inputParameter>
    !#   <name>H_0</name>
    !#   <variable>simpleDefaultConstructor%HubbleConstantValue</variable>
    !#   <defaultValue>69.7d0</defaultValue>
    !#   <defaultSource>(\citealt{hinshaw_nine-year_2012}; CMB$+H_0+$BAO)</defaultSource>
    !#   <description>The present day value of the Hubble parameter in units of km/s/Mpc.</description>
    !#   <type>real</type>
    !#   <cardinality>0..1</cardinality>
    !# </inputParameter>
    return
  end function simpleDefaultConstructor

  function simpleConstructor(HubbleConstant)
    implicit none
    type            (cosmologyParametersSimple)                :: simpleConstructor
    double precision                           , intent(in   ) :: HubbleConstant

    ! Construct an instance of this class using a value of the Hubble constant provided directly.
    simpleConstructor%HubbleConstantValue=HubbleConstant
    return
  end function simpleConstructor

  elemental subroutine simpleDestructor(self)
    implicit none
    type(cosmologyParametersSimple), intent(inout) :: self

    ! Do any clean-up required by this class when an instance goes out-of-scope.
    return
  end subroutine simpleDestructor

  double precision function simpleHubbleConstant(self,units)
    implicit none
    class  (cosmologyParametersSimple), intent(inout)           :: self
    integer                           , intent(in   ), optional :: units

    ! Do whatever is necessary to return the Hubble constant in the appropriate units.
    return
  end function simpleHubbleConstant
\end{lstlisting}

In the above example, we define a ``simple'' implementation of the {\normalfont cosmologyParameters} class. Key points are:
\begin{description}
\item[Name:] The name should always be prefixed with the function class name. In this case, we have a {\normalfont \ttfamily simple} implementation of the {\normalfont \ttfamily cosmologyParameters} function class, and so our name is {\normalfont \ttfamily cosmologyParametersSimple}.
\item[Extends:] The base class for the function class is always the function class name suffixed with {\normalfont \ttfamily Class}, in this case {\normalfont \ttfamily cosmologyParametersClass}. Implementations must always be extensions of either this base class, or of another implementation.
\item[Procedures:] The implementation must define procedures for all methods of the function class, \emph{except} for where a method specified a {\normalfont \ttfamily code} element in the function class directive (in which case a procedure for the method may still be optionally defined). Specification of a {\normalfont \ttfamily final} function is encouraged.
\item[Constructors:] The implementation must specify at least one constructor, which takes no arguments (usually known as the default constructor). This constructor must create an instance of the implementation, setting any parameters from the input parameter file as necessary. Additional constructors may be defined as required.
\item[Procedures:] All required procedures (including constructors and destructors) should be given after a line containing the {\normalfont \ttfamily contains} keyword. \glc\ coding policy is that all procedures associated with an implementation should be prefixed with the implementation name, {\normalfont \ttfamily simple} in this case.
\end{description}

Functionality to store and restore the state (see \href{https://github.com/galacticusorg/galacticus/releases/download/bleeding-edge/Galacticus_Usage.pdf\#sec.Restarting}{here}) of classes built via a {\normalfont \ttfamily functionClass} directive are automatically built. If variables of a given implementation should be restored to a specific state, this can be specified by adding a {\normalfont \ttfamily restoreTo} element to the directive declaring the implementation. The {\normalfont \ttfamily restoreTo} element should specify a comma-separated list of one or more variables to set in its {\normalfont \ttfamily variables} attribute, and the state to which they should be restored in its {\normalfont \ttfamily state} attribute. Any variables which should be excluded from state store/restore (e.g. if their values are known to be determined statically at construction) can be specified via a {\normalfont \ttfamily exclude} element---a list of variables to exclude should be given as a comma-separated list in its {\normalfont \ttfamily variables} attribute.

\subsection{Generic Programming}

The preprocessor supports generic programming by allowing generic types to be defined, which are automatically expanded to a set of specific types. Significant flexibility is provided to allow control over how each specific type is handled. A generic type is specific via a {\normalfont \ttfamily generic} directive such as:
\lstset{escapechar=@}
\begin{lstlisting}
  !# <generic identifier="Type">
  !#  <instance label="Logical"        intrinsic="logical"                         outputConverter="regEx@\textbrokenbar@(.*)@\textbrokenbar@char($1)@\textbrokenbar@"/>
  !#  <instance label="Integer"        intrinsic="integer"                         outputConverter="regEx@\textbrokenbar@(.*)@\textbrokenbar@$1@\textbrokenbar@"      />
  !#  <instance label="Double"         intrinsic="double precision"                outputConverter="regEx@\textbrokenbar@(.*)@\textbrokenbar@$1@\textbrokenbar@"      />
  !#  <instance label="LogicalRank1"   intrinsic="logical          , dimension(:)" outputConverter="regEx@\textbrokenbar@(.*)@\textbrokenbar@char($1)@\textbrokenbar@"/>
  !#  <instance label="IntegerRank1"   intrinsic="integer          , dimension(:)" outputConverter="regEx@\textbrokenbar@(.*)@\textbrokenbar@$1@\textbrokenbar@"      />
  !#  <instance label="DoubleRank1"    intrinsic="double precision , dimension(:)" outputConverter="regEx@\textbrokenbar@(.*)@\textbrokenbar@$1@\textbrokenbar@"      />
  !# </generic>
\end{lstlisting}
The above defines a generic type, which will be identified using the label ``{\normalfont \ttfamily Type}''. The directive contains several {\normalfont \ttfamily instance} elements, each of which specifies a specific type which should be implemented for the generic type. Each instance can contain an arbitrary number of attributes which specify strings or regular expressions which will be used to construct the specific implementation.

A generic directive applies to the entire unit within which it is scoped. The preprocessor will examine every element within that unit. If a generic tag (see below) is found in the opening of any subunit, that entire subunit is copied once for each instance, and any generic tags replaced with the appropriate content from the {\normalfont \ttfamily instance} element. Where a generic tag is found in a non-opening line (and that line is not contained within a subunit whose opener \emph{does} contain a generic tag), the line itself is replicated in the same way.

An example of the usage of generic tags using the above generic directive is:
\lstset{escapechar=@}
\begin{lstlisting}

  type :: exampleType
     private
     .
     .
     .
   contains
     final     ::        exampleTypeDestroy
     procedure ::        exampleTypeSet{Type@\textbrokenbar@label}
     generic   :: set => exampleTypeSet{Type@\textbrokenbar@label}
  end type exampleType

contains

  subroutine exampleTypeSet{Type@\textbrokenbar@label}(self,setValue)
    implicit none
    class           (exampleType), intent(in   ) :: self
    {Type@\textbrokenbar@intrinsic}             , intent(in   ) :: setValue

    {Type@\textbrokenbar@match@\textbrokenbar@^Logical@\textbrokenbar@! Do something to set a logical value.@\textbrokenbar@! Do something different to set a numerical value.@\textbrokenbar@}
    write (0,*) "Value is: ",{Type@\textbrokenbar@outputConverter@\textbrokenbar@setValue}
  end subroutine exampleTypeSet{Type@\textbrokenbar@label}
\end{lstlisting}

In this example, the {\normalfont \ttfamily exampleType} class is defined to have a generic {\normalfont \ttfamily set} method. The presence of the {\normalfont \ttfamily \{Type@\textbrokenbar@label\}} generic tag will cause those lines to be replicated with the tag replaced by the content of the {\normalfont \ttfamily label} attribute of each instance of the generic type. In the contained subroutine, a generic tag appears in the opener. As such, the entire subroutine will be replicated once for each instance of the generic type, and the generic tags replaced as appropriate.

When a generic instance attribute begins with {\normalfont \ttfamily regEx}, matching generic tags are handled differently. In particular, a match-and-replace regular expression is applied to the third element of the generic tag (elements are separated by broken vertical bar characters). The match and replace components of the regular expression are defined in the instance attribute, once again separated by broken vertical bars.

Finally, the special generic tag {\normalfont \ttfamily match} acts as a ternary operator. If the regular expression specified in the third element matches the {\normalfont \ttfamily label} attribute of a specific instance, the generic tag is replaced with its fourth element, otherwise it is replaced with its fifth element.

\section{Numerical Tools}

\glc\ provides a variety of tools to solve basic numerical problems. These can be found in files {\normalfont \ttfamily source/numerical.*}. \glc\ makes use of the \href{http://www.gnu.org/software/gsl/}{GNU Scientific Library} for many of these tools, but typically provides a higher-level wrapper around those functions, providing a cleaner interface and, in some cases, additional functionality.

\subsection{Finding Roots of Equations}\index{root finding}\index{numerical algorithms!root finding}

Tools for solving equations of the form $f(x)=0$ are provided by the {\normalfont \ttfamily rootFinder} object (available via the {\normalfont \ttfamily Root\_Finder} module). Typical use of this object is as follows:
\begin{lstlisting}
! Import the module.
use Root_Finder
...
! Create a rootFinder object - make it OpenMP threadprivate so it can be used
! simultaneously by all threads.
type(rootFinder), save :: finder
!$omp threadprivate(finder)                                                                                                      
...
! Check if our root finder has been initialized.
if (.not.finder%isInitialized()) then
  ! Specify the function that evaluates f(x).
  call finder%rootFunction   (myRootFunction                     )
  ! Specify the type of root-finding algorithm - this is optional (Brent's
  ! method will be used by default).
  call finder%type           (GSL_Root_fSolver_Brent             )
  ! Specify the tolerances to use in finding the root. Both arguments are
  !optional - values of 1.0d-10 will be used for both absolute and relative
  ! tolerance by default.
  call finder%tolerance      (toleranceAbsolute,toleranceRelative)
  ! Specify how the initially provided range can be expanded to bracket the
  ! root. This is optional - if not provided no range expansion will be attempted.
  call finder%rangeExpand                                               & 
       &  (                                                             &
       &   rangeExpandDownward          =0.5d0                        , &
       &   rangeExpandUpward            =2.0d0                        , &
       &   rangeExpandType              =rangeExpandMultiplicative    , &
       &   rangeDownwardLimit           =1.0d-3                       , &
       &   rangeUpwardLimit             =1.0d+3                       , &
       &   rangeExpandDownwardSignExpect=rangeExpandSignExpectPositive, &
       &   rangeExpandUpwardSignExpect  =rangeExpandSignExpectNegative
       &  )
end if
x=finder%find(rootGuess=1.0d0)
.
.
.
double precision function myRootFunction(x)
  implicit none
  double precision, intent(in   ) :: x
  ...
  return
end function myRootFunction
\end{lstlisting}
The above example begins by importing the {\normalfont \ttfamily Root\_Finder} module and then creating a {\normalfont \ttfamily rootFinder} object called {\normalfont \ttfamily finder}. This is made OpenMP {\normalfont \ttfamily threadprivate} so that it may be used simultaneously by all threads. The first step is to initialize {\normalfont \ttfamily finder}---the {\normalfont \ttfamily isInitialized} method tells us if this has already happened. The most important step is to specify the function that will evaluate $f(x)$. This is done via the {\normalfont \ttfamily rootFunction} method---once done, the {\normalfont \ttfamily rootFinder} object is marked as initialized (and the {\normalfont \ttfamily isInitialized} method will return {\normalfont \ttfamily true}). All other initialization steps are optional. In this example, we use the {\normalfont \ttfamily type} method to specify that the {\normalfont \ttfamily Brent} algorithm should be used for root finding. Any valid \href{http://www.gnu.org/software/gsl/manual/html_node/Root-Bracketing-Algorithms.html}{GSL-supported root finding algorithm} can be used. We then use the {\normalfont \ttfamily tolerance} method to specify both the absolute and relative tolerances in the $x$ variable that must be attained to declare the root to be found. Both arguments are optional---default values of $10^{-10}$ will be used if either tolerance is not specified. 

The final step of initialization is to call the {\normalfont \ttfamily rangeExpand} method. This specifies how the initial guessed value or range for $x$ should be expanded to bracket the root. If you plan to always specify an initial range, and know that it will always bracket the root, you do not need to specify how the range should be expanded. In this case we've specified that range expansion is multiplicative---that is, the lower and upper values of $x$ defining the range will be multiplied by fixed factors until the root is bracketed---via the {\normalfont \ttfamily rangeExpandType=rangeExpandMultiplicative} option. Alternatively, additive expansion is possible using {\normalfont \ttfamily rangeExpandType=rangeExpandAdditive}. The factors by which to multiply the lower and upper bounds of the range (or the factor to add in the case of additive expansion) are specified by the {\normalfont \ttfamily rangeExpandDownward} and {\normalfont \ttfamily rangeExpandUpward} options. It is possible to specify absolute lower/upper limits to the range via the {\normalfont \ttfamily rangeDownwardLimit} and {\normalfont \ttfamily rangeUpwardLimit} options. The range will not be expanded beyond these limits---if the root cannot be bracketed without exceeding these limits an error condition will occur. Finally, it is possible to indicate the expected sign of $f(x)$ at the lower and/or upper limits via the {\normalfont \ttfamily rangeExpandDownwardSignExpect} and {\normalfont \ttfamily rangeExpandUpwardSignExpect} options. Valid settings are {\normalfont \ttfamily rangeExpandSignExpectNegative}, {\normalfont \ttfamily rangeExpandSignExpectPositive}, and {\normalfont \ttfamily rangeExpandSignExpectNone} (the default---implying that there is no expectation for the sign). If the sign of $f(x)$ is specified, then range expansion will stop once the expected sign is found. This can often improve efficiency, by allowing the range expander to expand the range in only one direction, resulting in a narrower range in which to search for the root.

Finally, we use the {\normalfont \ttfamily find} method to return the value of the root. The first argument to {\normalfont \ttfamily find} is the name of the function that evaluates $f(x)$. Additionally, we must supply either {\normalfont \ttfamily rootGuess} (a scalar value guess to use as the initial value for both the lower and upper values of the range---note that range expansion must be allowed in this case), or {\normalfont \ttfamily rootRange} (a two-element array to use as the initial lower and upper values of the range bracketing the root).

The function evaluating $f(x)$ must have a form compatible with that shown for {\normalfont \ttfamily myRootFunction} in the above example.

\section{Computation Dependencies and Data Files}\label{sec:codeUniqueLabels}\index{dependencies, computation}\index{labels, unique}\index{unique labels}

In many situations, some module in \glc\ might want to perform a calculation and then store the results to a file so that they can be reused later. A good example is the \refPhysics{transferFunctionCAMB} transfer function model, which computes a transfer function using {\normalfont \scshape CAMB} and stores this function in a file so that it can be re-read next time, avoiding the need to recompute the transfer function. A problem arises in such cases as the calculation may depend on the values of parameters (in our example, the transfer function will depend on cosmological parameters for example). We would like to record which parameter values this calculation refers to, perhaps encoding these into the file name, so that we can reuse these data in a future run only if the parameter values are unchanged. Given the modular nature of \glc\ it is impossible to know in advance which parameters will be relevant (e.g. does the cosmological parameter implementation have a parameter that describes a time varying equation of state for dark energy?). 

To address this problem, \glc\ provides a mechanism to generate a unique descriptor for a given object. This descriptor encodes the parameter used to construct the object, and recursively includes the parameters used to construct any other object which is composited. A long-form (human readable) descriptor is returned by the {\normalfont \ttfamily descriptor} method associated with all {\normalfont \ttfamily functionClass} objects. Additionally, the {\normalfont \ttfamily hashedDescriptor} method will return an MD5 hash of the descriptor, which will be unique (up to collisions) and can be used to identify the object both internally and, for example, when used as a suffix to file names. If the optional {\normalfont \ttfamily includeSourceDigest} argument is set to true in the {\normalfont \ttfamily hashedDescriptor} method then the hashed descriptor will include a hash of the source code of the object (and all composited objects) such that the descriptor will change should the source code be changed.

\section{Optimization}\label{sec:Optimization}\index{optimization}

In designing \glc, we opted for simplicity and clarity over speed. However, there are numerous parts of the code where optimization has been performed without a significant loss of clarity. In this section we discuss some of the techniques used.

\subsection{Unique IDs and Stored Properties}\index{unique ID}

Frequently, a given property of a node may be required in many different aspects of the calculation. For example, the dark matter halo virial radius is used extensively in several distinct calculations within \glc. Frequently such calculations are performed for the same node, with the same properties several times\footnote{For example, \glc's ODE solver will fix the properties of a node and then request that derivatives of all properties be computed. Some functions will then be called multiple times for the same node with unchanged properties.}. Obviously this is inefficient. It can be advantageous in such cases to store the result of a calculation and, if the function is called again with the same unchanged node to simply return the stored value. \glc\ facilitates this by two features.

The first feature is the ``unique ID''---an integer number assigned to each node in \glc\ and which uniquely identifies a node (i.e. no two nodes processed in a \glc\ run will have the same unique ID). This number, which can be retrieved using the {\normalfont \ttfamily uniqueID} property of a tree node, can be recorded each time a function is called. If called again for a node with the same unique ID as the previous call, the function can simply return the same answer as on the previous call.

The second feature accounts for the fact that the properties of a node will change, so even if a function is called on a node with the same unique ID it may occasionally need to recompute its result. \glc\ provides a calculation reset task\index{calculation reset task}\index{task, calculation reset} (see \S\ref{sec:CalculationResetTask}). All such tasks are performed just prior to the computation of derivatives for a node being evolved. A function can register a calculation reset task and use it to flag that it must update its calculations even if called again with the same node.

\section{Global Functions}

In very exceptional circumstances it is necessary to subvert the module hierarchy used by \glc\ to permit one module to call a function in a higher level module\footnote{This usually arises because circular dependencies would arise if the called function were placed in a lower level module.}. Examples of where this approach is necessary usually involve initial bootstrapping (i.e. to establish halo density contrasts, which requires knowledge of the halo density profile, which in turn requires knowledge of the halo density contrast\ldots).

\glc\ provides for global functions which facilitate this---specifically, it is possible to generate function pointers to higher level functions which are accessible via a very low-level module. To create a globally callable copy of a function use add the follow prior to the function definition:
\begin{verbatim}
  !# <functionGlobal>
  !#  <unitName>myFunction</unitName>
  !#  <type>double precision</type>
  !#  <arguments>double precision , intent(in   ) :: mass, time</arguments>
  !# </functionGlobal>
\end{verbatim}
Here, {\normalfont \ttfamily myFunction} is the name of the function to make global, while the {\normalfont \ttfamily type} and {\normalfont \ttfamily arguments} (of which there may be more than one) elements are used to generate a suitable interface for the function. At run-time, a pointer to this function is then available from the {\normalfont \ttfamily Functions\_Global} module, named {\normalfont \ttfamily myFunction\_}. Note that {\normalfont \ttfamily Functions\_Global\_Set} provided by the {\normalfont \ttfamily Functions\_Global\_Utilities} module must be called once to initialize these global function pointers prior to their use.

\section{\glc\ Metadata}

\glc\ can collect metadata on its own activity. This is useful for profiling the code for example.

\subsection{OpenMP Critical Section Wait Times}

\glc\ makes use of \gls{openmp} for parallel operation. \gls{openmp} {\normalfont \ttfamily critical} sections are used throughout to limit access to parts of the code which must be executed in serial. Threads will block at each {\normalfont \ttfamily critical} section if another thread is currently within it. \glc\ can profile the amount of time spent waiting at each named \gls{openmp} {\normalfont \ttfamily critical} section across all threads. To enable this profiling \glc\ should be compiled with {\normalfont \ttfamily -DOMPPROFILE} added to the compile options (in {\normalfont \ttfamily GALACTICUS\_FCFLAGS}). This will cause profiling instructions to be added to the source code prior to compilation. When this instrumented executable is run an {\normalfont \ttfamily metaData/openMP/} group will be added to the output file. This group contains two datasets, {\normalfont \ttfamily criticalSectionNames} and {\normalfont \ttfamily criticalSectionWaitTimes}. The first lists the names of all named \gls{openmp} {\normalfont \ttfamily critical} sections, while the second lists the total number of seconds (across all threads) spent waiting at each section. A script is provided to analyze this metadata:
\begin{verbatim}
./scripts/aux/openMPCriticalWaitProfile.pl <modelFileName>
\end{verbatim}
This script will analyze the \gls{openmp} {\normalfont \ttfamily critical} section wait time metadata in the named file, reporting the total time spent waiting at critical sections followed by a rank ordered list of the top ten sections by wait time. This can be useful for assessing whether optimization might help to reduce \gls{openmp} {\normalfont \ttfamily critical} section wait times.

\section{Enumerations}

Enumerations are used to communicate options to many functions in \glc. All available enumerations, along with their members, are described below.

\input{autoEnumerationDefinitions}

\section{Defined Constants}

\glc\ defines numerous constants, including mathematical constants (e.g. $\pi$), physical constants (e.g. the speed of light), unit conversions (e.g. Angstroms to meters), and prefixes (e.g. ``kilo'', ``mega'', etc.). These should be used whenever a constant is needed in the code---it is bad practice to use the numerical value of a constant directly in the code\footnote{Both because it is prone to mistakes (the more times a numerical value is used directly, the more chances there are for typos and other errors), and because using a named constant makes it much easier to understand \emph{what} the code is doing and \emph{why}.}.

All defined constants are described below, along with references to their source, and the name of the module in which the constant is defined. To import a constant into a function, you would add a {\normalfont \ttfamily use} statement. For example, for the constant $\pi$, you would use:
\begin{verbatim}
   use :: Numerical_Constants_Math, only : Pi
\end{verbatim}
after which you can use this constant, e.g.:
\begin{verbatim}
   volumeSphere=4.0d0*Pi/3.0d0*radiusSphere**3
\end{verbatim}

\subsection{Indicating Units of Defined Constants}

The unit system in which a constant is defined should be indicated by a suffix starting with an underscore. The default, which requires no suffix, is the SI system. So, for example, the defined constant {\normalfont \ttfamily massSolar} (which has no suffix) will be in SI units of kilograms. The defined constant {\normalfont \ttfamily lymanSeriesLimitWavelengthHydrogen\_atomic} is in the ``atomic'' unit system, so will be in units of Angstroms.

Currently defined unit systems are:
\begin{description}
\item[no suffix:] With no suffix, the defined constant is in the \href{https://en.wikipedia.org/wiki/International_System_of_Units}{SI} unit system.
\item[{\normalfont \ttfamily \_internal}:] The ``internal'' suffix indicates a defined constant is in \glc's internal unit system of $\mathrm{M}_\odot$, Mpc, Gyr, and km/s.
\item[{\normalfont \ttfamily \_atomic}:] The ``atomic'' suffix indicates atomic units (\AA).
\item[{\normalfont \ttfamily \_cgs}:] The ``cgs'' suffix indicates a defined constant is in the \href{https://en.wikipedia.org/wiki/Centimetre%E2%80%93gram%E2%80%93second_system_of_units}{CGS} unit system.
\end{description}

\subsection{Available Defined Constants}\label{sec:definedConstants}

\input{constants}

\section{Classes}
 
The return type of each method, and the interfaces (i.e. the types and names of its arguments) are specified for each method of each object. A ``{\normalfont \ttfamily void}'' return type indicates a subroutine.

\subsection{Methods available to all {\normalfont \ttfamily functionClass}es}\label{sec:functionClassAll}

Although not methods directly bound to the {\normalfont \ttfamily functionClass} class, the following methods will be automatically created for all child classes of {\normalfont \ttfamily functionClass} when they are built by \glc.


\begin{description}
  
\item[]{\normalfont \ttfamily autoHook} Insert any event hooks required by this object. 
\begin{itemize}
\item Return type: {\normalfont \ttfamily void}
\item Interface: {\normalfont \ttfamily ()}\\
\end{itemize}

\item[]{\normalfont \ttfamily deepCopy}  Perform a deep copy of the object. Here, {\normalfont \ttfamily \textless functionClassBase \textgreater} refers to the base class of the created {\normalfont \ttfamily functionClass} child.
\begin{itemize}
\item Return type: {\normalfont \ttfamily void}
\item Interface: {\normalfont \ttfamily (destination)}\\
  {\normalfont \ttfamily class(\textless functionClassBase\textgreater), intent(inout) :: destination}\\
\end{itemize}

\item[]{\normalfont \ttfamily deepCopyReset} Reset deep copy pointers in this object and any objects that it uses.
\begin{itemize}
\item Return type: {\normalfont \ttfamily void}
\item Interface: {\normalfont \ttfamily ()}\\
\end{itemize}

\item[]{\normalfont \ttfamily descriptor} Return an input parameter list descriptor which could be used to recreate this object.
\begin{itemize}
\item Return type: {\normalfont \ttfamily void}
\item Interface: {\normalfont \ttfamily (descriptor, includeClass)}\\
  {\normalfont \ttfamily type(inputParameters), intent(inout) :: descriptor}\\
  {\normalfont \ttfamily logical, intent(in   ), optional :: includeClass}\\
\end{itemize}

\item[]{\normalfont \ttfamily hashedDescriptor} Return a hash of the descriptor for this object, optionally include the source code digest in the hash.
\begin{itemize}
\item Return type: {\normalfont \ttfamily type(varying\_string)}
\item Interface: {\normalfont \ttfamily (descriptor, includeClass)}\\
  {\normalfont \ttfamily logical, intent(in   ), optional :: includeSourceDigest}\\
\end{itemize}

\item[]{\normalfont \ttfamily objectType} Return the type of the object. 
\begin{itemize}
\item Return type: {\normalfont \ttfamily type(varying\_string)}
\item Interface: {\normalfont \ttfamily ()}\\
\end{itemize}

\item[]{\normalfont \ttfamily stateRestore} Restore the state of this object from file.
\begin{itemize}
\item Return type: {\normalfont \ttfamily void}
\item Interface: {\normalfont \ttfamily (stateFile, gslFile, stateOperationID)}\\
  {\normalfont \ttfamily integer, intent(in   ) :: stateFile}\\
  {\normalfont \ttfamily type(c\_ptr), intent(in   ) :: gslStateFile}\\
  {\normalfont \ttfamily integer(c\_size\_t), intent(in   ) :: stateOperationID}\\
\end{itemize}

\item[]{\normalfont \ttfamily stateStore} Store the state of this object to file.
\begin{itemize}
\item Return type: {\normalfont \ttfamily void}
\item Interface: {\normalfont \ttfamily (stateFile, gslFile, stateOperationID)}\\
  {\normalfont \ttfamily integer, intent(in   ) :: stateFile}\\
  {\normalfont \ttfamily type(c\_ptr), intent(in   ) :: gslStateFile}\\
  {\normalfont \ttfamily integer(c\_size\_t), intent(in   ) :: stateOperationID}\\
\end{itemize}

\end{description}

\input{dataMethods}


\chapter{Adding New Methods}

\section{Code Directives}\label{sec:CodeDirectives}\index{directives}\index{code!directives}

\glc\ is designed to be flexible and extensible, allowing you to add new methods and functionality without having to hack the code extensively. To achieve this it makes much use of embedded code directives which, for example, explain to the build system how a particular subroutine or function connects into the \glc\ code. Such code directives are indicated by lines beginning with {\tt !\#}, and take the form of short blocks of XML. For example, a typical code directive might look like:
\begin{verbatim}
 !# <accretionDisksMethod>
 !#  <unitName>Accretion_Disks_Shakura_Sunyaev_Initialize</unitName>
 !# </accretionDisksMethod>
\end{verbatim}
This directive would typically appear just prior to a subroutine which initializes the Shakura-Sunyaev accretion disk module (it could appear anywhere throughout that module, but it makes sense to keep it close to the subroutine that it references). The {\tt accretionDisksMethod} tag explains to the \glc\ build system that this module contains an implementation of black hole accretion disks. The {\tt unitName} tag specifies the name of a program unit which (in this case) should be called to initialize this accretion disk implementation. The build system will then insert appropriate {\tt use} and {\tt call} statements into the \glc\ code such that this routine will be called if and when accretion disks are required by \glc.

\section{Identifying Components and Mass Types}\label{sec:ComponentMassTypes}

Many functions can be applied to different components or groups of components and to different types of mass within a node. In general, these functions make use of a set of label defined in the \hyperlink{galactic_structure.options.F90:galactic_structure_options}{{\tt Galactic\_Structure\_Options}} module. Components are identified by a {\tt componentType} label which can take on the following values:
\begin{description}
 \item [{\tt componentTypeAll}] All components are matched;
 \item [{\tt componentTypeDisk}] Only disk components are matched;
 \item [{\tt componentTypeSpheroid}] Only spheroid components are matched.
 \item [{\tt componentTypeBlackHole}] Only black hole components are matched.
 \item [{\tt componentTypeHotHalo}] Only hot halo components are matched.
 \item [{\tt componentTypeDarkHalo}] Only dark matter halo components are matched.
\end{description}
Types of mass are identified by a {\tt massType} which can take one of the following values:
\begin{description}
 \item [{\tt massTypeAll}] All mass is included;
 \item [{\tt massTypeDark}] Only dark matter is included;
 \item [{\tt massTypeBaryonic}] Only baryonic mass is included;
 \item [{\tt massTypeGalactic}] Only galactic mass is included.
 \item [{\tt massTypeGaseous}] Only gaseous mass is included.
 \item [{\tt massTypeStellar}] Only stellar mass is included.
 \item [{\tt massTypeBlackHole}] Only black hole mass is included.
\end{description}

\section{Components}\index{components}

This section describes the internal structure of node components, and how a component is implemented.

\subsection{Component Structure}\index{components!structure}

Each node in the merger tree consists of an arbitrary number of ``components'', each of which can actually be an array, allowing multiple components of a given class. Each component represents a specific class of object, which could be a dark matter halo, a galactic disk or a black hole etc. A component of each class may be of one or more differnt implementations of that component class. Component classes are extensions of the {\tt nodeComponent} base class, while each implementation is an extension of its component class (or, sometimes, of another implementation of that same class). Each component implementation type consists of a set of data\footnote{Data objects in components can be real, integer, boolean or of derived type. For derived types, currently {\tt history}, {\tt abundances}, {\tt chemicals}, and {\tt keplerOrbit} are supported. Adding additional derived types is possible, providing that the type supports the required methods for output, serialization, etc. Data objects can currently be scalar or rank-1 arrays.}, representing the properties (mass, size etc.) of the component, along with the rates of change (and ODE solver tolerances) for any properties which are evolvable. Additionally, each component contains a large number of methods (functions) which can be used to access its properties, query its interfaces and which are used internally to perform ODE evolution, output etc. The {\tt nodeComponent} base class and all classes derived from it are built automatically by {\tt Galacticus::Build::Components} at compile time (take a look in {\tt work/build/objects.nodes.components.Inc} if you want to see the generated code).

\subsection{Extending Components}

It is possible to create a component which extends an existing component (see the discussion of the {\tt extends} element in \S\ref{sec:ComponentDefinition}). This capability is intended to allow new properties to be added to a component without having to create a whole new copy of the component. It is \emph{not} intended to allow changes in the way in which the component is evolved through the halo hierarchy. (With the exception that rules to describe how the newly added properties will evolve through the halo hierarchy can be added of course.)

A simple example of this extension capability can be found in the {\tt scaleShape} dark matter profile component (\S\ref{sec:DarkMatterProfileScaleShape}), which extends the {\tt scale} dark matter profile component (\S\ref{sec:DarkMatterProfileScale}). In this case, the {\tt scaleShape} component adds a new property, {\tt shape}, and specifies how it is to be initialized, evolved, output, and change by node promotion events. It \emph{does not} affect how the {\tt scale} property, inherited from the {\tt scale} dark matter profile component, is evolved.

\subsection{Implementing a New Component}\label{sec:ComponentImplement}\index{components!implementing}

Implementing a new component involves writing some modules and functions which contain a definition of the component and, if necessary, handle initialization, creation, evolution, and responses to any events. Frequently, the easiest way to make a new component is to copy a previously existing one and modify it as needed. Details of the various functions that component modules must perform are given below.

By convention, a component's implementation is split into three or four files, although some components might not need all of these files. These files are named as follows (with {\tt \textless component\textgreater} acting as a placeholder for the name of the component in question):
\begin{description}
 \item [{\tt objects.nodes.components.\textless component\textgreater.F90}] The primary file which describes the component and its properties, and which contains functions that manipulate the component as it evolves through a merger tree (ODE rates, behavior during mergers, etc.);
 \item [{\tt objects.nodes.components.\textless component\textgreater.bound\_functions.F90}] Contains functions which will be bound to the component object (i.e. the {\tt nodeComponent\textless Class\textgreater\textless Implementation\textgreater} class), and so will be available as type bound procedures. Generally, these functions will include any which get or set values of properties in the component, those which return information about its internal state (such as the density at some position in the component; see \S\ref{sec:GalacticComponentDensity}), and any other functions which we may want to be overridden by extensions to the component.
 \item [{\tt objects.nodes.components.\textless component\textgreater.data.F90}] Contains any data which may need to be shared between the above two files. This might contain parameters which control some property of the component that is the same for all instances (e.g. if spheroids are modelled as S\'ersic profiles all with the same value of the S\'ersic index, that value might be placed into this file).
 \item [{\tt objects.nodes.components.\textless component\textgreater.structure.F90}] Contains any functions which implement the structure (e.g. density, rotation curve) of the component and which cannot be placed in {\tt objects.nodes.components.\textless component\textgreater.bound\_functions.Inc} due to dependencies on modules which in turn depend on the {\tt Galacticus\_Nodes} module.
\end{description}
In general, {\tt objects.nodes.components.\textless component\textgreater.F90} is the place for the component definition and functions which process the component during tree evolution (including output), while {\tt objects.nodes.components.\textless component\textgreater.bound\_functions.Inc} is intended for functions which record or report the internal state of the component.

\subsubsection{Component Definition}\label{sec:ComponentDefinition}

Component definition itself takes the form of an embedded XML document. The following example illustrates such a document:
\begin{verbatim}
  !# <component>
  !#  <class>disk</class>
  !#  <name>exponential</name>
  !#  <isDefault>yes</isDefault>
  !#  <properties>
  !#   <property>
  !#     <name>isInitialized</name>
  !#     <type>logical</type>
  !#     <rank>0</rank>
  !#     <attributes isSettable="true" isGettable="true" isEvolvable="false" />
  !#   </property>
  !#   <property>
  !#     <name>massStellar</name>
  !#     <type>real</type>
  !#     <rank>0</rank>
  !#     <attributes isSettable="true" isGettable="true" isEvolvable="true" />
  !#     <output unitsInSI="massSolar" comment="Mass of stars in the exponential disk."/>
  !#   </property>
  !#   <property>
  !#     <name>abundancesStellar</name>
  !#     <type>abundances</type>
  !#     <rank>0</rank>
  !#     <attributes isSettable="true" isGettable="true" isEvolvable="true" />
  !#     <output unitsInSI="massSolar" comment="Mass of metals in the stellar phase of the exponential disk."/>
  !#   </property>
  !#   <property>
  !#     <name>massGas</name>
  !#     <type>real</type>
  !#     <rank>0</rank>
  !#     <attributes isSettable="true" isGettable="true" isEvolvable="true" createIfNeeded="true" makeGeneric="true" />
  !#     <output unitsInSI="massSolar" comment="Mass of gas in the exponential disk."/>
  !#   </property>
  !#   <property>
  !#     <name>coolingMass</name>
  !#     <attributes isSettable="false" isGettable="false" isEvolvable="true" isDeferred="rate" bindsTo="top" />
  !#     <type>real</type>
  !#     <rank>0</rank>
  !#     <isVirtual>yes</isVirtual>
  !#   </property>
  !#   <property>
  !#     <name>halfMassRadius</name>
  !#     <attributes isSettable="false" isGettable="true" isEvolvable="false" />
  !#     <type>real</type>
  !#     <rank>0</rank>
  !#     <isVirtual>yes</isVirtual>
  !#     <getFunction>Node_Component_Disk_Exponential_Half_Mass_Radius</getFunction>
  !#   </property>
  !#   <property>
  !#     <name>luminositiesStellar</name>
  !#     <type>real</type>
  !#     <rank>1</rank>
  !#     <attributes isSettable="true" isGettable="true" isEvolvable="true" />
  !#     <classDefault modules="Stellar_Population_Properties_Luminosities" count="Stellar_Population_Luminosities_Count()">0.0d0</classDefault>
  !#     <output labels="':'//Stellar_Population_Luminosities_Name({i})" count="Stellar_Population_Luminosities_Count()" condition="Stellar_Population_Luminosities_Output({i},time)" modules="Stellar_Population_Properties_Luminosities" unitsInSI="luminosityZeroPointAB" comment="Luminosity of disk stars."/>
  !#   </property>
  !#  </properties>
  !#  <bindings>
  !#   <binding method="attachPipes" function="Node_Component_Disk_Exponential_Attach_Pipes" type="void" bindsTo="component" />
  !#  </bindings>
  !#  <functions>objects.nodes.components.disk.exponential.custom_methods.inc</functions>
  !# </component>
\end{verbatim}

The elements of this document have the following meaning:
\begin{description}
\item [{\tt class}] \emph{[Required]} Specifies the component class of which this is an implementation.
\item [{\tt name}] \emph{[Required]} Specifies the name of this specific implementation.
\item [{\tt extends}] \emph{[Optional]} If present, this element must contain {\tt class} and {\tt name} elements which specify the type of component which should be extended. The component then automatically inherits all properties and type-bound functions of the extended type.
\item [{\tt isDefault}] \emph{[Required]} Specifies whether or not this should be the default implementation of this class. Note that only one implementation of each class can be declared to be the default. If no implementation of a given class is declared to be the default then the (automatically generated) {\tt null} implementation will be made the default.
\item [{\tt properties}] \emph{[Optional]} Contains an array of {\tt property} elements which specify the properties of this implementation. Each member {\tt property} has the following structure:
\begin{description}
\item [{\tt name}] \emph{[Required]} The name of the property. 
\item [{\tt type}] \emph{[Required]} The type (one of {\tt real}, {\tt integer}, {\tt logical}, {\tt history}, {\tt abundances}, {\tt chemicals}, or {\tt keplerOrbit} at present) of the property.
\item [{\tt rank}] \emph{[Required]} The rank of this property (currently {\tt 0} for a scalar or {\tt 1} for a 1-D array).
\item [{\tt attributes}] \emph{[Required]} Attributes of this property:
\begin{description}
\item [{\tt isSettable}] If {\tt true} then the value of this property can be set directory.
\item [{\tt isGettable}] If {\tt true} then the value of this property can be got directory.
\item [{\tt isEvolvable}] If {\tt true} this property evolves as part of the \glc\ \gls{ode} system.
\item [{\tt createIfNeeded}] If {\tt true} then any attempt to get, set, or adjust the rate of this property will cause the component to be created if it does not already exist. This is useful if the component should be created in response to mass transfer from some other component for example.
\item [{\tt makeGeneric}] If {\tt true} then any {\tt rate} method for this property will have a version created which binds to the base {\tt nodeComponent} class. This version is suitable for attaching to deferred rate functions of components of another class. For example, the disk gas mass rate function is made generic, and then attached to the deferred cooling rate of the hot halo using:
\begin{verbatim}
  call hotHalo%hotHaloCoolingMassRateFunction(DiskExponentialMassGasRateGeneric)
\end{verbatim}
\item [{\tt isDeferred}] Contains a ``{\tt :}'' separated list which can contain {\tt get}, {\tt set}, and {\tt rate}. The methods present in this list will not have functions bound to them at compile time. Instead a function will be created which allows a function to be bound to these methods at run time. For example:
\begin{verbatim}
  call myComponent%massFunction    (My_Component_Mass_Get_Function)
  call myComponent%massSetFunction (My_Component_Mass_Set_Function)
  call myComponent%massRateFunction(My_Component_Mass_Rate_Function)
\end{verbatim}
Additionally, a method is created which returns true or false depending on whether the method has been attached to a function yet, e.g.
\begin{verbatim}
 myComponent%massIsAttached    ()
 myComponent%massSetIsAttached ()
 myComponent%massRateIsAttached()
\end{verbatim}
\item [{\tt bindsTo}] Specifies to which level in the class hierarchy set, get and rate methods should be bound. Normally, these are bound to the component implementation itself. However, it can be useful to specify a binding of ``{\tt top}'' to bind to the base {\tt nodeComponent} class to make these methods interoperable with properties of other classes (see the discussion of the {\tt makeGeneric} element above).
\end{description}
\item [{\tt output}] \emph{[Optional]} If present, the property will be included in the \glc\ output file. The following attributes control the details of that output:
\begin{description}
\item [{\tt unitsInSI}] The units of the output quantity in the SI system.
\item [{\tt comment}] A comment to be included with the HDF5 dataset for this property.
\item [{\tt condition}] A statement which must evaluate to {\tt true} or {\tt false} and which will be used to determine if the property will be output. The present output time for is available as {\tt time}. In the case of an array property the construct ``{\tt \{i\}}'' can be used to pass the index of the element for which the condition should be evaluated.
\item [{\tt modules}] A comma-separated list of any modules required to perform the output (e.g. modules which contain functions or values that are used).
\end{description}
Additional attributes are required for array properties:
\begin{description}
\item [{\tt labels}] This can be an array, declared as ``{\tt [$L_1$,\ldots,$L_N$]}'', specifying the suffix to be added to the property name for each component of the array in the output, or a function which returns the suffix. In the case of a function the construct ``{\tt \{i\}}'' can be used to pass the index of the element for which the suffix is required.
\item [{\tt count}] A statement which evaluates the the number of elements to be output (i.e. the length of the array).
\end{description}
\item [{\tt isVirtual}] \emph{[Optional]} If present and set to ``{\tt yes}'', this property is a virtual property. A virtual property has no data associated with it and must supply its own functions for getting, setting and adjusting its rate of change (if allowed by the property's attributes). Virtual properties are used for quantities which are derived from actual properties of the component implementation (for example, a star formation rate could be a virtual property if it is derived from an actual gas mass property) or for adjusting the rates of several actual properties simultaneously.
\item [{\tt getFunction}] \emph{[Optional]} Specifies the function to be used for getting the value of the property, overriding the default get function. The function must be included in the \hyperlink{objects.nodes.F90:galacticus_nodes}{\tt Galacticus\_Nodes} module by use of the {\tt functions} element described below. Note that this function, by virtue of its priveleged access to the itnernal structure of node components, can access the value of the data associated with the propery using:
\begin{verbatim}
myComponent%<property>Data%value
\end{verbatim}
\item [{\tt setFunction}] \emph{[Optional]} The same as {\tt getFunction} but defines a function to set the value of the property.
\item [{\tt classDefault}] \emph{[Optional]} Specifies the default value for this property if the component class has not been created (i.e. has no specific implementation yet). The content of this element gives the default value (which can be a scalar, an array, a function, etc.). Additional, optional attributes control the use of this element:
\begin{description}
 \item [{\tt modules}] Specifies a comma-separated list of modules which are required to set the default values (e.g. modules which contain the value or function to be used).
 \item [{\tt count}] For array properties whose size is not known at compile-time, it is possible to specify a function which will return the appropriate size of the array at run-time. The scalar default value given in the {\tt classDefault} element will then be replicated the appropriate number of times.
\end{description}
\end{description}
\item [{\tt bindings}] \emph{[Optional]} Contains an array of {\tt binding} elements which specify functions to bind to this implementation. Each member {\tt binding} has the following structure:
\begin{description}
\item [{\tt method}] The name of the bound method, such that the function can be accessed using
\begin{verbatim}
 myComponent%<method>(...)
\end{verbatim}
\item [{\tt function}] The function to which the method should be bound. (This function must be included in the \hyperlink{objects.nodes.F90:galacticus_nodes}{\tt Galacticus\_Nodes} module by use of the {\tt functions} element described below.
\item [{\tt type}] The type of function.
\item [{\tt bindsTo}] Specifies where this method should be bound. ``{\tt component}'' specifies binding to the specific implementation of this component class, ``{\tt componentClass}'' specifies binding to the component class, while ``{\tt top}'' specifies binding to the base {\tt nodeComponent} class.
\end{description}
\item [{\tt functions}] \emph{[Optional]} Contains the name of a file which will be included into the \hyperlink{objects.nodes.F90:galacticus_nodes}{\tt Galacticus\_Nodes} module. This file can contain functions which will be bound to this implementation. By virtue of being included in the \hyperlink{objects.nodes.F90:galacticus_nodes}{\tt Galacticus\_Nodes} module these functions have privelged access to the internal structure of all node component objects.
\end{description}

\subsubsection{Component Initialization}\index{components!initialization}

Initialization of a component module (if necessary, for example, to read parameters or allocate workspace) can occur at a number of different points in the execution of \glc. Providing initialization occurs in advance of any calculations then any point is acceptable. One possibility is simply to call an initialization function at the head of all functions defined in the component module. This initialization function should return immediately if it has already been called (to avoid duplicate initialization). Another option is to use the {\tt mergerTreePreTreeConstructionTask} event (see \S\ref{sec:MergerTreePreConstructionTask}) to perform initialization just before merger trees are constructed (the initialization function must again return immediately if it has been previously called).

Optionally, a component may include a {\tt mergerTreeEvolveThreadInitialize} directive, which gives the name of a subroutine in its {\tt unitName} element. The routine specified by {\tt mergerTreeEvolveThreadInitialize} is called by all threads prior to merger tree evolution, and can therefore be used to perform any ``per thread'' initialization. Note that this routine will be called many times during a given \glc\ run---it is the responsibility of the routine to ensure that it performs any initialization only once.

\subsubsection{Component Access, Creation and Destruction}\index{components!creation}\index{components!destruction}

When a node is created, it initially contains no components. A component must therefore create itself on the fly as needed. Typically, a component is first created when an attempt is made to set a property value, or to adjust the rate of change of a property value or in response to some event (e.g. a satellite component may be created in response to a node merging with a larger node). Requests for property values frequently \emph{do not} require that the component exist, as a zero value can often be returned instead\footnote{Or some other value if a {\tt classDefault} has been specified (see \S\ref{sec:ComponentImplement}).}.

To access a component from a node, use:
\begin{verbatim}
 myComponent => thisNode%<class>([instance=<N>,autoCreate=<create>])
\end{verbatim}
where {\tt class} is the component class required, the optional {\tt instance} argument requests a specific instance of the component (relevant if the node contains more than one of a particular component, e.g. if it contains two supermassive black holes for example; if no {\tt instance} is specified the first instance will be returned), and the {\tt autoCreate} option specifies whether or not the component should be automatically created (assuming it does not already exist). {\tt autoCreate}$=${\tt true} should be used to create components initially.

A component of a node can be destroyed using:
\begin{verbatim}
call thisNode%<class>Destroy()
\end{verbatim}

\subsubsection{Component Methods}\label{sec:ComponentMethods}\index{components!methods}

Component implementations optionally provide functions to get and set their properties (and to set the rate of change of evolvable properties) so that other components and functions within \glc\ to can interact with them in a way that is independent of the specific component implementation chosen. To permit this, \glc\ creates functions for each property to access it in all permitted ways\footnote{Additionally, C wrappers are generated to the get methods for real scalar properties. See \S\protect\ref{sec:MixedLanguageCoding} for a discussion of includeing C code within \protect\glc.}. For example, the {\tt exponential} implementation of the {\tt disk} component class has a ``{\tt massStellar}'' property defined by:
\begin{verbatim}
 <method>
   <name>massStellar</name>
   <type>real</type>
   <rank>0</rank>
   <attributes isSettable="true" isGettable="true" isEvolvable="true" />
 </method>
\end{verbatim}
This causes \glc\ to define several functions bound to the {\tt nodeComponentDisk} class:
\begin{description}
\item [{\tt massStellarIsSettable}] Returns {\tt true} if this property is settable;
\item [{\tt massStellarIsGettable}] Returns {\tt true} if this property is gettable;
\item [{\tt massStellarSet}] Sets the value of this property to the supplied argument;
\item [{\tt massStellarGet}] Gets the value of this property;
\item [{\tt massStellarRate}] Cumulates its argument to the rate of change of this property;
\item [{\tt massStellarScale}] Sets the absolute scale for this property used in ODE error control;
\end{description}
along with several others used internally for output, serialization etc.

\subsubsection{Component Evolution}\label{sec:ComponentEvolution}\index{evolution}\index{components!evolution}

All component properties which have an {\tt isEvolvable} attribute set to {\tt true} are included in \glc's ODE solver as the node is evolved forward in time. As described in \S\ref{sec:ComponentMethods}, \glc\ will create two functions that permit the rate of change of a property adjusted and for the absolute scale used in ODE error control to be set.

A ``rate compute'' function should be defined to perform any calculations necessary to determine the rate of change of the property and adjust the rate appropriately. Below is an example of the rate compute subroutine for the stellar mass property of the exponential disk component, with only the basic structure shown:
\begin{lstlisting}[escapechar=@,breaklines,prebreak=\&,postbreak=\&]
  !# <rateComputeTask>
  !#  <unitName>Node_Component_Disk_Exponential_Rate_Compute</unitName>
  !# </rateComputeTask>
  subroutine Node_Component_Disk_Exponential_Rate_Compute(thisNode,interrupt,interruptProcedure)
    implicit none
    type     (treeNode             ), pointer, intent(inout) :: thisNode
    logical                         ,          intent(inout) :: interrupt
    procedure(                     ), pointer, intent(inout) :: interruptProcedure
    class    (nodeComponentDisk    ), pointer                :: thisDiskComponent
 
    ! Get the disk and check that it is of our class.
    thisDiskComponent => thisNode%disk()
    select type (thisDiskComponent)
    class is (nodeComponentDiskExponential)
      ...
      call thisDiskComponent%massStellarRate(stellarMassRate)
      ...
    end select
    return
  end subroutine Node_Component_Disk_Exponential_Rate_Comput
\end{lstlisting}
Here, we get the disk component and check that it of the {\tt exponential} variety. If it is, we compute the rates of change for one or more properties and then adjust their rates appropriately. If multiple instances of a component are used then the rate compute function should loop over all instances and adjust rates appropriately.

When evolving ODEs the ODE solver aims to keep the error on property $i$ below
\begin{equation}
 D_i = \epsilon_{\rm abs} s_i + \epsilon_{\rm rel} |y_i|,
\end{equation}
where $epsilon_{\rm abs}=${\tt [odeToleranceAbsolute]}, $epsilon_{\rm rel}=${\tt [odeToleranceRelative]}, $y_i$ is the value of property $i$ and $s_i$ is a scaling factor which controls the absolute tolerance for this property. By default, $s_i=1$, but this can be changed for a component utilizing the {\tt scaleSetTask} directive. This allows a function to be called in which the component sets suitable scale factors for each of its properties prior to any ODE evolution being carried out. This can be very useful, for example, in cases where two components are coupled. Consider a case where a disk is transferring material to a spheroid via a bar instability. If the disk is orders of magnitude more massive that the spheroid then the rate of mass transfer can be very high (i.e. $\dot{y}/y$ for the spheroid will be large). With just a relative tolerance (i.e. the $\epsilon_{\rm rel} |y_i|$ term) this would require very short timesteps for the spheroid. However, in such cases we don't care about such tiny tolerances for the spheroid (since it will grow to be substantially more massive). Therefore, it may be appropriate to set $s_i$ to be equal to the sum of the disk and spheroid properties for example. The scale set directive and associated subroutine should follow this template:
\begin{verbatim}
  !# <scaleSetTask>
  !#  <unitName>Node_Component_Disk_Exponential_Scale_Set</unitName>
  !# </scaleSetTask>
  subroutine Node_Component_Disk_Exponential_Scale_Set(thisNode)
    implicit none
    type (treeNode         ), pointer, intent(inout) :: thisNode
    class(nodeComponentDisk), pointer                :: thisDiskComponent

    ! Get the disk component.
    thisDiskComponent => thisNode%disk()
    ! Check if an exponential disk component exists.
    select type (thisDiskComponent)
    class is (nodeComponentDiskExponential)
      ...
      call thisDiskComponent%massStellarScale(massScale)
      ...
    end select
    return
  end subroutine Node_Component_Disk_Exponential_Scale_Set
\end{verbatim}
Sensible choices for the $s_i$ factors can significantly speed-up execution of \glc.

\subsubsection{Evolution Interrupts}\index{interrupts}\index{evolution!interrupt}

It is often necessary to interrupt the smooth ODE evolution of a node in \glc. This can happen if, for example, a galaxy mergers with another galaxy (in which case the merger must be processed prior to further evolution) or if a component must be created before evolution can continue. The rate adjust and rate compute subroutines allow for interrupts to be flagged via their {\tt interrupt} and {\tt interruptProcedure} arguments. If an interrupt is required then {\tt interrupt} should be set to true, while {\tt interruptProcedure} should be set to point to a procedure which will handle the interrupt. Then, providing no other interrupt occurred earlier, the evolution will be stopped and the interrupt procedure called before evolution is continued.

An interrupt procedure should have the form:
\begin{lstlisting}[escapechar=@,breaklines,prebreak=\&,postbreak=\&]
  subroutine My_Interrupt_Procedure(thisNode)
    implicit none
    type(treeNode), pointer, intent(inout) :: thisNode
  
    ! Do whatever needs to be done to handle the interrupt.

    return
  end subroutine My_Interrupt_Procedure
\end{lstlisting}

\section{Existing Method Types}

\subsection{Functions}

Functions implement basic calculations (e.g. computing the power spectrum).

\IfFileExists{./autoMethods.tex}{\input{autoMethods}}{}

\subsubsection{Accretion Disks}\label{sec:AccretionDisks}

Additional methods for accretion disk properties can be added using the {\tt accretionDisksMethod} directive. The directive should contain a single argument, giving the name of a subroutine to be called to initialize the method. For example, the {\tt Shakura-Sunyae} method is described by a directive:
\begin{verbatim}
 !# <accretionDisksMethod>
 !#  <unitName>Accretion_Disks_Shakura_Sunyaev_Initialize</unitName>
 !# </accretionDisksMethod>
\end{verbatim}
Here, {\tt Accretion\_Disks\_Shakura\_Sunyaev\_Initialize} is the name of a subroutine which will be called to initialize the method. The initialization subroutine must have the following form:
\begin{verbatim}
  subroutine Method_Initialize(accretionDisksMethod,Accretion_Disk_Radiative_Efficiency_Get,Black_Hole_Spin_Up_Rate_Get,Accretion_Disk_Jet_Power_Get)
    implicit none
    type(varying_string),          intent(in)    :: accretionDisksMethod
    procedure(),          pointer, intent(inout) :: Accretion_Disk_Radiative_Efficiency_Get,Black_Hole_Spin_Up_Rate_Get,Accretion_Disk_Jet_Power_Get
    
    if (accretionDisksMethod == 'myMethod') then
       Accretion_Disk_Radiative_Efficiency_Get => My_Accretion_Disk_Radiative_Efficiency_Get
       Black_Hole_Spin_Up_Rate_Get             => My_Black_Hole_Spin_Up_Rate_Get
       Accretion_Disk_Jet_Power_Get            => My_Accretion_Disk_Jet_Power_Get
    end if
    return
  end subroutine Method_Initialize
\end{verbatim}
where {\tt myMethod} is the name of this method as will be specified by the {\tt accretionDisksMethod} input parameter. The procedure pointers {\tt Accretion\_Disk\_Radiative\_Efficiency\_Get}, {\tt Black\_Hole\_Spin\_Up\_Rate\_Get} and {\tt Accretion\_Disk\_Jet\_Power\_Get} must be set to point to functions which return the radiative efficiency, black hole spin up rate and jet power for the accretion disk respectively as described below. The initialization subroutine should perform any other tasks required to initialize the module (such as reading parameters etc.).

The radiative efficiency function must have the form:
\begin{verbatim}
 double precision function My_Accretion_Disk_Radiative_Efficiency_Get(thisNode,massAccretionRate)
    implicit none
    type(treeNode),   intent(inout), pointer :: thisNode
    double precision, intent(in)             :: massAccretionRate
    .
    .
    .
    return
 end function My_Accretion_Disk_Radiative_Efficiency_Get
\end{verbatim}
The function must return the radiative efficiency for the accretion disk in {\tt thisNode}. The black hole spin function must have the form:
\begin{verbatim}
 double precision function My_Black_Hole_Spin_Up_Rate_Get(thisNode,massAccretionRate)
    implicit none
    type(treeNode),   intent(inout), pointer :: thisNode
    double precision, intent(in)             :: massAccretionRate
    .
    .
    .
    return
 end function My_Black_Hole_Spin_Up_Rate_Get
\end{verbatim}
The function must return the spin-up rate for the black hole in {\tt thisNode} given the {\tt massAccretionRate}. The jet power function must have the form:
\begin{verbatim}
 double precision function My_Accretion_Disk_Jet_Power_Get(thisNode,massAccretionRate)
    implicit none
    type(treeNode),   intent(inout), pointer :: thisNode
    double precision, intent(in)             :: massAccretionRate
    .
    .
    .
    return
 end function My_Accretion_Disk_Jet_Power_Get
\end{verbatim}
The function must return (in units of $M_\odot$ (km/s)$^2$ Gyr$^{-1}$) the jet power for the black hole/accretion disk system in {\tt thisNode} given the {\tt massAccretionRate}.

Currently defined accretion disk methods are:
\begin{description}
 \item [{\tt Shakura-Sunyaev}] Computes the properties of a thin, radiatively efficiency accretion disk.
 \item [{\tt ADAF}] Computes the properties of an ADAF using the model of \cite{benson_maximum_2009}.
 \item [{\tt switched}] Select either {\tt Shakura-Sunyaev} or {\tt ADAF} accretion disks based on the accretion rate:
 \begin{eqnarray}
  \dot{m}_{\rm minimum} < \dot{M}_{\bullet, 0}/\dot{M}_{\rm Eddington} < \dot{m}_{\rm maximum} &\rightarrow& \hbox{ Shakura-Sunyaev} \nonumber \\
  \hbox{otherwise } &\rightarrow& \hbox{ ADAF},
 \end{eqnarray}
 where $\dot{m}_{\rm minimum}$={\tt accretionRateThinDiskMinimum} and $\dot{m}_{\rm maximum}$={\tt accretionRateThinDiskMaximum} are input parameters.
 \item [{\tt eddingtonLimited}] Assumes no specific disk structure, instead setting the radiative efficiency to a fixed number and the jet power to a fixed fraction of the Eddington luminosity.
\end{description}

\subsubsection{Accretion Onto Halos}

Additional methods for accretion of baryons onto halos can be added using the {\tt accretionHalosMethod} directive. The directive should contain a single argument, giving the name of a subroutine to be called to initialize the method. For example, the {\tt simple} method is described by a directive:
\begin{verbatim}
 !# <accretionHalosMethod>
 !#  <unitName>Accretion_Halos_Simple_Initialize</unitName>
 !# </accretionHalosMethod>
\end{verbatim}
Here, {\tt Accretion\_Halos\_Simple\_Initialize} is the name of a subroutine which will be called to initialize the method. The initialization subroutine must have the following form:
\begin{verbatim}
  subroutine Method_Initialize(accretionHalosMethod,Halo_Baryonic_Accretion_Rate_Get,Halo_Baryonic_Accreted_Mass_Get &
   & ,Halo_Baryonic_Failed_Accretion_Rate_Get,Halo_Baryonic_Failed_Accreted_Mass_Get &
   & ,Halo_Baryonic_Accretion_Rate_Abundances_Get,Halo_Baryonic_Accreted_Abundances_Get, &
   & ,Halo_Baryonic_Accretion_Rate_Chemicals_Get,Halo_Baryonic_Accreted_Chemicals_Get)
    implicit none
    type(varying_string),          intent(in)    :: accretionHalosMethod
    procedure(),          pointer, intent(inout) :: Halo_Baryonic_Accretion_Rate_Get,Halo_Baryonic_Accreted_Mass_Get,Halo_Baryonic_Failed_Accretion_Rate_Get &
   & ,Halo_Baryonic_Failed_Accreted_Mass_Get,Halo_Baryonic_Accretion_Rate_Abundances_Get &
   & ,Halo_Baryonic_Accreted_Abundances_Get,Halo_Baryonic_Accretion_Rate_Chemicals_Get &
   & ,Halo_Baryonic_Accreted_Chemicals_Get
    
    if (accretionHalosMethod == 'myMethod') then
       Halo_Baryonic_Accretion_Rate_Get            => My_Accretion_Rate_Get
       Halo_Baryonic_Accreted_Mass_Get             => My_Accreted_Mass_Get
       Halo_Baryonic_Failed_Accretion_Rate_Get     => My_Failed_Accretion_Rat_Get
       Halo_Baryonic_Failed_Accreted_Mass_Get      => My_Failed_Accreted_Mass_Get
       Halo_Baryonic_Accretion_Rate_Abundances_Get => My_Accretion_Rate_Abundances_Get
       Halo_Baryonic_Accreted_Abundances_Get       => My_Accreted_Abundances_Get
       Halo_Baryonic_Accretion_Rate_Chemicals_Get  => My_Accretion_Rate_Chemicals_Get
       Halo_Baryonic_Accreted_Chemicals_Get        => My_Accreted_Chemicals_Get
    end if
    return
  end subroutine Method_Initialize
\end{verbatim}
where {\tt myMethod} is the name of this method as will be specified by the {\tt accretionHalosMethod} input parameter. The procedure pointers {\tt Halo\_Baryonic\_Accretion\_Rate\_Get}, {\tt Halo\_Baryonic\_Accreted\_Mass\_Get}, {\tt Halo\_Baryonic\_Failed\_Accretion\_Rate\_Get} and {\tt Halo\_Baryonic\_Failed\_Accreted\_Mass\_Get} must be set to point to functions which return accretion rate, total accreted mass (assuming no progenitors), failed accretion rate and total failed accreted mass (assuming no progenitors) respectively as described below. The procedure pointers {\tt Halo\_Baryonic\_Accretion\_Rate\_Abundances\_Get}, {\tt Halo\_Baryonic\_Accreted\_Abundances\_Get}, {\tt Halo\_Baryonic\_Accretion\_Rate\_Chemicals\_Get}, {\tt Halo\_Baryonic\_Accreted\_Chemicals\_Get} must be set to point to functions which return the accretion rates nad masses (assuming no progenitors) of heavy element abundances and chemicals respectively. The initialization subroutine should perform any other tasks required to 
initialize the module (such as reading parameters etc.).

The mass functions must have the form:
\begin{verbatim}
 double precision function My_Accretion_Get(thisNode)
    implicit none
    type(treeNode),   intent(inout), pointer :: thisNode
    .
    .
    .
    return
 end function My_Accretion_Get
\end{verbatim}
In the case of the accretion rate functions, the function must return the accretion rate of baryons from the \gls{igm} onto {\tt thisNode} in $M_\odot$ Gyr$^{-1}$. For total accreted mass functions, the total mass of baryons (in $M_\odot$) accreted onto {\tt thisNode} should be returned under the assumption that {\tt thisNode} formed instataneously with no progenitors. The ``failed'' accretion refers to mass which would have been accreted onto the halo if it simply traced the growth of overall mass. That is:
\begin{equation}
 \dot{M}_{\rm failed} = {\Omega_{\rm b} \over \Omega_{\rm M}} \dot{M} - \dot{M}_{\rm accreted},
\end{equation}
where $\dot{M}$ is the growth rate of total halo mass and $\dot{M}_{\rm accreted}$ is the accretion rate of baryons onto the halo. If desired, this failed mass can be transferred back into the accreted component once the halo is deemed able to accrete, by simply adjusting the accretion rates returned appropriately.

For abundances and chemicals, the subroutines should have the form:
\begin{verbatim}
 subroutine My_Abundances_Get(thisNode,accretionAbundances)
    implicit none
    type(treeNode),            intent(inout), pointer :: thisNode
    type(abundancesStructure), intent(inout)          :: accretionAbundances

    return
  end subroutine My_Abundances_Get
\end{verbatim}
and
\begin{verbatim}
 subroutine My_Chemicals_Get(thisNode,accretionChemicals)
    implicit none
    type(treeNode),                    intent(inout), pointer :: thisNode
    type(chemicalAbundancesStructure), intent(inout)          :: accretionChemicals

    return
  end subroutine My_Chemicals_Get
\end{verbatim}
respectively.

Currently defined accretion disk methods are:
\begin{description}
 \item [{\tt simple}] Assumes that halos accrete all available baryons if they have virial velocities above {\tt reionizationSuppressionVelocity} or exist prior to redshift {\tt reionizationSuppressionRedshift}. This is a simple model of the effects of reionization on gas accretion. In halos which cannot accrete, accretion is placed into the failed mode. In halos which can accrete, any gas in the failed reservoir is returned to the accreted channel on a timescale of $\dot{M}/M$. Abundances are computed assuming a pristine \gls{igm} (i.e. abundances are always zero) and chemicals are computed using the chemical state functions (see \S\ref{sec:ChemicalStateMethods}).
 \item [{\tt null}] Assumes no accretion onto halos.
\end{description}

\subsubsection{Atomic Collisional Ionization Rates}

Additional methods for atomic collisional ionization rate calculations can be added using the {\tt atomicCollisionalIonizationMethod} directive. The directive should contain a single argument, giving the name of a subroutine to be called to initialize the method. For example, the {\tt Verner} method is described by a directive:
\begin{verbatim}
 !# <atomicCollisionalIonizationMethod>
 !#  <unitName>Atomic_Rate_Ionization_Collisional_Verner_Initialize</unitName>
 !# </atomicCollisionalIonizationMethod>
\end{verbatim}
Here, {\tt Atomic\_Rate\_Ionization\_Collisional\_Verner\_Initialize} is the name of a subroutine which will be called to initialize the method. The initialization subroutine must have the following form:
\begin{verbatim}
  subroutine Method_Initialize(atomicCollisionalIonizationMethod,Atomic_Rate_Ionization_Collisional_Get)
    implicit none
    type(varying_string),          intent(in)    :: atomicCollisionalIonizationMethod
    procedure(),          pointer, intent(inout) :: Atomic_Rate_Ionization_Collisional_Get
    
    if (atomicCollisionalIonizationMethod == 'myMethod') Atomic_Rate_Ionization_Collisional_Get => My_Method_Get_Procedure
    return
  end subroutine Method_Initialize
\end{verbatim}
where {\tt myMethod} is the name of this method as will be specified by the {\tt atomicCollisionalIonizationMethod} input parameter. The procedure pointer {\tt Atomic\_Rate\_Ionization\_Collisional\_Get} must be set to point to a function which returns the rate coefficient of atomic collisional ionization under given physical conditions. The initialization subroutine should perform any other tasks required to initialize the module (such as reading parameters etc.).

The collisional ionization rate function must have the form:
\begin{verbatim}
 double precision function My_Method_Get_Procedure(atomicNumber,ionizationState,temperature)
    implicit none
    integer,          intent(in) :: atomicNumber,ionizationState
    double precision, intent(in) :: temperature
    .
    .
    .
    return
 end function My_Method_Get_Procedure
\end{verbatim}
The function must return the collisional ionization rate coefficient (in units of cm$^3$ s$^{-1}$) of ions of the given {\tt atomicNumber}, {\tt ionizationState} (where ionization state is the atomic number plus 1 minus the number of electrons) and {\tt temperature}.

Currently defined collisional ionization rate methods are:
\begin{description}
 \item [{\tt Verner}]  Computes the rate coefficient of direct collisional ionization by use of the fits from \citeauthor{voronov_practical_1997}~(\citeyear{voronov_practical_1997}; Version 2, March 24, 1997).
\end{description}

\subsubsection{Atomic Photoionization Cross-Sections}

Additional methods for atomic photoionization cross-section calculations can be added using the {\tt atomicPhotoIonizationMethod} directive. The directive should contain a single argument, giving the name of a subroutine to be called to initialize the method. For example, the {\tt Verner} method is described by a directive:
\begin{verbatim}
 !# <atomicPhotoIonizationMethod>
 !#  <unitName>Atomic_Cross_Section_Ionization_Photo_Verner_Initialize</unitName>
 !# </atomicPhotoIonizationMethod>
\end{verbatim}
Here, {\tt Atomic\_Cross\_Section\_Ionization\_Photo\_Verner\_Initialize} is the name of a subroutine which will be called to initialize the method. The initialization subroutine must have the following form:
\begin{verbatim}
  subroutine Method_Initialize(atomicPhotoIonizationMethod,Atomic_Cross_Section_Ionization_Photo_Get)
    implicit none
    type(varying_string),          intent(in)    :: atomicPhotoIonizationMethod
    procedure(),          pointer, intent(inout) :: Atomic_Cross_Section_Ionization_Photo_Get
    
    if (atomicPhotoIonizationMethod == 'myMethod') Atomic_Cross_Section_Ionization_Photo_Get => My_Method_Get_Procedure
    return
  end subroutine Method_Initialize
\end{verbatim}
where {\tt myMethod} is the name of this method as will be specified by the {\tt atomicPhotoIonizationMethod} input parameter. The procedure pointer {\tt Atomic\_Cross\_Section\_Ionization\_Photo\_Get} must be set to point to a function which returns the cross-section (in units of cm$^2$) for photoionization. The initialization subroutine should perform any other tasks required to initialize the module (such as reading parameters etc.).

The cross-section function must have the form:
\begin{verbatim}
 double precision function My_Method_Get_Procedure(atomicNumber,ionizationState,shellNumber,wavelength)
    implicit none
    integer,          intent(in) :: atomicNumber,ionizationState,shellNumber
    double precision, intent(in) :: wavelength
    .
    .
    .
    return
 end function My_Method_Get_Procedure
\end{verbatim}
The function must return the cross-section for photoionization (in units of cm$^2$) of electrons in the specified {\tt shellNumber} for ions of the given {\tt atomicNumber} and {\tt ionizationState} (where ionization state is the atomic number plus 1 minus the number of electrons) at the specified {\tt wavelength} (given in units of \AA).

Currently defined photoionization cross-section methods are:
\begin{description}
 \item [{\tt Verner}]  Computes the cross-sections by use of the fits from \citeauthor{verner_atomic_1996_1}~(\citeyear{verner_atomic_1996_1}; Version 2, March 25, 1996).
\end{description}

\subsubsection{Atomic Radiative Recombination Rates}

Additional methods for atomic radiative recombination rate calculations can be added using the {\tt atomicRadiativeRecombinationMethod} directive. The directive should contain a single argument, giving the name of a subroutine to be called to initialize the method. For example, the {\tt Verner} method is described by a directive:
\begin{verbatim}
 !# <atomicRadiativeRecombinationMethod>
 !#  <unitName>Atomic_Rate_Recombination_Radiative_Verner_Initialize</unitName>
 !# </atomicRadiativeRecombinationMethod>
\end{verbatim}
Here, {\tt Atomic\_Rate\_Recombination\_Radiative\_Verner\_Initialize} is the name of a subroutine which will be called to initialize the method. The initialization subroutine must have the following form:
\begin{verbatim}
  subroutine Method_Initialize(atomicRadiativeRecombinationMethod,Atomic_Rate_Recombination_Radiative_Get)
    implicit none
    type(varying_string),          intent(in)    :: atomicRadiativeRecombinationMethod
    procedure(),          pointer, intent(inout) :: Atomic_Rate_Recombination_Radiative_Get
    
    if (atomicRadiativeRecombinationMethod == 'myMethod') Atomic_Rate_Recombination_Radiative_Get => My_Method_Get_Procedure
    return
  end subroutine Method_Initialize
\end{verbatim}
where {\tt myMethod} is the name of this method as will be specified by the {\tt atomicRadiativeRecombinationMethod} input parameter. The procedure pointer {\tt Atomic\_Rate\_Recombination\_Radiative\_Get} must be set to point to a function which returns the rate coefficient of atomic radiative recombination under given physical conditions. The initialization subroutine should perform any other tasks required to initialize the module (such as reading parameters etc.).

The radiative recombination rate function must have the form:
\begin{verbatim}
 double precision function My_Method_Get_Procedure(atomicNumber,ionizationState,temperature)
    implicit none
    integer,          intent(in) :: atomicNumber,ionizationState
    double precision, intent(in) :: temperature
    .
    .
    .
    return
 end function My_Method_Get_Procedure
\end{verbatim}
The function must return the radiative recombination rate coefficient (in units of cm$^3$ s$^{-1}$) to ions of the given {\tt atomicNumber}, {\tt ionizationState} (where ionization state is the atomic number plus 1 minus the number of electrons) and {\tt temperature}.

Currently defined radiative recombination rate methods are:
\begin{description}
 \item [{\tt Verner}]  Computes the rate coefficient of radiative recombination using the compilation of results from Dima Verner as originally encapsulated in \href{ftp://gradj.pa.uky.edu//dima//rec//rrfit.f}{{\tt rrfit.f}}.
\end{description}

\subsubsection{Bar Instabilities}

Additional methods for bar instabilities in disks can be added using the {\tt barInstabilityMethod} directive. The directive should contain a single argument, giving the name of a subroutine to be called to initialize the method. For example, the {\tt ELN} method is described by a directive:
\begin{verbatim}
 !# <barInstabilityMethod>
 !#  <unitName>Galactic_Dynamics_Bar_Instabilities_ELN_Initialize</unitName>
 !# </barInstabilityMethod>
\end{verbatim}
Here, {\tt Galactic\_Dynamics\_Bar\_Instabilities\_ELN\_Initialize} is the name of a subroutine which will be called to initialize the method. The initialization subroutine must have the following form:
\begin{verbatim}
  subroutine Method_Initialize(barInstabilityMethod,Bar_Instability_Timescale_Get)
    implicit none
    type(varying_string),          intent(in)    :: barInstabilityMethod
    procedure(),          pointer, intent(inout) :: Bar_Instability_Timescale_Get
    
    if (barInstabilityMethod == 'myMethod') then
       Bar_Instability_Timescale_Get => My_Bar_Instability_Timescale_Get
       .
       .
       .
    end if
    return
  end subroutine Method_Initialize
\end{verbatim}
where {\tt myMethod} is the name of this method as will be specified by the {\tt barInstabilityMethod} input parameter. The procedure pointer {\tt Bar\_Instability\_Timescale\_Get} must be set to point to a function which returns the timescale on which the bar instability depletes material from the disk to the pseudo-bulge. The initialization subroutine should perform any other tasks required to initialize the module (such as reading parameters etc.).

The bar instability timesacale function must have the form:
\begin{verbatim}
  subroutine My_Bar_Instability_Timescale(thisNode,barInstabilityTimeScale,barInstabilityExternalDrivingSpecificTorque)
    implicit none
    type            (treeNode), intent(inout), pointer :: thisNode
    double precision          , intent(  out)          :: barInstabilityTimeScale,barInstabilityExternalDrivingSpecificTorque
    .
    .
    .
    return
  end subroutine My_Bar_Instability_Timescale
\end{verbatim}
The function should compute and return, in {\tt barInstabilityTimeScale} the timescale (in Gyr) for the bar instability in the disk in {\tt thisNode} to transfer material from the disk to the pseudo-bulge. If no instability is present, a negative timescale should be returned. Additionally, any specific torque external to the galaxy driving the instability should be returned in {\tt barInstabilityExternalDrivingSpecificTorque}.

Currently defined bar instability methods are:
\begin{description}
 \item [{\tt null}] A null method in which disks are never bar unstable;
 \item [{\tt ELN}] The bar instability is determined using the algorithm of \cite{efstathiou_stability_1982}.
 \item [{\tt ELN+tidal}] The bar instability is determined using the algorithm of \cite{efstathiou_stability_1982} with an additional term to account for an external tidal field.
\end{description}

\subsubsection{Black Hole Binaries: Initial Separation}

Additional methods for black hole binary initial separation calculations can be added using the {\tt blackHoleBinaryInitialRadiiMethod} directive. The directive should contain a single argument, giving the name of a subroutine to be called to initialize the method. For example, the {\tt spheroidRadiusFraction} method is described by a directive:
\begin{verbatim}
 !# <blackHoleBinaryInitialRadiiMethod>
 !#  <unitName>Black_Hole_Binary_Initial_Radii_Spheroid_Size_Initialize</unitName>
 !# </blackHoleBinaryInitialRadiiMethod>
\end{verbatim}
Here, {\tt Black\_Hole\_Binary\_Initial\_Radii\_Spheroid\_Size\_Initialize} is the name of a subroutine which will be called to initialize the method. The initialization subroutine must have the following form:
\begin{verbatim}
  subroutine Method_Initialize(blackHoleBinaryInitialRadiiMethod,Black_Hole_Binary_Initial_Radius_Get)
    implicit none
    type(varying_string),          intent(in)    :: blackHoleBinaryInitialRadiiMethod
    procedure(),          pointer, intent(inout) :: Black_Hole_Binary_Initial_Radius_Get
    
    if (blackHoleBinaryInitialRadiiMethod == 'myMethod') Black_Hole_Binary_Initial_Radius_Get => My_Method_Get
    return
  end subroutine Method_Initialize
\end{verbatim}
where {\tt myMethod} is the name of this method as will be specified by the {\tt blackHoleBinaryInitialRadiiMethod} input parameter. The procedure pointer {\tt Black\_Hole\_Binary\_Initial\_Radius\_Get} must be set to point to a function which returns the initial separation of a just-formed black hole binary. The initialization subroutine should perform any other tasks required to initialize the module (such as reading parameters etc.).

The initial separation function must have the form:
\begin{verbatim}
 double precision function My_Method_Get(thisNode,hostNode)
    implicit none
    type(treeNode), intent(inout), pointer :: thisNode,hostNode
    .
    .
    .
    return
 end subroutine My_Method_Get
\end{verbatim}
The function must return the initial separation (in Mpc) of the active black hole in {\tt thisNode} as it merges into {\tt hostNode}.

Currently defined black hole binary initial separation methods are:
\begin{description}
 \item [{\tt spheroidRadiusFraction}] Assumes that the initial separation is equal to a fraction {\tt [blackHoleInitialRadiusSpheroidRadiusRatio]} of the larger of the spheroid scale radii in {\tt thisNode} and {\tt hostNode}.
 \item [{\tt Volonteri2003}] Assumes that the initial separation follows the relationship described in \cite{volonteri_assembly_2003} following the black hole masses in {\tt thisNode} and {\tt hostNode}.
\item [{\tt tidalRadius}] Solves the radius at which the satellite galaxy is stripped of its stars, and assume only the central black hole remains, at that specific radius. This uses the masses in {\tt thisNode} and {\tt hostNode}.
\end{description}

\subsubsection{Black Hole Binaries: Separation Growth Rate}\label{sec:SMBHRadialMotion}

Additional methods for black hole binary separation growth rate calculations can be added using the {\tt blackHoleBinarySeparationGrowthRateMethod} directive. The directive should contain a single argument, giving the name of a subroutine to be called to initialize the method. For example, the {\tt Volonteri2003} method is described by a directive:

\begin{verbatim}
  !# <blackHoleBinarySeparationGrowthRateMethod>
  !#  <unitName>Black_Hole_Binary_Separation_Growth_Rate_Standard_Init</unitName>
  !# </blackHoleBinarySeparationGrowthRateMethod>
\end{verbatim}

Currently defined black hole binary separation growth rate methods are:
\begin{description}
 \item [{\tt null}] Assumes that the initial separation stays constant.
 \item [{\tt standard}] Assumes that the separation growth rate follows \cite{volonteri_assembly_2003} following the black hole masses in {\tt thisNode}. Although it innovates as it encompasses all three influences: Dynamical Friction, Hardening due to stars and finally due to Gravitational Wave expulsion. Dynamical friction here occurs until a certain hardening separation is reached, it then is replaced by the (faster) three-body interactions with stars.
\end{description}

\subsubsection{Black Hole Binaries: Recoil Velocity}

Additional methods for the recoil velocity of a binary black hole can be added using the {\tt blackHoleBinaryRecoilVelocityMethod} directive. The directive should contain a single argument, giving the name of a subroutine to be called to initialize the method. For example, the {\tt Standard} method is described by a directive:

\begin{verbatim}
  !# <blackHoleBinaryRecoilVelocityMethod>
  !#  <unitName>Black_Hole_Binary_Recoil_Velocity_Standard_Initialize</unitName>
  !# </blackHoleBinaryRecoilVelocityMethod>
\end{verbatim}

Currently defined black hole binary recoil velocity methods are:
\begin{description}
 \item [{\tt null}] Assumes that there is zero recoil velocity.
 \item [{\tt Campanelli2008}] Assumes that the recoil velocity follows \cite{campanelli_large_2007}, utilizing the black hole masses and spins in {\tt thisNode}. For now it does not take the direction of the spin into account, and assumes a zero perpendicular velocity.
\end{description}

\subsubsection{Black Hole Binaries: Mergers}\label{sec:BlackHoleBinaryMergers}

Additional methods for black hole binary merger calculations can be added using the {\tt blackHoleBinaryMergersMethod} directive. The directive should contain a single argument, giving the name of a subroutine to be called to initialize the method. For example, the {\tt Rezzolla2008} method is described by a directive:
\begin{verbatim}
 !# <blackHoleBinaryMergersMethod>
 !#  <unitName>Black_Hole_Binary_Merger_Initialize</unitName>
 !# </blackHoleBinaryMergersMethod>
\end{verbatim}
Here, {\tt Black\_Hole\_Binary\_Merger\_Initialize} is the name of a subroutine which will be called to initialize the method. The initialization subroutine must have the following form:
\begin{verbatim}
  subroutine Method_Initialize(blackHoleBinaryMergersMethod,Black_Hole_Binary_Merger_Do)
    implicit none
    type(varying_string),          intent(in)    :: blackHoleBinaryMergersMethod
    procedure(),          pointer, intent(inout) :: Black_Hole_Binary_Merger_Do
    
    if (blackHoleBinaryMergersMethod == 'myMethod') Black_Hole_Binary_Merger_Do => My_Method_Do_Procedure
    return
  end subroutine Method_Initialize
\end{verbatim}http://www.facebook.com/groups/172995192751085?view=search&query=beawwch
where {\tt myMethod} is the name of this method as will be specified by the {\tt blackHoleBinaryMergersMethod} input parameter. The procedure pointer {\tt Black\_Hole\_Binary\_Merger\_Do} must be set to point to a function which returns the properties (mass and spin) of the merged black hole as described below. The initialization subroutine should perform any other tasks required to initialize the module (such as reading parameters etc.).

The cooling radius function must have the form:
\begin{verbatim}
 subroutine My_Method_Do_Procedure(blackHoleMassA,blackHoleMassB,blackHoleSpinA,blackHoleSpinB,blackHoleMassFinal,blackHoleSpinFinal)
    implicit none
    double precision, intent(in)  :: blackHoleMassA,blackHoleMassB,blackHoleSpinA,blackHoleSpinB
    double precision, intent(out) :: blackHoleMassFinal,blackHoleSpinFinal
    .
    .
    .
    return
 end subroutine My_Method_Do_Procedure
\end{verbatim}
The function must return the mass and spin (in {\tt blackHoleMassFinal} and {\tt blackHoleSpinFinal} respectively) of the black hole resulting from the merger of black holes with masses {\tt blackHoleMassA} and {\tt blackHoleMassB} and spins {\tt blackHoleSpinA} and {\tt blackHoleSpinB}. The subroutine should make no assumptions about the mass ordering of the input black holes (i.e. A could be more massive than B or vice versa).

Currently defined black hole binary merger methods are:
\begin{description}
 \item [{\tt Rezzolla2008}] Computes the properties of the merged black hole using the approximations of \cite{rezzolla_final_2008}.
\end{description}

\subsubsection{Chemical State}\label{sec:ChemicalStateMethods}

Additional methods for chemical states can be added using the {\tt chemicalStateMethod} directive. The directive should contain a single argument, giving the name of a subroutine to be called to initialize the method. For example, the {\tt atomic\_CIE\_Cloudy} method is described by a directive:
\begin{verbatim}
  !# <chemicalStateMethod>
  !#  <unitName>Chemical_State_Atomic_CIE_Cloudy_Initialize</unitName>
  !# </chemicalStateMethod>
\end{verbatim}
Here, {\tt Chemical\_State\_Atomic\_CIE\_Cloudy\_Initialize} is the name of a subroutine which will be called to initialize the method. The initialization subroutine must have the following form:
\begin{verbatim}
  subroutine Method_Initialize(chemicalStateMethod,Electron_Density_Get,Electron_Density_Temperature_Log_Slope_Get,Electron_Density_Density_Log_Slope_Get,Chemical_Densities_Get)
    implicit none
    type(varying_string),          intent(in)    :: chemicalStateMethod
    procedure(),          pointer, intent(inout) :: Electron_Density_Get,Electron_Density_Temperature_Log_Slope_Get,Electron_Density_Density_Log_Slope_Get
    
    if (chemicalStateMethod == 'myMethod') then
      Electron_Density_Get                       => My_Method_Procedure
      Electron_Density_Temperature_Log_Slope_Get => My_Method_Temperature_Log_Slope_Procedure
      Electron_Density_Density_Log_Slope_Get     => My_Method_Density_Log_Slope_Procedure
      Chemical_Densities_Get                     => My_Method_Chemical_Densities_Procedure
    end if
    return
  end subroutine Method_Initialize
\end{verbatim}
where {\tt myMethod} is the name of this method as will be specified by the {\tt chemicalStateMethod} input parameter. The procedure pointer {\tt Electron\_Density\_Get} must be set to point to a function which returns the electron density as described below. The other two electron density procedure pointers should point to functions which return the logarithmic gradients of the electron density with respect to temperature and density respectively. The initialization subroutine should perform any other tasks required to initialize the module (such as reading parameters etc.). The {\tt Chemical\_Densities\_Get} pointer should be set to point to a subroutine that returns the densities of all ``chemicals'' active in the chemical subsystem (see \S\ref{sec:ChemicalSubsystem}).

The electron density function must have the form:
\begin{verbatim}
 double precision function Electron_Density_Get(temperature,numberDensityHydrogen,abundances,radiation)
    implicit none
    double precision,          intent(in) :: temperature,numberDensityHydrogen
    type(abundancesStructure), intent(in) :: abundances
    type(radiationStructure),  intent(in) :: radiation
    .
    .
    .
    return
 end function Electron_Density_Get
\end{verbatim}
The function must return the electron density (in units of cm$^{-3}$) for gas at the given {\tt temperature} (in Kelvin), with hydrogen number density {\tt numberDensityHydrogen} (in cm$^{-3}$), composition as described by the {\tt abundances} structure and in the presence of a radiation field described by the {\tt radiation} structure. The logarithmic slope functions should have the same template, but return the appropriate logarithmic slope instead.

The chemical densities subroutine must have the form:
\begin{verbatim}
 subroutine Chemical_Densities_Get(theseAbundances,temperature,numberDensityHydrogen,abundances,radiation)
    implicit none
    type(chemicalAbundancesStructure), intent(inout) :: theseAbundances
    double precision,                  intent(in)    :: temperature,numberDensityHydrogen
    type(abundancesStructure)          intent(in)    :: abundances
    type(radiationStructure)           intent(in)    :: radiation
    .
    .
    .
    return
 end subroutine Chemical_Densities_Get
\end{verbatim}
The function must return the density (in units of cm$^{-3}$) of each chemical species for gas at the given {\tt temperature} (in Kelvin), with hydrogen number density {\tt numberDensityHydrogen} (in cm$^{-3}$), composition as described by the {\tt abundances} structure and in the presence of a radiation field described by the {\tt radiation} structure.

Currently defined chemical state methods are:
\begin{description}
 \item [\hyperlink{chemical.state.CIE_file.F90:chemical_states_cie_file}{{\tt CIE\_from\_file}}] Reads a tabulated CIE chemical state from a file and interpolates in the table to give a result. The XML file containing the table should have the following form:
 \begin{verbatim}
  <chemicalStates>
  <chemicalState>
    <temperature>
      <datum>10000.0</datum>
      <datum>15000.0</datum>
      .
      .
      .
    </temperature>
    <electronDensity>
      <datum>1.0e-23</datum>
      <datum>1.7e-23</datum>
      .
      .
      .
    </electronDensity>
    <hiDensity>
      <datum>0.966495864314214</datum>
      <datum>0.965828463162061</datum>
      .
      .
      .
    </hiDensity>
    <hiiDensity>
      <datum>0.033504135685786</datum>
      <datum>0.0341715368379391</datum>
      .
      .
      .
    </hiiDensity>
    <metallicity>-4.0</metallicity>
  </chemicalState>
  <chemicalState>
  .
  .
  .
  </chemicalState>
  <description>Some description of what this chemical state is.</description>
  <extrapolation>
    <metallicity>
      <limit>low</limit>
      <method>fixed</method>
    </metallicity>
    <metallicity>
      <limit>high</limit>
      <method>fixed</method>
    </metallicity>
    <temperature>
      <limit>low</limit>
      <method>fixed</method>
    </temperature>
    <temperature>
      <limit>high</limit>
      <method>fixed</method>
    </temperature>
  </extrapolation>
 </chemicalStates>
 \end{verbatim}
 Each {\tt chemicalState} element should contain two lists (inside {\tt temperature} and {\tt electronDensity} tags) of {\tt datum} elements which specify temperature (in Kelvin) and electron density (by number, relative to hydrogen) respectively, and a {\tt metallicity} element which gives the logarithmic metallcity relative to Solar (a value of -999 or less is taken to imply zero metallicity). Optionally, {\tt hiDensity} and {\tt hiiDensity} elements may be added containing lists of H{\sc i} and H{\sc ii} densities (by number, relative to hydrogen) respectively. Any number of {\tt coolingFunction} elements may appear, but they must be in order of increasing metallicity and must all contain the same set of temperatures. The {\tt extrapolation} element defines how the table is to be extrapolated in the {\tt low} and {\tt high} limits of {\tt temperature} and {\tt metallicity}. The {\tt method} elements can take the following values:
 \begin{description}
  \item[{\tt zero}] The electron density is set to zero beyond the relevant limit.
  \item[{\tt fixed}] The electron density is held fixed at the value at the relevant limit.
  \item[{\tt power law}] The electron density is extrapolated assuming a power-law dependence beyond the relevant limit. This option is only allowed if the electron density is everywhere positive.
 \end{description}
 If the electron density is everywhere positive the interpolation will be done in the logarithmic of temperature, metallicity\footnote{The exception is if the first electron density is tabulated for zero metallicity. In that case, a linear interpolation in metallicity is always used between zero and the first non-zero tabulated metallicity.} and electron density. Otherwise, interpolation is linear in these quantities. The electron density is scaled assuming a linear dependence on hydrogen density.
 \item [{\tt atomicCIECloudy}] Uses the {\sc Cloudy} software to compute the chemical state for atomic gas in collisional ionization equilibrium. {\sc Cloudy} will be downloaded, compiled and run automatically if necessary\footnote{{\sc Cloudy} is used to generate a file which contains a tabulation of the chemical state suitable for reading by the {\tt CIE from file} method. Generation of the tabulation typically takes several hours, but only needs to be done once as the stored table is simply read back in on later runs.}.
 \item [{\tt PIE\_from\_file}]  Reads a tabulated PIE ionization state from a file and interpolates in the table to give a result. The HDF5 file containing the table should have the following form:
\begin{verbatim}
GROUP "/" {
   DATASET "neutralHydrogenRatio" {
      DATATYPE  H5T_IEEE_F64BE
      DATASPACE  SIMPLE { ( <ratioCount>, <redshiftCount>, <temperatureCount>, <densityCount> ) }
   }
   DATASET "density" {
      DATATYPE  H5T_IEEE_F64BE
      DATASPACE  SIMPLE { ( <densityCount> ) }
   }
   DATASET "electronRatio" {
      DATATYPE  H5T_IEEE_F64BE
      DATASPACE  SIMPLE { ( <ratioCount>, <redshiftCount>, <temperatureCount>, <densityCount> ) }
   }
   DATASET "heliumToHydrogenRatio" {
      DATATYPE  H5T_IEEE_F64BE
      DATASPACE  SIMPLE { ( <ratioCount> ) }
      ATTRIBUTE "extrapolationHigh" {
         DATATYPE  H5T_STRING {}
         DATASPACE  SCALAR
      }
      ATTRIBUTE "extrapolationLow" {
         DATATYPE  H5T_STRING {}
         DATASPACE  SCALAR
      }
   }
   DATASET "redshift" {
      DATATYPE  H5T_IEEE_F64BE
      DATASPACE  SIMPLE { ( <redshiftCount> ) }
      ATTRIBUTE "extrapolationHigh" {
         DATATYPE  H5T_STRING {}
         DATASPACE  SCALAR
      }
      ATTRIBUTE "extrapolationLow" {
         DATATYPE  H5T_STRING {}
         DATASPACE  SCALAR
      }
   }
   DATASET "temperature" {
      DATATYPE  H5T_IEEE_F64BE
      DATASPACE  SIMPLE { ( <temperatureCount> ) }
      ATTRIBUTE "extrapolationHigh" {
         DATATYPE  H5T_STRING {}
         DATASPACE  SCALAR
      }
      ATTRIBUTE "extrapolationLow" {
         DATATYPE  H5T_STRING {}
         DATASPACE  SCALAR
      }
   }
}
\end{verbatim}
The datasets should contain the following information:
\begin{description}
\item [{\tt temperature}] A list of temperatures (in units of Kelvin) at which cooling functions are tabulated. If present, the {\tt extrapolationLow} and {\tt extrapolationHigh} attributes specify how the data should be extrapolated to lower and higher temperatures (see below);
\item [{\tt redshift}] A list of redshifts at which cooling functions are tabulated. If present, the {\tt extrapolationLow} and {\tt extrapolationHigh} attributes specify how the data should be extrapolated to lower and higher redshifts (see below);
\item [{\tt density}] A list of hydrogen number densities (in units of cm$^{-3}$) at which cooling functions are tabulated. If present, the {\tt extrapolationLow} and {\tt extrapolationHigh} attributes specify how the data should be extrapolated to lower and higher densities (see below);
\item [{\tt heliumToHydrogenRatio}] A list of helium-to-hydrogen number density ratios at which cooling functions are tabulated. If present, the {\tt extrapolationLow} and {\tt extrapolationHigh} attributes specify how the data should be extrapolated to lower and higher ratios (see below);
\item [{\tt elements}] A list of the atomic numbers of elements for which cooling functions are tabulated;
\item [{\tt neutralHydrogenRatio}] The neutral hydrogen fraction on the grid of temperature, density, redshift and helium-to-hydrogen number density ratio;
\item [{\tt electronRatio}] The electron to hydrogen number density ratio on the grid of temperature, density, redshift and helium-to-hydrogen number density ratio.
\end{description}
\end{description}

\subsubsection{Conditional Stellar Mass Functions}

Additional methods for empirical conditional mass functions can be added using the {\tt conditionalStellarMassFunctionMethod} directive. The directive should contain a single argument, giving the name of a subroutine to be called to initialize the method. For example, the {\tt Behroozi2010} method is described by a directive:
\begin{verbatim}
  !# <conditionalStellarMassFunctionMethod>
  !#  <unitName>Conditional_Stellar_Mass_Functions_Behroozi2010_Initialize</unitName>
  !# </conditionalStellarMassFunctionMethod>
\end{verbatim}
Here, {\tt Conditional\_Stellar\_Mass\_Functions\_Behroozi2010\_Initialize} is the name of a subroutine which will be called to initialize the method. The initialization subroutine must have the following form:
\begin{verbatim}
  subroutine Method_Initialize(conditionalStellarMassFunctionMethod&
       &,Cumulative_Conditional_Stellar_Mass_Function_Get,Cumulative_Conditional_Stellar_Mass_Function_Var_Get)
    implicit none
    type(varying_string),                 intent(in)    :: conditionalStellarMassFunctionMethod
    procedure(double precision), pointer, intent(inout) :: Cumulative_Conditional_Stellar_Mass_Function_Get,Cumulative_Conditional_Stellar_Mass_Function_Var_Get
    
    if (conditionalStellarMassFunctionMethod == 'myMethod') then
       Cumulative_Conditional_Stellar_Mass_Function_Get     => My_Cumulative_Conditional_Stellar_Mass_Function
       Cumulative_Conditional_Stellar_Mass_Function_Var_Get => My_Cumulative_Conditional_Stellar_Mass_Function_Var
       .
       .
       .
    end if
    return
  end subroutine Method_Initialize
\end{verbatim}
where {\tt myMethod} is the name of this method as will be specified by the {\tt timePerTreeMethod} input parameter. The procedure pointer {\tt Galacticus\_Time\_Per\_Tree\_Get} must be set to point to a function which returns an estimate of the time taken (in seconds) to process a merger tree. The initialization subroutine should perform any other tasks required to initialize the module (such as reading parameters etc.).

The functionw must have the form:
\begin{verbatim}
   double precision function My_Cumulative_Conditional_Stellar_Mass_Function(massHalo,massStellar)
    implicit none
    double precision, intent(in) :: massHalo,massStellar
    .
    .
    .
    return
   end function My_Cumulative_Conditional_Stellar_Mass_Function_Var

   double precision My_Cumulative_Conditional_Stellar_Mass_Function_Var My_Cumulative_Conditional_Stellar_Mass_Function(massHalo,massStellarLow,massStellarHigh)
    implicit none
    double precision, intent(in) :: massHalo,massStellarLow,massStellarHigh
    .
    .
    .
    return
   end function My_Cumulative_Conditional_Stellar_Mass_Function_Var 
\end{verbatim}
The first function must return the number of galaxies of mass greater than {\tt massStellar} in halos of mass {\tt massHalo}. The second function should return the variance in the number of galaxies in the stellar mass range {\tt massStellarLow} to {\tt massStellarHigh} in halos of mass {\tt massHalo}

Currently defined tree timing methods are:
\begin{description}
 \item [{\tt Behroozi2010}] This method uses the fitting function of \cite{behroozi_comprehensive_2010} to compute the conditional stellar mass function. To compute the variance in the mass function, this method assumes that the number of satellite galaxies follows a Poisson distribution, while central galaxies follow a \gls{Bernoulli distribution}.
\end{description}

\subsubsection{Cooling Rate}

Additional methods for the cooling rate from the hot halo can be added using the {\tt coolingRateMethod} directive. The directive should contain a single argument, giving the name of a subroutine to be called to initialize the method. For example, the {\tt White-Frenk1991} method is described by a directive:
\begin{verbatim}
 !# <coolingRateMethod>
 !#  <unitName>Cooling_Rate_White_Frenk_Initialize</unitName>
 !# </coolingRateMethod>
\end{verbatim}
Here, {\tt Cooling\_Rate\_White\_Frenk\_Initialize} is the name of a subroutine which will be called to initialize the method. The initialization subroutine must have the following form:
\begin{verbatim}
  subroutine Method_Initialize(coolingRateMethod,Cooling_Rate_Get)
    implicit none
    type(varying_string),          intent(in)    :: coolingRateMethod
    procedure(),          pointer, intent(inout) :: Cooling_Rate_Get
    
    if (coolingRateMethod == 'myMethod') Cooling_Rate_Get => My_Method_Get
    return
  end subroutine Method_Initialize
\end{verbatim}
where {\tt myMethod} is the name of this method as will be specified by the {\tt coolingRateMethod} input parameter. The procedure pointer {\tt Cooling\_Rate\_Get} must be set to point to a function which returns the cooling rate from the hot halo. The initialization subroutine should perform any other tasks required to initialize the module (such as reading parameters etc.).

The cooling rate function must have the form:
\begin{verbatim}
 double precision function Cooling_Rate_Get(thisNode)
    implicit none
    type(treeNode),   intent(inout), pointer :: thisNode
    .
    .
    .
    return
 end function Cooling_Rate_Get
\end{verbatim}
The function must return the rate of mass drop-out from the hot halo (in units of $M_\odot$/Gyr) for {\tt thisNode}. 

Currently defined cooling rate methods are:
\begin{description}
 \item [\hyperlink{cooling.cooling_rate.White-Frenk.F90:cooling_rates_white_frenk:cooling_rate_white_frenk}{{\tt White-Frenk1991}}] Implements something similar to that proposed by \cite{white_galaxy_1991}. Namely, the cooling rate is set equal to
 \begin{equation}
  \dot{M}_{\rm cool} = 4 \pi \rho(r_{\rm infal}) r_{\rm infall}^2 \dot{r}_{\rm infall}
 \end{equation}
 if the infall radius is within the outer radius of the hot halo and
 \begin{equation}
  \dot{M}_{\rm cool} = {M_{\rm hot} \over \tau_{\rm dynamical,halo}}
 \end{equation}
 otherwise.
\item [\hyperlink{cooling.cooling_rate.Cole2000.F90:cooling_rates_cole2000:cooling_rate_cole2000}{{\tt Cole2000}}] Implements the cooling rate algorithm from \cite{cole_hierarchical_2000}.
\item [\hyperlink{cooling.cooling_rate.simple.F90:cooling_rates_simple:cooling_rate_simple}{{\tt simple}}] Implements a simple algorithm in which the cooling rate is determined from a fixed timescale.
\item [\hyperlink{cooling.cooling_rate.simple_scaling.F90:cooling_rates_simple_scaling:cooling_rate_simple_scaling}{{\tt simpleScaling}}] Implements a simple algorithm in which the cooling rate is determined from a timescale which is a function of halo mass and redshift.
\end{description}

\subsubsection{Cooling Function}\label{sec:CoolingFunctionMethods}

Additional methods for cooling functions can be added using the {\tt coolingFunctionMethods}, {\tt coolingFunctionCompute}, {\tt coolingFunctionDensitySlopeCompute} and {\tt coolingFunctionTemperatureSlopeCompute} directives. Each directive should contain a single argument, giving the name of a subroutine to be called to either initialize the method or compute the relevant quantity. For example, the {\tt atomicCIECloudy} method is initialized by a directive:
\begin{verbatim}
  !# <coolingFunctionMethods>
  !#  <unitName>Cooling_Function_Atomic_CIE_Cloudy_Initialize</unitName>
  !# </coolingFunctionMethods>
\end{verbatim}
Here, {\tt Cooling\_Function\_Atomic\_CIE\_Cloudy\_Initialize} is the name of a subroutine which will be called to initialize the method. The initialization subroutine must have the following form:
\begin{verbatim}
  subroutine Method_Initialize(coolingFunctionMethods,coolingFunctionsMatched)
    implicit none
    type(varying_string), intent(in   ) :: coolingFunctionMethods(:)
    integer,              intent(inout) :: coolingFunctionsMatched
    
    if (any(coolingFunctionMethods == 'myMethod')) then
       .
       .
       .
    end if
    return
  end subroutine Method_Initialize
\end{verbatim}
where {\tt myMethod} is the name of this method as will be specified by the {\tt coolingFunctionMethods} input parameter. The initialization routine should record that whether this cooling function was selected, and perform any other initialization necessary. For each cooling function matched by this method, the value of {\tt coolingFunctionsMatched} should be incremented by one---this permits a check that all cooling functions were matched.

The other directives should specify subroutines with the following template:
\begin{verbatim}
  subroutine Cooling_Function_PropertyCompute(coolingFunctionProperty,temperature,numberDensityHydrogen,abundances&
       &,chemicalDensities,radiation)
    implicit none
    double precision,                  intent(in)  :: temperature,numberDensityHydrogen
    type(abundancesStructure),         intent(in)  :: abundances
    type(chemicalAbundancesStructure), intent(in)  :: chemicalDensities
    type(radiationStructure),          intent(in)  :: radiation
    double precision,                  intent(out) :: coolingFunctionProperty
    .
    .
    .
    return
  end subroutine Cooling_Function_PropertyCompute
\end{verbatim}
and each should return the relevant quantity in {\tt coolingFunctionProperty}. The {\tt coolingFunctionCompute} subroutine should return the cooling function, the {\tt coolingFunctionDensitySlopeCompute} subroutine should return the partial derivative with respect to hydrogen density and the {\tt coolingFunctionTemperatureSlopeCompute} subroutine should return the partial derivative with respect to temperature.

Currently defined cooling function methods are:
\begin{description}
 \item [\hyperlink{cooling.cooling_function.CIE_file.F90:cooling_functions_cie_file:cooling_function_cie_file}{{\tt CIE\_from\_file}}] Reads a tabulated CIE cooling function from a file and interpolates in the table to give a result. The XML file containing the table should have the following form:
 \begin{verbatim}
  <coolingFunctions>
  <coolingFunction>
    <temperature>
      <datum>10000.0</datum>
      <datum>15000.0</datum>
      .
      .
      .
    </temperature>
    <coolingRate>
      <datum>1.0e-23</datum>
      <datum>1.7e-23</datum>
      .
      .
      .
    </coolingRate>
    <metallicity>-4.0</metallicity>
  </coolingFunction>
  <coolingFunction>
  .
  .
  .
  </coolingFunction>
  <description>Some description of what this cooling function is.</description>
  <extrapolation>
    <metallicity>
      <limit>low</limit>
      <method>power law</method>
    </metallicity>
    <metallicity>
      <limit>high</limit>
      <method>power law</method>
    </metallicity>
    <temperature>
      <limit>low</limit>
      <method>power law</method>
    </temperature>
    <temperature>
      <limit>high</limit>
      <method>power law</method>
    </temperature>
  </extrapolation>
 </coolingFunctions>
 \end{verbatim}
 Each {\tt coolingFunction} element should contain two lists (inside {\tt temperature} and {\tt coolingRate} tags) of {\tt datum} elements which specify temperature (in Kelvin) and cooling function (in ergs cm$^3$ s$^{-1}$ computed for a hydrogen density of 1 cm$^{-3}$) respectively, and a {\tt metallicity} element which gives the logarithmic metallcity relative to Solar (a value of -999 or less is taken to imply zero metallicity). Any number of {\tt coolingFunction} elements may appear, but they must be in order of increasing metallicity and must all contain the same set of temperatures. The {\tt extrapolation} element defines how the table is to be extrapolated in the {\tt low} and {\tt high} limits of {\tt temperature} and {\tt metallicity}. The {\tt method} elements can take the following values:
 \begin{description}
  \item[{\tt zero}] The cooling function is set to zero beyond the relevant limit.
  \item[{\tt fixed}] The cooling function is held fixed at the value at the relevant limit.
  \item[{\tt power law}] The cooling function is extrapolated assuming a power-law dependence beyond the relevant limit. This option is only allowed if the cooling function is everywhere positive.
 \end{description}
 If the cooling function is everywhere positive the interpolation will be done in the logarithmic of temperature, metallicity\footnote{The exception is if the first cooling function is tabulated for zero metallicity. In that case, a linear interpolation in metallicity is always used between zero and the first non-zero tabulated metallicity.} and cooling function. Otherwise, interpolation is linear in these quantities. The cooling function is scaled assuming a quadratic dependence on hydrogen density.
 \item [{\tt atomic\_CIE\_Cloudy}] Uses the {\sc Cloudy} software to compute a cooling function for atomic gas in collisional ionization equilibrium. {\sc Cloudy} will be downloaded, compiled and run automatically if necessary\footnote{{\sc Cloudy} is used to generate a file which contains a tabulation of the cooling function suitable for reading by the {\tt CIE from file} method. Generation of the tabulation typically takes several hours, but only needs to be done once as the stored table is simply read back in on later runs.}. Figure~\ref{fig:atomicCIECloudyCoolingFunction} shows the cooling function from this method.
 \item [{\tt CMB\_Compton}] Computes the cooling function due to Compton scattering off of \gls{cmb} photons.
 \item [{\tt molecularHydrogenGalliPalla}] Implements the molecular hydrogen cooling function from \cite{galli_chemistry_1998}.
\end{description}

\begin{figure}
 \begin{center}
 \includegraphics[width=160mm]{../plots/cooling_function_Atomic_CIE_Cloudy.pdf}
 \end{center}
 \caption{Cooling function for atomic gas in collisional ionization equilibrium computed using Cloudy 08.00.}
 \label{fig:atomicCIECloudyCoolingFunction}
\end{figure}

\subsubsection{Cooling Radius}

Additional methods for cooling radius calculations can be added using the {\tt coolingRadiusMethod} directive. The directive should contain a single argument, giving the name of a subroutine to be called to initialize the method. For example, the {\tt simple} method is described by a directive:
\begin{verbatim}
 !# <coolingRadiusMethod>
 !#  <unitName>Cooling_Radius_Simple_Initialize</unitName>
 !# </coolingRadiusMethod>
\end{verbatim}
Here, {\tt Cooling\_Radius\_Simple\_Initialize} is the name of a subroutine which will be called to initialize the method. The initialization subroutine must have the following form:
\begin{verbatim}
  subroutine Method_Initialize(coolingRadiusMethod,Cooling_Radius_Get,Cooling_Radius_Growth_Rate_Get)
    implicit none
    type(varying_string),          intent(in)    :: coolingRadiusMethod
    procedure(),          pointer, intent(inout) :: Cooling_Radius_Get,Cooling_Radius_Growth_Rate_Get
    
    if (coolingRadiusMethod == 'myMethod') then
       Cooling_Radius_Get => My_Method_Get_Procedure
       Cooling_Radius_Growth_Rate_Get => My_Method_Growth_Rate_Get_Procedure
    end if
    return
  end subroutine Method_Initialize
\end{verbatim}
where {\tt myMethod} is the name of this method as will be specified by the {\tt coolingRadiusMethod} input parameter. The procedure pointer {\tt Cooling\_Radius\_Get} must be set to point to a function which returns the cooling function as described below while {\tt Cooling\_Radius\_Growth\_Rate\_Get} should be set to point to a function which returns the rate at which the cooling radius is growing. The initialization subroutine should perform any other tasks required to initialize the module (such as reading parameters etc.).

The cooling radius function must have the form:
\begin{verbatim}
 double precision function Cooling_Radius_Get(thisNode)
    implicit none
    type(treeNode), intent(inout), pointer :: thisNode
    .
    .
    .
    return
 end function Cooling_Radius_Get
\end{verbatim}
The function must return the cooling radius (in units of Mpc) for {\tt thisNode}. The cooling radius growth rate function should have the same template but return the rate at which the cooling radius grows in units of Mpc/Gyr.

Currently defined cooling radius methods are:
\begin{description}
 \item [\hyperlink{cooling.cooling_radius.simple.F90:cooling_radii_simple:cooling_radius_simple}{{\tt simple}}] Computes the cooling radius by seeking the radius at which the time available for cooling equals the cooling time. The growth rate is determined consistently based on the slope of the density profile, the density dependence of the cooling function and the rate at which the time available for cooling is increasing. This method assumes that the cooling time is a monotonic function of radius.
 \item [\hyperlink{cooling.cooling_radius.isothermal_profile.F90:cooling_radii_isothermal:cooling_radius_isothermal}{{\tt isothermal}}] Computes the cooling radius by assuming that the hot gas density profile is an isothermal profile ($\rho(r) \propto r^{-2}$), and that the cooling rate scales as density squared, $\dot{E}\propto \rho^2$, such that the cooling time scales as inverse density, $t_{\rm cool} \propto \rho^{-1}$. Consequently, the cooling radius grows as the square root of the time available for cooling.
\end{description}

\subsubsection{Cooling: Freefall Radius}

Additional methods for freefall radius in cooling calculations can be added using the {\tt freefallRadiusMethod} directive. The directive should contain a single argument, giving the name of a subroutine to be called to initialize the method. For example, the {\tt simple} method is described by a directive:
\begin{verbatim}
 !# <freefallRadiusMethod>
 !#  <unitName>Freefall_Radius_Dark_Matter_Halo_Initialize</unitName>
 !# </freefallRadiusMethod>
\end{verbatim}
Here, {\tt Freefall\_Radius\_Dark\_Matter\_Halo\_Initialize} is the name of a subroutine which will be called to initialize the method. The initialization subroutine must have the following form:
\begin{verbatim}
  subroutine Method_Initialize(freefallRadiusMethod,Freefall_Radius_Get,Freefall_Radius_Growth_Rate_Get)
    implicit none
    type(varying_string),          intent(in)    :: freefallRadiusMethod
    procedure(),          pointer, intent(inout) :: Freefall_Radius_Get,Freefall_Radius_Growth_Rate_Get
    
    if (freefallRadiusMethod == 'myMethod') then
       Freefall_Radius_Get             => My_Method_Get_Procedure
       Freefall_Radius_Growth_Rate_Get => My_Method_Growth_Rate_Get_Procedure
    end if
    return
  end subroutine Method_Initialize
\end{verbatim}
where {\tt myMethod} is the name of this method as will be specified by the {\tt freefallRadiusMethod} input parameter. The procedure pointer {\tt Freefall\_Radius\_Get} must be set to point to a function which returns the freefall radius as described below while {\tt Freefall\_Radius\_Growth\_Rate\_Get} should be set to point to a function which returns the rate at which the freefall radius is growing. The initialization subroutine should perform any other tasks required to initialize the module (such as reading parameters etc.).

The freefall radius function must have the form:
\begin{verbatim}
 double precision function Freefall_Radius_Get(thisNode)
    implicit none
    type(treeNode), intent(inout), pointer :: thisNode
    .
    .
    .
    return
 end function Freefall_Radius_Get
\end{verbatim}
The function must return the freefall radius (in units of Mpc) for {\tt thisNode}. The freefall radius growth rate function should have the same template but return the rate at which the freefall radius grows in units of Mpc/Gyr.

Currently defined freefall radius methods are:
\begin{description}
 \item [\hyperlink{cooling.freefall_radii.dark_matter_halo.F90:freefall_radii_dark_matter_halo:freefall_radius_dark_matter_halo}{{\tt darkMatterHalo}}] Computes the freefall radius by finding the radius in the dark matter halo profile from which a test particle could have free-fallen to zero radius (assuming it began at rest) in the time available for freefall.
\end{description}

\subsubsection{Cooling: Infall Radius}

Additional methods for the infall radius in cooling calculations can be added using the {\tt infallRadiusMethod} directive. The directive should contain a single argument, giving the name of a subroutine to be called to initialize the method. For example, the {\tt coolingRadius} method is described by a directive:
\begin{verbatim}
 !# <infallRadiusMethod>
 !#  <unitName>Infall_Radius_Cooling_Radius_Initialize</unitName>
 !# </infallRadiusMethod>
\end{verbatim}
Here, {\tt Infall\_Radius\_Cooling\_Radius\_Initialize} is the name of a subroutine which will be called to initialize the method. The initialization subroutine must have the following form:
\begin{verbatim}
  subroutine Method_Initialize(infallRadiusMethod,Infall_Radius_Get,Infall_Radius_Growth_Rate_Get)
    implicit none
    type(varying_string),          intent(in)    :: infallRadiusMethod
    procedure(),          pointer, intent(inout) :: Infall_Radius_Get,Infall_Radius_Growth_Rate_Get
    
    if (infallRadiusMethod == 'myMethod') then
       Infall_Radius_Get             => My_Method_Get_Procedure
       Infall_Radius_Growth_Rate_Get => My_Method_Growth_Rate_Get_Procedure
    end if
    return
  end subroutine Method_Initialize
\end{verbatim}
where {\tt myMethod} is the name of this method as will be specified by the {\tt infallRadiusMethod} input parameter. The procedure pointer {\tt Infall\_Radius\_Get} must be set to point to a function which returns the infall radius as described below while {\tt Infall\_Radius\_Growth\_Rate\_Get} should be set to point to a function which returns the rate at which the infall radius is growing. The initialization subroutine should perform any other tasks required to initialize the module (such as reading parameters etc.).

The infall radius function must have the form:
\begin{verbatim}
 double precision function Infall_Radius_Get(thisNode)
    implicit none
    type(treeNode), intent(inout), pointer :: thisNode
    .
    .
    .
    return
 end function Infall_Radius_Get
\end{verbatim}
The function must return the infall radius (in units of Mpc) for {\tt thisNode}, i.e. the radius from which gas in the hot halo that is currently accreting onto the galaxy originated. The infall radius growth rate function should have the same template but return the rate at which the infall radius grows in units of Mpc/Gyr.

Currently defined infall radius methods are:
\begin{description}
 \item [\hyperlink{cooling.infall_radius.cooling_radius.F90:infall_radii_cooling_radius:infall_radius_cooling_radius}{{\tt coolingRadius}}] Assumes that the infall radius equals the cooling radius.
 \item [\hyperlink{cooling.infall_radius.cooling_and_freefall.F90:infall_radii_cooling_freefall:infall_radius_cooling_freefall}{{\tt cooling and freefall}}] Assumes that the infall radius is equal to the smaller of the cooling and freefall radii.
\end{description}

\subsubsection{Cooling Specific Angular Momentum}

Additional methods for calculations of the specific angular momentum of cooling gas can be added using the {\tt coolingSpecificAngularMomentumMethod} directive. The directive should contain a single argument, giving the name of a subroutine to be called to initialize the method. For example, the {\tt simple} method is described by a directive:
\begin{verbatim}
 !# <coolingSpecificAngularMomentumMethod>
 !#  <unitName>Cooling_Specific_AM_Constant_Rotation_Initialize</unitName>
 !# </coolingSpecificAngularMomentumMethod>
\end{verbatim}
Here, {\tt Cooling\_Time\_Simple\_Initialize} is the name of a subroutine which will be called to initialize the method. The initialization subroutine must have the following form:
\begin{verbatim}
  subroutine Method_Initialize(coolingSpecificAngularMomentumMethod,Cooling_Specific_Angular_Momentum_Get)
    implicit none
    type(varying_string),          intent(in)    :: coolingSpecificAngularMomentumMethod
    procedure(),          pointer, intent(inout) :: Cooling_Specific_Angular_Momentum_Get
    
    if (coolingSpecificAngularMomentumMethod == 'myMethod') then
       Cooling_Specific_Angular_Momentum_Get => My_Method_Get_Procedure
    end if
    return
  end subroutine Method_Initialize
\end{verbatim}
where {\tt myMethod} is the name of this method as will be specified by the {\tt coolingSpecificAngularMomentumMethod} input parameter. The procedure pointer {\tt Cooling\_Specific\_Angular\_Momentum\_Get} must be set to point to a function which returns the specific angular momentum of cooling gas. The initialization subroutine should perform any other tasks required to initialize the module (such as reading parameters etc.).

The specific angular momentum of cooling gas function must have the form:
\begin{verbatim}
 double precision function Cooling_Specific_Angular_Momentum_Get(thisNode)
    implicit none
    type(treeNode), intent(inout), pointer :: thisNode
    .
    .
    .
    return
 end function Cooling_Specific_Angular_Momentum_Get
\end{verbatim}
The function must return the specific angular momentum (in units of km/s Mpc) of gas that is cooling in {\tt thisNode}.

Currently defined specific angular momentum of cooling gas methods are:
\begin{description}
 \item [\hyperlink{cooling.specific_angular_momentum.constant_rotation.F90:cooling_specific_angular_momenta_constant_rotation:cooling_specific_angular_momentum_constant_rotation}{{\tt constantRotation}}] Computes the specific angular momentum of the cooling gas based on the cooling radius, mean specific angular momentum and the assumption of a constant mean rotation speed in the cooling gas as a function of radius.
 \item [\hyperlink{cooling.specific_angular_momentum.mean.F90:cooling_specific_angular_momenta_mean:cooling_specific_angular_momentum_mean}{{\tt mean}}] Assumes that the specific angular momentum of the cooling gas always equals the mean specific angular momentum of the hot halo.
\end{description}

\subsubsection{Cooling Time Available}

Additional methods for the time available for cooling can be added using the {\tt coolingTimeAvailableMethod} directive. The directive should contain a single argument, giving the name of a subroutine to be called to initialize the method. For example, the {\tt White-Frenk} method is described by a directive:
\begin{verbatim}
 !# <coolingTimeAvailableMethod>
 !#  <unitName>Cooling_Time_Available_WF_Initialize</unitName>
 !# </coolingTimeAvailableMethod>
\end{verbatim}
Here, {\tt Cooling\_Time\_Available\_WF\_Initialize} is the name of a subroutine which will be called to initialize the method. The initialization subroutine must have the following form:
\begin{verbatim}
  subroutine Method_Initialize(coolingTimeAvailableMethod,Cooling_Time_Available_Get&
       &,Cooling_Time_Available_Increase_Rate_Get)
    implicit none
    type(varying_string),          intent(in)    :: coolingTimeAvailableMethod
    procedure(),          pointer, intent(inout) :: Cooling_Time_Available_Get,Cooling_Time_Available_Increase_Rate_Get
    
    if (coolingTimeAvailableMethod == 'myMethod') then
      Cooling_Time_Available_Get               => My_Method_Get
      Cooling_Time_Available_Increase_Rate_Get => My_Method_Increase_Rate_Get
    end if
    return
  end subroutine Method_Initialize
\end{verbatim}
where {\tt myMethod} is the name of this method as will be specified by the {\tt coolingTimeAvailableMethod} input parameter. The procedure pointers {\tt Cooling\_Time\_Available\_Get} and {\tt Cooling\_Time\_Available\_Increase\_Rate\_Get} must be set to point to functions which return the time available for cooling and the rate of increase of this time respectively. The initialization subroutine should perform any other tasks required to initialize the module (such as reading parameters etc.).

The cooling time available functions must have the form:
\begin{verbatim}
 double precision function Cooling_Time_Available_Get(thisNode)
    implicit none
    type(treeNode),   intent(inout), pointer :: thisNode
    .
    .
    .
    return
 end function Cooling_Time_Available_Get
\end{verbatim}
The first function must return the time available for cooling (in units of Gyr) for {\tt thisNode}, while the second must return the rate of increase of this time. 

Currently defined cooling time available methods are:
\begin{description}
 \item [\hyperlink{cooling.time_available.White-Frenk.F90:cooling_time_available_white_frenk:cooling_time_available_wf}{{\tt White-Frenk}}] The time available is set to a value between the age of the Universe and the dynamical time of the halo, depending on the interpolating parameter {\tt [coolingTimeAvailableAgeFactor]};
 \item [\hyperlink{cooling.time_available.halo_formation.F90:cooling_times_available_halo_formation:cooling_time_available_halo_formation}{{\tt haloFormation}}] The time available for cooling is set equal to the current time minus the formation time of the halo.
\end{description}

\subsubsection{Cooling Time Available For Freefall}

Additional methods for the time available for freefall in cooling calculations can be added using the {\tt freefallTimeAvailableMethod} directive. The directive should contain a single argument, giving the name of a subroutine to be called to initialize the method. For example, the {\tt haloFormation} method is described by a directive:
\begin{verbatim}
 !# <freefallTimeAvailableMethod>
 !#  <unitName>Freefall_Time_Available_Halo_Formation_Initialize</unitName>
 !# </freefallTimeAvailableMethod>
\end{verbatim}
Here, {\tt Freefall\_Time\_Available\_Halo\_Formation\_Initialize} is the name of a subroutine which will be called to initialize the method. The initialization subroutine must have the following form:
\begin{verbatim}
  subroutine Method_Initialize(freefallTimeAvailableMethod,Freefall_Time_Available_Get&
       &,Freefall_Time_Available_Increase_Rate_Get)
    implicit none
    type(varying_string),          intent(in)    :: freefallTimeAvailableMethod
    procedure(),          pointer, intent(inout) :: Freefall_Time_Available_Get,Freefall_Time_Available_Increase_Rate_Get
    
    if (freefallTimeAvailableMethod == 'myMethod') then
      Freefall_Time_Available_Get               => My_Method_Get
      Freefall_Time_Available_Increase_Rate_Get => My_Method_Increase_Rate_Get
    end if
    return
  end subroutine Method_Initialize
\end{verbatim}
where {\tt myMethod} is the name of this method as will be specified by the {\tt freefallTimeAvailableMethod} input parameter. The procedure pointers {\tt Freefall\_Time\_Available\_Get} and {\tt Freefall\_Time\_Available\_Increase\_Rate\_Get} must be set to point to functions which return the time available for freefal in cooling calculations and the rate of increase of this time respectively. The initialization subroutine should perform any other tasks required to initialize the module (such as reading parameters etc.).

The freefall time available functions must have the form:
\begin{verbatim}
 double precision function Freefall_Time_Available_Get(thisNode)
    implicit none
    type(treeNode),   intent(inout), pointer :: thisNode
    .
    .
    .
    return
 end function Freefall_Time_Available_Get
\end{verbatim}
The first function must return the time available for freefall cooling calculations (in units of Gyr) for {\tt thisNode}, while the second must return the rate of increase of this time. 

Currently defined freefall time available methods are:
\begin{description}
 \item [\hyperlink{cooling.freefall_time_available.halo_formation.F90:freefall_times_available_halo_formation:freefall_time_available_halo_formation}{{\tt haloFormation}}] The time available for cooling is set equal to the current time minus the formation time of the halo.
\end{description}

\subsubsection{Cooling Time}

Additional methods for cooling time calculations can be added using the {\tt coolingTimeMethod} directive. The directive should contain a single argument, giving the name of a subroutine to be called to initialize the method. For example, the {\tt simple} method is described by a directive:
\begin{verbatim}
 !# <coolingTimeMethod>
 !#  <unitName>Cooling_Time_Simple_Initialize</unitName>
 !# </coolingTimeMethod>
\end{verbatim}
Here, {\tt Cooling\_Time\_Simple\_Initialize} is the name of a subroutine which will be called to initialize the method. The initialization subroutine must have the following form:
\begin{verbatim}
  subroutine Method_Initialize(coolingTimeMethod,Cooling_Time_Get,Cooling_Time_Density_Log_Slope_Get,Cooling_Time_Temperature_Log_Slope_Get)
    implicit none
    type(varying_string),          intent(in)    :: coolingTimeMethod
    procedure(),          pointer, intent(inout) :: Cooling_Time_Get,Cooling_Time_Density_Log_Slope_Get,Cooling_Time_Temperature_Log_Slope_Get
    
    if (coolingTimeMethod == 'myMethod') then
       Cooling_Time_Get => My_Method_Get_Procedure
       Cooling_Time_Density_Log_Slope_Get     => My_Method_Density_Log_Slope_Procedure
       Cooling_Time_Temperature_Log_Slope_Get => My_Method_Temperature_Log_Slope_Procedure
    end if
    return
  end subroutine Method_Initialize
\end{verbatim}
where {\tt myMethod} is the name of this method as will be specified by the {\tt coolingTimeMethod} input parameter. The procedure pointer {\tt Cooling\_Time\_Get} must be set to point to a function which returns the cooling function as described below while the other two pointers should point to functions which return the appropriate logarithmic slope. The initialization subroutine should perform any other tasks required to initialize the module (such as reading parameters etc.).

The cooling time function must have the form:
\begin{verbatim}
 double precision function Cooling_Time_Get(temperature,density,abundances,chemicalDensities,radiation)
    implicit none
    double precision,                  intent(in) :: temperature,density
    type(abundancesStructure),         intent(in) :: abundances
    type(chemicalAbundancesStructure), intent(in) :: chemicalDensities
    type(radiationStructure),          intent(in) :: radiation
    .
    .
    .
    return
 end function Cooling_Time_Get
\end{verbatim}
The function must return the cooling time (in units of Gyr) for at the specified {\tt temperature}, {\tt density} and for composition and radiation field as specified by the {\tt abundances}, {\tt chemicalDensities} and {\tt radiation} structures. The logarithmic slope functions should have the same template, but return the logarithmic slope of the cooling time with respect to the appropriate variable instead.

Currently defined cooling time methods are:
\begin{description}
 \item [\hyperlink{cooling.cooling_time.simple.F90:cooling_times_simple:cooling_time_simple}{{\tt simple}}] Compute the cooling time as the ratio of the gas thermal energy density to the volume rate of radiative energy loss. The gas is assumed to have an effective number of degrees of freedom specified by the {\tt coolingTimeSimpleDegreesOfFreedom} parameter.
\end{description}

\subsubsection{Cosmological Mass Root Variance}

Additional methods for computing the cosmological mass root variance, $\sigma(M)$, can be added using the {\tt cosmologicalMassVarianceMethod} directive. The directive should contain a single argument, giving the name of a subroutine to be called to initialize the method. For example, the {\tt filteredPowerSpectrum} method is described by a directive:
\begin{verbatim}
  !# <cosmologicalMassVarianceMethod>
  !#  <unitName>Cosmological_Mass_Variance_Filtered_Power_Spectrum_Initialize</unitName>
  !# </cosmologicalMassVarianceMethod>
\end{verbatim}
Here, {\tt Cosmological\_Mass\_Variance\_Filtered\_Power\_Spectrum\_Initialize} is the name of a subroutine which will be called to initialize the method. The initialization subroutine must have the following form:
\begin{verbatim}
  subroutine Method_Initialize(cosmologicalMassVarianceMethod,Cosmological_Mass_Variance_Tabulate)
    implicit none
    type     (varying_string),          intent(in   ) :: cosmologicalMassVarianceMethod
    procedure(               , pointer, intent(inout) :: Cosmological_Mass_Variance_Tabulate
    
    if (cosmologicalMassVarianceMethod == 'myMethod') then
       Cosmological_Mass_Variance_Tabulate => My_Method_Tabulate
       .
       .
       .
    end if
    return
  end subroutine Method_Initialize
\end{verbatim}
where {\tt myMethod} is the name of this method as will be specified by the {\tt cosmologicalMassVarianceMethod} input parameter. The procedure pointer {\tt Cosmological\_Mass\_Variance\_Tabulate} must be set to point to a function which populates a {\tt table1D} object with a tabulation of $\sigma(M)$. The initialization subroutine should perform any other tasks required to initialize the module (such as reading parameters etc.).

The tabulation function must have the form:
\begin{verbatim}
   subroutine Cosmological_Mass_Variance_Filtered_Power_Spectrum(mass,massNormalization,sigmaNormalization,sigmaTable)
    implicit none
    double precision         , intent(in   )              :: mass,massNormalization
    double precision         , intent(inout)              :: sigmaNormalization
    class           (table1D), intent(inout), allocatable :: sigmaTable
    .
    .
    return
   end subroutine Cosmological_Mass_Variance_Filtered_Power_Spectrum
\end{verbatim}
The function should allocate {\tt sigmaTable} to a suitable type of {\tt table1D} object and populate it with a tabulation of $\sigma(M)$ which includes the given {\tt mass}. On input, the required normalization of $\sigma(M)$ at mass {\tt massNormalization} is given by {\tt sigmaNormalization}. The function should divide this value by the unnormalized value of $\sigma(M)$---this is used to normalize the cosmological power spectrum.

Currently defined cosmological mass root variance methods are:
\begin{description}
 \item [{\tt filteredPowerSpectrum}] The mass root variance is found by integrating over the transferred linear power spectrum multiplied by the selected window function (see \S\ref{sec:PowerSpectrumWindowFunction}).
\end{description}

\subsubsection{Critical Overdensity for Halo Collapse}

Additional methods for the critical linear theory overdensity for halo collapse can be added using the {\tt criticalOverdensityMethod} directive. The directive should contain a single argument, giving the name of a subroutine to be called to initialize the method. For example, the {\tt sphericalTopHat} method is described by a directive:
\begin{verbatim}
  !# <criticalOverdensityMethod>
  !#  <unitName>Spherical_Collape_Delta_Critical_Initialize</unitName>
  !# </criticalOverdensityMethod>
\end{verbatim}
Here, {\tt Spherical\_Collape\_Delta\_Critical\_Initialize} is the name of a subroutine which will be called to initialize the method. The initialization subroutine must have the following form:
\begin{verbatim}
  subroutine Method_Initialize(criticalOverdensityMethod,Critical_Overdensity_Tabulate)
    implicit none
    type(varying_string),          intent(in)    :: criticalOverdensityMethod
    procedure(),          pointer, intent(inout) :: Critical_Overdensity_Tabulate
    
    if (criticalOverdensityMethod.eq.'myMethod') then
       Critical_Overdensity_Tabulate => My_Do_Tabulate
       .
       .
       .
    end if
    return
  end subroutine Method_Initialize
\end{verbatim}
where {\tt myMethod} is the name of this method as will be specified by the {\tt criticalOverdensityMethod} input parameter. The procedure pointer {\tt Critical\_Overdensity\_Tabulate} must be set to point to a subroutine which tabulates the critical overdensity as described below. The initialization subroutine should perform any other tasks required to initialize the module (such as reading parameters etc.).

The tabulation subroutine must have the form:
\begin{verbatim}
   subroutine Critical_Overdensity_Tabulate(time,deltaCritNumberPoints,deltaCritTime,deltaCritDeltaCrit)
    implicit none
    double precision, intent(in)                               :: time
    integer,          intent(out)                              :: deltaCritNumberPoints
    double precision, intent(inout), allocatable, dimension(:) :: deltaCritTime,deltaCritDeltaVirial
    .
    .
    .
    return
   end subroutine Critical_Overdensity_Tabulate
\end{verbatim}
The subroutine must tabulate the critical overdensity in array {\tt deltaCritDeltaVirial()} as a function of wavenumber {\tt deltaCritTime()} (these arrays must be allocated to the correct size, and may be prevously allocated, therefore requiring a deallocation). The number of tabulated points should be returned in {\tt deltaCritNumberPoints}. The subroutine should ensure that the currently requested {\tt time} is within the range of the tabulated function (preferably with some buffer).

Currently defined critical overdensity methods are:
\begin{description}
 \item [{\tt sphericalTopHat}] The critical overdensity is computed for a Universe containing collisionless matter and a cosmological constant following the spherical top hat collapse model (see, for example, \citealt{percival_cosmological_2005}).
 \item [{\tt Kitayama-Suto1996}] The critical overdensity is computed using the fitting formula of \cite{kitayama_semianalytic_1996}, and is therefore valid only for flat cosmological models.
\end{description}

\subsubsection{Critical Overdensity for Halo Collapse: Mass Scaling}

Additional methods for the mass scaling of the critical linear theory overdensity for halo collapse can be added using the {\tt criticalOverdensityMassScalingMethod} directive. The directive should contain a single argument, giving the name of a subroutine to be called to initialize the method. For example, the {\tt warm dark matter} method is described by a directive:
\begin{verbatim}
  !# <criticalOverdensityMassScalingMethod>
  !#  <unitName>Critical_Overdensity_Mass_Scaling_WDM_Initialize</unitName>
  !# </criticalOverdensityMassScalingMethod>
\end{verbatim}
Here, {\tt Critical\_Overdensity\_Mass\_Scaling\_WDM\_Initialize} is the name of a subroutine which will be called to initialize the method. The initialization subroutine must have the following form:
\begin{verbatim}
  subroutine Method_Initialize(criticalOverdensityMassScalingMethod, &
   & Critical_Overdensity_Mass_Scaling_Get,Critical_Overdensity_Mass_Scaling_Gradient_Get)
    implicit none
    type(varying_string),          intent(in)    :: criticalOverdensityMassScalingMethod
    procedure(),          pointer, intent(inout) :: Critical_Overdensity_Mass_Scaling_Get, &
   &                                                Critical_Overdensity_Mass_Scaling_Gradient_Get
    
    if (criticalOverdensityMassScalingMethod == 'myMethod') then
       Critical_Overdensity_Mass_Scaling_Get          => My_Do_Tabulate
       Critical_Overdensity_Mass_Scaling_Gradient_Get => My_Do_Gradient_Tabulate
       .
       .
       .
    end if
    return
  end subroutine Method_Initialize
\end{verbatim}
where {\tt myMethod} is the name of this method as will be specified by the {\tt criticalOverdensityMassScalingMethod} input parameter. The procedure pointers {\tt Critical\_Overdensity\_Mass\_Scaling\_Get} and {\tt Critical\_Overdensity\_Mass\_Scaling\_Gradient\_Get} must be set to point to functions which return the critical overdensity mass scaling and its gradient as described below. The initialization subroutine should perform any other tasks required to initialize the module (such as reading parameters etc.).

The mass scaling function must have the form:
\begin{verbatim}
   double precision function Critical_Overdensity_Mass_Scaling_Get(mass)
    implicit none
    double precision, intent(in) :: mass
    .
    .
    .
    return
   end function Critical_Overdensity_Mass_Scaling_Get
\end{verbatim}
The function should return the factor by which the critical overdensity for collapse at the given {\tt mass} scale (given in units of $M_{\odot}$) differs from that for the case $M\rightarrow\infty$. The mass scaling gradient function should have the same form, but should return the derivative of the scaling with respect to mass.

Currently defined critical overdensity mass scaling methods are:
\begin{description}
 \item [{\tt null}] The critical overdensity is assumed to have no scaling with mass;
 \item [{\tt warm dark matter}] The mass scaling is computed for warm dark matter using a fitting function to the results of \cite{barkana_constraints_2001}.
\end{description}

\subsubsection{Dark Matter Density Profile}

Additional methods for the dark matter density profile can be added using the {\tt darkMatterProfileMethod} directive. The directive should contain a single argument, giving the name of a subroutine to be called to initialize the method. For example, the {\tt NFW} method is described by a directive:
\begin{verbatim}
 !# <darkMatterProfileMethod>
 !#  <unitName>Dark_Matter_Profile_NFW_Initialize</unitName>
 !# </darkMatterProfileMethod>
\end{verbatim}
Here, {\tt Dark\_Matter\_Profile\_NFW\_Initialize} is the name of a subroutine which will be called to initialize the method. The initialization subroutine must have the following form:
\begin{verbatim}
  subroutine Method_Initialize(darkMatterProfileMethod,Dark_Matter_Profile_Density_Get,Dark_Matter_Profile_Energy_Get&
       & ,Dark_Matter_Profile_Energy_Growth_Rate_Get,Dark_Matter_Profile_Rotation_Normalization_Get &
       & ,Dark_Matter_Profile_Radius_from_Specific_Angular_Momentum_Get,Dark_Matter_Profile_Circular_Velocity_Get&
       & ,Dark_Matter_Profile_Potential_Get,Dark_Matter_Profile_Enclosed_Mass_Get,Dark_Matter_Profile_kSpace_Get&
       & ,Dark_Matter_Profile_Freefall_Radius_Get ,Dark_Matter_Profile_Freefall_Radius_Increase_Rate_Get)
    implicit none
    type(varying_string),          intent(in)    :: darkMatterProfileMethod
    procedure(),          pointer, intent(inout) :: Dark_Matter_Profile_Density_Get,Dark_Matter_Profile_Energy_Get,Dark_Matter_Profile_Energy_Growth_Rate_Get&
         &,Dark_Matter_Profile_Rotation_Normalization_Get,Dark_Matter_Profile_Radius_from_Specific_Angular_Momentum_Get&
         &,Dark_Matter_Profile_Circular_Velocity_Get,Dark_Matter_Profile_Potential_Get,Dark_Matter_Profile_Enclosed_Mass_Get&
         &,Dark_Matter_Profile_kSpace_Get,Dark_Matter_Profile_Freefall_Radius_Get&
         &,Dark_Matter_Profile_Freefall_Radius_Increase_Rate_Get

    if (darkMatterProfileMethod == 'myMethod') then
       Dark_Matter_Profile_Density_Get                               => My_Dark_Matter_Profile_Density
       Dark_Matter_Profile_Energy_Get                                => My_Dark_Matter_Profile_Energy
       Dark_Matter_Profile_Energy_Growth_Rate_Get                    => My_Dark_Matter_Profile_Energy_Growth_Rate
       Dark_Matter_Profile_Rotation_Normalization_Get                => My_Dark_Matter_Profile_Rotation_Normalization
       Dark_Matter_Profile_Radius_from_Specific_Angular_Momentum_Get => My_Radius_from_Specific_Angular_Momentum
       Dark_Matter_Profile_Circular_Velocity_Get                     => My_Dark_Matter_Profile_Circular_Velocity
       Dark_Matter_Profile_Potential_Get                             => My_Dark_Matter_Profile_Potential
       Dark_Matter_Profile_Enclosed_Mass_Get                         => My_Dark_Matter_Profile_Enclosed_Mass
       Dark_Matter_Profile_kSpace_Get                                => My_Dark_Matter_Profile_kSpace
       Dark_Matter_Profile_Freefall_Radius_Get                       => My_Dark_Matter_Profile_Freefall_Radius
       Dark_Matter_Profile_Freefall_Radius_Increase_Rate_Get         => My_Dark_Matter_Profile_Freefall_Radius_Increase_Rate
       .
       .
       .
    end if
    return
  end subroutine Method_Initialize
\end{verbatim}
where {\tt myMethod} is the name of this method as will be specified by the {\tt darkMatterProfileMethod} input parameter. The procedure pointers must be set to point to functions which return properties of the dark matter density profile as described below. The initialization subroutine should perform any other tasks required to initialize the module (such as reading parameters etc.).

The density, enclosed mass, potential and circular velocity functions must have the form:
\begin{verbatim}
  double precision function My_Dark_Matter_Profile_Property(thisNode,radius)
    implicit none
    type(treeNode),   intent(inout), pointer :: thisNode
    double precision, intent(in)             :: radius
    .
    .
    .
    return
  end function My_Dark_Matter_Profile_Property
\end{verbatim}
These functions should compute and return the density (in units of $M_\odot$ Mpc$^{-3}$), enclosed dark matter mass (in units of $M_\odot$), gravitational potential (in units of (km/s)$^2$) and circular velocity (in units of km/s) due to dark matter at the given {\tt radius} (in units of Mpc) for {\tt thisNode} respectively.

The freefall radius functions must have the form:
\begin{verbatim}
  double precision function My_Dark_Matter_Profile_Property(thisNode,time)
    implicit none
    type(treeNode),   intent(inout), pointer :: thisNode
    double precision, intent(in)             :: time
    .
    .
    .
    return
  end function My_Dark_Matter_Profile_Property
\end{verbatim}
These functions should compute and return the freefall radius (in units of Mpc), or its growth rate given a {\tt time} available for freefall (in Gyr) for {\tt thisNode} respectively.

The energy, energy growth rate and rotation velocity normalization functions must have the form:
\begin{verbatim}
  double precision function My_Dark_Matter_Profile_Property(thisNode)
    implicit none
    type(treeNode),   intent(inout), pointer :: thisNode
    .
    .
    .
    return
  end function My_Dark_Matter_Profile_Property
\end{verbatim}
The energy functions should compute and return the energy (potential plus kinetic; in units of $M_\odot$ (km/s)$^2$) of the dark matter halo out to the virial radius of {\tt thisNode} and the rate of change of that energy (in units of $M_\odot$ (km/s)$^2$ Gyr$^{-1}$) respectively. The rotation normalization function should compute and return the normalization between rotation speed and mean specific angular momentum (in units of Mpc$^{-1}$) of {\tt thisNode} assuming that the dark matter halo rotates at the same velocity at all radii.

Finally, the radius from specific angular momentum function must have the form:
\begin{verbatim}
  double precision function My_Dark_Matter_Profile_Radius_From_Specific_Angular_Momentum(thisNode,specificAngularMomentum)
    implicit none
    type(treeNode),   intent(inout), pointer :: thisNode
    double precision, intent(in)             :: specificAngularMomentum
    .
    .
    .
    return
  end function My_Dark_Matter_Profile_Radius_From_Specific_Angular_Momentum
\end{verbatim}
This function should compute and return the radius (in units of Mpc) in {\tt thisNode} at which a circular orbit would have the given {\tt specificAngularMomentum} (in units of km/s Mpc).

The ``kSpace'' function must have the form:
\begin{verbatim}
  double precision function My_Dark_Matter_Profile_kSpace(thisNode,wavenumber)
    implicit none
    type(treeNode),   intent(inout), pointer :: thisNode
    double precision, intent(in)             :: wavenumber
    .
    .
    .
    return
  end function My_Dark_Matter_Profile_kSpace
\end{verbatim}
This function should compute and return the Fourier transform of the dark matter halo density profile (normalized to unity at small wavenumber---as defined in \citealt{cooray_halo_2002}) for {\tt thisNode} at the given {\tt wavenumber} (specified in Mpc$^{-1}$).

Currently defined dark matter density profile methods are:
\begin{description}
 \item [{\tt Isothermal}] The density profile is a singular isothermal sphere;
 \item [{\tt NFW}] The density profile proposed by \cite{navarro_universal_1997}\index{Navarro-Frenk-White profile}\index{density profile!Navarro-Frenk-White}.
 \item [{\tt Einasto}] The Einasto density profile, described, for example, by \cite{cardone_spherical_2005}\index{Einasto profile}\index{density profile!Einasto}.
\end{description}

\subsubsection{Dark Matter Halo Mass Accretion History}\index{dark matter halo!mass accretion history}\index{mass accretion history!dark matter halo}\label{sec:HaloMassAccretionHistory}

Additional methods for dark matter halo mass accretion histories can be added using the {\tt darkMatterAccretionHistoryMethod} directive. The directive should contain a single argument, giving the name of a subroutine to be called to initialize the method. For example, the {\tt Zhao2009} method is described by a directive:
\begin{verbatim}
 !# <darkMatterAccretionHistoryMethod>
 !#  <unitName>Dark_Matter_Mass_Accretion_Zhao2009_Initialize</unitName>
 !# </darkMatterAccretionHistoryMethod>
\end{verbatim}
Here, {\tt Dark\_Matter\_Mass\_Accretion\_Zhao2009\_Initialize} is the name of a subroutine which will be called to initialize the method. The initialization subroutine must have the following form:
\begin{verbatim}
  subroutine Method_Initialize(darkMatterAccretionHistoryMethod,Dark_Matter_Halo_Mass_Accretion_Time_Get)
    implicit none
    type(varying_string),          intent(in)    :: darkMatterAccretionHistoryMethod
    procedure(),          pointer, intent(inout) :: Dark_Matter_Halo_Mass_Accretion_Time_Get
    
    if (darkMatterAccretionHistoryMethod == 'myMethod') then
       Dark_Matter_Halo_Mass_Accretion_Time_Get => My_Time_Get
       .
       .
       .
    end if
    return
  end subroutine Method_Initialize
\end{verbatim}
where {\tt myMethod} is the name of this method as will be specified by the {\tt darkMatterAccretionHistoryMethod} input parameter. The procedure pointer {\tt Dark\_Matter\_Halo\_Mass\_Accretion\_Time\_Get} must be set to point to a function which returns the time at which a given mass is reached in the mass accretion history. The initialization subroutine should perform any other tasks required to initialize the module (such as reading parameters etc.).

The time function must have the form:
\begin{verbatim}
  double precision function My_Dark_Matter_Halo_Mass_Accretion_Time(baseNode,nodeMass)
    implicit none
    type(treeNode),   intent(inout), pointer :: baseNode
    double precision, intent(in)             :: nodeMass
    .
    .
    .
    return
  end function My_Dark_Matter_Halo_Mass_Accretion_Time
\end{verbatim}
The function should compute and return the time at which the mass accretion history of {\tt baseNode} reaches the specified {\tt nodeMass}.

Currently defined mass accretion history methods are:
\begin{description}
 \item [{\tt Wechsler2002}] Uses the fitting function from \cite{wechsler_concentrations_2002} to compute the mass accretion history. If {\tt [accretionHistoryWechslerFormationRedshiftCompute]} is set to true then the formation redshift for each history is set sing the method of \cite{bullock_profiles_2001}, otherwise it can be set directly via the {\tt [accretionHistoryWechslerFormationRedshift]} parameter;
 \item [{\tt Zhao2009}] Uses the algorithm of \cite{zhao_accurate_2009} to compute the mass accretion history.
\end{description}

\subsubsection{Dark Matter Density Profile Concentration}

Additional methods for the dark matter density profile concentration can be added using the {\tt darkMatterConcentrationMethod} directive. The directive should contain a single argument, giving the name of a subroutine to be called to initialize the method. For example, the {\tt Gao2008} method is described by a directive:
\begin{verbatim}
 !# <darkMatterConcentrationMethod>
 !#  <unitName>Dark_Matter_Concentrations_Gao20008_Initialize</unitName>
 !# </darkMatterConcentrationMethod>
\end{verbatim}
Here, {\tt Dark\_Matter\_Concentrations\_Gao20008\_Initialize} is the name of a subroutine which will be called to initialize the method. The initialization subroutine must have the following form:
\begin{verbatim}
  subroutine Method_Initialize(darkMatterConcentrationMethod,Dark_Matter_Profile_Concentration_Get)
    implicit none
    type(varying_string),          intent(in)    :: darkMatterConcentrationMethod
    procedure(),          pointer, intent(inout) :: Dark_Matter_Profile_Concentration_Get
    
    if (darkMatterConcentrationMethod == 'myMethod') then
       Dark_Matter_Profile_Concentration_Get => My_Concentration_Get
       .
       .
       .
    end if
    return
  end subroutine Method_Initialize
\end{verbatim}
where {\tt myMethod} is the name of this method as will be specified by the {\tt darkMatterConcentrationMethod} input parameter. The procedure pointer {\tt Dark\_Matter\_Profile\_Concentration\_Get} must be set to point to a function which returns the concentration of a node. The initialization subroutine should perform any other tasks required to initialize the module (such as reading parameters etc.).

The concentration function must have the form:
\begin{verbatim}
  double precision function My_Dark_Matter_Profile_Concentration(thisNode)
    implicit none
    type(treeNode), intent(inout), pointer :: thisNode
    .
    .
    .
    return
  end function My_Dark_Matter_Profile_Concentration
\end{verbatim}
The function should compute and return the concentration for {\tt thisNode}.

Currently defined darl matter profile concentration methods are:
\begin{description}
 \item [{\tt NFW1996}] The concentration is computed using the algorithm given by \cite{navarro_structure_1996};
 \item [{\tt Gao2008}] The concentration is computed using a fitting function from \cite{gao_redshift_2008};
 \item [{\tt Zhao2009}] The concentration is computed using a fitting function from \cite{zhao_accurate_2009}. Halo formation times (as defined by \citealt{zhao_accurate_2009}) are computed directly from the merger tree branch associated with each node. Where the tree does not exist or has insufficient resolution to contain the merging time the merging time will be found by extrapolation of the earliest resolved progenitor halo using the selected mass accretion history method (see \S\ref{sec:HaloMassAccretionHistory});
 \item [{\tt Munoz-Cuartas2011}] The concentration is computed using a fitting function from \cite{munoz-cuartas_redshift_2011};
 \item [{\tt Prada2011}] The concentration is computed using a fitting function from \cite{prada_halo_2011};
\end{description}

\subsubsection{Dark Matter Density Profile Shape}\label{sec:darkMatterProfileShape}

Additional methods for the dark matter density profile shape paramter can be added using the {\tt darkMatterShapeMethod} directive. The directive should contain a single argument, giving the name of a subroutine to be called to initialize the method. For example, the {\tt Gao2008} method is described by a directive:
\begin{verbatim}
 !# <darkMatterShapeMethod>
 !#  <unitName>Dark_Matter_Shapes_Gao20008_Initialize</unitName>
 !# </darkMatterShapeMethod>
\end{verbatim}
Here, {\tt Dark\_Matter\_Shapes\_Gao20008\_Initialize} is the name of a subroutine which will be called to initialize the method. The initialization subroutine must have the following form:
\begin{verbatim}
  subroutine Method_Initialize(darkMatterShapeMethod,Dark_Matter_Profile_Shape_Get)
    implicit none
    type(varying_string),          intent(in)    :: darkMatterShapeMethod
    procedure(),          pointer, intent(inout) :: Dark_Matter_Profile_Shape_Get
    
    if (darkMatterShapeMethod == 'myMethod') then
       Dark_Matter_Profile_Shape_Get => My_Shape_Get
       .
       .
       .
    end if
    return
  end subroutine Method_Initialize
\end{verbatim}
where {\tt myMethod} is the name of this method as will be specified by the {\tt darkMatterShapeMethod} input parameter. The procedure pointer {\tt Dark\_Matter\_Profile\_Shape\_Get} must be set to point to a function which returns the shape parameter of a node. The initialization subroutine should perform any other tasks required to initialize the module (such as reading parameters etc.).

The shape parameter function must have the form:
\begin{verbatim}
  double precision function My_Dark_Matter_Profile_Shape(thisNode)
    implicit none
    type(treeNode), intent(inout), pointer :: thisNode
    .
    .
    .
    return
  end function My_Dark_Matter_Profile_Shape
\end{verbatim}
The function should compute and return the shape parameter for {\tt thisNode}.

Currently defined darl matter profile shape parameter methods are:
\begin{description}
 \item [{\tt Gao2008}] The shape parameter is computed using a fitting function from \cite{gao_redshift_2008} - see \S\ref{sec:DarkMatterProfileShape} for details.
\end{description}

\subsubsection{Dark Matter Halo Spin Distribution}\label{sec:HaloSpinDistribution}\index{dark matter halo!spin!distribution}\index{spin!dark matter halo}

Additional methods for the dark matter density profile concentration can be added using the {\tt haloSpinDistributionMethod} directive. The directive should contain a single argument, giving the name of a subroutine to be called to initialize the method. For example, the {\tt Gao2008} method is described by a directive:
\begin{verbatim}
 !# <haloSpinDistributionMethod>
 !#  <unitName>Halo_Spin_Distribution_Bett2007_Initialize</unitName>
 !# </haloSpinDistributionMethod>
\end{verbatim}
Here, {\tt Halo\_Spin\_Distribution\_Bett2007\_Initialize} is the name of a subroutine which will be called to initialize the method. The initialization subroutine must have the following form:
\begin{verbatim}
  subroutine Method_Initialize(haloSpinDistributionMethod,Halo_Spin_Sample_Get)
    implicit none
    type(varying_string),          intent(in)    :: haloSpinDistributionMethod
    procedure(),          pointer, intent(inout) :: Halo_Spin_Sample_Get
    
    if (haloSpinDistributionMethod == 'myMethod') then
       Halo_Spin_Sample_Get => My_Spin_Sample_Get
       .
       .
       .
    end if
    return
  end subroutine Method_Initialize
\end{verbatim}
where {\tt myMethod} is the name of this method as will be specified by the {\tt haloSpinDistributionMethod} input parameter. The procedure pointer {\tt Halo\_Spin\_Sample\_Get} must be set to point to a function which returns a spin parameter drawn at random from a distribution. The initialization subroutine should perform any other tasks required to initialize the module (such as reading parameters etc.).

The spin parameter function must have the form:
\begin{verbatim}
  double precision function My_Spin_Distribution_Sample(thisNode)
    implicit none
    type(treeNode), intent(inout), pointer :: thisNode
    .
    .
    .
    return
  end function My_Spin_Distribution_Sample
\end{verbatim}
The function should compute and return a spin parameter for {\tt thisNode} drawn at random from a distribution.

Currently defined spin distribution methods are:
\begin{description}
 \item [{\tt lognormal}] The spin is drawn from a lognormal distribution with median {\tt [lognormalSpinDistributionMedian]} and width {\tt [lognormalSpinDistributionSigma]}.
 \item [{\tt Bett2007}] The spin is drawn from the distribution found by \cite{bett_spin_2007}. The $\lambda_0$ and $\alpha$ parameter of Bett et al.'s distribution are set by the {\tt [spinDistributionBett2007Lambda0]} and {\tt [spinDistributionBett2007Alpha]} input parameters.
 \item [{\tt deltaFunction}] The spin is drawn from a delta function distribution, i.e. a value equal to {\tt [deltaFunctionSpinDistributionSpin]} is always returned.
\end{description}

\subsubsection{Dark Matter Halo Mass Loss Rates}\label{sec:HaloMassLossRates}\index{dark matter halo!mass loss}\index{mass loss!dark matter halo}

Additional methods for dark matter halo mass loss rates can be added using the {\tt darkMatterHaloMassLossRateMethod} directive. The directive should contain a single argument, giving the name of a subroutine to be called to initialize the method. For example, the {\tt van den Bosch 2005} method is described by a directive:
\begin{verbatim}
 !# <darkMatterHaloMassLossRateMethod>
 !#  <unitName>Dark_Matter_Halos_Mass_Loss_Rate_vanDenBosch_Initialize</unitName>
 !# </darkMatterHaloMassLossRateMethod>
\end{verbatim}
Here, {\tt Dark\_Matter\_Halos\_Mass\_Loss\_Rate\_vanDenBosch\_Initialize} is the name of a subroutine which will be called to initialize the method. The initialization subroutine must have the following form:
\begin{verbatim}
  subroutine Method_Initialize(darkMatterHaloMassLossRateMethod,Dark_Matter_Halos_Mass_Loss_Rate_Get)
    implicit none
    type(varying_string),          intent(in)    :: darkMatterHaloMassLossRateMethod
    procedure(),          pointer, intent(inout) :: Dark_Matter_Halos_Mass_Loss_Rate_Get
    
    if (darkMatterHaloMassLossRateMethod == 'myMethod') then
       Dark_Matter_Halos_Mass_Loss_Rate_Get => My_Mass_Loss_Rate_Get
       .
       .
       .
    end if
    return
  end subroutine Method_Initialize
\end{verbatim}
where {\tt myMethod} is the name of this method as will be specified by the {\tt darkMatterHaloMassLossRateMethod} input parameter. The procedure pointer {\tt Dark\_Matter\_Halos\_Mass\_Loss\_Rate\_Get} must be set to point to a function which returns a spin parameter drawn at random from a distribution. The initialization subroutine should perform any other tasks required to initialize the module (such as reading parameters etc.).

The mass loss rate function must have the form:
\begin{verbatim}
  double precision function My_Mass_Loss_Rate(thisNode)
    implicit none
    type(treeNode), intent(inout), pointer :: thisNode
    .
    .
    .
    return
  end function My_Mass_Loss_Rate
\end{verbatim}
The function should compute and return the mass loss rate from {\tt thisNode} in units of $M_\odot/$Gyr.

Currently defined halo mass loss rate methods are:
\begin{description}
 \item [{\tt null}] Always returns zero mass loss rate.
 \item [{\tt vanDenBosch2005}] Uses the algorithm of \cite{van_den_bosch_mass_2005} to compute the mass loss rate.
\end{description}

\subsubsection{Excursion Set Barrier}\label{sec:excursionSetBarrierMethod}

Additional methods for the excursion set barrier can be added using the {\tt excursionSetBarrierMethod} directive. The directive should contain a single argument, giving the name of a subroutine to be called to initialize the method. For example, the {\tt linear} method is described by a directive:
\begin{verbatim}
 !# <excursionSetBarrierMethod>
 !#  <unitName>Excursion_Sets_Barriers_Linear_Initialize</unitName>
 !# </excursionSetBarrierMethod>
\end{verbatim}
Here, {\tt Excursion\_Sets\_Barriers\_Linear\_Initialize} is the name of a subroutine which will be called to initialize the method. The initialization subroutine must have the following form:
\begin{verbatim}
  subroutine Method_Initialize(excursionSetBarrierMethodExcursion_Sets_Barrier_Get,Excursion
_Sets_Barrier_Gradient_Get,barrierName)
    implicit none
    type     (varying_string  ),         intent(in   ) :: excursionSetBarrierMethod
    procedure(double precision),pointer, intent(inout) :: Excursion_Sets_Barrier_Get,Excursion_Sets_Barrier_Gradient
_Get
    type(varying_string),                intent(inout) :: barrierName

    if (excursionSetBarrierMethod == 'myMethod') then
      Excursion_Sets_Barrier_Get          => My_Barrier
      Excursion_Sets_Barrier_Gradient_Get => My_Barrier_Gradient
      barrierName=barrierName//":myLabel"
    end if
    return
  end subroutine Method_Initialize
\end{verbatim}
where {\tt myMethod} is the name of this method as will be specified by the {\tt excursionSetBarrierMethod} input parameter. The procedure pointers {\tt Excursion\_Sets\_Barrier\_Get}, and {\tt Excursion\_Sets\_Barrier\_Gradient\_Get} must be set to point functions which return the barrier and its gradient respectively, as described below. The initialization subroutine should also append a descriptive label to the {\tt barrierName} argument. The initialization subroutine should perform any other tasks required to initialize the module (such as reading parameters etc.).

The barrier and barrier gradient functions must have the form:
\begin{verbatim}
 double precision function Excursion_Sets_Barrier(variance,time)
    implicit none
    double precision, intent(in) :: variance,time
    .
    .
    .
    return
 end function Excursion_Sets_Barrier
\end{verbatim}
The barrier function must return the barrier at the specified {\tt variance} and {\tt time}, while the barrier gradient function should return the derivative with respect to variance of the same barrier.

Currently defined excursion set barrier methods are:
\begin{description}
 \item [{\tt linear}] A linear ($1^{\rm st}$-order polynomial) barrier;
  \item [{\tt quadratic}] A quadratic ($2^{\rm nd}$-order polynomial);
  \item [{\tt criticalOverdensity}] A barrier equal to the critical overdensity for halo collapse.
\end{description}

\subsubsection{Excursion Set Barrier First Crossing Distribution}\label{sec:excursionSetFirstCrossingMethod}

Additional methods for the excursion set barrier first crossing distribution can be added using the {\tt excursionSetFirstCrossingMethod} directive. The directive should contain a single argument, giving the name of a subroutine to be called to initialize the method. For example, the {\tt linearBarrier} method is described by a directive:
\begin{verbatim}
 !# <excursionSetFirstCrossingMethod>
 !#  <unitName>Excursion_Sets_First_Crossing_Linear_Barrier_Initialize</unitName>
 !# </excursionSetFirstCrossingMethod>
\end{verbatim}
Here, {\tt Excursion\_Sets\_First\_Crossing\_Linear\_Barrier\_Initialize} is the name of a subroutine which will be called to initialize the method. The initialization subroutine must have the following form:
\begin{verbatim}
  subroutine Method_Initialize(excursionSetFirstCrossingMethod,Excursion_Sets_First_Crossing_Probability_Get&
         &,Excursion_Sets_First_Crossing_Rate_Get,Excursion_Sets_Non_Crossing_Rate_Get)
    implicit none
    type     (varying_string  ),         intent(in   ) :: excursionSetFirstCrossingMethod
    procedure(double precision),pointer, intent(inout) :: Excursion_Sets_First_Crossing_Probability_Get&
         &,Excursion_Sets_First_Crossing_Rate_Get,Excursion_Sets_Non_Crossing_Rate_Get
    
    if (excursionSetFirstCrossingMethod == 'myMethod') then
       Excursion_Sets_First_Crossing_Probability_Get => My_First_Crossing_Probability
       Excursion_Sets_First_Crossing_Rate_Get        => My_First_Crossing_Rate
       Excursion_Sets_Non_Crossing_Rate_Get          => My_Non_Crossing_Rate
    end if
    return
  end subroutine Method_Initialize
\end{verbatim}
where {\tt myMethod} is the name of this method as will be specified by the {\tt excursionSetFirstCrossingMethod} input parameter. The procedure pointers {\tt Excursion\_Sets\_First\_Crossing\_Probability\_Get}, {\tt Excursion\_Sets\_First\_Crossing\_Rate\_Get}, and {\tt Excursion\_Sets\_Non\_Crossing\_Rate\_Get} must be set to point functions which return the first crossing probability, first crossing probability rate and noncrossing rate as described below. The initialization subroutine should perform any other tasks required to initialize the module (such as reading parameters etc.).

The first crossing probability function must have the form:
\begin{verbatim}
 double precision function Excursion_Sets_First_Crossing_Probability(variance,time)
    implicit none
    double precision, intent(in) :: variance,time
    .
    .
    .
    return
 end function Excursion_Sets_First_Crossing_Probability
\end{verbatim}
The function must return the first crossing probability per unit variance at the specified {\tt variance} and {\tt time}.

The first crossing probability rate function must have the form:
\begin{verbatim}
 double precision function Excursion_Sets_First_Crossing_Rate(variance,varianceProgenitor,time)
    implicit none
    double precision, intent(in) :: variance,varianceProgenitor,time
    .
    .
    .
    return
 end function Excursion_Sets_First_Crossing_Rate
\end{verbatim}
The function must return the rate of first crossing per unit variance at the specified {\tt variance} and {\tt time} for a progenitor of the specified {\tt varianceProgenitor}.

The non-crossing probability rate function must have the form:
\begin{verbatim}
 double precision function Excursion_Sets_First_Non_Crossing_Rate(variance,time)
    implicit none
    double precision, intent(in) :: variance,time
    .
    .
    .
    return
 end function Excursion_Sets_First_Non_Crossing_Rate
\end{verbatim}
The function must return the rate of trajectories which never cross the barrier at the specified {\tt variance} and {\tt time}..

Currently defined excursion set barrier first crossing methods are:
\begin{description}
 \item [{\tt linearBarrier}] Assumes the solution for a linear barrier;
 \item [{\tt Farahi}] Solves the first crossing problem using the methodology of \cite{benson_dark_2012};
 \item [{\tt ZhangHui2006}] Solves the first crossing problem using the methodology of \cite{zhang_random_2006};
 \item [{\tt ZhangHui2006HighOrder}] Solves the first crossing problem using a higher order extension of the methodology of \cite{zhang_random_2006}.
\end{description}

\subsubsection{Excursion Set Barrier Remapping}\label{sec:excursionSetBarrierRemapInitialize}

Additional methods for the excursion set barrier can be added using the {\tt excursionSetBarrierRemapInitialize} directive. The directive should contain a single argument, giving the name of a subroutine to be called to initialize the method. For example, the {\tt scale} method is described by a directive:
\begin{verbatim}
 !# <excursionSetBarrierRemapInitialize>
 !#  <unitName>Excursion_Sets_Barriers_Remap_Scale_Initialize</unitName>
 !# </excursionSetBarrierRemapInitialize>
\end{verbatim}
Here, {\tt Excursion\_Sets\_Barriers\_Remap\_Scale\_Initialize} is the name of a subroutine which will be called to initialize the method. The initialization subroutine must have the following form:
\begin{verbatim}
  subroutine Method_Initialize(excursionSetBarrierRemapMethods,barrierName, &
  & ratesCalculation,matchedCount)
    implicit none
    type(varying_string), intent(in   ), dimension(:) :: excursionSetBarrierRemapMethods
    type(varying_string), intent(inout)               :: barrierName
    logical             , intent(in   )               :: ratesCalculation
    integer             , intent(inout)               :: matchedCount

    if (any(excursionSetBarrierRemapMethods == 'myMethod')) then
       position=-1
       do i=1,size(excursionSetBarrierRemapMethods)
          if (excursionSetBarrierRemapMethods(i) == 'myMethod') then
             position=i
             exit
          end if
       end do
       if (ratesCalculation) then
          methodRatesPosition=position
       else
          methodPosition     =position
       end if
       matchedCount=matchedCount+1
       barrierName=barrierName//":myLabel"
    end if
    return
  end subroutine Method_Initialize
\end{verbatim}
where {\tt myMethod} is the name of this method as will be specified by the {\tt excursionSetBarrierRemapMethods} input parameter. The initialization subroutine should identify the position of the matched method in the {\tt excursionSetBarrierRemapMethods()} array and record that it is active for standard barrier calculations ({\tt ratesCalculation}$=${\tt false}) or for barriers used in crossing rate calculations ({\tt ratesCalculation}$=${\tt true}). It should also increment the {\tt matchedCount} argument (to allow checking that all specified barriers were matched) and append a descriptive label to the{\tt barrierName} argument. The initialization subroutine should perform any other tasks required to initialize the module (such as reading parameters etc.).

The method must provide a subroutine to compute remapping of the barrier as follows:
\begin{verbatim}
  !# <excursionSetBarrierRemap>
  !#  <unitName>Method_Barrier_Remap</unitName>
  !# </excursionSetBarrierRemap>
  subroutine Method_Barrier_Remap(barrier,variance,time,ratesCalculation,iRemap)
    implicit none
    double precision, intent(inout) :: barrier
    double precision, intent(in   ) :: variance,time
    logical         , intent(in   ) :: ratesCalculation
    integer         , intent(in   ) :: iRemap

    if ((ratesCalculation.and.iRemap == methodRatesPosition).or.(.not.ratesCalculation.and.iRemap == methodPosition)) then
     ! Do remapping.
     .
     .
     .
    end if
    return
  end subroutine Method_Barrier_Remap
\end{verbatim}
and a subroutine to compute remapping of the barrier gradient as follows:
\begin{verbatim}
  !# <excursionSetBarrierRemapGradient>
  !#  <unitName>Method_Barrier_Gradient_Remap</unitName>
  !# </excursionSetBarrierRemapGradient>
  subroutine Method_Barrier_Gradient_Remap(barrier,barrierGradient,variance,time,ratesCalculation,iRemap)
    implicit none
    double precision, intent(inout) :: barrier,barrierGradient
    double precision, intent(in   ) :: variance,time
    logical         , intent(in   ) :: ratesCalculation
    integer         , intent(in   ) :: iRemap

    if ((ratesCalculation.and.iRemap == methodRatesPosition).or.(.not.ratesCalculation.and.iRemap == methodPosition)) then
     ! Do remapping.
     .
     .
     .
    end if
    return
  end subroutine Method_Barrier_Gradient_Remap
\end{verbatim}

Currently defined excursion set barrier remapping methods are:
\begin{description}
 \item [{\tt null}] A null method which leaves the barrier unchanged;
 \item [{\tt scale}] Scales the barrier by a multiplicative factor;
 \item [{\tt Sheth-Mo-Tormen}] Remaps the barrier according to the algorithm of \cite{sheth_ellipsoidal_2001}.
\end{description}

\subsubsection{Galactic Component Radii Solver}\label{sec:galactic_radii_solvers}

Additional methods for solving for radii of galactic components can be added using the {\tt galacticStructureRadiusSolverMethod} directive. The directive should contain a single argument, giving the name of a subroutine to be called to initialize the method. For example, the {\tt simple} method is described by a directive:
\begin{verbatim}
 !# <galacticStructureRadiusSolverMethod>
 !#  <unitName>Galactic_Structure_Radii_Simple_Initialize</unitName>
 !# </galacticStructureRadiusSolverMethod>
\end{verbatim}
Here, {\tt Galactic\_Structure\_Radii\_Simple\_Initialize} is the name of a subroutine which will be called to initialize the method. The initialization subroutine must have the following form:
\begin{verbatim}
  subroutine Method_Initialize(galacticStructureRadiusSolverMethod,Galactic_Structure_Radii_Solve_Do)
    implicit none
    type(varying_string),          intent(in)    :: galacticStructureRadiusSolverMethod
    procedure(),          pointer, intent(inout) :: Galactic_Structure_Radii_Solve_Do
    
    if (galacticStructureRadiusSolverMethod == 'myMethod') Galactic_Structure_Radii_Solve_Do => My_Method_Do_Procedure
    return
  end subroutine Method_Initialize
\end{verbatim}
where {\tt myMethod} is the name of this method as will be specified by the {\tt galacticStructureRadiusSolverMethod} input parameter. The procedure pointer {\tt Galactic\_Structure\_Radii\_Solve\_Do} must be set to point to a subroutine which solves for the radii of components in a node as described below. The initialization subroutine should perform any other tasks required to initialize the module (such as reading parameters etc.).

The radii solving subroutine must have the form:
\begin{verbatim}
 subroutine Radii_Solver_Do(thisNode)
    implicit none
    type(treeNode), intent(in), pointer :: thisNode
    .
    .
    .
    return
 end subroutine Radii_Solver_Do
\end{verbatim}
The function must set the radii (and corresponding circular velocities) of all components that have a radius property in {\tt thisNode}.

Currently defined radius solver methods are:
\begin{description}
 \item [\hyperlink{galactic_structure.radius_solver.simple.F90:galactic_structure_radii_simple:galactic_structure_radii_solve_simple}{{\tt simple}}] This solver computes radii assuming that the gravitational potential is dominated by dark matter (i.e. no baryonic self-gravity is included) and that dark matter does not respond to the presence of baryons (i.e. no adiabatic contraction). It uses the ``radius solver'' (see \S\ref{sec:radius_solver}) task to interact with the node.
 \item [\hyperlink{galactic_structure.radius_solver.adiabatic.F90:galactic_structure_radii_adiabatic:galactic_structure_radii_solve_adiabatic}{{\tt adiabatic}}] This solver computes radii including the effects of self-gravity of the baryonic component and adiabatic contraction of the dark matter halo using the method of \cite{gnedin_response_2004}. It uses the ``radius solver'' (see \S\ref{sec:radius_solver}) task to interact with the node.
 \item [\hyperlink{galactic_structure.radius_solver.linear.F90:galactic_structure_radii_linear:galactic_structure_radii_solve_linear}{{\tt linear}}] This solver assumes that radii scale linearly with specific angular momentum, equalling the virial radius when the specific angular momentum equals the product of virial radii and velocities. It uses the ``radius solver'' (see \S\ref{sec:radius_solver}) task to interact with the node.
 \item [\hyperlink{galactic_structure.radius_solver.fixed.F90:galactic_structure_radii_fixed:galactic_structure_radii_solve_fixed}{{\tt fixed}}] This solver assumes that radii equal the product of virial radius of the halo and its spin parameter (with an adjustable coefficient). It uses the ``radius solver'' (see \S\ref{sec:radius_solver}) task to interact with the node.
\end{description}

\subsubsection{Galactic Component Radius Solver Initial Radius}

Additional methods for computing the initial radius in the dark matter profile when solving for adiabatic contraction of the halo can be added using the {\tt galacticStructureRadiusSolverInitialRadiusMethod} directive. The directive should contain a single argument, giving the name of a subroutine to be called to initialize the method. For example, the {\tt adiabatic} method is described by a directive:
\begin{verbatim}
 !# <galacticStructureRadiusSolverInitialRadiusMethod>
 !#  <unitName>Galactic_Structure_Initial_Radii_Adiabatic_Initialize</unitName>
 !# </galacticStructureRadiusSolverInitialRadiusMethod>
\end{verbatim}
Here, {\tt Galactic\_Structure\_Initial\_Radii\_Adiabatic\_Initialize} is the name of a subroutine which will be called to initialize the method. The initialization subroutine must have the following form:
\begin{verbatim}
  subroutine Method_Initialize(galacticStructureRadiusSolverInitialRadiusMethod,Galactic_Structure_Radius_Initial_Get,Galactic_Structure_Radius_Initial_Derivative_Get)
    implicit none
    type(varying_string),          intent(in)    :: galacticStructureRadiusSolverInitialRadiusMethod
    procedure(),          pointer, intent(inout) :: Galactic_Structure_Radius_Initial_Get,Galactic_Structure_Radius_Initial_Derivative_Get
    
    if (galacticStructureRadiusSolverInitialRadiusMethod == 'myMethod') then
       Galactic_Structure_Radius_Initial_Get            => My_Method_Get
       Galactic_Structure_Radius_Initial_Derivative_Get => My_Method_Derivative_Get
    end if
    return
  end subroutine Method_Initialize
\end{verbatim}
where {\tt myMethod} is the name of this method as will be specified by the {\tt galacticStructureRadiusSolverInitialRadiusMethod} input parameter. The procedure pointers {\tt Galactic\_Structure\_Radius\_Initial\_Get} and {\tt Galactic\_Structure\_Radius\_Initial\_Derivative\_Get} must be set to point to functions which compute the initial radius in the dark matter halo given the final radius, and the deriative of this quantity with respect to the final radius as described below. The initialization subroutine should perform any other tasks required to initialize the module (such as reading parameters etc.).

The initial radius function must have the form:
\begin{verbatim}
 double precision function Radius_Initial_Get(thisNode,radius)
    implicit none
    type            (treeNode), intent(in   ), pointer :: thisNode
    double precision          , intent(in   )          :: radius
    .
    .
    .
    return
 end function Radius_Initial_Get
\end{verbatim}
The function must return the initial radius in the dark matter halo of {\tt thisNode} corresponding to the final {\tt radius} after accounting for the effects of adiabatic contraction.

The initial radius derivative function must have the form:
\begin{verbatim}
 double precision function Radius_Initial_Derivative_Get(thisNode,radius)
    implicit none
    type            (treeNode), intent(in   ), pointer :: thisNode
    double precision          , intent(in   )          :: radius
    .
    .
    .
    return
 end function Radius_Initial_Derivative_Get
\end{verbatim}
The function must return the derivative with respect to the final {\tt radius} of initial radius in the dark matter halo of {\tt thisNode} corresponding to the final {\tt radius} after accounting for the effects of adiabatic contraction.

Currently defined initial radius methods are:
\begin{description}
 \item [\hyperlink{galactic_structure.radius_solver.initial_radii.static.F90:galactic_structure_initial_radii_static:galactic_structure_radius_initial_static}{{\tt static}}] This method assumes a static dark matter halo, and so the initial radius always equals the final radius.
 \item [\hyperlink{galactic_structure.radius_solver.initial_radii.adiabatic.F90:galactic_structure_initial_radii_adiabatic:galactic_structure_radius_initial_adiabatic}{{\tt adiabatic}}] This method assumes adiabatic contraction follows the model of \cite{gnedin_response_2004}.
\end{description}


\subsubsection{Hot Halo Ram Pressure Force}

Additional methods for the ram pressure stripping force due to hot halos can be added using the {\tt hotHaloRamPressureForceMethod} directive. The directive should contain a single argument, giving the name of a subroutine to be called to initialize the method. For example, the {\tt Font2008} method is described by a directive:
\begin{verbatim}
 !# <hotHaloRamPressureForceMethod>
 !#  <unitName>Hot_Halo_Ram_Pressure_Force_Font2008_Initialize</unitName>
 !# </hotHaloRamPressureForceMethod>
\end{verbatim}
Here, {\tt Hot\_Halo\_Ram\_Pressure\_Force\_Font2008\_Initialize} is the name of a subroutine which will be called to initialize the method. The initialization subroutine must have the following form:
\begin{verbatim}
  subroutine Method_Initialize(hotHaloRamPressureForceMethod,Hot_Halo_Ram_Pressure_Force_Get)
    implicit none
    type(varying_string),          intent(in)    :: hotHaloRamPressureForceMethod
    procedure(),          pointer, intent(inout) :: Hot_Halo_Ram_Pressure_Force_Get
    
    if (hotHaloRamPressureForceMethod == 'myMethod') Hot_Halo_Ram_Pressure_Force_Get => My_Hot_Halo_Ram_Pressure_Force_Get
    return
  end subroutine Method_Initialize
\end{verbatim}
where {\tt myMethod} is the name of this method as will be specified by the {\tt hotHaloRamPressureForceMethod} input parameter. The procedure pointer {\tt Hot\_Halo\_Ram\_Pressure\_Force\_Get} must be set to point to a function which returns ram pressure force due to the hot halo. The initialization subroutine should perform any other tasks required to initialize the module (such as reading parameters etc.).

The ram pressure force function must have the form:
\begin{verbatim}
 double precision function Hot_Halo_Ram_Pressure_Force_Get(thisNode)
    implicit none
    type(treeNode), intent(inout), pointer :: thisNode
    .
    .
    .
    return
 end function Hot_Halo_Ram_Pressure_Force_Get
\end{verbatim}
The function must return the ram pressure force acting on {\tt thisNode} due to the hot halo of its host node (in units of $M_\odot \hbox{km}^2 \hbox{s}^{-1} \hbox{Mpc}^{-3}$.

Currently defined hot halo ram pressure force methods are:
\begin{description}
 \item [\hyperlink{hot_halo.ram_pressure_force.null.F90:hot_halo_ram_pressure_force_null:hot_halo_ram_pressure_force_null_get}{{\tt null}}] Returns a zero ram pressure force.
 \item [\hyperlink{hot_halo.ram_pressure_force.Font2008.F90:hot_halo_ram_pressure_force_font2008:hot_halo_ram_pressure_force_font2008_get}{{\tt Font2008}}] Computes the ram pressure stripping radius using the algorithm of \cite{font_colours_2008}.
\end{description}

\subsubsection{Hot Halo Ram Pressure Stripping Radius}

Additional methods for the ram pressure stripping radius in hot halos can be added using the {\tt hotHaloRamPressureStrippingMethod} directive. The directive should contain a single argument, giving the name of a subroutine to be called to initialize the method. For example, the {\tt virialRadius} method is described by a directive:
\begin{verbatim}
 !# <hotHaloRamPressureStrippingMethod>
 !#  <unitName>Hot_Halo_Ram_Pressure_Stripping_Virial_Radii_Initialize</unitName>
 !# </hotHaloRamPressureStrippingMethod>
\end{verbatim}
Here, {\tt Hot\_Halo\_Ram\_Pressure\_Stripping\_Virial\_Radii\_Initialize} is the name of a subroutine which will be called to initialize the method. The initialization subroutine must have the following form:
\begin{verbatim}
  subroutine Method_Initialize(hotHaloRamPressureStrippingMethod,Hot_Halo_Ram_Pressure_Stripping_Get)
    implicit none
    type(varying_string),          intent(in)    :: hotHaloRamPressureStrippingMethod
    procedure(),          pointer, intent(inout) :: Hot_Halo_Ram_Pressure_Stripping_Get
    
    if (hotHaloRamPressureStrippingMethod == 'myMethod') Hot_Halo_Ram_Pressure_Stripping_Get => My_Hot_Halo_Ram_Pressure_Stripping_Get
    return
  end subroutine Method_Initialize
\end{verbatim}
where {\tt myMethod} is the name of this method as will be specified by the {\tt hotHaloRamPressureStrippingMethod} input parameter. The procedure pointer {\tt Hot\_Halo\_Ram\_Pressure\_Stripping\_Get} must be set to point to a function which returns the radius to which the hot halo is stripped by ram pressure forces. The initialization subroutine should perform any other tasks required to initialize the module (such as reading parameters etc.).

The ram pressure stripping radius function must have the form:
\begin{verbatim}
 double precision function Hot_Halo_Ram_Pressure_Stripping_Get(thisNode)
    implicit none
    type(treeNode), intent(inout), pointer :: thisNode
    .
    .
    .
    return
 end function Hot_Halo_Ram_Pressure_Stripping_Get
\end{verbatim}
The function must return the radius (in units of Mpc) to which the hot halo of {\tt thisNode} is stripped by ram pressure forces.

Currently defined hot halo ram pressure stripping radii methods are:
\begin{description}
 \item [\hyperlink{hot_halo.ram_pressure_stripping.virial_radius.F90:hot_halo_ram_pressure_stripping_virial_radii:hot_halo_ram_pressure_stripping_virial_radius}{{\tt virialRadius}}] Sets the ram pressure stripping radius equal to the virial radius always---effectively resulting in no ram pressure stripping.
 \item [\hyperlink{hot_halo.ram_pressure_stripping.Font2008.F90:hot_halo_ram_pressure_stripping_font2008:hot_halo_ram_pressure_stripping_font2008_get}{{\tt Font2008}}] Computes the ram pressure stripping radius using the algorithm of \cite{font_colours_2008}.
\end{description}

\subsubsection{Hot Halo Ram Pressure Stripping Timescale}

Additional methods for the ram pressure stripping timescale in hot halos can be added using the {\tt hotHaloRamPressureStrippingTimescaleMethod} directive. The directive should contain a single argument, giving the name of a subroutine to be called to initialize the method. For example, the {\tt virialRadius} method is described by a directive:
\begin{verbatim}
 !# <hotHaloRamPressureStrippingTimescaleMethod>
 !#  <unitName>Hot_Halo_Ram_Pressure_Timescales_Halo_DynTime_Initialize</unitName>
 !# </hotHaloRamPressureStrippingTimescaleMethod>
\end{verbatim}
Here, {\tt Hot\_Halo\_Ram\_Pressure\_Timescales\_Halo\_DynTime\_Initialize} is the name of a subroutine which will be called to initialize the method. The initialization subroutine must have the following form:
\begin{verbatim}
  subroutine Method_Initialize(hotHaloRamPressureStrippingTimescaleMethod,Hot_Halo_Ram_Pressure_Stripping_Get)
    implicit none
    type(varying_string),          intent(in)    :: hotHaloRamPressureStrippingTimescaleMethod
    procedure(),          pointer, intent(inout) :: Hot_Halo_Ram_Pressure_Timescale_Get
    
    if (hotHaloRamPressureStrippingTimescaleMethod == 'myMethod') Hot_Halo_Ram_Pressure_Timescale_Get => My_Hot_Halo_Ram_Pressure_Timescale_Get
    return
  end subroutine Method_Initialize
\end{verbatim}
where {\tt myMethod} is the name of this method as will be specified by the {\tt hotHaloRamPressureStrippingTimescaleMethod} input parameter. The procedure pointer {\tt Hot\_Halo\_Ram\_Pressure\_Timescale\_Get} must be set to point to a function which returns the timescale on which material is removed from the hot halo due to ram pressure forces. The initialization subroutine should perform any other tasks required to initialize the module (such as reading parameters etc.).

The ram pressure stripping timescale function must have the form:
\begin{verbatim}
 double precision function Hot_Halo_Ram_Pressure_Timescale_Get(thisNode)
    implicit none
    type(treeNode), intent(inout), pointer :: thisNode
    .
    .
    .
    return
 end function Hot_Halo_Ram_Pressure_Timescale_Get
\end{verbatim}
The function must return the timescale (in units of Gyr) on which the hot halo of {\tt thisNode} is being stripped by ram pressure forces.

Currently defined hot halo ram pressure stripping timescale methods are:
\begin{description}
 \item [\hyperlink{hot_halo.ram_pressure_stripping.timescale.halo_dynamical_time.F90:hot_halo_ram_pressure_timescales_halo_dyntime:hot_halo_ram_pressure_timescale_halo_dyntime}{{\tt haloDynamicalTime}}] Sets the ram pressure stripping timescale equal to the dynamical time of the node's dark matter halo
 \item [\hyperlink{hot_halo.ram_pressure_stripping.timescale.ram_pressure_acceleration.F90:hot_halo_ram_pressure_timescales_ram_pressure_accel:hot_halo_ram_pressure_timescale_ram_pressure_accel}{{\tt ramPressureAcceleration}}] Computes the timescale from the ram pressure acceleration following \cite{roediger_ram_2007}.
\end{description}

\subsubsection{Hot Halo Temperature Profile}

Additional methods for the hot halo temperature profile can be added using the {\tt hotHaloTemperatureMethod} directive. The directive should contain a single argument, giving the name of a subroutine to be called to initialize the method. For example, the {\tt virial} method is described by a directive:
\begin{verbatim}
 !# <hotHaloTemperatureMethod>
 !#  <unitName>Hot_Halo_Temperature_Virial</unitName>
 !# </hotHaloTemperatureMethod>
\end{verbatim}
Here, {\tt Hot\_Halo\_Temperature\_Virial} is the name of a subroutine which will be called to initialize the method. The initialization subroutine must have the following form:
\begin{verbatim}
  subroutine Method_Initialize(hotHaloTemperatureMethod,Hot_Halo_Temperature_Get,Hot_Halo_Temperature_Logarithmic_Slope_Get)
    implicit none
    type(varying_string),          intent(in)    :: hotHaloTemperatureMethod
    procedure(),          pointer, intent(inout) :: Hot_Halo_Temperature_Get,Hot_Halo_Temperature_Logarithmic_Slope_Get
    
    if (hotHaloTemperatureMethod == 'myMethod') then
       Hot_Halo_Temperature_Get => My_Method_Get
       Hot_Halo_Temperature_Logarithmic_Slope_Get => My_Method_Logarithmic_Slope_Get
    end if
    return
  end subroutine Method_Initialize
\end{verbatim}
where {\tt myMethod} is the name of this method as will be specified by the {\tt hotHaloTemperatureMethod} input parameter. The procedure pointer {\tt Hot\_Halo\_Temperature\_Get} must be set to point to a function which returns the hot halo density as described below while {\tt Hot\_Halo\_Temperature\_Logarithmic\_Slope\_Get} must point to a function which returns the logarithmic slope of the temperature profile. The initialization subroutine should perform any other tasks required to initialize the module (such as reading parameters etc.).

The temperature function must have the form:
\begin{verbatim}
 double precision function Hot_Halo_Temperature_Get(thisNode,radius)
    implicit none
    type(treeNode),   intent(inout), pointer :: thisNode
    double precision, intent(in)             :: radius
    .
    .
    .
    return
 end function Hot_Halo_Temperature_Get
\end{verbatim}
The function must return the temperature (in Kelvin) of the hot halo at the specified radius (given in Mpc) for {\tt thisNode}. The logarithmic slope function should have the same template, but return $\d\ln T / \d \ln r$.

Currently defined hot halo density profile methods are:
\begin{description}
 \item [\hyperlink{hot_halo.temperature_profile.virial.F90:hot_halo_temperature_profile_virial:hot_halo_temperature_virial_get}{{\tt virial}}] Implements an isothermal profile with temperature equal to the virial temperature.
\end{description}

\subsubsection{Halo Bias}

Additional methods for the halo bias (i.e. the linear theory bias) can be added using the {\tt darkMatterHaloBiasMethod} directive. The directive should contain a single argument, giving the name of a subroutine to be called to initialize the method. For example, the {\tt Press-Schechter} method is described by a directive:
\begin{verbatim}
 !# <darkMatterHaloBiasMethod>
 !#  <unitName>Dark_Matter_Halo_Bias_Press_Schechter_Initialize</unitName>
 !# </darkMatterHaloBiasMethod>
\end{verbatim}
Here, {\tt Dark\_Matter\_Halo\_Bias\_Press\_Schechter\_Initialize} is the name of a subroutine which will be called to initialize the method. The initialization subroutine must have the following form:
\begin{verbatim}
  subroutine Method_Initialize(darkMatterHaloBiasMethod,Dark_Matter_Halo_Bias_Node_Get,Dark_Matter_Halo_Bias_Get)
    implicit none
    type(varying_string),          intent(in)    :: darkMatterHaloBiasMethod
    procedure(),          pointer, intent(inout) :: Dark_Matter_Halo_Bias_Get
    
    if (darkMatterHaloBiasMethod == 'myMethod') then
       Dark_Matter_Halo_Bias_Node_Get => My_Method_Node_Get
       Dark_Matter_Halo_Bias_Get      => My_Method_Get
    end if
    return
  end subroutine Method_Initialize
\end{verbatim}
where {\tt myMethod} is the name of this method as will be specified by the {\tt darkMatterHaloBiasMethod} input parameter. The procedure pointers {\tt Dark\_Matter\_Halo\_Bias\_Node\_Get} and {\tt Dark\_Matter\_Halo\_Bias\_Get} must be set to point to functions which return the bias of the specified halo. The initialization subroutine should perform any other tasks required to initialize the module (such as reading parameters etc.).

The halo bias functions must have the following forms.
\begin{verbatim}
 double precision function Dark_Matter_Halo_Bias_Node(thisNode)
    implicit none
    type(treeNode), intent(inout), pointer :: thisNode
    .
    .
    .
    return
 end function Dark_Matter_Halo_Bias_Node
\end{verbatim}
The function should return the linear theory bias for {\tt thisNode}.
\begin{verbatim}
 double precision function Dark_Matter_Halo_Bias(mass,time)
    implicit none
    double precision, intent(in   ) :: mass,time
    .
    .
    .
    return
 end function Dark_Matter_Halo_Bias
\end{verbatim}
The function should return the linear theory bias for the given {\tt mass} and {\tt time}. Two versions of these functions are provided because a common assumption is that the bias depends only on mass and time, while in reality it may depend on other properties of the halo (environment, formation time etc.). The first version of the function allows for arbitrary dependence on properties of the node.

Currently defined halo bias methods are:
\begin{description}
 \item [\hyperlink{structure_formation.halo_bias.Press-Schechter.F90:dark_matter_halo_biases_press_schechter:dark_matter_halo_bias_press_schechter}{{\tt Press-Schechter}}] Implements the bias resulting from the Press-Schechter \citep{press_formation_1974} mass function \citep{mo_analytic_1996}.
 \item [\hyperlink{structure_formation.halo_bias.SMT.F90:dark_matter_halo_biases_smt:dark_matter_halo_bias_smt}{{\tt SMT}}] Implements the Sheth-Tormen \citep{sheth_ellipsoidal_2001} bias.
 \item [\hyperlink{structure_formation.halo_bias.Tinker2010.F90:dark_matter_halo_biases_tinker2010:dark_matter_halo_bias_tinker2010}{{\tt Tinker2010}}] Implements the bias described by \cite{tinker_large_2010}.
\end{description}

\subsubsection{Halo Mass Functions}

Additional methods for the halo mass function can be added using the {\tt haloMassFunctionMethod} directive. The directive should contain a single argument, giving the name of a subroutine to be called to initialize the method. For example, the {\tt Press-Schechter} method is described by a directive:
\begin{verbatim}
 !# <haloMassFunctionMethod>
 !#  <unitName>Halo_Mass_Function_Press_Schechter_Initialize</unitName>
 !# </haloMassFunctionMethod>
\end{verbatim}
Here, {\tt Halo\_Mass\_Function\_Press\_Schechter\_Initialize} is the name of a subroutine which will be called to initialize the method. The initialization subroutine must have the following form:
\begin{verbatim}
  subroutine Method_Initialize(haloMassFunctionMethod,Halo_Mass_Function_Differential_Get)
    implicit none
    type(varying_string),          intent(in)    :: haloMassFunctionMethod
    procedure(),          pointer, intent(inout) :: Halo_Mass_Function_Tabulate
    
    if (haloMassFunctionMethod == 'myMethod') Halo_Mass_Function_Differential_Get => My_Method_Get
    return
  end subroutine Method_Initialize
\end{verbatim}
where {\tt myMethod} is the name of this method as will be specified by the {\tt haloMassFunctionMethod} input parameter. The procedure pointer {\tt Halo\_Mass\_Function\_Differential\_Get} must be set to point to a subrouine which returns the differential form of the halo mass function. The initialization subroutine should perform any other tasks required to initialize the module (such as reading parameters etc.).

The halo mass function function must have the form:
\begin{verbatim}
 double precision function Halo_Mass_Function_Differential_Get(time,mass)
    implicit none
    double precision, intent(in   ) :: time,mass
    .
    .
    .
    return
 end function Halo_Mass_Function_Differential_Get
\end{verbatim}
The function should return the halo mass function, $\d n/\d M$ (in units of Mpc$^{-3} M_\odot^-1$) at mass {\tt mass} and time {\tt time}.

Currently defined halo mass function methods are:
\begin{description}
 \item [\hyperlink{structure_formation.halo_mass_function.Press-Schechter.F90:halo_mass_function_press_schechter:halo_mass_function_differential_press_schechter}{{\tt Press-Schechter}}] Implements the Press-Schechter \citep{press_formation_1974} mass function.
 \item [\hyperlink{structure_formation.halo_mass_function.Sheth-Tormen.F90:halo_mass_function_sheth_tormen:halo_mass_function_sheth_tormen_differential}{{\tt Sheth-Tormen}}] Implements the Sheth-Tormen \citep{sheth_ellipsoidal_2001} mass function.
 \item [\hyperlink{structure_formation.halo_mass_function.Tinker2008.F90:halo_mass_function_tinker2008:halo_mass_function_differential_tinker2008}{{\tt Tinker2008}}] Implements the mass function described by \cite{tinker_towardhalo_2008}.
\end{description}

\subsubsection{Halo Mass Sampling Density Functions}

Additional methods for halo mass sampling density functions can be added using the {\tt haloMassFunctionSamplingMethod} directive. The directive should contain a single argument, giving the name of a subroutine to be called to initialize the method. For example, the {\tt powerLaw} method is described by a directive:
\begin{verbatim}
 !# <haloMassFunctionSamplingMethod>
 !#  <unitName>Merger_Trees_Mass_Function_Sampling_Power_Law_Initialize</unitName>
 !# </haloMassFunctionSamplingMethod>
\end{verbatim}
Here, {\tt Merger\_Trees\_Mass\_Function\_Sampling\_Power\_Law\_Initialize} is the name of a subroutine which will be called to initialize the method. The initialization subroutine must have the following form:
\begin{verbatim}
  subroutine Method_Initialize(haloMassFunctionSamplingMethod,Merger_Tree_Construct_Mass_Function_Sampling_Get)
    implicit none
    type(varying_string),          intent(in)    :: haloMassFunctionSamplingMethod
    procedure(),          pointer, intent(inout) :: Merger_Tree_Construct_Mass_Function_Sampling_Get
    
    if (haloMassFunctionSamplingMethod == 'myMethod') Merger_Tree_Construct_Mass_Function_Sampling_Get => My_Mass_Function_Sampling
    return
  end subroutine Method_Initialize
\end{verbatim}
where {\tt myMethod} is the name of this method as will be specified by the {\tt haloMassFunctionSamplingMethod} input parameter. The procedure pointer {\tt Merger\_Tree\_Construct\_Mass\_Function\_Sampling\_Get} must be set to point to a function which returns the sampling rate per unit decade of halo mass.

The halo mass sampling density function must have the form:
\begin{verbatim}
 double precision function My_Mass_Function_Sampling(mass,time,massMinimum,massMaximum)
    implicit none
    double precision, intent(in) :: mass,time,massMinimum,massMaximum
    .
    .
    .
    return
 end function My_Mass_Function_Sampling
\end{verbatim}
The function should return the halo mass sampling density function (the relative number of halos per decade of halo mass to sample) for halos of the given {\tt mass}. Halos are defined at the given {\tt time} and will be sampled in the mass range {\tt massMinimum} to {\tt massMaximum}.

Currently defined halo mass sampling density function methods are:
\begin{description}
 \item [{\tt powerLaw}] The distribution of halo masses is such that the mass of the $i^{\rm th}$ halo is
\begin{equation}
 M_{\rm halo,i} = \exp\left[ \ln(M_{\rm halo,min}) + \ln\left({M_{\rm halo,max}/M_{\rm halo,min}}\right) x_i^{1+\alpha} \right].
\end{equation}
Here, $x_i$ is a number between 0 and 1 and $\alpha=${\tt mergerTreeBuildTreesHaloMassExponent} is an input parameter that controls the relative number of low and high mass tree produced. 
\item [{\tt haloMassFunction}] The sampling density is set equal to the dark matter halo mass function, defined per decade of halo mass.
\item [{\tt stellarMassFunction}] The sampling density is chosen to give optimally minimal errors on the model stellar mass function (see \S\ref{sec:OptimalSamplingStellarMassFunction} for full details).
\end{description}

\subsubsection{Halo Spin Distribution}

Additional methods for the halo spin distribution can be added using the {\tt haloSpinDistributionMethod} directive. The directive should contain a single argument, giving the name of a subroutine to be called to initialize the method. For example, the {\tt lognormal} method is described by a directive:
\begin{verbatim}
 !# <haloSpinDistributionMethod>
 !#  <unitName>Halo_Spin_Distribution_Lognormal_Initialize</unitName>
 !# </haloSpinDistributionMethod>
\end{verbatim}
Here, {\tt Halo\_Spin\_Distribution\_Lognormal\_Initialize} is the name of a subroutine which will be called to initialize the method. The initialization subroutine must have the following form:
\begin{verbatim}
  subroutine Method_Initialize(haloSpinDistributionMethod,Halo_Spin_Sample_Get)
    implicit none
    type(varying_string),          intent(in)    :: haloSpinDistributionMethod
    procedure(),          pointer, intent(inout) :: Halo_Spin_Sample_Get
    
    if (haloSpinDistributionMethod == 'myMethod') Halo_Spin_Sample_Get => My_Method_Get
    return
  end subroutine Method_Initialize
\end{verbatim}
where {\tt myMethod} is the name of this method as will be specified by the {\tt haloSpinDistributionMethod} input parameter. The procedure pointer {\tt Halo\_Spin\_Sample\_Get} must be set to point to a function which returns a halo spin drawn at random from the distribution. The initialization subroutine should perform any other tasks required to initialize the module (such as reading parameters etc.).

The halo spin sampling function must have the form:
\begin{verbatim}
 double precision function Halo_Spin_Sample_Get(thisNode)
    implicit none
    type(treeNode),   intent(inout), pointer :: thisNode
    .
    .
    .
    return
 end function Halo_Spin_Sample_Get
\end{verbatim}
The function must return a halo spin drawn at random for the distribution appropriate to {\tt thisNode}. 

Currently defined halo spin distribution methods are:
\begin{description}
 \item [\hyperlink{dark_matter_halos.spins.distributions.lognormal.F90:halo_spin_distributions_lognormal:halo_spin_distribution_lognormal}{{\tt lognormal}}] Implements a lognormal distribution with median {\tt lognormalSpinDistributionMedian} and dispersion in $\ln\lambda$ of {\tt lognormalSpinDistributionSigma}, both of which are input parameters to \glc.
 \item [\hyperlink{dark_matter_halos.spins.distributions.Bett2007.F90:halo_spin_distributions_bett2007}{{\tt Bett2007}}] Implements distribution from \cite{bett_spin_2007} with parameter $\lambda_0=${\tt [spinDistributionBett2007Lambda0]} and $\alpha=${\tt [spinDistributionBett2007Alpha]}, both of which are input parameters to \glc.
\end{description}

\subsubsection{Halo Profiles}

Additional methods for the halo profile can be added using the {\tt haloProfileMethod} directive. The directive should contain a single argument, giving the name of a subroutine to be called to initialize the method. For example, the {\tt isothermal} method is described by a directive:
\begin{verbatim}
 !# <haloProfileMethod>
 !#  <unitName>Dark_Matter_Profile_Isothermal_Initialize</unitName>
 !# </haloProfileMethod>
\end{verbatim}
Here, {\tt Dark\_Matter\_Profile\_Isothermal\_Initialize} is the name of a subroutine which will be called to initialize the method. The initialization subroutine must have the following form:
\begin{verbatim}
  subroutine Method_Initialize(darkMatterProfileMethod,Dark_Matter_Profile_Energy_Get,Dark_Matter_Profile_Energy_Growth_Rate_Get,Dark_Matter_Profile_Rotation_Normalization_Get,Dark_Matter_Profile_Radius_from_Specific_Angular_Momentum_Get,Dark_Matter_Profile_Circular_Velocity_Get,Dark_Matter_Profile_Potential_Get,Dark_Matter_Profile_Enclosed_Mass_Get)
    implicit none
    type(varying_string),          intent(in)    :: darkMatterProfileMethod
    procedure(),          pointer, intent(inout) :: Dark_Matter_Profile_Energy_Get,Dark_Matter_Profile_Energy_Growth_Rate_Get,Dark_Matter_Profile_Rotation_Normalization_Get,Dark_Matter_Profile_Radius_from_Specific_Angular_Momentum_Get,Dark_Matter_Profile_Circular_Velocity_Get,Dark_Matter_Profile_Potential_Get,Dark_Matter_Profile_Enclosed_Mass_Get
    
    if (darkMatterProfileMethod == 'myMethod') then
       Dark_Matter_Profile_Energy_Get                                => My_Energy_Procedure
       Dark_Matter_Profile_Energy_Growth_Rate_Get                    => My_Energy_Growth_Rate_Procedure
       Dark_Matter_Profile_Rotation_Normalization_Get                => My_Rotation_Normalization_Procedure
       Dark_Matter_Profile_Radius_from_Specific_Angular_Momentum_Get => My_Radius_from_Specific_Angular_Momentum_Procedure
       Dark_Matter_Profile_Circular_Velocity_Get                     => My_Circular_Velocity_Procedure
       Dark_Matter_Profile_Potential_Get                             => My_Potential_Procedure
       Dark_Matter_Profile_Enclosed_Mass_Get                         => My_Enclosed_Mass_Procedure
    end if
    return
  end subroutine Method_Initialize
\end{verbatim}
where {\tt myMethod} is the name of this method as will be specified by the {\tt haloProfileMethod} input parameter. The procedure pointer {\tt Dark\_Matter\_Profile\_Energy\_Get} must be set to point to a function which returns the energy of the halo in units of $M_\odot$ km$^2$ s$^{-1}$ while {\tt Dark\_Matter\_Profile\_Energy\_Growth\_Rate\_Get} must point to a function which returns the rate of change of that energy. The {\tt Dark\_Matter\_Profile\_Rotation\_Normalization\_Get} should point to a function which provides the normalization of the rotation velocity vs. specific angular momentum relation for the halo such that, when multiplied by $4 \pi r^2 \d r \rho(r) V_{\rm rot}$ it gives the angular momentum of material between $r$ and $r+\d r$, i.e. it should return $A$ such that:
\begin{equation}
 J = A \langle j \rangle \int_0^{r_{\rm vir}} 4 \pi r^2 \d r \rho(r) r,
\end{equation}
where we ignore any variation of angular momentum within angle in each spherical shell of matter (since it will cancel out in later calculations anyway). {\tt Dark\_Matter\_Profile\_Radius\_from\_Specific\_Angular\_Momentum\_Get} must be set to point to a procedure which returns the radius in the halo at which circular orbits have the specified specific angular momentum. {\tt Dark\_Matter\_Profile\_Circular\_Velocity\_Get} must be set to point to a procedure which returns the circular velocity in the halo at a given radius.  {\tt Dark\_Matter\_Profile\_Potential\_Get} must be set to point to a procedure which returns the gravitational potential in the halo at a given radius.  {\tt Dark\_Matter\_Profile\_Enclosed\_Mass\_Get} must be set to point to a procedure which returns the mass enclosed in the halo at a given radius. The initialization subroutine should perform any other tasks required to initialize the module (such as reading parameters etc.).

The halo energy function must have the form:
\begin{verbatim}
 double precision function Dark_Matter_Profile_Energy_Get(thisNode)
    implicit none
    type(treeNode),   intent(inout), pointer :: thisNode
    .
    .
    .
    return
 end function Dark_Matter_Profile_Energy_Get
\end{verbatim}
The function must return the energy of {\tt thisNode} in units of $M_\odot$ km$^2$ s$^{-1}$. The energy growth rate function should have the same template but return the rate of change of the energy in units of $M_\odot$ km$^2$ s$^{-1}$ Gyr$^{-1}$. The rotation normalization function has the same template but should return the normalization, $A$, in the relation $V_{\rm rot} = A j$ where $j$ is the mean specific angular momentum of the halo. $A$ should have units of Mpc$^{-1}$.

The halo circular velocity function must have the form:
\begin{verbatim}
 double precision function Dark_Matter_Profile_Circular_Velocity_Get(thisNode,radius)
    implicit none
    type(treeNode),   intent(inout), pointer :: thisNode
    double precision, intent(in)             :: radius
    .
    .
    .
    return
 end function Dark_Matter_Profile_Circular_Velocity_Get
\end{verbatim}
and should return the circular velocity (in km/s) at the specified {\tt radius} (in Mpc) in {\tt thisNode}. 
The halo enclosed mass function must have the form:
\begin{verbatim}
 double precision function Dark_Matter_Profile_Enclosed_Mass_Get(thisNode,radius)
    implicit none
    type(treeNode),   intent(inout), pointer :: thisNode
    double precision, intent(in)             :: radius
    .
    .
    .
    return
 end function Dark_Matter_Profile_Enclosed_Mass_Get
\end{verbatim}
and should return the enclosed mass (in $M_\odot$) at the specified {\tt radius} (in Mpc) in {\tt thisNode}. The halo potential function must have the form:
\begin{verbatim}
 double precision function Dark_Matter_Profile_Potential_Get(thisNode,radius,status)
    implicit none
    type(treeNode),   intent(inout), pointer  :: thisNode
    double precision, intent(in)              :: radius
    integer         , intent(  out), optional :: status
    .
    .
    .
    return
 end function Dark_Matter_Profile_Potential_Get
\end{verbatim}
and should return the gravitational potential (in km$^2$/s$^2$) at the specified {\tt radius} (in Mpc) in {\tt thisNode}. The potential is conventionally chosen such that $\Phi(r_{\rm virial}=-V_{\rm virial}^2$ so that the potential at infinity is zero if the halo profile is truncated at the virial radius. If the {\tt status} argument is present, it should be set to one of the codes provided by the \hyperlink{dark_matter_profiles.error_codes.F90:dark_matter_profiles_error_codes}{{\tt Dark\_Matter\_Profiles\_Error\_Codes}} module to indicate success of (the specific reason for) failure. The ``radius from specific angular momentum'' function should have the form:
\begin{verbatim}
 double precision function Dark_Matter_Profile_Radius_from_Specific_Angular_Momentum_Get(thisNode,specificAngularMomentum)
    implicit none
    type(treeNode),   intent(inout), pointer :: thisNode
    double precision, intent(in)             :: specificAngularMomentum
    .
    .
    .
    return
 end function Dark_Matter_Profile_Radius_from_Specific_Angular_Momentum_Get
\end{verbatim}
and should return the radius (in Mpc) at which a circular orbit in the halo has the specified {\tt specificAngularMomentum} (in units of km s$^{-1}$ Mpc).

Currently defined halo profile methods are:
\begin{description}
 \item [\hyperlink{dark_matter_profiles.isothermal.F90:dark_matter_profiles_isothermal:dark_matter_profile_isothermal_initialize}{{\tt isothermal}}] Implements an isothermal ($\rho \propto r^{-2}$) halo density profile.
\end{description}

\subsubsection{Halo Virial Density Contrast}

Additional methods for the halo virial density contrast can be added using the {\tt virialDensityContrastMethod} directive. The directive should contain a single argument, giving the name of a subroutine to be called to initialize the method. For example, the {\tt sphericalTopHat} method is described by a directive:
\begin{verbatim}
  !# <virialDensityContrastMethod>
  !#  <unitName>Spherical_Collape_Delta_Virial_Initialize</unitName>
  !# </virialDensityContrastMethod>
\end{verbatim}
Here, {\tt Spherical\_Collape\_Delta\_Virial\_Initialize} is the name of a subroutine which will be called to initialize the method. The initialization subroutine must have the following form:
\begin{verbatim}
  subroutine Method_Initialize(virialDensityContrastMethod,Virial_Density_Contrast_Tabulate)
    implicit none
    type(varying_string),          intent(in)    :: virialDensityContrastMethod
    procedure(),          pointer, intent(inout) :: Virial_Density_Contrast_Tabulate

    if (virialDensityContrastMethod.eq.'myMethod') then
       Virial_Density_Contrast_Tabulate => My_Do_Tabulate
       .
       .
       .
    end if
    return
  end subroutine Method_Initialize
\end{verbatim}
where {\tt myMethod} is the name of this method as will be specified by the {\tt virialDensityContrastMethod} input parameter. The procedure pointer {\tt Virial\_Density\_Contrast\_Tabulate} must be set to point to a subroutine which tabulates the critical overdensity as described below. The initialization subroutine should perform any other tasks required to initialize the module (such as reading parameters etc.).

The tabulation subroutine must have the form:
\begin{verbatim}
   subroutine Virial_Density_Contrast_Tabulate(time,deltaVirialTableNumberPoints,deltaVirialTime,deltaVirialDeltaVirial)
    implicit none
    double precision, intent(in)                               :: time
    double precision, allocatable, dimension(:), intent(inout) :: deltaVirialTime,deltaVirialDeltaVirial
    integer,                                     intent(out)   :: deltaVirialTableNumberPoints
    .
    .
    return
   end subroutine Virial_Density_Contrast_Tabulate
\end{verbatim}
The subroutine must tabulate the virial overdensity in array {\tt deltaVirialDeltaVirial()} as a function of wavenumber {\tt deltaVirialTime()} (these arrays must be allocated to the correct size, and may be prevously allocated, therefore requiring a deallocation). The number of tabulated points should be returned in {\tt deltaVirialNumberPoints}. The subroutine should ensure that the currently requested {\tt time} is within the range of the tabulated function (preferably with some buffer).

Currently defined virial density contrast methods are:
\begin{description}
 \item [{\tt sphericalTopHat}] The virial density contrast is computed for a Universe containing collisionless matter and a cosmological constant following the spherical top hat collapse model (see, for example, \citealt{percival_cosmological_2005}).
 \item [{\tt Bryan-Norman1998}] The virial density contrast is computed using the fitting functions of \cite{bryan_statistical_1998}. As such, it is valid only for $\Omega_\Lambda=0$ or $\Omega_{\rm M}+\Omega_\Lambda=1$ cosmologies and will abort on other cosmologies.
 \item [{\tt fixed}] The virial density contrast is fixed at {\tt [virialDensityConstrastFixed]}, defined relative to {\tt criticalDensity} and {\tt meanDensity} as specified by {\tt [virialDensityConstrastFixedType]}.
 \item [{\tt Kitayama-Suto1996}] The virial density constrast is computed using the fitting formula of \cite{kitayama_semianalytic_1996}, and is therefore valid only for flat cosmological models.
\end{description}

\subsubsection{Initial Mass Function Functions}\label{sec:IMF_functions}

Each registered \gls{imf} must provide multiple functions, specified by the following directives:
\begin{verbatim}
 !# <imfRecycledInstantaneous>
 !#  <unitName>Star_Formation_IMF_Recycled_Instantaneous_My_IMF</unitName>
 !# </imfRecycledInstantaneous>

 !# <imfYieldInstantaneous>
 !#  <unitName>Star_Formation_IMF_Yield_Instantaneous_My_IMF</unitName>
 !# </imfYieldInstantaneous>

 !# <imfTabulate>
 !#  <unitName>Star_Formation_IMF_Tabulate_My_IMF</unitName>
 !# </imfTabulate>

 !# <imfMinimumMass>
 !#  <unitName>Star_Formation_IMF_Minimum_Mass_My_IMF</unitName>
 !# </imfMinimumMass>

 !# <imfMaximumMass>
 !#  <unitName>Star_Formation_IMF_Maximum_Mass_My_IMF</unitName>
 !# </imfMaximumMass>

 !# <imfPhi>
 !#  <unitName>Star_Formation_IMF_Phi_My_IMF</unitName>
 !# </imfPhi>
\end{verbatim}

These functions/subroutines should have the following forms:
\begin{verbatim}
  subroutine Star_Formation_IMF_Recycled_Instantaneous_My_IMF(imfSelected,imfMatched,recycledFraction)
    integer,          intent(in)    :: imfSelected
    logical,          intent(inout) :: imfMatched
    double precision, intent(out)   :: recycledFraction

    if (imfSelected == imfIndex) then
       .
       .
       .
       imfMatched=.true.
    end if
    return
  end subroutine Star_Formation_IMF_Recycled_Instantaneous_My_IMF

  subroutine Star_Formation_IMF_Yield_Instantaneous_My_IMF(imfSelected,imfMatched,yield)
    integer,          intent(in)    :: imfSelected
    logical,          intent(inout) :: imfMatched
    double precision, intent(out)   :: yield

    if (imfSelected == imfIndex) then
       .
       .
       .
       imfMatched=.true.
    end if
    return
  end subroutine Star_Formation_IMF_Yield_Instantaneous_My_IMF

  subroutine Star_Formation_IMF_Tabulate_My_IMF(imfSelected,imfMatched,imfMass,imfPhi)
    integer,          intent(in)                               :: imfSelected
    logical,          intent(inout)                            :: imfMatched
    double precision, intent(inout), allocatable, dimension(:) :: imfMass,imfPhi

    if (imfSelected == imfIndex) then
       .
       .
       .
       imfMatched=.true.
    end if
    return
  end subroutine Star_Formation_IMF_Tabulate_My_IMF

  subroutine Star_Formation_IMF_Minimum_Mass_My_IMF(imfSelected,imfMatched,minimumMass)
    implicit none
    integer,          intent(in)    :: imfSelected
    logical,          intent(inout) :: imfMatched
    double precision, intent(out)   :: minimumMass
    
    if (imfSelected == imfIndex) then
       .
       .
       .
       imfMatched=.true.
    end if
    return
  end subroutine Star_Formation_IMF_Minimum_Mass_My_IMF

  subroutine Star_Formation_IMF_Maximum_Mass_My_IMF(imfSelected,imfMatched,minimumMass)
    implicit none
    integer,          intent(in)    :: imfSelected
    logical,          intent(inout) :: imfMatched
    double precision, intent(out)   :: maximumMass
    
    if (imfSelected == imfIndex) then
       .
       .
       .
       imfMatched=.true.
    end if
    return
  end subroutine Star_Formation_IMF_Maximum_Mass_My_IMF

  subroutine Star_Formation_IMF_Phi_My_IMF(imfSelected,imfMatched,imfMass,imfPhi)
    integer,          intent(in)                               :: imfSelected
    logical,          intent(inout)                            :: imfMatched
    double precision, intent(in)                               :: imfMass
    double precision, intent(out)                              :: imfPhi

    if (imfSelected == imfIndex) then
       .
       .
       .
       imfMatched=.true.
    end if
    return
  end subroutine Star_Formation_IMF_Phi_My_IMF
\end{verbatim}
In each case the procedure should check if the supplied {\tt imfSelected} index matches the index which this \gls{imf} was given when it was registered. If it is, then {\tt imfMatched} should be set to true. The procedures should then perform as follows:
\begin{description}
 \item [{\tt Star\_Formation\_IMF\_Yield\_Instantaneous\_My\_IMF}] Return a suitable metal yield in {\tt yield} for this \gls{imf} in the instantaneous recyclying approximation.
 \item [{\tt Star\_Formation\_IMF\_Recycled\_Instantaneous\_My\_IMF}] Return a suitable recycled fraction in {\tt recycledFraction} for this \gls{imf} in the instantaneous recyclying approximation.
 \item [{\tt Star\_Formation\_IMF\_Tabulate\_My\_IMF}] Allocate the {\tt imfMass()} and {\tt imfPhi()} arrays and fill them with a tabulation of the \gls{imf}. The routine can choose the size of the tabulation and should ensure that it is suffient to resolve any features in the \gls{imf}.
 \item [{\tt Star\_Formation\_IMF\_Minimum\_Mass\_My\_IMF}] Return the lowest mass for which the \gls{imf} is non-zero.
 \item [{\tt Star\_Formation\_IMF\_Maximum\_Mass\_My\_IMF}] Return the largest mass for which the \gls{imf} is non-zero.
 \item [{\tt Star\_Formation\_IMF\_Phi\_My\_IMF}] Return the \gls{imf} for the specified {\tt imfMass} initial stellar mass.
\end{description}
Currently defined IMFs are described in \S\ref{sec:physicsIMF}.

\subsubsection{Initial Mass Function Selection}

Additional methods for selection of initial mass functions can be added using the {\tt imfSelectionMethod} directive. The directive should contain a single argument, giving the name of a subroutine to be called to initialize the method. For example, the {\tt fixed} method is described by a directive:
\begin{verbatim}
 !# <imfSelectionMethod>
 !#  <unitName>IMF_Select_Fixed_Initialize</unitName>
 !# </imfSelectionMethod>
\end{verbatim}
Here, {\tt IMF\_Select\_Fixed\_Initialize} is the name of a subroutine which will be called to initialize the method. The initialization subroutine must have the following form:
\begin{verbatim}
  subroutine Method_Initialize(imfSelectionMethod,IMF_Select,imfNames)
    implicit none
    type(varying_string),          intent(in)    :: imfSelectionMethod,imfNames(:)
    procedure(),          pointer, intent(inout) :: IMF_Select

    if (imfSelectionMethod == 'myMethod') then
       IMF_Select_Fixed => My_Selection_Procedure
       .
       .
       .
    end if
    return
  end subroutine Method_Initialize
\end{verbatim}
where {\tt myMethod} is the name of this method as will be specified by the {\tt imfSelectionMethod} input parameter. The procedure pointer {\tt IMF\_Select} must be set to point to a function which returns the index of the selected \gls{imf} as described below. The initialization subroutine should perform any other tasks required to initialize the module (such as reading parameters etc.). The input array {\tt imfNames()} contains a list of all available \gls{imf} names and can be used for \hyperlink{star_formation.IMF.utilities.F90:star_formation_imf_utilities:imf_index_lookup}{index determination}.

The selection function must have the form:
\begin{verbatim}
 integer function IMF_Select(starFormationRate,fuelAbundances,component)
    double precision,          intent(in) :: starFormationRate
    type(abundancesStructure), intent(in) :: fuelAbundances
    integer,                   intent(in) :: component
    .
    .
    .
    return
 end function IMF_Select
\end{verbatim}
The function must return the index of the \gls{imf} appropriate for the given {\tt starFormationRate} (in $M_\odot$ Gyr$^{-1}$), {\tt fuelAbundances} and {\tt component} (using the component labels provided by the \hyperlink{galactic_structure.options.F90:galactic_structure_options}{{\tt Galactic\_Structure\_Options}} module).

Currently defined \gls{imf} selection methods are:
\begin{description}
 \item [{\tt fixed}] A fixed \gls{imf} is used irrespective of physical conditions. The \gls{imf} is specified by the input parameter {\tt imfSelectionFixed}.
 \item [{\tt diskSpheroid}] Uses different {\gls{imf}}s for star formation in disks and in spheroids irrespective of other physical conditions. The {\gls{imf}}s are specified by the input parameters {\tt imfSelectionDisk} and {\tt imfSelectionSpheroid}.
\end{description}

\subsubsection{Intergalactic Medium State}\label{sec:IntergalacticMediumStateMethods}\index{intergalactic medium}

Additional methods for the intergalactic medium state can be added using the {\tt intergalaticMediumStateMethod} directive. The directive should contain a single argument, giving the name of a subroutine to be called to initialize the method. For example, the {\tt RecFast} method is described by a directive:
\begin{verbatim}
  !# <intergalaticMediumStateMethod>
  !#  <unitName>Intergalactic_Medium_State_RecFast_Initialize</unitName>
  !# </intergalaticMediumStateMethod>
\end{verbatim}
Here, {\tt Intergalactic\_Medium\_State\_RecFast\_Initialize} is the name of a subroutine which will be called to initialize the method. The initialization subroutine must have the following form:
\begin{verbatim}
  subroutine Method_Initialize(intergalaticMediumStateMethod,Intergalactic_Medium_Electron_Fraction_Get&
       &,Intergalactic_Medium_Temperature_Get)
    implicit none
    type(varying_string),          intent(in)    :: intergalaticMediumStateMethod
    procedure(),          pointer, intent(inout) :: Intergalactic_Medium_Electron_Fraction_Get,Intergalactic_Medium_Temperature_Get
    
    if (intergalaticMediumStateMethod == 'myMethod') then
       Intergalactic_Medium_Electron_Fraction_Get => My_Intergalactic_Medium_Electron_Fraction
       Intergalactic_Medium_Temperature_Get       => My_Intergalactic_Medium_Temperature
    end if
    return
  end subroutine Method_Initialize
\end{verbatim}
where {\tt myMethod} is the name of this method as will be specified by the {\tt intergalaticMediumStateMethod} input parameter. The procedure pointer {\tt Intergalactic\_Medium\_Electron\_Fraction\_Get} must be set to point to a function which returns the electron fraction as described below, while {\tt Intergalactic\_Medium\_Temperature\_Get} must be set to point to a function which returns the temperature of the intergalactic medium.

Both functions must have the form:
\begin{verbatim}
 double precision function Property_Get(time)
    implicit none
    double precision, intent(in) :: time
    .
    .
    .
    return
 end function Property_Get
\end{verbatim}
The electron fraction function must return the electron number density in units of the hydrogen number density in the intergalactic medium, while the temperature function should return the temperature of the intergalactic medium (in Kelvin) at the specified time.

Currently defined intergalactic medium state methods are:
\begin{description}
 \item [{\tt file}] Reads a tabulation of the intergalactic medium state from a file and interpolates in the table to give a result. The XML file containing the table should have the following form:
 \begin{verbatim}
<igm>
  <electronFraction>
    <datum>1.1590278</datum>
    <datum>1.1590278</datum>
    .
    .
    .
  </electronFraction>
  <matterTemperature>
    <datum>27230.271</datum>
    <datum>27203.016</datum>
    .
    .
    .
  </matterTemperature>
  <redshift>
    <datum>9990.00</datum>
    <datum>9980.00</datum>
    .
    .
    .
  </redshift>
</igm>
 \end{verbatim}
 The {\tt electronFraction} and {\tt matterTemperature} elements should contain {\tt datum} elements listing the relevant quantity for each redshift listed in the {\tt redshift} element.
 \item [{\tt RecFast}] The state of the intergalactic medium is computed using the \href{http://www.astro.ubc.ca/people/scott/recfast.html}{{\sc RecFast}} code \cite{seager_how_2000,wong_how_2008}, which will be automatically downloaded, patched and built if necessary. The results from {\sc RecFast} are stored in a file which is then processed as in the {\tt file} method described above.
\end{description}

\subsubsection{Linear Growth Function}

Additional methods for the linear growth factor can be added using the {\tt linearGrowthMethod} directive. The directive should contain a single argument, giving the name of a subroutine to be called to initialize the method. For example, the {\tt simple} method is described by a directive:
\begin{verbatim}
  !# <linearGrowthMethod>
  !#  <unitName>Growth_Factor_Simple_Initialize</unitName>
  !# </linearGrowthMethod>
\end{verbatim}
Here, {\tt Growth\_Factor\_Simple\_Initialize} is the name of a subroutine which will be called to initialize the method. The initialization subroutine must have the following form:
\begin{verbatim}
  subroutine Method_Initialize(linearGrowthMethod,Linear_Growth_Tabulate)
    implicit none
    type(varying_string),          intent(in)    :: linearGrowthMethod
    procedure(),          pointer, intent(inout) :: Linear_Growth_Tabulate
    
    if (linearGrowthMethod.eq.'myMethod') then
       Linear_Growth_Tabulate => My_Do_Tabulate
       .
       .
       .
    end if
    return
  end subroutine Method_Initialize
\end{verbatim}
where {\tt myMethod} is the name of this method as will be specified by the {\tt linearGrowthMethod} input parameter. The procedure pointer {\tt Linear\_Growth\_Tabulate} must be set to point to a subroutine which tabulates the linear growth factor as described below. The initialization subroutine should perform any other tasks required to initialize the module (such as reading parameters etc.).

The tabulation subroutine must have the form:
\begin{verbatim}
   subroutine Linear_Growth_Tabulate(time,growthTableNumberPoints,growthTableTime,growthTableWavenumber &
   & ,growthTableGrowthFactor,normalizationMatterDominated)
    implicit none
    double precision, intent(in)                                   :: time
    integer,          intent(out)                                  :: growthTableNumberPoints
    double precision, intent(inout), allocatable, dimension(:)     :: growthTableTime,growthTableWavenumber
    double precision, intent(inout), allocatable, dimension(:,:,:) :: growthTableGrowthFactor
    double precision, intent(out),                dimension(3)     :: normalizationMatterDominated
    .
    .
    return
   end subroutine Linear_Growth_Tabulate
\end{verbatim}
The subroutine must tabulate the linear growth factor in array {\tt growthTableGrowthFactor()} for dark matter, baryons and radiation (entries 1, 2 and 3 of the first dimension respectively) as a function of time {\tt growthTableTime()} (second dimension) and wavenumber {\tt growthTableWavenumber()} (third dimension). These arrays must be allocated to the correct size, and may be prevously allocated, therefore requiring a deallocation. The number of tabulated points in the time dimension should be returned in {\tt growthTableNumberPoints}. It is permissible to tabulate for just a single wavenumber if the growth function is independent of wavenumber. The subroutine should ensure that the currently requested {\tt time} is within the range of the tabulated function (preferably with some buffer). The linear growth factors must be normalized to unity at $a=1$. Additionally, {\tt normalizationMatterDominated} should be set to the factor by which the tabulated growth factor (for the smallest wavenumber tabulated) 
must be multiplied such that it scales as $(9 \Omega_{\rm M} / 4.0d0)^{1/3} (H_0 t)^{2/3}$ during the matter dominated regime.

Currently defined linear growth factor methods are:
\begin{description}
 \item [{\tt simple}] The linear growth factor is computed for a Universe containing collisionless matter and a cosmological constant.
\end{description}

\subsubsection{Merger Tree Branching}

Additional methods for merger tree branching can be added using the {\tt treeBranchingMethod} directive. The directive should contain a single argument, giving the name of a subroutine to be called to initialize the method. For example, the {\tt modifiedPress-Schechter} method is described by a directive:
\begin{verbatim}
 !# <treeBranchingMethod>
 !#  <unitName>Modified_Press_Schechter_Branching_Initialize</unitName>
 !# </treeBranchingMethod>
\end{verbatim}
Here, {\tt Modified\_Press\_Schechter\_Branching\_Initialize} is the name of a subroutine which will be called to initialize the method. The initialization subroutine must have the following form:
\begin{verbatim}
 subroutine Method_Initialize(treeBranchingMethod,Tree_Branching_Probability,Tree_Subresolution_Fraction,Tree_Branch_Mass,Tree_Maximum_Step)
    type(varying_string),          intent(in)    :: treeBranchingMethod
    procedure(),          pointer, intent(inout) :: Tree_Branching_Probability,Tree_Subresolution_Fraction,Tree_Branch_Mass,Tree_Maximum_Step
    
    if (treeBranchingMethod == 'myMethod') then
       Tree_Branching_Probability  => My_Branching_Probability
       Tree_Subresolution_Fraction => My_Subresolution_Fraction
       Tree_Maximum_Step           => My_Maximum_Step
       Tree_Branch_Mass            => My_Branch_Mass
       .
       .
       .
    end if
    return
  end subroutine Method_Initialize
\end{verbatim}
where {\tt myMethod} is the name of this method as will be specified by the {\tt treeBranchingMethod} input parameter. The procedure pointers must be set to point to routines which perform various functions as described below. The initialization subroutine should perform any other tasks required to initialize the module (such as reading parameters etc.).

The procedure pointers must point to functions with the following templates:
\begin{verbatim}
  double precision function My_Branch_Mass(haloMass,deltaCritical,massResolution,probability)
    double precision, intent(in) :: haloMass,deltaCritical,massResolution,probability
    .
    .
    .
    return
 end function My_Branch_Mass

 double precision function My_Branching_Maximum_Step(haloMass,deltaCritical,massResolution)
    double precision, intent(in) :: haloMass,deltaCritical,massResolution
    .
    .
    .
    return
 end function My_Branching_Maximum_Step

  double precision function My_Branching_Probability(haloMass,deltaCritical,massResolution)
    double precision, intent(in) :: haloMass,deltaCritical,massResolution
    .
    .
    .
    return
 end function My_Branching_Probability

  double precision function My_Subresolution_Fraction(haloMass,deltaCritical,massResolution)
    double precision, intent(in) :: haloMass,deltaCritical,massResolution
    .
    .
    .
    return
 end function My_Subresolution_Fraction
\end{verbatim}
{\tt Tree\_Branching\_Probability} must point to a function which returns the probability per unit change in $\delta_{\rm crit}$ that a halo of mass {\tt haloMass} at time {\tt deltaCritical} will undergo a branching to progenitors with mass greater than {\tt massResolution}. {\tt Tree\_Subresolution\_Fraction} must point to a function which returns the fraction of mass accreted in subresolution halos, i.e. those below {\tt massResolution}, per unit change in $\delta_{\rm crit}$ for a halo of mass {\tt haloMass} at time {\tt deltaCritical}, or a negative value if the halo is so close to the resolution limit that this number cannot be determined accurately. {\tt Tree\_Maximum\_Step} must point to a function which returns the maximum allowed step in $\delta_{\rm crit}$ that a halo of mass {\tt haloMass} at time {\tt deltaCritical} should be allowed to take. {\tt Tree\_Branch\_Mass} must point to a function which returns the mass of one of the halos to which the given halo branches, given the branching 
probability, {\tt probability}.

Currently defined merger tree branching methods are:
\begin{description}
 \item [{\tt modifiedPress-Schechter}] Branching probabilities are computed using the method of \cite{parkinson_generating_2008}. Progenitor mass functions generated using \glc's implementation of this algorithm (and the \cite{cole_hierarchical_2000} merger tree building algorithm) are shown in Fig.~\ref{fig:PCH_Progenitor_MFs}.
 \item [{\tt generalizedPress-Schechter}] Branching probabilities are computed using excursion set barrier first crossing rates (computed using the selected {\tt excursionSetFirstCrossingMethod}; see \S\ref{sec:excursionSetFirstCrossingMethod}), modified by the selected {\tt treeBranchingModifierMethod} (see \S\ref{sec:treeBranchingModifierMethod}).
\end{description}

\begin{figure}
 \begin{center}
 \includegraphics[height=160mm,angle=90]{../plots/progenitorMassFunction.pdf}
 \end{center}
 \caption{Progenitor mass functions at redshifts $z=0.5$, 1, 2 and 4 (bottom to top) for halos of mass $10^{12\pm0.151}$, $10^{13.5\pm0.151}$ and $10^{15\pm0.151}h^{-1}M_\odot$ (left to right) are shown. Green lines are measured from the Millennium Simulation, while red lines are computed using \glc's merger tree building routines (with the \cite{parkinson_generating_2008} branching algorithm and the \cite{cole_hierarchical_2000} tree building algorithm).}
 \label{fig:PCH_Progenitor_MFs}
\end{figure}

\subsubsection{Merger Tree Branching Modifiers}\label{sec:treeBranchingModifierMethod}

Additional methods for merger tree branching probability modifiers can be added using the {\tt treeBranchingModifierMethod} directive. The directive should contain a single argument, giving the name of a subroutine to be called to initialize the method. For example, the {\tt null} method is described by a directive:
\begin{verbatim}
 !# <treeBranchingModifierMethod>
 !#  <unitName>Merger_Tree_Branching_Modifiers_Null_Initialize</unitName>
 !# </treeBranchingModifierMethod>
\end{verbatim}
Here, {\tt Merger\_Tree\_Branching\_Modifiers\_Null\_Initialize} is the name of a subroutine which will be called to initialize the method. The initialization subroutine must have the following form:
\begin{verbatim}
 subroutine Method_Initialize(treeBranchingModifierMethod,Merger_Tree_Branching_Modifier_Get)
    type     (varying_string  ),          intent(in   ) :: treeBranchingModifierMethod
    procedure(double precision), pointer, intent(inout) :: Merger_Tree_Branching_Modifier_Get
    
    if (treeBranchingModifierMethod == 'myMethod') then
       Merger_Tree_Branching_Modifier_Get => My_Get
       .
       .
       .
    end if
    return
  end subroutine Method_Initialize
\end{verbatim}
where {\tt myMethod} is the name of this method as will be specified by the {\tt treeBranchingModifierMethod} input parameter. The procedure pointer {\tt Merger\_Tree\_Branching\_Modifier\_Get} must build and return the modifier to the branching probability as described below. The initialization subroutine should perform any other tasks required to initialize the module (such as reading parameters etc.).

The branching probability modifier function should have the interface:
\begin{verbatim}
  subroutine Merger_Tree_Branching_Modifier_Get(parentDelta,childSigma,parentSigma)
    double precision, intent(in) :: parentDelta,childSigma,parentSigma
    .
    .
    .
    return
  end subroutine Merger_Tree_Branching_Modifier_Get
\end{verbatim}
and should return the multiplicative modifier to the branching probability for the given {\tt parentDelta}, {\tt childSigma} and {\tt parentSigma}.

Currently defined merger tree branching probability modifier methods are:
\begin{description}
 \item [{\tt null}] Makes no modification;
 \item [{\tt Parkinson-Cole-Helly2008}] Modifies branching rates according to the algorithm of \cite{parkinson_generating_2008}.
\end{description}

\subsubsection{Merger Tree Building}\label{sec:MergerTreeBuildMethod}

Additional methods for merger tree building can be added using the {\tt mergerTreeBuildMethod} directive. The directive should contain a single argument, giving the name of a subroutine to be called to initialize the method. For example, the {\tt Cole2000} method is described by a directive:
\begin{verbatim}
 !# <mergerTreeBuildMethod>
 !#  <unitName>Merger_Tree_Build_Cole2000_Initialize</unitName>
 !# </mergerTreeBuildMethod>
\end{verbatim}
Here, {\tt Merger\_Tree\_Build\_Cole2000\_Initialize} is the name of a subroutine which will be called to initialize the method. The initialization subroutine must have the following form:
\begin{verbatim}
 subroutine Method_Initialize(mergerTreeBuildMethod,Merger_Tree_Build)
    type(varying_string),          intent(in)    :: mergerTreeBuildMethod
    procedure(),          pointer, intent(inout) :: Merger_Tree_Build
    
    if (mergerTreeBuildMethod == 'myMethod') then
       Merger_Tree_Build => My_Do_Tabulate
       .
       .
       .
    end if
    return
  end subroutine Method_Initialize
\end{verbatim}
where {\tt myMethod} is the name of this method as will be specified by the {\tt mergerTreeBuildMethod} input parameter. The procedure pointer {\tt Merger\_Tree\_Build} must build and return a merger tree given the a base node as described below. The initialization subroutine should perform any other tasks required to initialize the module (such as reading parameters etc.).

\begin{verbatim}
  subroutine Merger_Tree_Build_Do(thisTree)
    type(mergerTree), intent(inout) :: thisTree
    .
    .
    .
    return
  end subroutine Merger_Tree_Build_Do
\end{verbatim}
and should return a full merger tree in {\tt thisTree} built from the base node which will already be set in {\tt thisTree}. The tree must have at least masses, times and parent/child/sibling links created. Other properties (e.g. spins) can be optionally included also.

Currently defined merger tree building methods are:
\begin{description}
 \item [{\tt Cole2000}] Uses the \cite{cole_hierarchical_2000} merger tree building algorithm.
\end{description}

\subsubsection{Merger Tree Construction}\label{sec:MergerTreeConstruction}

Additional methods for merger tree construction can be added using the {\tt mergerTreeConstructMethod} directive. The directive should contain a single argument, giving the name of a subroutine to be called to initialize the method. For example, the {\tt build} method is described by a directive:
\begin{verbatim}
 !# <mergerTreeConstructMethod>
 !#  <unitName>Merger_Tree_Build_Initialize</unitName>
 !# </mergerTreeConstructMethod>
\end{verbatim}
Here, {\tt Merger\_Tree\_Build\_Initialize} is the name of a subroutine which will be called to initialize the method. The initialization subroutine must have the following form:
\begin{verbatim}
   subroutine Method_Initialize(mergerTreeConstructMethod,Merger_Tree_Construct)
    type(varying_string),          intent(in)    :: mergerTreeConstructMethod
    procedure(),          pointer, intent(inout) :: Merger_Tree_Construct
    
    if (mergerTreeConstructMethod == 'myMethod') then
       Merger_Tree_Construct => My_Do_Tabulate
       .
       .
       .
    end if
    return
  end subroutine Method_Initialize
\end{verbatim}
where {\tt myMethod} is the name of this method as will be specified by the {\tt mergerTreeConstructMethod} input parameter. The procedure pointer {\tt Merger\_Tree\_Construct} must be set to point to a function which returns a fully constructed merger tree as described below. The initialization subroutine should perform any other tasks required to initialize the module (such as reading parameters etc.).

The construction subroutine should have the following form:
\begin{verbatim}
  subroutine Merger_Tree_Construct_Do(thisTree,skipTree)
    type(mergerTree), intent(inout) :: thisTree
    logical,          intent(in)    :: skipTree
    .
    .
    .
    return
  end subroutine Merger_Tree_Construct_Do
\end{verbatim}
and should return a full merger tree in {\tt thisTree}, unless {\tt skipTree} is true, in which case this tree will be skipped (i.e. not evolved or output) and so it suffices to merely allocate the base node---there is no need to create the entire tree (although it is permissible to do so)---and update any internal data (e.g. a count of trees constructed) as required. The tree must have at least masses, times and parent/child/sibling links created. Other properties (e.g. spins) can be optionally included also. By default, the tree is assumed to be ``uninitialized'', such that the merger tree initialization function will be called prior to the tree being evolve. If the tree construction method returns a fully initialized tree it should set {\tt thisTree\%initialized=.true.}.

Currently defined merger tree construction methods are:
\begin{description}
 \item [{\tt build}] Generates a set of halo masses distributed between {\tt mergerTreeBuildHaloMassMinimum} and {\tt mergerTreeBuildHaloMassMaximum} (with {\tt mergerTreeBuildTreesPerDecade} halos per decade of mass) at redshift {\tt mergerTreeBuildTreesBaseRedshift}, or with masses read from a file, and then uses the selected merger tree build method (see \S\ref{sec:MergerTreeBuildMethod}) to build trees from these base nodes;
 \item [{\tt read}] Reads merger tree data from an HDF5 file (see \S\ref{sec:MergerTreeFiles}). The file to read is specified by the {\tt [mergerTreeReadFileName]} parameter.
 \item [{\tt smoothAccretion}] Constructs a branchless merger tree with a smooth accretion history using the selected mass accretion history method (see \S\ref{sec:HaloMassAccretionHistory}). See \S\ref{sec:SmoothAccretion} for details.
 \item [{\tt stateRestore}] Intended primarily for debugging purposes, this method will restore a tree whose complete internal state was written to file. See \S\ref{sec:TreeConstructStateRestore} for details of how to use this method.
 \item [{\tt fullySpecified}] Intended primarily for constructing test cases, this method allows the full state of the merger tree (and all components of nodes) to be specified via an XML document. See \S\ref{sec:TreeConstructFullySpecified} for details of how to use this method.
\end{description}

\subsubsection{Non-linear Matter Power Spectrum}\index{power spectrum!non-linear}

Additional methods for the non-linear matter power spectrum can be added using the {\tt powerSpectrumNonlinearMethod} directive. The directive should contain a single argument, giving the name of a subroutine to be called to initialize the method. For example, the {\tt Peacock-Dodds1996} method is described by a directive:
\begin{verbatim}
  !# <powerSpectrumNonlinearMethod>
  !#  <unitName>Nonlinear_Power_Spectrum_Power_Law_Initialize</unitName>
  !# </powerSpectrumNonlinearMethod>
\end{verbatim}
Here, {\tt Power\_Spectrum\_Nonlinear\_PeacockDodds1996\_Initialize} is the name of a subroutine which will be called to initialize the method. The initialization subroutine must have the following form:
\begin{verbatim}
  subroutine Method_Initialize(powerSpectrumNonlinearMethod,Power_Spectrum_Nonlinear_Get)
    implicit none
    type     (varying_string  ),          intent(in   ) :: powerSpectrumNonlinearMethod
    procedure(double precision), pointer, intent(inout) :: Power_Spectrum_Nonlinear_Get
    
    if (powerSpectrumNonlinearMethod == 'myMethod') then
       Power_Spectrum_Nonlinear_Get => My_Get
       .
       .
       .
    end if
    return
  end subroutine Method_Initialize
\end{verbatim}
where {\tt myMethod} is the name of this method as will be specified by the {\tt powerSpectrumNonlinearMethod} input parameter. The procedure pointer {\tt Power\_Spectrum\_Nonlinear\_Get} must be set to point to a subroutine which computes the non-linear matter power spectrum as described below. The initialization subroutine should perform any other tasks required to initialize the module (such as reading parameters etc.).

The non-linear matter power spectrum function must have the form:
\begin{verbatim}
   double precision function Nonlinear_Power_Spectrum(wavenumber,time)
    implicit none
    double precision, intent(in   ) :: wavenumber,time
    .
    .
    .
    return
   end subroutine Nonlinear_Power_Spectrum
\end{verbatim}
The function must return the non-linear matter power spectrum, $P_{\rm nl}(k)$, at the requested {\tt wavenumber} (in units of Mpc$^{-1}$) and {\tt time} (in units of Gyr).

Currently defined non-linear matter power spectrum methods are:
\begin{description}
 \item [{\tt linear}] Simply returns the linear matter power spectrum. Intended primarily for testing purposes.
 \item [{\tt Peacock-Dodds1996}] Uses the fitting function of \cite{peacock_non-linear_1996} to compute the non-linear matter power spectrum.
 \item [{\tt CosmicEmu}] Utilizes the cosmic emulator (``CosmicEmu'') code of \cite{lawrence_coyote_2010} to compute the non-linear matter power spectrum.
\end{description}

\subsubsection{Chemical Reaction Rates}

Additional methods for chemical species reaction rates can be added using the {\tt chemicalReactionRatesMethods} directive. Note that more than one method can be specified in which cases rates are cumulative over all selected methods. The directive should contain a single argument, giving the name of a subroutine to be called to initialize the method. For example, the {\tt hydrogenNetwork} method is described by a directive:
\begin{verbatim}
  !# <chemicalReactionRatesMethods>
  !#  <unitName>Chemical_Hydrogen_Rates_Initialize</unitName>
  !# </chemicalReactionRatesMethods>
\end{verbatim}
Here, {\tt Chemical\_Hydrogen\_Rates\_Initialize} is the name of a subroutine which will be called to initialize the method. The initialization subroutine must have the following form:
\begin{verbatim}
  subroutine Method_Initialize(chemicalReactionRatesMethods)
    implicit none
    type(varying_string), intent(in) :: chemicalReactionRatesMethods
    
    if (chemicalReactionRatesMethods == 'myMethod') then
       ratesSelected = .true.
       .
       .
       .
    end if
    return
  end subroutine Method_Initialize
\end{verbatim}
where {\tt myMethod} is the name of this method as will be specified by the {\tt chemicalReactionRatesMethods} input parameter. The {\tt ratesSelected} variable is set to true if the method is active and will be checked on all subsequent calls to the module such that rates are computed only if {\tt ratesSelected} is true. The initialization subroutine should perform any other tasks required to initialize the module (such as reading parameters etc.).

The method must provide a subroutine to compute the chemical reaction rates. This subroutine is specified by the {\tt chemicalRatesCompute} directive. The directive should contain a single argument, giving the name of a subroutine to be called to compute rates. For example, the {\tt hydrogenNetwork} method uses:
\begin{verbatim}
  !# <chemicalRatesCompute>
  !#  <unitName>Chemical_Hydrogen_Rates_Compute</unitName>
  !# </chemicalRatesCompute> 
\end{verbatim}
Here, {\tt Chemical\_Hydrogen\_Rates\_Compute} is the name of a subroutine which will be called to compute the rates. The rates subroutine must have the following form:
\begin{verbatim}
  subroutine Compute_Rates(temperature,chemicalDensity,radiation,chemicalRates)
    implicit none
    type(chemicalAbundancesStructure), intent(in)    :: chemicalDensity
    double precision,                  intent(in)    :: temperature
    type(radiationStructure),          intent(in)    :: radiation
    type(chemicalAbundancesStructure), intent(inout) :: chemicalRates

    ! Exit immediately if this method is not active.
    if (.not.ratesSelected) return

    ! Compute rates for all species present.
    .
    .
    .
    return
  end subroutine Compute_Rates
\end{verbatim}
Here, {\tt temperature} is the temperature of the gas, {\tt chemicalDensity} provides the densities (in cm$^{-3}$) of all chemicals, the radiation field is described by the {\tt radiation} object and any reaction rates should be \emph{added to} the {\tt chemicalRates} object in units of cm$^{-3}$ s$^{-1}$.

Currently defined chemical reaction rate methods are:
\begin{description}
 \item [{\tt null}] A null method which does not affect any rates.
 \item [{\tt hydrogenNetwork}] Computes rates using the network of reactions and fitting functions from \cite{abel_modeling_1997} and \cite{tegmark_small_1997}.
\end{description}

\subsubsection{Population III Supernovae}

Additional methods for Population III supernovae can be added using the {\tt supernovaePopIIIMethod} directive. The directive should contain a single argument, giving the name of a subroutine to be called to initialize the method. For example, the {\tt Heger-Woosley2002} method is described by a directive:
\begin{verbatim}
  !# <supernovaePopIIIMethod>
  !#  <unitName>Supernovae_Population_III_HegerWoosley_Initialize</unitName>
  !# </supernovaePopIIIMethod>
\end{verbatim}
Here, {\tt Supernovae\_Population\_III\_HegerWoosley\_Initialize} is the name of a subroutine which will be called to initialize the method. The initialization subroutine must have the following form:
\begin{verbatim}
  subroutine Method_Initialize(supernovaePopIIIMethod,SNePopIII_Cumulative_Energy_Get)
    implicit none
    type(varying_string),          intent(in)    :: supernovaePopIIIMethod
    procedure(),          pointer, intent(inout) :: SNePopIII_Cumulative_Energy_Get
    
    if (supernovaePopIIIMethod == 'myMethod') then
       SNePopIII_Cumulative_Energy_Get => My_SNePopIII_Cumulative_Energy_Get
       .
       .
       .
    end if
    return
  end subroutine Method_Initialize
\end{verbatim}
where {\tt myMethod} is the name of this method as will be specified by the {\tt supernovaePopIIIMethod} input parameter. The procedure pointer {\tt SNePopIII\_Cumulative\_Energy\_Get} must be set to point to a function which returns the cumulative energy input from Population III supernovae as described below. The initialization subroutine should perform any other tasks required to initialize the module (such as reading parameters etc.).

The functions must have the form:
\begin{verbatim}
   double precision function PopIII_Cumulative_Energy(initialMass,age,metallicity)
    implicit none
    double precision, intent(in) :: initialMass,age,metallicity
    .
    .
    .
    return
   end function PopIII_Cumulative_Energy 
\end{verbatim}
This function must return the cumulative energy (in $M_\odot$ (km/s)$^2$) from Population III supernovae resulting from a star with given {\tt initialMass} and {\tt metallicity} after a time {\tt age}.

Currently defined population III supernovae methods are:
\begin{description}
 \item [{\tt Heger-Woosley2002}] Computes the energy input from the pair-instability results of \cite{heger_nucleosynthetic_2002}.
\end{description}

\subsubsection{Power Spectrum Variance Window Function}\label{sec:PowerSpectrumWindowFunction}

Additional methods for the window function used to compute variance from the power spectrum can be added using the {\tt powerSpectrumWindowFunctionMethod} directive. The directive should contain a single argument, giving the name of a subroutine to be called to initialize the method. For example, the {\tt topHat} method is described by a directive:
\begin{verbatim}
  !# <powerSpectrumWindowFunctionMethod>
  !#  <unitName>Power_Spectrum_Window_Functions_Top_Hat_Initialize</unitName>
  !# </powerSpectrumWindowFunctionMethod>
\end{verbatim}
Here, {\tt Power\_Spectrum\_Window\_Functions\_Top\_Hat\_Initialize} is the name of a subroutine which will be called to initialize the method. The initialization subroutine must have the following form:
\begin{verbatim}
  subroutine Method_Initialize(powerSpectrumWindowFunctionMethod,Power_Spectrum_Window_Function_Get)
    implicit none
    type     (varying_string  ),          intent(in   ) :: powerSpectrumWindowFunctionMethod
    procedure(double precision), pointer, intent(inout) :: Power_Spectrum_Window_Function_Get
    procedure(double precision), pointer, intent(inout) :: Power_Spectrum_Window_Function_Wavenumber_Maximum_Get
    
    if (powerSpectrumWindowFunctionMethod == 'myMethod') then
       Power_Spectrum_Window_Function_Get                    => My_Window_Function
       Power_Spectrum_Window_Function_Wavenumber_Maximum_Get => My_Window_Function_Maximum_Wavelength
       .
       .
       .
    end if
    return
  end subroutine Method_Initialize
\end{verbatim}
where {\tt myMethod} is the name of this method as will be specified by the {\tt powerSpectrumWindowFunctionMethod} input parameter. The procedure pointers {\tt Power\_Spectrum\_Window\_Function\_Get}, and {\tt Power\_Spectrum\_Window\_Function\_Wavenumber\_Maximum\_Get} must be set to point to functions which return the window function for a given wavenumber and mass, and the maximum wavenumber for which that window function is non-zero respectively. The initialization subroutine should perform any other tasks required to initialize the module (such as reading parameters etc.).

The window function function must have the form:
\begin{verbatim}
   subroutine Power_Spectrum_Window_Function(wavenumber,smoothingMass)
    implicit none
    double precision, intent(in   ) :: wavenumber,smoothingMass
     .
    .
    .
    return
   end subroutine Power_Spectrum_Window_Function
\end{verbatim}
The function should return the window function for the specified {\tt wavenumber} (given in Mpc$^{-1}$) and the given {\tt smoothingMass} (given in $M_\odot$).

The window function maximum wavenumber function must have the form:
\begin{verbatim}
   subroutine Power_Spectrum_Window_Function_Wavenumber_Maximum(smoothingMass)
    implicit none
    double precision, intent(in   ) :: smoothingMass
     .
    .
    .
    return
   end subroutine Power_Spectrum_Window_Function_Wavenumber_Maximum
\end{verbatim}
The function should return the largest wavenumber for which the window function is non-zero for the given {\tt smoothingMass} (given in $M_\odot$). If the window function is non-zero as $k\rightarrow\infty$ then a suitably large value (e.g. $10^{30}$Mpc$^{-1}$) should be returned.

Currently defined power spectrum variance window function methods are:
\begin{description}
 \item [{\tt topHat}] The window function is a top-hat in real-space.
 \item [{\tt kSpaceSharp}] The window function is a top-hat in $k$-space.
 \item [{\tt topHatKSpaceSharpHybrid}] A convolution of top-hat in real space and top-hat in $k$-space window functions.
\end{description}

\subsubsection{Primordial Power Spectrum}

Additional methods for the primordial power spectrum can be added using the {\tt powerSpectrumMethod} directive. The directive should contain a single argument, giving the name of a subroutine to be called to initialize the method. For example, the {\tt powerLaw} method is described by a directive:
\begin{verbatim}
  !# <powerSpectrumMethod>
  !#  <unitName>CDM_Primordial_Power_Spectrum_Power_Law_Initialize</unitName>
  !# </powerSpectrumMethod>
\end{verbatim}
Here, {\tt CDM\_Primordial\_Power\_Spectrum\_Power\_Law\_Initialize} is the name of a subroutine which will be called to initialize the method. The initialization subroutine must have the following form:
\begin{verbatim}
  subroutine Method_Initialize(powerSpectrumMethod,Power_Spectrum_Tabulate)
    implicit none
    type(varying_string),          intent(in)    :: powerSpectrumMethod
    procedure(),          pointer, intent(inout) :: Power_Spectrum_Tabulate
    
    if (powerSpectrumMethod.eq.'myMethod') then
       Power_Spectrum_Tabulate => My_Do_Tabulate
       .
       .
       .
    end if
    return
  end subroutine Method_Initialize
\end{verbatim}
where {\tt myMethod} is the name of this method as will be specified by the {\tt powerSpectrumMethod} input parameter. The procedure pointer {\tt Power\_Spectrum\_Tabulate} must be set to point to a subroutine which tabulates the power spectrum as described below. The initialization subroutine should perform any other tasks required to initialize the module (such as reading parameters etc.).

The tabulation subroutine must have the form:
\begin{verbatim}
   subroutine Power_Spectrum_Tabulate(wavenumber,powerSpectrumNumberPoints,powerSpectrumLogWavenumber,powerSpectrumLogP)
    implicit none
    double precision,                            intent(in)    :: wavenumber
    double precision, allocatable, dimension(:), intent(inout) :: powerSpectrumLogWavenumber,powerSpectrumLogP
    integer,                                     intent(out)   :: powerSpectrumNumberPoints
    .
    .
    .
    return
   end subroutine Power_Spectrum_Tabulate
\end{verbatim}
The subroutine must tabulate the natural log of the power spectrum in array {\tt powerSpectrumLogP()} as a function of the natural log of wavenumber {\tt powerSpectrumLogWavenumber()} (these arrays must be allocated to the correct size, and may be prevously allocated, therefore requiring a deallocation). The number of tabulated points should be returned in {\tt powerSpectrumNumberPoints}. The subroutine should ensure that the currently requested {\tt wavenumber} is within the range of the tabulated function (preferably with some buffer).

Currently defined power spectrum methods are:
\begin{description}
 \item [{\tt powerLaw}] The power spectrum is assumed to be a power law, possibly with a running index. It is defined by
\begin{equation}
 P(k)\propto k^{n_{\rm s} + \ln(k/k_{\rm ref}) [\d n /\d\ln k]},
\end{equation}
where the parameters are specified by input parameters $n_{\rm s}\equiv${\tt powerSpectrumIndex}, $k_{\rm ref}\equiv${\tt powerSpectrumReferenceWavenumber} and $\d n / \d \ln k \equiv${\tt powerSpectrumRunning}.
\end{description}

\subsubsection{Radiation Components}\index{radiation}\label{sec:radiationComponents}

Radiation components (i.e. types of radiation field that may be added to any radiation object; see \S\ref{sec:RadiationSubsystem}) are defined using a combination of several directives: {\tt radiationLabel}, {\tt radiationSet}, {\tt radiationTemperature} and {\tt radiationFlux}. For example, the cosmic microwave background radiation component is defined by the following set of directives:
\begin{verbatim}
  !# <radiationLabel>
  !#  <label>CMB</label>
  !# </radiationLabel>

  !# <radiationSet>
  !#  <unitName>Radiation_Set_CMB</unitName>
  !#  <label>CMB</label>
  !# </radiationSet>

  !# <radiationTemperature>
  !#  <unitName>Radiation_Temperature_CMB</unitName>
  !#  <label>CMB</label>
  !# </radiationTemperature>

  !# <radiationFlux>
  !#  <unitName>Radiation_Flux_CMB</unitName>
  !#  <label>CMB</label>
  !# </radiationFlux>
\end{verbatim}
The first of these, {\tt radiationLabel}, should contain a single element, {\tt label}, which gives a label that will be used to identify this component, both in other directives and also in the internal parameters used to select this radiation component (e.g. in this case, a parameter {\tt radiationTypeCMB} will be available within \glc\ to select the cosmic microwave background component). The other directives must all specify the same {\tt label} element and additional give, in a {\tt unitName} element, the name of a function/subroutine to be called to perform the relevant calculation.

The {\tt radiationSet} directive must specify a subroutine with the following template:
\begin{verbatim}
  subroutine Radiation_Set(componentMatched,thisNode,radiationProperties)
   implicit none
   logical,          intent(in)                               :: componentMatched
   type(treeNode),   intent(inout), pointer                   :: thisNode
   double precision, intent(inout), allocatable, dimension(:) :: radiationProperties

   if (.not.componentMatched) return
   .
   .
   .
   return
  end subroutine Radiation_Set
\end{verbatim}
If {\tt componentMatched} is true, then the subroutine should set the radiation component, otherwise it should exit immediately. If the radiation component is to be set, then the routine can allocate the {\tt radiationProperties} array as necessary to store any data needed to specify the radiation field. These data should then be set using, if necessary, any relevant information from {\tt thisNode}.

The {\tt radiationTemperature} directive should specify a subroutine with the following template:
\begin{verbatim}
  subroutine Radiation_Temperature(requestedType,ourType,radiationProperties,radiationTemperature,radiationType)
   implicit none
   integer,          intent(in)                            :: requestedType,ourType
   double precision, intent(in),              dimension(:) :: radiationProperties
   double precision, intent(inout)                         :: radiationTemperature
   integer,          intent(in),    optional, dimension(:) :: radiationType

   if (requestedType /= ourType) return
   if (present(radiationType)) then
      if (all(radiationType /= ourType)) return
   end if
   .
   .
   .
   return
  end subroutine Radiation_Temperature
\end{verbatim}
The tests in the above should always be included so that the subroutine exits immediately if the component type is not active or not requested. Once these tests have been made, the subroutine should set the temperature (in units of Kelvin) of the radiation field (if applicable).

The {\tt radiationFlux} directive should specify a subroutine with the following template:
\begin{verbatim}
  subroutine Radiation_Flux(requestedType,ourType,radiationProperties,wavelength,radiationFlux,radiationType)
    implicit none
    integer,          intent(in)                           :: requestedType,ourType
    double precision, intent(in)                           :: wavelength
    double precision, intent(in),             dimension(:) :: radiationProperties
    double precision, intent(inout)                        :: radiationFlux
    integer,          intent(in),   optional, dimension(:) :: radiationType

    if (requestedType /= ourType) return
    if (present(radiationType)) then
       if (all(radiationType /= ourType)) return
    end if
    .
    .
    .
    return
  end subroutine Radiation_Flux
\end{verbatim}
The tests in the above should always be included so that the subroutine exits immediately if the component type is not active or not requested. Once these tests have been made, the subroutine should add the flux (in units of ergs cm$^2$ s$^{-1}$ Hz$^{-1}$ ster$^{-1}$) at the specified {\tt wavelength} (in units of \AA) of the radiation field to that in {\tt radiationFlux}.

Currently defined radiation component types are:
\begin{description}
 \item [{\tt null}] A null component with no radiation.
 \item [{\tt CMB}] The cosmic microwave background, assumed to be a perfect blackbody spectrum with a temperature equal to {\tt [T\_CMB]}$(1+z)$.
 \item [{\tt IGB}] The intergalactic background light, set using the method selected by {\tt [radiationIntergalacticBackgroundMethod]; see \S\ref{sec:radiationIGB}}.
\end{description}

\subsubsection{Radiation Components: Intergalactic Background}\label{sec:radiationIGB}

Additional methods for the intergalactic background radiation component can be added using the {\tt radiationIntergalacticBackgroundMethod} directive. The directive should contain a single argument, giving the name of a subroutine to be called to initialize the method. For example, the {\tt file} method is described by a directive:
\begin{verbatim}
 !# <radiationIntergalacticBackgroundMethod>
 !#  <unitName>Radiation_IGB_File_Initialize</unitName>
 !# </radiationIntergalacticBackgroundMethod>
\end{verbatim}
Here, {\tt Radiation\_IGB\_File\_Initialize} is the name of a subroutine which will be called to initialize the method. The initialization subroutine must have the following form:
\begin{verbatim}
  subroutine Method_Initialize(radiationIntergalacticBackgroundMethod,Radiation_Set_Intergalactic_Background_Do,Radiation_Flux_Intergalactic_Background_Do)
    implicit none
    type(varying_string),          intent(in)    :: radiationIntergalacticBackgroundMethod
    procedure(),          pointer, intent(inout) :: Radiation_Set_Intergalactic_Background_Do,Radiation_Flux_Intergalactic_Background_Do
    
    if (radiationIntergalacticBackgroundMethod == 'myMethod') then
      Radiation_Set_Intergalactic_Background_Do  => My_Method_Set
      Radiation_Flux_Intergalactic_Background_Do => My_Method_Flux
    end if
    return
  end subroutine Method_Initialize
\end{verbatim}
where {\tt myMethod} is the name of this method as will be specified by the {\tt radiationIntergalacticBackgroundMethod} input parameter. The procedure pointers {\tt Radiation\_Set\_Intergalactic\_Background\_Do} and {\tt Radiation\_Flux\_Intergalactic\_Background\_Do} must be set to point to subroutines which set the radiation field and return its flux as described below. The initialization subroutine should perform any other tasks required to initialize the module (such as reading parameters etc.).

The set subroutine must have the form:
\begin{verbatim}
  subroutine My_Method_Set(thisNode,radiationProperties)
    implicit none
    type(treeNode),   intent(inout), pointer                   :: thisNode
    double precision, intent(inout), allocatable, dimension(:) :: radiationProperties

    return
  end subroutine My_Method_Set
\end{verbatim}
and should set the radiation component as described in \S\ref{sec:radiationComponents}. The flux subroutine must have the form:
\begin{verbatim}
   subroutine My_Method_Flux(radiationProperties,wavelength,radiationFlux)
    implicit none
    double precision, intent(in)                 :: wavelength
    double precision, intent(in),   dimension(:) :: radiationProperties
    double precision, intent(inout)              :: radiationFlux

    return
   end subroutine My_Method_Flux
\end{verbatim}
and should increment {\tt radiationFlux} as described in \S\ref{sec:radiationComponents}.

Currently defined intergalactic background radiation methods are:
\begin{description}
 \item [{\tt file}] The intergalatic background radiation field, specified as a function of cosmic time, is read from a file. The flux is determined by linearly interpolating to the required time and wavelength. The XML file to read is specified by {\tt [radiationIGBFileName]}. An example of the required file structure is:
 \begin{verbatim}
<spectrum>
  <URL>http://adsabs.harvard.edu/abs/1996ApJ...461...20H</URL>
  <description>Cosmic background radiation spectrum from quasars alone.</description>
  <reference>Haardt, F. &amp; Madau, P. 1996, ApJ, 461, 20</reference>
  <source>Francesco Haardt on Aug 6 2005, via Cloudy 08.00</source>
  <wavelengths>
    <datum>0.0002481</datum>
    <datum>0.001489</datum>
    .
    .
    .
    <units>Angstroms</units>
  </wavelengths>
  <spectra>
    <datum>7.039E-49</datum>
    <datum>8.379E-48</datum>
    <datum>1.875E-39</datum>
    <datum>7.583E-38</datum>
    .
    .
    .
    <redshift>0</redshift>
    <units>erg cm^-2 s^-1 Hz^-1 sr^-1</units>
  </spectra>
</spectrum>
 \end{verbatim}
\end{description}
The optional {\tt URL}, {\tt description}, {\tt reference} and {\tt source} elements can be used to give the provenance of the data. The {\tt wavelengths} element should contain a set of {\tt datum} elements each containing a wavelength (in increasing order) at which the spectrum will be tabulated. Wavelengths must be given in Angstroms. Multiple {\tt spectra} elements can be given, each specifying the spectrum at a redshift as given in the {\tt redshift} element. Each {\tt spectra} element must contain an array of {\tt datum} elements that gives the spectrum at each wavelength listed in the {\tt wavelength} element. Spectra must be in units of erg cm$^{-2}$ s$^{-1}$ Hz$^{-1}$ sr$^{-1}$.

\subsubsection{Ram Pressure Mass Loss Rates in Disks/Spheroids}

Additional methods for computing ram pressure induced mass loss rates in disks/spheroids can be added using the {\tt ramPressureStrippingMassLossRate(Disks|Spheroids)Method} directive. The directive should contain a single argument, giving the name of a subroutine to be called to initialize the method. For example, the {\tt simple} method for disks is described by a directive:
\begin{verbatim}
 !# <ramPressureStrippingMassLossRateDisksMethod>
 !#  <unitName>Ram_Pressure_Stripping_Mass_Loss_Rate_Disks_Simple_Init</unitName>
 !# </ramPressureStrippingMassLossRateDisksMethod>
\end{verbatim}
Here, {\tt Ram\_Pressure\_Stripping\_Mass\_Loss\_Rate\_Disks\_Simple\_Init} is the name of a subroutine which will be called to initialize the method. The initialization subroutine must have the following form:
\begin{verbatim}
  subroutine Method_Initialize(ramPressureStrippingMassLossRateDisksMethod,Ram_Pressure_Stripping_Mass_Loss_Rate_Disk_Get)
    implicit none
    type(varying_string),          intent(in)    :: starFormationTimescaleDisksMethod
    procedure(),          pointer, intent(inout) :: Ram_Pressure_Stripping_Mass_Loss_Rate_Disk_Get
    
    if (ramPressureStrippingMassLossRateDisksMethod == 'myMethod') Ram_Pressure_Stripping_Mass_Loss_Rate_Disk_Get => My_Ram_Pressure_Stripping_Mass_Loss_Rate_Disk_Get
    return
  end subroutine Method_Initialize
\end{verbatim}
where {\tt myMethod} is the name of this method as will be specified by the {\tt ramPressureStrippingMassLossRate(Disks|Spheroids)Method} input parameter. The procedure pointer {\tt Ram\_Pressure\_Stripping\_Mass\_Loss\_Rate\_(Disk|Spheroid)\_Get} must be set to point to a function which returns mass loss rate due to ram pressure as described below. The initialization subroutine should perform any other tasks required to initialize the module (such as reading parameters etc.).

The mass loss rate function must have the form:
\begin{verbatim}
 double precision function Ram_Pressure_Stripping_Mass_Loss_Rate_Disk_Get(thisNode)
    implicit none
    type(treeNode), intent(in) :: thisNode
    .
    .
    .
    return
 end function Ram_Pressure_Stripping_Mass_Loss_Rate_Disk_Get
\end{verbatim}
The function must return the mass loss rate induced by ram pressure forces (in units of $M_\odot$/Gyr) for the disk/spheroid in {\tt thisNode}.

Currently defined ram pressure mass loss rate methods are:
\begin{description}
 \item [\hyperlink{ram_pressure_stripping.mass_loss_rate.disks.simple.F90:ram_pressure_stripping_mass_loss_rate_disks_simple:ram_pressure_stripping_mass_loss_rate_disk_simple}{{\tt simple}}] The mass loss rate scales in proportion to the ratio of ram pressure and gravitational restoring forces;
 \item [\hyperlink{ram_pressure_stripping.mass_loss_rate.disks.null.F90:ram_pressure_stripping_mass_loss_rate_disks_null:ram_pressure_stripping_mass_loss_rate_disk_null}{{\tt null}}] The mass loss rate is assumed to be always zero.
\end{description}

\subsubsection{Satellite Merging Mass Movements}

Additional methods for the satellite merging mass movements can be added using the {\tt satelliteMergingMassMovementsMethod} directive. The directive should contain a single argument, giving the name of a subroutine to be called to initialize the method. For example, the {\tt simple} method is described by a directive:
\begin{verbatim}
 !# <satelliteMergingMassMovementsMethod>
 !#  <unitName>Satellite_Merging_Mass_Movements_Simple_Initialize</unitName>
 !# </satelliteMergingMassMovementsMethod>
\end{verbatim}
Here, {\tt Satellite\_Merging\_Mass\_Movements\_Simple\_Initialize} is the name of a subroutine which will be called to initialize the method. The initialization subroutine must have the following form:
\begin{verbatim}
  subroutine Method_Initialize(satelliteMergingMassMovementsMethod,Satellite_Merging_Mass_Movement_Get)
    implicit none
    type(varying_string),          intent(in)    :: satelliteMergingMassMovementsMethod
    procedure(),          pointer, intent(inout) :: Satellite_Merging_Mass_Movement_Get
    
    if (satelliteMergingMassMovementsMethod == 'simple') Satellite_Merging_Mass_Movement_Get => My_Method_Get
    return
  end subroutine Method_Initialize
\end{verbatim}
where {\tt myMethod} is the name of this method as will be specified by the {\tt satelliteMergingMassMovementsMethod} input parameter. The procedure pointer {\tt Satellite\_Merging\_Mass\_Movement\_Get} must be set to point to a function which sets the mass movement descriptors as described below. The initialization subroutine should perform any other tasks required to initialize the module (such as reading parameters etc.).

The mass movement subroutine must have the form:
\begin{verbatim}
  subroutine My_Method_Get(thisNode,gasMovesTo,starsMoveTo,hostGasMovesTo,hostStarsMoveTo,mergerIsMajor)
    implicit none
    type(treeNode), intent(inout), pointer  :: thisNode
    integer,        intent(out)             :: gasMovesTo,starsMoveTo,hostGasMovesTo,hostStarsMoveTo
    logical,        intent(out)             :: mergerIsMajor
    .
    .
    .
    return
  end subroutine My_Method_Get
\end{verbatim}
The subroutine must return values for each of the ``{\tt MoveTo}'' descriptors to specify where stars and gas from {\tt thisNode} and {\tt thisNode}'s host node should move to in the host. Allowed values are:
\begin{description}
 \item [{\tt movesToDisk}] The material in question moves to the disk of the host node;
 \item [{\tt movesToSpheroid}] The material in question moves to the spheroid of the host node;
 \item [{\tt doesNotMove}] The material in question does not move (allowed only for host node descriptors).
\end{description}
Additionally, the {\tt mergerIsMajor} flag should be set to indicate whether this merger is deemed to be ``major'' (typically defined as one which redistributes mass from a disk into a spheroidal component).

Currently defined satellite merger mass movement methods are:
\begin{description}
 \item [{\tt verySimple}] In this case, the satellite is always added to the disk of the host, while material in the host does not move.
 \item [{\tt simple}] If the baryonic mass of the satellite exceeds a fraction {\tt majorMergerMassRatio} of the baryonic mass of the host then all material is moved to the spheroid of the host. Otherwise, satellite gas moves to the component given by {\tt minorMergerGasMovesTo}, satellite stars move to the host spheroid and host material does not move.
 \item [{\tt Baugh2005}] If the baryonic mass of the satellite exceeds a fraction {\tt majorMergerMassRatio} of the baryonic mass of the host then all material is moved to the spheroid of the host. Otherwise, if the baryonic mass of the satellite exceeds a fraction {\tt burstMassRatio} of the baryonic mass of the host and the gas fraction in the host exceeds {\tt burstCriticalGasFraction} then all gas is moved to the host spheroid, while the host stellar disk remains in place. For mergers failing both criteria, satellite gas moves to the component given by {\tt minorMergerGasMovesTo}, satellite stars move to the host spheroid and host material does not move. 
\end{description}

\subsubsection{Satellite Merging Remnant Sizes}\label{sec:satelliteMergerMassMovementMethod}

Additional methods for the satellite merging remnant sizes can be added using the {\tt satelliteMergingRemnantSizeMethod} directive. The directive should contain a single argument, giving the name of a subroutine to be called to initialize the method. For example, the {\tt Cole2000} method is described by a directive:
\begin{verbatim}
 !# <satelliteMergingRemnantSizeMethod>
 !#  <unitName>Satellite_Merging_Remnant_Sizes_Cole2000_Initialize</unitName>
 !# </satelliteMergingRemnantSizeMethod>
\end{verbatim}
Here, {\tt Satellite\_Merging\_Remnant\_Sizes\_Cole2000\_Initialize} is the name of a subroutine which will be called to initialize the method. The initialization subroutine must have the following form:
\begin{verbatim}
  subroutine Method_Initialize(satelliteMergingRemnantSizeMethod,Satellite_Merging_Remnant_Size_Do)
    implicit none
    type(varying_string),          intent(in)    :: satelliteMergingRemnantSizeMethod
    procedure(),          pointer, intent(inout) :: Satellite_Merging_Remnant_Size_Do
    
    if (satelliteMergingRemnantSizeMethod == 'myMethod') Satellite_Merging_Remnant_Size_Do => My_Method_Do
    return
  end subroutine Method_Initialize
\end{verbatim}
where {\tt myMethod} is the name of this method as will be specified by the {\tt satelliteMergingRemnantSizeMethod} input parameter. The procedure pointer {\tt Satellite\_Merging\_Remnant\_Size\_Do} must be set to point to a function which computes the size of the merger remnant and stores the properties (e.g. radius, circular velocity and specific angular momentum at the half-mass radius) of the \emph{host} node. The initialization subroutine should perform any other tasks required to initialize the module (such as reading parameters etc.).

The remnant size subroutine must have the form:
\begin{verbatim}
  subroutine My_Method_Do(thisNode)
    implicit none
    type(treeNode), intent(inout), pointer  :: thisNode
    .
    .
    .
    return
  end subroutine My_Method_Do
\end{verbatim}
The subroutine must compute the properties of the merger remnant. Typically these are stored in the {\tt Satellite\_Merging\_Remnant\_Sizes\_Properties} module for later retrieval by the appropriate component.

Currently defined satellite merger remnant size methods are:
\begin{description}
 \item [{\tt null}] This is a null method which does nothing. It is useful for runs where no baryonic components are included (e.g. for studying dark matter only).
 \item [{\tt Cole2000}] Implements the algorithm of \cite{cole_hierarchical_2000} to compute the remnant size. The orbital energy assumed can be adjusted using the {\tt mergerRemnantSizeOrbitalEnergy} parameter, which is equivalent to the $f_{\rm orbit}$ parameter of \cite{cole_hierarchical_2000}.
 \item [{\tt Covington2008}] Implements the algorithm of \cite{covington_predicting_2008} to compute the remnant size. The orbital energy assumed can be adjusted using the {\tt mergerRemnantSizeOrbitalEnergy} parameter, which is equivalent to the $f_{\rm orbit}$ parameter of \cite{cole_hierarchical_2000}.
\end{description}

\subsubsection{Satellite Merging Remnants: Progenitor Properties}

Additional methods for satellite merging remnant progenitor properties can be added using the {\tt satelliteMergingRemnantProgenitorPropertiesMethod} directive. The directive should contain a single argument, giving the name of a subroutine to be called to initialize the method. For example, the {\tt standard} method is described by a directive:
\begin{verbatim}
 !# <satelliteMergingRemnantProgenitorPropertiesMethod>
 !#  <unitName>Satellite_Merging_Remnant_Progenitor_Properties_Standard_Init</unitName>
 !# </satelliteMergingRemnantProgenitorPropertiesMethod>
\end{verbatim}
Here, {\tt Satellite\_Merging\_Remnant\_Progenitor\_Properties\_Standard\_Init} is the name of a subroutine which will be called to initialize the method. The initialization subroutine must have the following form:
\begin{verbatim}
  subroutine Method_Initialize(satelliteMergingRemnantProgenitorPropertiesMethod,Satellite_Merging_Remnant_Progenitor_Properties_Get)
    implicit none
    type(varying_string),          intent(in)    :: satelliteMergingRemnantProgenitorPropertiesMethod
    procedure(),          pointer, intent(inout) :: Satellite_Merging_Remnant_Progenitor_Properties_Get
    
    if (satelliteMergingRemnantProgenitorPropertiesMethod == 'myMethod') Satellite_Merging_Remnant_Progenitor_Properties_Get => My_Method_Get
    return
  end subroutine Method_Initialize
\end{verbatim}
where {\tt myMethod} is the name of this method as will be specified by the {\tt satelliteMergingRemnantProgenitorPropertiesMethod} input parameter. The procedure pointer {\tt Satellite\_Merging\_Remnant\_Progenitor\_Properties\_Get} must be set to point to a subroutine which computes various properties of the progenitor galaxies involved in the merger, as described below. The initialization subroutine should perform any other tasks required to initialize the module (such as reading parameters etc.).

The progenitor properties subroutine must have the form:
\begin{verbatim}
  subroutine My_Method_Get(satelliteNode,hostNode,satelliteMass,hostMass,satelliteSpheroidMass &
       & ,hostSpheroidMass,hostSpheroidMassPreMerger,satelliteRadius,hostRadius &
       & ,angularMomentumFactor,remnantSpheroidMass,remnantSpheroidGasMass)
    implicit none
    type(treeNode),   intent(inout), pointer :: satelliteNode,hostNode
    double precision, intent(out)            :: satelliteMass,hostMass,satelliteSpheroidMass, &
       & hostSpheroidMass,hostSpheroidMassPreMerger,satelliteRadius,hostRadius, &
       & angularMomentumFactor,remnantSpheroidMass,remnantSpheroidGasMass
    .
    .
    .
    return
  end subroutine My_Method_Do
\end{verbatim}
The subroutine must compute properties of the merger progenitor galaxies in {\tt satelliteNode} and {\tt hostNode}: {\tt satelliteMass} and {\tt hostMass} are the total masses of the two galaxies; {\tt satelliteSpheroidMass} and {\tt hostSpheroidMass} are the masses of each galaxy that will end up in the spheroid of the merger remnant; {\tt hostSpheroidMassPreMerger} is the mass of the host spheroid prior to the merger; {\tt satelliteRadius} and {\tt hostRadius} are radii of the two galaxies for use in merger remnant size calculations (and so should typically refer to the radius of material that will end up in the merger remnant spheroid); {\tt remnantSpheroidMass} is the mass of the spheroid in the remnant; {\tt remnantSpheroidGasMass} is the mass of gas in the spheroid of the remnant; and {\tt angularMomentumFactor} gives the pseudo-specific angular momentum of the remnant in units of $({\rm G} M_{\rm remnant,spheroid} r_{\rm remnant,spheroid})^{1/2}$ where $M_{\rm remnant,spheroid}$ is the mass of the 
remnant spheroid and $r_{\rm remnant,spheroid}$ is the radius of the remnant spheroid.

Currently defined satellite merger progenitor properties methods are:
\begin{description}
 \item [{\tt Cole2000}] Implements the algorithm of \cite{cole_hierarchical_2000} to compute the remnant properties. Masses of host and spheroid are set equal to their stellar plus cold gas masses utilizing, while radii are the half-mass radii of each galaxy, including only those components which end up in the remnant spheroid. The angular momentum factor is set to a mass-weighted average of the corresponding factor for each component which will end up in the merger remnant spheroid.
 \item [{\tt standard}] Masses of host and spheroid are set equal to their stellar plus cold gas masses utilizing, while radii are a mass-weighted average of the half-mass radii of the components which end up in the merger remnant spheroid. The angular momentum factor is similarly set to a mass-weighted average of the corresponding factor for each component which will end up in the merger remnant spheroid.
\end{description}

\subsubsection{Satellite Tidal Fields}

Additional methods for the satellite tidal fields can be added using the {\tt satellitesTidalFieldMethod} directive. The directive should contain a single argument, giving the name of a subroutine to be called to initialize the method. For example, the {\tt sphericalSymmetry} method is described by a directive:
\begin{verbatim}
  !# <satellitesTidalFieldMethod>
  !#  <unitName></unitName>
  !# </satellitesTidalFieldMethod>
\end{verbatim}
Here, {\tt Satellites\_Tidal\_Fields\_Spherical\_Symmetry\_Initialize} is the name of a subroutine which will be called to initialize the method. The initialization subroutine must have the following form:
\begin{verbatim}
  subroutine Method_Initialize(satellitesTidalFieldMethod,Satellites_Tidal_Field_Get)
    implicit none
    type     (varying_string     )         , intent(in   ) :: satellitesTidalFieldMethod
    procedure(My_Method_Procedure), pointer, intent(inout) :: Satellites_Tidal_Field_Get
    
    if (satellitesTidalFieldMethod == 'myMethod') Satellites_Tidal_Field_Get => My_Method_Procedure
    return
  end subroutine Method_Initialize
\end{verbatim}
where {\tt myMethod} is the name of this method as will be specified by the {\tt satellitesTidalFieldMethod} input parameter. The procedure pointer {\tt Satellites\_Tidal\_Field\_Get} must be set to point to a function which returns the time until merging as described below. The initialization subroutine should perform any other tasks required to initialize the module (such as reading parameters etc.).

The satellite tidal field function must have the form:
\begin{verbatim}
 double precision function Satellites_Tidal_Fields(thisNode)
    implicit none
    type(treeNode), pointer, intent(inout) :: thisNode
    .
    .
    .
    return
 end function Satellites_Tidal_Fields
\end{verbatim}
The function must return the magnitude of the tidal field for {\tt thisNode} in units of (km/s)$^2$ Mpc$^{-2}$.

Currently defined satellite tidal field methods are:
\begin{description}
 \item [\hyperlink{satellites.tidal_fields.null.F90:satellites_tidal_fields_null}{{\tt null}}] Assumes a zero tidal field.
 \item [\hyperlink{satellites.tidal_fields.spherical_symmetry.F90:satellites_tidal_fields_spherical_symmetry}{{\tt sphericalSymmetry}}] Computes the tidal field assuming a spherically-symmetric host halo.
\end{description}

\subsubsection{Satellite Virial Orbits}\label{sec:SatelliteVirialOrbits}

Additional methods for the satellite virial orbits (i.e. orbital parameters at virial radius crossing) can be added using the {\tt virialOrbitsMethod} directive. The directive should contain a single argument, giving the name of a subroutine to be called to initialize the method. For example, the {\tt Benson2005} method is described by a directive:
\begin{verbatim}
  !# <virialOrbitsMethod>
  !#  <unitName>Virial_Orbital_Parameters_Benson2005_Initialize</unitName>
  !# </virialOrbitsMethod>
\end{verbatim}
Here, {\tt Virial\_Orbital\_Parameters\_Benson2005\_Initialize} is the name of a subroutine which will be called to initialize the method. The initialization subroutine must have the following form:
\begin{verbatim}
  subroutine Method_Initialize(virialOrbitsMethod,Virial_Orbital_Parameters_Get)
    implicit none
    type(varying_string),          intent(in)    :: virialOrbitsMethod
    procedure(),          pointer, intent(inout) :: Virial_Orbital_Parameters_Get
    
    if (virialOrbitsMethod.eq.'myMethod') Virial_Orbital_Parameters_Get => My_Method_Procedure
    return
  end subroutine Method_Initialize
\end{verbatim}
where {\tt myMethod} is the name of this method as will be specified by the {\tt virialOrbitsMethod} input parameter. The procedure pointer {\tt Virial\_Orbital\_Parameters\_Get} must be set to point to a subroutine which returns orbital parameters as described below. The initialization subroutine should perform any other tasks required to initialize the module (such as reading parameters etc.).

The orbital parameter subroutine must have the form:
\begin{verbatim}
  function Virial_Orbital_Parameters(thisNode,hostNode,acceptUnboundOrbits) result (thisOrbit)
    implicit none
    type(keplerOrbit)                         :: thisOrbit
    type(treeNode),   intent(inout), pointer  :: thisNode,hostNode
    logical,          intent(in)              :: acceptUnboundOrbits
    .
    .
    .
    return
  end subroutine Virial_Orbital_Parameters
\end{verbatim}
The subroutine must return a fully-defined Kepler orbit object (i.e. the orbit must have at least three parameters defined in addition to the node masses) initialized to the orbital parameters for {\tt thisNode} orbitting in {\tt hostNode} at the point of virial radius crossing. If {\tt acceptUnboundOrbits} is true, then unbound orbits may be returned, otherwise, the routine must ensure that the returned orbit is bound. Velocities should be returned in units of km/s, lengthscales in units of Mpc and masses in $M_\odot$. Note that the usual conventions of the {\tt keplerOrbit} object should be followed, namely the that orbitting bodies are treated as point masses, with the host being stationary and the usual reduced mass used.

Currently defined satellite virial orbit methods are:
\begin{description}
 \item [\hyperlink{satellites.merging.virial_orbits.Benson2005.F90:virial_orbits_benson2005:virial_orbital_parameters_benson2005}{{\tt Benson2005}}] The orbital parameters are select from the distribution found by \cite{benson_orbital_2005}.
 \item [\hyperlink{satellites.merging.virial_orbits.Wetzel2010.F90:virial_orbits_wetzel2010:virial_orbital_parameters_wetzel2010}{{\tt Wetzel2010}}] The orbital parameters are select from the distribution found by \cite{wetzel_orbits_2010}.
 \item [\hyperlink{satellites.merging.virial_orbits.fixed.F90:virial_orbits_fixed:virial_orbital_parameters_fixed}{{\tt fixed}}] The orbital parameters are set to fixed values, with $v_{\rm r}=${\tt [virialOrbitsFixedRadialVelocity]}$V_{\rm virial}$ and  $v_\phi=${\tt [virialOrbitsFixedTangentialVelocity]}$V_{\rm virial}$.
\end{description}

\subsubsection{Star Formation Feedback in Disks/Spheroids}

Additional methods for computing feedback from star formation in disks/spheroids can be added using the {\tt starFormationFeedback[Disks|Spheroids]Method} directive. The directive should contain a single argument, giving the name of a subroutine to be called to initialize the method. For example, the {\tt powerLaw} method is described by a directive:
\begin{verbatim}
 !# <starFormationFeedbackSpheroidsMethod>
 !#  <unitName>Star_Formation_Feedback_Spheroids_Power_Law_Initialize</unitName>
 !# </starFormationFeedbackSpheroidsMethod>
\end{verbatim}
Here, {\tt Star\_Formation\_Feedback\_Spheroids\_Power\_Law\_Initialize} is the name of a subroutine which will be called to initialize the method. The initialization subroutine must have the following form:
\begin{verbatim}
  subroutine Method_Initialize(starFormationFeedbackDisksMethod,Star_Formation_Feedback_Disk_Outflow_Rate_Get)
    implicit none
    type(varying_string),          intent(in)    :: starFormationFeedbackDisksMethod
    procedure(),          pointer, intent(inout) :: Star_Formation_Feedback_Disk_Outflow_Rate_Get
    
    if (starFormationFeedbackDisksMethod == 'myMethod') Star_Formation_Feedback_Disk_Outflow_Rate_Get => My_Star_Formation_Feedback_Disk_Outflow_Rate_Get
    return
  end subroutine Method_Initialize
\end{verbatim}
where {\tt myMethod} is the name of this method as will be specified by the {\tt starFormationFeedback[Disks|Spheroids]Method} input parameter. The procedure pointer {\tt Star\_Formation\_Feedback\_Disk\_Outflow\_Rate\_Get} (or {\tt Star\_Formation\_Feedback\_Spheroid\_Outflow\_Rate\_Get} for the spheroid case) must be set to point to a function which returns the mass outflow rate due to star formation as described below. The initialization subroutine should perform any other tasks required to initialize the module (such as reading parameters etc.).

The outflow rate function must have the form:
\begin{verbatim}
 double precision function Star_Formation_Feedback_Outflow_Rate_Get(thisNode,starFormationRate,energyInputRate)
    implicit none
    type(treeNode),   intent(inout), pointer :: thisNode
    double precision, intent(in)             :: starFormationRate,energyInputRate
    .
    .
    .
    return
 end function Star_Formation_Feedback_Outflow_Rate_Get
\end{verbatim}
The function must return the mass outflow rate (in $M_\odot$ Gyr$^{-1}$) for {\tt thisNode}.

Currently defined star formation feedback methods are:
\begin{description}
 \item [\hyperlink{star_formation.feedback.disks.fixed.F90:star_formation_feedback_disks_fixed}{{\tt fixed}}] The outflow rate is a fixed multiple of the the star formation rate.
 \item [\hyperlink{star_formation.feedback.spheroids.power_law.F90:star_formation_feedback_spheroids_power_law:star_formation_feedback_spheroid_outflow_rate_power_law}{{\tt powerLaw}}] The outflow rate is given by
\begin{equation}
 \dot{M}_{\rm outflow} = \left({V_{\rm outflow} \over V}\right)^{\alpha_{\rm outflow}} {\dot{E} \over E_{\rm canonical}},
\end{equation}
where $V_{\rm outflow}=${\tt [disk|spheroid]OutflowVelocity} (in km/s) and $\alpha_{\rm outflow}=${\tt [disk|spheroid]OutflowVelocity} are input parameters, $V$ is the characteristic velocity of the component, $\dot{E}$ is the rate of energy input from stellar populations and $E_{\rm canonical}$ is the total energy input by a canonical stellar population normalized to $1 M_\odot$ after infinite time.
 \item [\hyperlink{star_formation.feedback.disks.Creasey2012.F90:star_formation_feedback_disks_creasey2012}{{\tt Creasey2012}}] The outflow rate computed using the model of \cite{creasey_how_2012}.
\end{description}

\subsubsection{(Expulsive) Star Formation Feedback in Disks/Spheroids}\index{feedback!expulsive}

Additional methods for computing expulsive feedback\footnote{``Expulsive'' feedback implies outflows in which gas is driven not only out of a galaxy but also out of its host dark matter halo.} from star formation in disks/spheroids can be added using the {\tt starFormationExpulsiveFeedback[Disks|Spheroids]Method} directive. The directive should contain a single argument, giving the name of a subroutine to be called to initialize the method. For example, the {\tt powerLaw} method is described by a directive:
\begin{verbatim}
 !# <starFormationExpulsiveFeedbackSpheroidsMethod>
 !#  <unitName>Star_Formation_Expulsive_Feedback_Spheroids_Power_Law_Initialize</unitName>
 !# </starFormationExpulsiveFeedbackSpheroidsMethod>
\end{verbatim}
Here, {\tt Star\_Formation\_Expulsive\_Feedback\_Spheroids\_Power\_Law\_Initialize} is the name of a subroutine which will be called to initialize the method. The initialization subroutine must have the following form:
\begin{verbatim}
  subroutine Method_Initialize(starFormationExpulsiveFeedbackDisksMethod,Star_Formation_Expulsive_Feedback_Disk_Outflow_Rate_Get)
    implicit none
    type(varying_string),          intent(in)    :: starFormationFeedbackDisksMethod
    procedure(),          pointer, intent(inout) :: Star_Formation_Expulsive_Feedback_Disk_Outflow_Rate_Get
    
    if (starFormationExpulsiveFeedbackDisksMethod == 'myMethod') Star_Formation_Expulsive_Feedback_Disk_Outflow_Rate_Get => My_Star_Formation_Expulsive_Feedback_Disk_Outflow_Rate_Get
    return
  end subroutine Method_Initialize
\end{verbatim}
where {\tt myMethod} is the name of this method as will be specified by the {\tt starFormationExpulsiveFeedback[Disks|Spheroids]Method} input parameter. The procedure pointer {\tt Star\_Formation\_Expulsive\_Feedback\_Disk\_Outflow\_Rate\_Get} (or {\tt Star\_Formation\_Expulsive\_Feedback\_Spheroid\_Outflow\_Rate\_Get} for the spheroid case) must be set to point to a function which returns the expulsive mass outflow rate due to star formation as described below. The initialization subroutine should perform any other tasks required to initialize the module (such as reading parameters etc.).

The outflow rate function must have the form:
\begin{verbatim}
 double precision function Star_Formation_Expulsive_Feedback_Outflow_Rate_Get(thisNode,starFormationRate,energyInputRate)
    implicit none
    type(treeNode),   intent(inout), pointer :: thisNode
    double precision, intent(in)             :: starFormationRate,energyInputRate
    .
    .
    .
    return
 end function Star_Formation_Expulsive_Feedback_Outflow_Rate_Get
\end{verbatim}
The function must return the expulsive mass outflow rate (in $M_\odot$ Gyr$^{-1}$) for {\tt thisNode}.

Currently defined star formation expulsive feedback methods are:
\begin{description}
 \item [\hyperlink{star_formation.feedback_expulsion.spheroids.null.F90:star_formation_expulsive_feedback_spheroids_null:star_formation_expulsive_feedback_spheroid_outflow_rate_null}{{\tt null}}] Assumes a zero outflow rate.
 \item [\hyperlink{star_formation.feedback_expulsion.spheroids.superwind.F90:star_formation_expulsive_feedback_spheroids_superwind:star_formation_expulsive_feedback_spheroid_outflow_rate_sw}{{\tt superwind}}] The outflow rate is given by
\begin{equation}
 \dot{M}_{\rm outflow} = \beta_{\rm superwind} {\dot{E} \over E_{\rm canonical}} \left\{ \begin{array}{ll} \left( V_{\rm superwind}/V\right)^2 & \hbox{ if } V > V_{\rm superwind} \\ 1 & \hbox{ otherwise,} \end{array} \right.
\end{equation}
where $V_{\rm superwind}=${\tt [disk|spheroid]SuperwindVelocity} (in km/s) and $\beta_{\rm superwind}=${\tt [disk|spheroid]SuperwindMassLoading} are input parameters, $V$ is the characteristic velocity of the component, $\dot{E}$ is the rate of energy input from stellar populations and $E_{\rm canonical}$ is the total energy input by a canonical stellar population normalized to $1 M_\odot$ after infinite time.
\end{description}

\subsubsection{Star Formation Rate Surface Density in Disks}

Additional methods for computing the surface density of star formation rate in disks can be added using the {\tt starFormationRateSurfaceDensityDisksMethod} directive. The directive should contain a single argument, giving the name of a subroutine to be called to initialize the method. For example, the {\tt Blitz-Rosolowsky2006} method is described by a directive:
\begin{verbatim}
 !# <starFormationRateSurfaceDensityDisksMethod>
 !#  <unitName>Star_Formation_Rate_Surface_Density_Disks_BR_Initialize</unitName>
 !# </starFormationRateSurfaceDensityDisksMethod>
\end{verbatim}
Here, {\tt Star\_Formation\_Rate\_Surface\_Density\_Disks\_BR\_Initialize} is the name of a subroutine which will be called to initialize the method. The initialization subroutine must have the following form:
\begin{verbatim}
  subroutine Method_Initialize(starFormationRateSurfaceDensityDisksMethod,Star_Formation_Rate_Surface_Density_Disk_Get)
    implicit none
    type(varying_string),          intent(in)    :: starFormationRateSurfaceDensityDisksMethod
    procedure(),          pointer, intent(inout) :: Star_Formation_Rate_Surface_Density_Disk_Get
    
    if (starFormationRateSurfaceDensityDisksMethod == 'myMethod') Star_Formation_Rate_Surface_Density_Disk_Get => My_Method_Get
    return
  end subroutine Method_Initialize
\end{verbatim}
where {\tt myMethod} is the name of this method as will be specified by the {\tt starFormationRateSurfaceDensityDisksMethod} input parameter. The procedure pointer {\tt Star\_Formation\_Rate\_Surface\_Density\_Disk\_Get} must be set to point to a function which returns the surface density of star formation rate at a specified radius as described below. The initialization subroutine should perform any other tasks required to initialize the module (such as reading parameters etc.).

The star formation rate surface density function must have the form:
\begin{verbatim}
 double precision function Star_Formation_Rate_Surface_Density_Get(thisNode,radius)
    implicit none
    type            (treeNode), intent(inout), pointer :: thisNode
    double precision          , intent(in   )          :: radius
    .
    .
    .
    return
 end function Star_Formation_Rate_Surface_Density_Get
\end{verbatim}
The function must return the surface density of star formation rate (in units of $M_\odot$ Gyr$^{-1}$ Mpc$^{-2}$) for {\tt thisNode}.

Currently defined star formation rate surface density methods are:
\begin{description}
 \item [\hyperlink{star_formation.rate_surface_density.disks.Kennicutt-Schmidt.F90:star_formation_rate_surface_density_disks_ks:star_formation_rate_surface_density_disk_ks}{{\tt Kennicutt-Schmidt}}] The rate is given by the Kennicutt-Schmidt law (\citealt{schmidt_rate_1959,kennicutt_global_1998}; see \S\ref{sec:StarFormationKennicuttSchmidt}).
 \item [\hyperlink{star_formation.rate_surface_density.disks.extended_Schmidt.F90:star_formation_rate_surface_density_disks_exschmidt:star_formation_rate_surface_density_disk_exschmidt}{{\tt extendedSchmidt}}] The rate is given by the extended Schmidt law (\citealt{shi_extended_2011}; see \S\ref{sec:StarFormationExtendedSchmidt}).
 \item [\hyperlink{star_formation.rate_surface_density.disks.Blitz-Rosolowsky.F90:star_formation_rate_surface_density_disks_br:star_formation_rate_surface_density_disk_br}{{\tt Blitz-Rosolowsky2006}}] The rate is given by the Blitz-Rosolowsky rule (\citealt{blitz_role_2006}; see \S\ref{sec:StarFormationBlitzRosolowsky}).
 \item [\hyperlink{star_formation.rate_surface_density.disks.KMT09.F90:star_formation_rate_surface_density_disks_kmt09:star_formation_rate_surface_density_disk_kmt09}{{\tt KMT09}}] The rate is given by the Krumholz-McKee-Tumlinson (\citealt{krumholz_star_2009}; see \S\ref{sec:StarFormationKMT09});
\end{description}

\subsubsection{Star Formation Timescale in Disks/Spheroids}

Additional methods for computing star formation timescales in disks/spheroids can be added using the {\tt starFormationTimescale[Disks|Spheroids]Method} directive. The directive should contain a single argument, giving the name of a subroutine to be called to initialize the method. For example, the {\tt dynamicalTime} method is described by a directive:
\begin{verbatim}
 !# <starFormationTimescaleDisksMethod>
 !#  <unitName>Star_Formation_Timescale_Disks_Dynamical_Time_Initialize</unitName>
 !# </starFormationTimescaleDisksMethod>
\end{verbatim}
Here, {\tt Star\_Formation\_Timescale\_Disks\_Dynamical\_Time\_Initialize} is the name of a subroutine which will be called to initialize the method. The initialization subroutine must have the following form:
\begin{verbatim}
  subroutine Method_Initialize(starFormationTimescaleDisksMethod,Star_Formation_Timescale_Disk_Get)
    implicit none
    type(varying_string),          intent(in)    :: starFormationTimescaleDisksMethod
    procedure(),          pointer, intent(inout) :: Star_Formation_Timescale_Disk_Get
    
    if (starFormationTimescaleDisksMethod == 'myMethod') Star_Formation_Timescale_Disk_Get => My_Method_Get_Procedure
    return
  end subroutine Method_Initialize
\end{verbatim}
where {\tt myMethod} is the name of this method as will be specified by the {\tt starFormationTimescale[Disks|Spheroids]Method} input parameter. The procedure pointer {\tt Star\_Formation\_Timescale\_Disk\_Get} (or {\tt Star\_Formation\_Timescale\_Spheroid\_Get} for the spheroid case) must be set to point to a function which returns the timescale for star formation as described below. The initialization subroutine should perform any other tasks required to initialize the module (such as reading parameters etc.).

The star formation timescale function must have the form:
\begin{verbatim}
 double precision function Star_Formation_Timescale_Get(thisNode)
    implicit none
    type(treeNode), intent(in) :: thisNode
    .
    .
    .
    return
 end function Star_Formation_Timescale_Get
\end{verbatim}
The function must return the star formation timescale (in units of Gyr) for {\tt thisNode}.

Currently defined star formation timescale methods are:
\begin{description}
 \item [\hyperlink{star_formation.timescales.disks.halo_scaling.F90:star_formation_timescale_disks_halo_scaling}{{\tt haloScaling}}]  The timescale scales with halo virial velocity and redshift;
 \item [\hyperlink{star_formation.timescales.disks.fixed.F90:star_formation_timescale_disks_fixed}{{\tt fixed}}]  The timescale is a fixed quantity;
 \item [\hyperlink{star_formation.timescales.disks.dynamical_time.F90:star_formation_timescale_disks_dynamical_time:star_formation_timescale_disk_dynamical_time}{{\tt dynamicalTime}}]  The timescale is given by
\begin{equation}
 \tau_\star = \epsilon_\star^{-1} \tau_{\rm dynamical, [disk|spheroid]} \left( {V_{\rm [disk|spheroid]} \over 200\hbox{km/s}} \right)^{\alpha_\star},
\end{equation}
where $\epsilon_\star$(={\tt starFormation[Disk|Spheroid]Efficiency}) is a star formation efficiency and $\alpha_\star$(={\tt starFormation[Disk|Spheroid]VelocityExponent}) controls the scaling with velocity. Note that $\tau_{\rm dynamical,[disk|spheroid]}=R_{\rm [disk|spheroid]}/V_{\rm [disk|spheroid]}$ where the radius and velocity are whatever characteristic values returned by the disk/spheroid method. This scaling is functionally similar to that adopted by \cite{cole_hierarchical_2000}, but they specifically used the half-mass radius and circular velocity at that radius.
 \item {\tt Baugh2005}] The timescale is given by $\tau_0 (V_{\rm disk}/200\hbox{km/s})^\alpha a^\beta$.
 \item {\tt integratedSurfaceDensity}] The timescale is given by $\tau_\star = M_{\rm cold}/\int_0^\infty 2 \pi r \dot{\Sigma}_\star(r) {\rm d}r$ where $\dot{\Sigma}_\star(r)$ is the surface density of star formation rate (see \S\ref{sec:StarFormationRateSurfaceDensity})
\end{description}

\subsubsection{Stellar Astrophysics}

Additional methods for stellar astrophysical properties can be added using the {\tt stellarAstrophysicsMethod} directive. The directive should contain a single argument, giving the name of a subroutine to be called to initialize the method. For example, the {\tt file} method is described by a directive:
\begin{verbatim}
  !# <stellarAstrophysicsMethod>
  !#  <unitName>Stellar_Astrophysics_File_Initialize</unitName>
  !# </stellarAstrophysicsMethod>
\end{verbatim}
Here, {\tt Stellar\_Astrophysics\_File\_Initialize} is the name of a subroutine which will be called to initialize the method. The initialization subroutine must have the following form:
\begin{verbatim}
  subroutine Method_Initialize(stellarAstrophysicsMethod,Star_Ejected_Mass_Get,Star_Initial_Mass_Get,Star_Metal_Yield_Mass_Get,Star_Lifetime_Get)
    implicit none
    type(varying_string),          intent(in)    :: stellarAstrophysicsMethod
    procedure(),          pointer, intent(inout) :: Star_Ejected_Mass_Get,Star_Initial_Mass_Get,Star_Metal_Yield_Mass_Get,Star_Lifetime_Get
    
    if (stellarAstrophysicsMethod == 'myMethod') then
      Star_Ejected_Mass_Get     => My_Star_Ejected_Mass
      Star_Initial_Mass_Get     => My_Star_Initial_Mass
      Star_Metal_Yield_Mass_Get => My_Star_Metal_Yield_Mass
      Star_Lifetime_Get         => My_Star_Lifetime
    end if
    return
  end subroutine Method_Initialize
\end{verbatim}
where {\tt myMethod} is the name of this method as will be specified by the {\tt stellarAstrophysicsMethod} input parameter. The procedure pointers must be set to point to functions which return stellar properties as described below. The initialization subroutine should perform any other tasks required to initialize the module (such as reading parameters etc.).

The ejected mass and lifetime functions must have the form:
\begin{verbatim}
 double precision function Star_Property(initialMass,metallicity)
    implicit none
    double precision, intent(in) :: initialMass,metallicity
    .
    .
    .
    return
 end function Star_Property
\end{verbatim}
These functions must return the total ejected mass (in $M_\odot$), total metal yield (in $M_\odot$) and lifetime (in Gyr) for a star of the specified {\tt initialMass} and {\tt metallicity}.

The metal yield function must have the form:
\begin{verbatim}
 double precision function Star_Metal_Yield(initialMass,metallicity,atomIndex)
    implicit none
    double precision, intent(in)           :: initialMass,metallicity
    integer,          intent(in), optional :: atomIndex
    .
    .
    .
    return
 end function Star_Property
\end{verbatim}
This function must return the yield (in $M_\odot$) of the element identified by {\tt atomIndex} (as returned by the \hyperlink{atomic.data.F90:atomic_data:atom_lookup}{{\tt Atom\_Lookup()}} function from the \hyperlink{atomic.data.F90:atomic_data}{{\tt Atomic\_Data}} module) if present, or total metal yield otherwise for a star of the specified {\tt initialMass} and {\tt metallicity}.

The initial mass function must have the form:
\begin{verbatim}
 double precision function Star_Initial_Mass(lifetime,metallicity)
    implicit none
    double precision, intent(in) :: lifetime,metallicity
    .
    .
    .
    return
 end function Star_Initial_Mass
\end{verbatim}
and should return the initial mass (in $M_\odot$) of a star of given {\tt lifetime} (specified in Gyr) and {\tt metallicity}.

Currently defined stellar astrophysics methods are:
\begin{description}
 \item [{\tt file}] Stellar properties are read from an XML file and interpolated. The structure of the XML file is described in \S\ref{sec:StellarAstrophysicsFile}.
\end{description}

\subsubsection{Stellar Population Properties}

Additional methods for computing properties of stellar populations can be added using the {\tt stellarPopulationPropertiesMethod} directive. The directive should contain a single argument, giving the name of a subroutine to be called to initialize the method. For example, the {\tt instantaneous} method is described by a directive:
\begin{verbatim}
 !# <stellarPopulationPropertiesMethod>
 !#  <unitName>Stellar_Population_Properties_Instantaneous_Initialize</unitName>
 !# </stellarPopulationPropertiesMethod>
\end{verbatim}
Here, {\tt Stellar\_Population\_Properties\_Instantaneous\_Initialize} is the name of a subroutine which will be called to initialize the method. The initialization subroutine must have the following form:
\begin{verbatim}
  subroutine Method_Initialize(stellarPopulationPropertiesMethod,Stellar_Population_Properties_Rates_Get &
 & ,Stellar_Population_Properties_Scales_Get,Stellar_Population_Properties_History_Count_Get             &
 & ,Stellar_Population_Properties_History_Create_Do)
    implicit none
    type(varying_string),          intent(in)    :: stellarPopulationPropertiesMethod
    procedure(),          pointer, intent(inout) :: Stellar_Population_Properties_Rates_Get  &
 & ,Stellar_Population_Properties_Scales_Get,Stellar_Population_Properties_History_Count_Get &
 & ,Stellar_Population_Properties_History_Create_Do
    
    if (stellarPopulationPropertiesMethod == 'myMethod') then
       Stellar_Population_Properties_Rates_Get         => My_Method_Rates_Get_Procedure
       Stellar_Population_Properties_Scales_Get        => My_Method_Scales_Procedure
       Stellar_Population_Properties_History_Count_Get => My_Method_History_Count_Get_Procedure
       Stellar_Population_Properties_History_Create_Do => My_Method_History_Create_Procedure
    end if
    return
  end subroutine Method_Initialize
\end{verbatim}
where {\tt myMethod} is the name of this method as will be specified by the {\tt stellarPopulationPropertiesMethod} input parameter. The procedure pointers {\tt Stellar\_Population\_Properties\_Rates\_Get} and {\tt Stellar\_Population\_Properties\_Scales\_Get} must be set to point to subroutines which return properties of a stellar population and set scaling factors for ODE error control as described below, while the {\tt Stellar\_Population\_Properties\_History\_Count\_Get} and {\tt Stellar\_Population\_Properties\_History\_Create\_Do} procedure pointers must be set to point to functions which return the number of histories that will be required by this method and create a suitable history object respectively. The initialization subroutine should perform any other tasks required to initialize the module (such as reading parameters etc.).

The stellar populations properties subroutine must have the form:
\begin{verbatim}
 subroutine Stellar_Population_Properties_Rates(starFormationRate,fuelAbundances,component,thisNode,thisHistory,stellarMassRate&
       &,stellarAbundancesRates,stellarLuminositiesRates,fuelMassRate,fuelAbundancesRates,energyInputRate)
    implicit none
    implicit none
    double precision,          intent(out)                 :: stellarMassRate,fuelMassRate,energyInputRate
    type(abundancesStructure), intent(out)                 :: stellarAbundancesRates,fuelAbundancesRates
    integer,                   intent(in)                  :: component
    double precision,          intent(out),   dimension(:) :: stellarLuminositiesRates
    double precision,          intent(in)                  :: starFormationRate
    type(abundancesStructure), intent(in)                  :: fuelAbundances
    type(treeNode),            intent(inout), pointer      :: thisNode
    type(history),             intent(inout)               :: thisHistory
    .
    .
    .
    return
 end subroutine Stellar_Population_Properties_Rates
\end{verbatim}
The subroutine is given the {\tt starFormationRate} (in $M_\odot$ Gyr$^{-1}$) in {\tt thisNode}. Any history information required by this method must be passed in via the {\tt history} argument. Stars are forming from fuel material with composition specified by {\tt fuelAbundances} and occurring in the specified galactic {\tt component} (using the labels provided by the \hyperlink{galactic_structure.options.F90:galactic_structure_options}{{\tt Galactic\_Structure\_Options}} module). The subroutine must return the rates of change of stellar and fuel mass (in $M_\odot$ Gyr$^{-1}$) in {\tt stellarMassRate} and {\tt fuelMassRate} respectively, and the corresponding rates (also in $M_\odot$ Gyr$^{-1}$) of abundance change in {\tt stellarAbundancesRates} and {\tt fuelAbundancesRates} respectively. Finally, it should return rates of change (in $L_{\rm AB}$ Gyr$^{-1}$) of stellar luminosities for all requested output bands in {\tt stellarLuminositiesRates}. Additionally, the rate of energy input from stellar 
populations must be returned in {\tt energyInputRate}.

The scales procedure should have the form:
\begin{verbatim}
  subroutine Stellar_Population_Properties_Scales_Noninstantaneous(thisHistory,stellarMass,stellarAbundances)
   implicit none
   double precision,          intent(in)    :: stellarMass
   type(abundancesStructure), intent(in)    :: stellarAbundances
   type(history),             intent(inout) :: thisHistory
   .
   .
   .
   return
 end subroutine Stellar_Population_Properties_Scales_Noninstantaneous
\end{verbatim}
and should set scale factors for ODE error control (see \S\ref{sec:ComponentEvolution}) in the stellar population properties history {\tt thisHistory}. The {\tt stellarMass} and {\tt stellarAbundances} (both in $M_\odot$) are provided as input as they are often useful in choosing appropriate scale factors.

The history count function must have the form
\begin{verbatim}
 integer function Stellar_Population_Properties_History_Count()
    implicit none
    .
    .
    return
 end function Stellar_Population_Properties_History_Count
\end{verbatim}
and should return the number of histories that will be required by this method. The history create function must have the form
\begin{verbatim}
  subroutine Stellar_Population_Properties_History_Create(thisNode,thisHistory)
   type(treeNode),  intent(inout), pointer :: thisNode
   type(history),   intent(inout)          :: thisHistory
   .
   .
   .
   return
 end subroutine Stellar_Population_Properties_History_Create
\end{verbatim}
and should create {\tt thisHistory} with a suitable set of time steps for {\tt thisNode}.

Currently defined stellar population properties methods are:
\begin{description}
 \item [\hyperlink{stellar_populations.properties.instantaneous.F90:stellar_population_properties_instantaneous:stellar_population_properties_rates_instantaneous}{{\tt instantaneous}}] Computes stellar population properties using an instantaneous recyclying approximation.
\end{description}

\subsubsection{Stellar Population Spectra}\label{sec:StellarPopulationSpectra}

Additional methods for computing specta of stellar populations can be added using the {\tt stellarPopulationSpectraMethod} directive. The directive should contain a single argument, giving the name of a subroutine to be called to initialize the method. For example, the {\tt Conroy-White-Gunn2009} method is described by a directive:
\begin{verbatim}
 !# <stellarPopulationSpectraMethod>
 !#  <unitName>Stellar_Population_Spectra_Conroy_Initialize</unitName>
 !# </stellarPopulationSpectraMethod>
\end{verbatim}
Here, {\tt Stellar\_Population\_Spectra\_Conroy\_Initialize} is the name of a subroutine which will be called to initialize the method. The initialization subroutine must have the following form:
\begin{verbatim}
  subroutine Method_Initialize(stellarPopulationSpectraMethod,Stellar_Population_Spectra_Get,Stellar_Population_Spectrum_Tabulation_Get)
    implicit none
    type(varying_string),          intent(in)    :: stellarPopulationSpectraMethod
    procedure(),          pointer, intent(inout) :: Stellar_Population_Spectra_Get,Stellar_Population_Spectrum_Tabulation_Get
    
    if (stellarPopulationSpectraMethod == 'myMethod') then
        Stellar_Population_Spectra_Get             => My_Method_Spectra_Get
        Stellar_Population_Spectrum_Tabulation_Get => My_Method_Spectrum_Tabulation_Get
    end if
    return
  end subroutine Method_Initialize
\end{verbatim}
where {\tt myMethod} is the name of this method as will be specified by the {\tt stellarPopulationSpectraMethod} input parameter. The procedure pointer {\tt Stellar\_Population\_Spectra\_Get} must be set to point to a function which returns the spectrum of a stellar population as described below while the {\tt Stellar\_Population\_Spectrum\_Tabulation\_Get} pointer must be set to point to a subroutine which returns a tabulation of ages and metallicities on which stellar spectra should be tabulated. The initialization subroutine should perform any other tasks required to initialize the module (such as reading parameters etc.).

The stellar spectra function must have the form:
\begin{verbatim}
  double precision function Stellar_Population_Spectra_Get(abundances,age,wavelength,imfIndex)
    implicit none
    type(abundancesStructure), intent(in) :: abundances
    double precision,          intent(in) :: age,wavelength
    integer,                   intent(in) :: imfIndex
    .
    .
    .
    return
  end function Stellar_Population_Spectra_Get
\end{verbatim}
The function is given the {\tt abundances}, {\tt age} (in Gyr), and {\tt imfIndex} of the stellar population and the {\tt wavelength} (in \AA) at which the spectrum should be computed. The spectrum should be returned in units of $L_\odot$ Hz$^{-1}$.

The tabulation subroutine must have the form:
\begin{verbatim}
  subroutine Stellar_Population_Spectrum_Tabulation(imfIndex,agesCount,metallicitiesCount,age,metallicity)
    implicit none
    integer,          intent(in)                             :: imfIndex
    integer,          intent(out)                            :: agesCount,metallicitiesCount
    double precision, intent(out), allocatable, dimension(:) :: age,metallicity
    .
    .
    .
    return
  end subroutine Stellar_Population_Spectrum_Tabulation
\end{verbatim}
and should return the number of ages and metallicities at which stellar population spectra should be tabualted for the specified \gls{imf}, and should allocate the {\tt age} and {\tt metallicity} arrays appropriately and should fill them with the ages and metallicities at which to tabulate.

Currently defined stellar population properties methods are:
\begin{description}
 \item [\hyperlink{stellar_populations.spectra.Conroy_et_al.F90:stellar_population_spectra_conroy}{{\tt Conroy-White-Gunn2009}}] Uses the {\tt FSPS} code of \cite{conroy_propagation_2009} to compute stellar spectra. If necessary, the {\tt FSPS} code will be downloaded, patched and compiled and run to generate spectra. These tabulations are then stored to file for later retrieval.
 \item [\hyperlink{stellar_populations.spectra.file.F90:stellar_population_spectra_file}{{\tt file}}] Stellar spectra for a given \gls{imf} are read from the file specified by the {\tt stellarPopulationSpectraForXXXXIMF} where {\tt XXXX} is the name of the \gls{imf}. This should specify an HDF5 file with the following structure:
\begin{verbatim}
 ages                     Dataset {ageCount}
 metallicities            Dataset {metallicityCount}
 spectra                  Dataset {metallicityCount, ageCount, metallicityCount}
 wavelengths              Dataset {wavelengthCount}
\end{verbatim}
where the datasets contain the tabulated ages (in Gyr), metallicities (logarithmic, relative to Solar), wavelengths (in \AA) and spectra (in $L_\odot$ Hz$^{-1}$).
\end{description}
Currently, the following pre-computed stellar spectra files are available as a separate download from \href{http://users.obs.carnegiescience.edu/abenson/galacticus/data/Galacticus_SSP_Data.tar.bz2}{\tt http://users.obs.carnegiescience.edu/abenson/galacticus/data/Galacticus\_SSP\_Data.tar.bz2}:
\begin{description}
 \item [{\tt data/stellarPopulations/SSP\_Spectra\_Conroy-et-al\_v2.0\_imfSalpeter.hdf5}] Corresponds to a Salpeter IMF computed using v2.0 of the {\tt FSPS} code;
 \item [{\tt data/stellarPopulations/SSP\_Spectra\_Conroy-et-al\_v2.1\_imfSalpeter.hdf5}]  Corresponds to a Salpeter IMF computed using v2.1 of the {\tt FSPS} code;
 \item [{\tt data/stellarPopulations/SSP\_Spectra\_Conroy-et-al\_v2.1\_imfChabrier.hdf5}]  Corresponds to a Chabrier IMF computed using v2.1 of the {\tt FSPS} code;
 \item [{\tt data/stellarPopulations/SSP\_Spectra\_Conroy-et-al\_v2.2\_imfChabrier.hdf5}]  Corresponds to a Chabrier IMF computed using v2.2 of the {\tt FSPS} code;
 \item [{\tt data/stellarPopulations/SSP\_Spectra\_Conroy-et-al\_v2.2\_imfKennicutt.hdf5}]  Corresponds to a Kennicutt IMF computed using v2.2 of the {\tt FSPS} code;
 \item [{\tt data/stellarPopulations/SSP\_Spectra\_Conroy-et-al\_v2.2\_imfBaugh2005TopHeavy.hdf5}]  Corresponds to the top-heavy IMF of \cite{baugh_can_2005} computed using v2.2 of the {\tt FSPS} code;
 \item [{\tt data/stellarPopulations/SSP\_Spectra\_Maraston\_hbMorphologyRed\_imfKroupa.hdf5}] The spectra from \cite{maraston_evolutionary_2005} for a Kroupa IMF and a red horizontal branch morphology;
 \item [{\tt data/stellarPopulations/SSP\_Spectra\_Maraston\_hbMorphologyRed\_imfSalpeter.hdf5}] The spectra from \cite{maraston_evolutionary_2005} for a Salpeter IMF and a red horizontal branch morphology; 
 \item [{\tt data/stellarPopulations/SSP\_Spectra\_BC2003\_highResolution\_imfChabrier.hdf5}] The (high resolution) spectra from \cite{bruzual_stellar_2003} for a Chabrier IMF, using Padova 1994 tracks;
 \item [{\tt data/stellarPopulations/SSP\_Spectra\_BC2003\_highResolution\_imfSalpeter.hdf5}] The (high resolution) spectra from \cite{bruzual_stellar_2003} for a Salpeter IMF, using Padova 1994 tracks;
 \item [{\tt data/stellarPopulations/SSP\_Spectra\_BC2003\_lowResolution\_imfChabrier.hdf5}] The (low resolution) spectra from \cite{bruzual_stellar_2003} for a Chabrier IMF, using Padova 1994 tracks;
 \item [{\tt data/stellarPopulations/SSP\_Spectra\_BC2003\_lowResolution\_imfSalpeter.hdf5}] The (low resolution) spectra from \cite{bruzual_stellar_2003} for a Salpeter IMF, using Padova 1994 tracks;
 \item [{\tt data/stellarPopulations/SSP\_Spectra\_Grasil\_gkn15rd\_ken.hdf5}] The spectra used by {\sc Grasil} for a Kennicutt IMF;
 \item [{\tt data/stellarPopulations/SSP\_Spectra\_Grasil\_gkn1rd\_ken.hdf5}] The spectra used by {\sc Grasil} for a Kennicutt IMF;
 \item [{\tt data/stellarPopulations/SSP\_Spectra\_Grasil\_gsrdk0b\_sal.hdf5}] The spectra used by {\sc Grasil} for a Salpeter IMF;
 \item [{\tt data/stellarPopulations/SSP\_Spectra\_Grasil\_imf27\_kro.hdf5}] Spectra used by {\sc Grasil}.
\end{description}
Note that the high resolution spectra from \cite{bruzual_stellar_2003} may require you to adjust the {\tt [stellarPopulationLuminosityIntegrationToleranceRelative]} parameter to a larger value. The sharp features in these high resolution spectra can be difficult to integrate. Scripts to convert the data provided by \cite{maraston_evolutionary_2005} and \cite{bruzual_stellar_2003} into \glc's format are provided in the {\tt scripts/ssps} folder. Spectra for other initial mass functions will be computed automatically when using the \cite{conroy_propagation_2009} population synthesis models.

\subsubsection{Stellar Population Spectra Postprocessing}

Additional methods for postprocessing specta of stellar populations can be added using the {\tt stellarPopulationSpectraPostprocessMethod} directive. The directive should contain a single argument, giving the name of a subroutine to be called to initialize the method. For example, the {\tt Meiksin2006} method is described by a directive:
\begin{verbatim}
 !# <stellarPopulationSpectraPostprocessInitialize>
 !#  <unitName>Stellar_Population_Spectra_Postprocess_Meiksin2006_Initialize</unitName>
 !# </stellarPopulationSpectraPostprocessInitialize>
\end{verbatim}
Here, {\tt Stellar\_Population\_Spectra\_Postprocess\_Meiksin2006\_Initialize} is the name of a subroutine which will be called to initialize the method. The initialization subroutine must have the following form:
\begin{verbatim}
  subroutine Method_Initialize(stellarPopulationSpectraPostprocessMethod,postprocessingFunction)
    implicit none
    type     (varying_string),          intent(in   ) :: stellarPopulationSpectraPostprocessMethods
    procedure(              ), pointer, intent(inout) :: postprocessingFunction

    if (stellarPopulationSpectraPostprocessMethod == 'myMethod') postprocessingFunction => myPostprocessor
    return
  end subroutine Method_Initialize
\end{verbatim}
where {\tt myMethod} is the name of this method as will be specified by the {\tt stellarPopulationSpectraPostprocessMethod} input parameter. The procedure pointer {\tt postprocessingFunction} should be set to point to function with the following form:
\begin{verbatim}
  subroutine Method_Compute(wavelength,age,redshift,modifier)
    implicit none
    double precision, intent(in   ) :: wavelength,age,redshift
    double precision, intent(inout) :: modifier

    return
  end subroutine Method_Compute
\end{verbatim}
This function must multiply {\tt modifier} by the factor by which stellar spectra will be scaled. The function is given the rest-frame {\tt wavelength} (in \AA), {\tt age} of the population (in Gyr), and the {\tt redshift} of the radiation and should return a multiplicative factor by which the spectrum will be scaled.

Currently defined stellar population postprocessing methods are:
\begin{description}
\item [\hyperlink{stellar_populations.spectra.postprocess.Meiksin2006.F90:stellar_population_spectra_postprocessing_meiksin2006}{{\tt Meiksin2006}}] Postprocesses spectra through absorption by the \gls{igm} using the results of \cite{meiksin_colour_2006}.
\item [\hyperlink{stellar_populations.spectra.postprocess.Madau1995.F90:stellar_population_spectra_postprocessing_madau1995}{{\tt Madau1995}}] Postprocesses spectra through absorption by the \gls{igm} using the results of \cite{madau_radiative_1995}.
\item [\hyperlink{stellar_populations.spectra.postprocess.Lyman_continuum_suppress.F90:stellar_population_spectra_postprocessing_lyc_suppress}{{\tt lymanContinuumSuppress}}] Suppresses all luminosity in the Lyman continuum.
\item [\hyperlink{stellar_populations.spectra.postprocess.recent.F90:stellar_population_spectra_postprocessing_recent}{{\tt recent}}] Suppresses all from populations with ages in excess of {\tt [recentPopulationsTimeLimit]}.
\item [\hyperlink{stellar_populations.spectra.postprocess.identity.F90:stellar_population_spectra_postprocessing_identity}{{\tt identity}}] This is an identity operator, i.e. it leaves the spectrum unchanged.
\end{description}

\subsubsection{Stellar Feedback}

Additional methods for stellar feedback can be added using the {\tt stellarFeedbackMethod} directive. The directive should contain a single argument, giving the name of a subroutine to be called to initialize the method. For example, the {\tt standard} method is described by a directive:
\begin{verbatim}
  !# <stellarFeedbackMethod>
  !#  <unitName>Stellar_Feedback_Standard_Initialize</unitName>
  !# </stellarFeedbackMethod>
\end{verbatim}
Here, {\tt Stellar\_Feedback\_Standard\_Initialize} is the name of a subroutine which will be called to initialize the method. The initialization subroutine must have the following form:
\begin{verbatim}
  subroutine Method_Initialize(stellarFeedbackMethod,Stellar_Feedback_Cumulative_Energy_Input_Get)
    implicit none
    type(varying_string),          intent(in)    :: stellarFeedbackMethod
    procedure(),          pointer, intent(inout) :: Stellar_Feedback_Cumulative_Energy_Input_Get
    
    if (stellarFeedbackMethod == 'myMethod') then
       Stellar_Feedback_Cumulative_Energy_Input_Get => My_Stellar_Feedback_Cumulative_Energy_Input_Get
       .
       .
       .
    end if
    return
  end subroutine Method_Initialize
\end{verbatim}
where {\tt myMethod} is the name of this method as will be specified by the {\tt stellarFeedbackMethod} input parameter. The procedure pointer {\tt Stellar\_Feedback\_Cumulative\_Energy\_Input\_Get} must be set to point to a function which returns the cumulative energy input from stars as described below. The initialization subroutine should perform any other tasks required to initialize the module (such as reading parameters etc.).

The function must have the form:
\begin{verbatim}
   double precision function Stellar_Feedback_Cumulative_Energy_Input(initialMass,age,metallicity)
    implicit none
    double precision, intent(in) :: initialMass,age,metallicity
    .
    .
    .
    return
   end function Stellar_Feedback_Cumulative_Energy_Input 
\end{verbatim}
The function must return the cumulative energy input (in $M_\odot$ (km/s)$^2$ from stars of given {\tt initialMass} and {\tt metallicity} after a time {\tt age}.

Currently defined stellar feedback methods are:
\begin{description}
 \item [{\tt standard}] This method assumes that the energy input has contributions from stellar winds, Type Ia, Type II and Population III supernovae. The minimum mass required for a star to produce a Type II supernova is specified via {\tt initialMassForSupernovaeTypeII} (in $M_\odot$), while the energy per Type II or Ia supernova is specified via {\tt supernovaEnergy} (in ergs).
\end{description}

\subsubsection{Stellar Tracks}

Additional methods for stellar tracks can be added using the {\tt stellarTracksMethod} directive. The directive should contain a single argument, giving the name of a subroutine to be called to initialize the method. For example, the {\tt file} method is described by a directive:
\begin{verbatim}
  !# <stellarTracksMethod>
  !#  <unitName>Stellar_Tracks_Initialize_File</unitName>
  !# </stellarTracksMethod>
\end{verbatim}
Here, {\tt Stellar\_Tracks\_Initialize\_File} is the name of a subroutine which will be called to initialize the method. The initialization subroutine must have the following form:
\begin{verbatim}
  subroutine Method_Initialize(stellarTracksMethod,Stellar_Luminosity_Get,Stellar_Effective_Temperature_Get)
    implicit none
    type(varying_string),          intent(in)    :: stellarTracksMethod
    procedure(),          pointer, intent(inout) :: Stellar_Luminosity_Get,Stellar_Effective_Temperature_Get
    
    if (stellarTracksMethod == 'myMethod') then
       Stellar_Luminosity_Get            => My_Stellar_Luminosity_Get
       Stellar_Effective_Temperature_Get => My_Stellar_Effective_Temperature_Get
       .
       .
       .
    end if
    return
  end subroutine Method_Initialize
\end{verbatim}
where {\tt myMethod} is the name of this method as will be specified by the {\tt stellarTracksMethod} input parameter. The procedure pointers {\tt Stellar\_Luminosity\_Get} and {\tt Stellar\_Effective\_Temperature\_Get} must be set to point to functions which return the luminosity and effective temperatures of stars as described below. The initialization subroutine should perform any other tasks required to initialize the module (such as reading parameters etc.).

The functions must have the form:
\begin{verbatim}
   double precision function Stellar_Tracks_Function(initialMass,age,metallicity)
    implicit none
    double precision, intent(in) :: initialMass,age,metallicity
    .
    .
    .
    return
   end function Stellar_Tracks_Function 
\end{verbatim}
The luminosity function must return the bolometric luminosity (in $L_\odot$) of a star of given {\tt initialMass} and {\tt metallicity} after a time {\tt age}. The effective temperature function should give the effective temperature (in Kelvin) for the same star.

Currently defined stellar tracks methods are:
\begin{description}
 \item [{\tt file}] Stellar tracks are read from an HDF5 file and interpolated in. The structure of the HDF5 is described in \S\ref{sec:StellarTracksFile}.
\end{description}

\subsubsection{Stellar Winds}

Additional methods for stellar winds can be added using the {\tt stellarWindsMethod} directive. The directive should contain a single argument, giving the name of a subroutine to be called to initialize the method. For example, the {\tt Leitherer1992} method is described by a directive:
\begin{verbatim}
  !# <stellarWindsMethod>
  !#  <unitName>Stellar_Winds_Leitherer1992_Initialize</unitName>
  !# </stellarWindsMethod>
\end{verbatim}
Here, {\tt Stellar\_Winds\_Leitherer1992\_Initialize} is the name of a subroutine which will be called to initialize the method. The initialization subroutine must have the following form:
\begin{verbatim}
  subroutine Method_Initialize(stellarWindsMethod,Stellar_Winds_Mass_Loss_Rate_Get,Stellar_Winds_Terminal_Velocity_Get)
    implicit none
    type(varying_string),          intent(in)    :: stellarWindsMethod
    procedure(),          pointer, intent(inout) :: Stellar_Winds_Mass_Loss_Rate_Get,Stellar_Winds_Terminal_Velocity_Get
    
    if (stellarWindsMethod == 'myMethod') then
       Stellar_Winds_Mass_Loss_Rate_Get    => My_Stellar_Winds_Mass_Loss_Rate_Get
       Stellar_Winds_Terminal_Velocity_Get => My_Stellar_Winds_Terminal_Velocity_Get
       .
       .
       .
    end if
    return
  end subroutine Method_Initialize
\end{verbatim}
where {\tt myMethod} is the name of this method as will be specified by the {\tt stellarWindsMethod} input parameter. The procedure pointers {\tt Stellar\_Winds\_Mass\_Loss\_Rate\_Get} and {\tt Stellar\_Winds\_Terminal\_Velocity\_Get} must be set to point to functions which return the mass loss rate and terminal velocity of winds as described below. The initialization subroutine should perform any other tasks required to initialize the module (such as reading parameters etc.).

The functions must have the form:
\begin{verbatim}
   double precision function Stellar_Wind_Function(initialMass,age,metallicity)
    implicit none
    double precision, intent(in) :: initialMass,age,metallicity
    .
    .
    .
    return
   end function Stellar_Wind_Function 
\end{verbatim}
The mass loss function must return the rate of mass loss (in $M_\odot$/Gyr) from stars of given {\tt initialMass} and {\tt metallicity} after a time {\tt age}. The terminal velocity function should give the velocity (in km/s) at infinity of the wind for the same stars.

Currently defined stellar winds methods are:
\begin{description}
 \item [{\tt Leitherer1992}] Computes wind properties using the fitting functions of \cite{leitherer_deposition_1992} and \glc\ stellar tracks.
\end{description}

\subsubsection{Supernovae Type Ia}

Additional methods for Type 1a supernovae can be added using the {\tt supernovaeIaMethod} directive. The directive should contain a single argument, giving the name of a subroutine to be called to initialize the method. For example, the {\tt Nagashima} method is described by a directive:
\begin{verbatim}
  !# <supernovaeIaMethod>
  !#  <unitName>Supernovae_Type_Ia_Nagashima_Initialize</unitName>
  !# </supernovaeIaMethod>
\end{verbatim}
Here, {\tt Supernovae\_Type\_Ia\_Nagashima\_Initialize} is the name of a subroutine which will be called to initialize the method. The initialization subroutine must have the following form:
\begin{verbatim}
  subroutine Method_Initialize(supernovaeIaMethod,SNeIa_Cumulative_Number_Get,SNeIa_Cumulative_Yield_Get)
    implicit none
    type(varying_string),          intent(in)    :: supernovaeIaMethod
    procedure(),          pointer, intent(inout) :: SNeIa_Cumulative_Number_Get,SNeIa_Cumulative_Yield_Get
    
    if (supernovaeIaMethod == 'myMethod') then
       SNeIa_Cumulative_Number_Get => My_SNeIa_Cumulative_Number_Get
       SNeIa_Cumulative_Yield_Get  => My_SNeIa_Cumulative_Yield_Get
       .
       .
       .
    end if
    return
  end subroutine Method_Initialize
\end{verbatim}
where {\tt myMethod} is the name of this method as will be specified by the {\tt supernovaeIaMethod} input parameter. The procedure pointers {\tt SNeIa\_Cumulative\_Number\_Get} and {\tt SNeIa\_Cumulative\_Yield\_Get} must be set to point to functions which return the cumulative number of and cumulative yield from Type Ia supernovae as described below. The initialization subroutine should perform any other tasks required to initialize the module (such as reading parameters etc.).

The cumulative number function must have the form:
\begin{verbatim}
   double precision function SNeIa_Cumulative_Number(initialMass,age,metallicity)
    implicit none
    double precision, intent(in) :: initialMass,age,metallicity
    .
    .
    .
    return
   end function SNeIa_Cumulative_Number 
\end{verbatim}
and must return the number of Type Ia supernovae resulting per $M_\odot$ of stars formed with given {\tt initialMass} and {\tt metallicity} after a time {\tt age}. (Since Type Ia's form in binary systems this function should specifically return the number such that when integrated over the \gls{imf} it gives the correct total number of Type Ia supernovae formed from a single stellar population.)

The cumulative yield function must have the form:
\begin{verbatim}
   double precision function SNeIa_Cumulative_Yield(initialMass,age,metallicity,atomIndex)
    implicit none
    double precision, intent(in)           :: initialMass,age,metallicity
    integer,          intent(in), optional :: atomIndex
    .
    .
    .
    return
   end function SNeIa_Cumulative_Yield 
\end{verbatim}
and should return the yield of the element identified by {\tt atomIndex} (as returned by the \hyperlink{atomic.data.F90:atomic_data:atom_lookup}{{\tt Atom\_Lookup()}} function from the \hyperlink{atomic.data.F90:atomic_data}{{\tt Atomic\_Data}} module) if present, or total metal yield otherwise from Type Ia's resulting from stars defined in the same way as for the cumulative number function.

Currently defined type Ia supernovae methods are:
\begin{description}
 \item [{\tt Nagashima}] Computes Type Ia properties using the methods described by \cite{nagashima_metal_2005}.
\end{description}

\subsubsection{Tidal Mass Loss Rates in Disks/Spheroids}

Additional methods for computing tidal induced mass loss rates in disks/spheroids can be added using the {\tt tidalStrippingMassLossRate(Disks|Spheroids)Method} directive. The directive should contain a single argument, giving the name of a subroutine to be called to initialize the method. For example, the {\tt simple} method for disks is described by a directive:
\begin{verbatim}
 !# <tidalStrippingMassLossRateDisksMethod>
 !#  <unitName>Tidal_Stripping_Mass_Loss_Rate_Disks_Simple_Init</unitName>
 !# </tidalStrippingMassLossRateDisksMethod>
\end{verbatim}
Here, {\tt Tidal\_Stripping\_Mass\_Loss\_Rate\_Disks\_Simple\_Init} is the name of a subroutine which will be called to initialize the method. The initialization subroutine must have the following form:
\begin{verbatim}
  subroutine Method_Initialize(tidalStrippingMassLossRateDisksMethod,Tidal_Stripping_Mass_Loss_Rate_Disk_Get)
    implicit none
    type(varying_string),          intent(in)    :: starFormationTimescaleDisksMethod
    procedure(),          pointer, intent(inout) :: Tidal_Stripping_Mass_Loss_Rate_Disk_Get
    
    if (tidalStrippingMassLossRateDisksMethod == 'myMethod') Tidal_Stripping_Mass_Loss_Rate_Disk_Get => My_Tidal_Stripping_Mass_Loss_Rate_Disk_Get
    return
  end subroutine Method_Initialize
\end{verbatim}
where {\tt myMethod} is the name of this method as will be specified by the {\tt tidalStrippingMassLossRate(Disks|Spheroids)Method} input parameter. The procedure pointer {\tt Tidal\_Stripping\_Mass\_Loss\_Rate\_(Disk|Spheroid)\_Get} must be set to point to a function which returns mass loss rate due to tidal forces as described below. The initialization subroutine should perform any other tasks required to initialize the module (such as reading parameters etc.).

The mass loss rate function must have the form:
\begin{verbatim}
 double precision function Tidal_Stripping_Mass_Loss_Rate_Disk_Get(thisNode)
    implicit none
    type(treeNode), intent(in) :: thisNode
    .
    .
    .
    return
 end function Tidal_Stripping_Mass_Loss_Rate_Disk_Get
\end{verbatim}
The function must return the mass loss rate induced by tidal forces (in units of $M_\odot$/Gyr) for the disk/spheroid in {\tt thisNode}.

Currently defined ram pressure mass loss rate methods are:
\begin{description}
 \item [\hyperlink{tidal_stripping.mass_loss_rate.disks.simple.F90:tidal_stripping_mass_loss_rate_disks_simple:tidal_stripping_mass_loss_rate_disk_simple}{{\tt simple}}] The mass loss rate scales in proportion to the ratio of tidal and gravitational restoring forces;
 \item [\hyperlink{tidal_stripping.mass_loss_rate.disks.null.F90:tidal_stripping_mass_loss_rate_disks_null:tidal_stripping_mass_loss_rate_disk_null}{{\tt null}}] The mass loss rate is assumed to be always zero.
\end{description}


\subsubsection{Tree Timing}

Additional methods for tree timing (i.e. the time taken to process a given merger tree) can be added using the {\tt timePerTreeMethod} directive. The directive should contain a single argument, giving the name of a subroutine to be called to initialize the method. For example, the {\tt file} method is described by a directive:
\begin{verbatim}
  !# <timePerTreeMethod>
  !#  <unitName>Galacticus_Time_Per_Tree_File_Initialize</unitName>
  !# </timePerTreeMethod>
\end{verbatim}
Here, {\tt Galacticus\_Time\_Per\_Tree\_File\_Initialize} is the name of a subroutine which will be called to initialize the method. The initialization subroutine must have the following form:
\begin{verbatim}
  subroutine Method_Initialize(timePerTreeMethod,Galacticus_Time_Per_Tree_Get)
    implicit none
    type(varying_string),          intent(in)    :: timePerTreeMethod
    procedure(),          pointer, intent(inout) :: Galacticus_Time_Per_Tree_Get
    
    if (timePerTreeMethod == 'myMethod') then
       Galacticus_Time_Per_Tree_Get => My_Time_Per_Tree_Get
       .
       .
       .
    end if
    return
  end subroutine Method_Initialize
\end{verbatim}
where {\tt myMethod} is the name of this method as will be specified by the {\tt timePerTreeMethod} input parameter. The procedure pointer {\tt Galacticus\_Time\_Per\_Tree\_Get} must be set to point to a function which returns an estimate of the time taken (in seconds) to process a merger tree. The initialization subroutine should perform any other tasks required to initialize the module (such as reading parameters etc.).

The function must have the form:
\begin{verbatim}
   double precision function Time_Per_Tree(treeRootMass)
    implicit none
    double precision, intent(in) :: treeRootMass
    .
    .
    .
    return
   end function Time_Per_Tree 
\end{verbatim}
The function must return an estimate of the time taken (in seconds) to process a merger tree with the given {\tt treeRootMass}.

Currently defined tree timing methods are:
\begin{description}
 \item [{\tt file}] This method reads coefficients of a simple fitting formula for the processing time from a file, specified via the {\tt [timePerTreeFitFileName]} parameter (see \S\ref{sec:TreeTimingFile}).
\end{description}

\subsubsection{Transfer Function}\label{sec:TransferFunctionMethod}

Additional methods for transfer function can be added using the {\tt transferFunctionMethod} directive. The directive should contain a single argument, giving the name of a subroutine to be called to initialize the method. For example, the {\tt file} method is described by a directive:
\begin{verbatim}
  !# <transferFunctionMethod>
  !#  <unitName>Transfer_Function_File_Initialize</unitName>
  !# </transferFunctionMethod>
\end{verbatim}
Here, {\tt Transfer\_Function\_File\_Initialize} is the name of a subroutine which will be called to initialize the method. The initialization subroutine must have the following form:
\begin{verbatim}
  subroutine Method_Initialize(transferFunctionMethod,Transfer_Function_Tabulate)
    implicit none
    type(varying_string),          intent(in)    :: transferFunctionMethod
    procedure(),          pointer, intent(inout) :: Transfer_Function_Tabulate
    
    if (transferFunctionMethod == 'myMethod') then
       Transfer_Function_Tabulate => My_Do_Tabulate
       .
       .
       .
    end if
    return
  end subroutine Method_Initialize
\end{verbatim}
where {\tt myMethod} is the name of this method as will be specified by the {\tt transferFunctionMethod} input parameter. The procedure pointer {\tt Transfer\_Function\_Tabulate} must be set to point to a subroutine which tabulates the transfer function as described below. The initialization subroutine should perform any other tasks required to initialize the module (such as reading parameters etc.).

The tabulation subroutine must have the form:
\begin{verbatim}
   subroutine Transfer_Function_Tabulate(wavenumber,transferFunctionNumberPoints,transferFunctionWavenumber,transferFunctionT)
    implicit none
    double precision,                            intent(in)    :: wavenumber
    double precision, allocatable, dimension(:), intent(inout) :: transferFunctionLogWavenumber,transferFunctionLogT
    integer,                                     intent(out)   :: transferFunctionNumberPoints
    .
    .
    .
    return
   end subroutine Transfer_Function_Tabulate
\end{verbatim}
The subroutine must tabulate the natural log of the transfer function in array {\tt transferFunctionLogT()} as a function of the natural log of wavenumber {\tt transferFunctionLogWavenumber()} (these arrays must be allocated to the correct size, and may be prevously allocated, therefore requiring a deallocation). The number of tabulated points should be returned in {\tt transferFunctionNumberPoints}. The subroutine should ensure that the currently requested {\tt wavenumber} is within the range of the tabulated function (preferably with some buffer).

Currently defined transfer function methods are:
\begin{description}
 \item [{\tt null}] $T(k)=1$.
 \item [{\tt file}] The transfer function is read from an XML file specified by input parameter {\tt transferFunctionFile}.
 \item [{\tt CAMB}] The transfer function is generated by \href{http://camb.info/}{\sc CAMB} using the specified cosmological parameters. The transfer function is written out to a file in the {\tt data/} directory and will be re-read later if needed.
 \item [{\tt BBKS}] The transfer function is computed using the fitting formula of \cite{bardeen_statistics_1986}.
 \item [{\tt Eisenstein-Hu1999}] The transfer function is computed using the fitting formula of \cite{eisenstein_power_1999}. The effective number of neutrino species and the summed mass (in electron volts) of all neutrino species are specified via the {\tt effectiveNumberNeutrinos} and {\tt summedNeutrinoMasses} parameters respectively.
\end{description}

The XML file format for transfer functions looks like:
\begin{verbatim}
 <data>
  <column>k [Mpc^{-1}] - wavenumber</column>
  <column>T(k) - transfer function</column>
  <datum>1.111614e-05 0.218866E+08</datum>
  <datum>1.228521e-05 0.218866E+08</datum>
  <datum>1.357727e-05 0.218866E+08</datum>
  <datum>1.50052e-05 0.218866E+08</datum>
  <datum>1.658335e-05 0.218866E+08</datum>
  <datum>1.83274e-05 0.218865E+08</datum>
  .
  .
  .
  <description>Cold dark matter power spectrum created by CAMB.</description>
  <fileFormat>1</fileFormat>
  <parameter>
    <name>Omega_b</name>
    <value>0.0450</value>
  </parameter>
  <parameter>
    <name>Omega_Matter</name>
    <value>0.250</value>
  </parameter>
  <parameter>
    <name>Omega_DE</name>
    <value>0.750</value>
  </parameter>
  <parameter>
    <name>H_0</name>
    <value>70.0</value>
  </parameter>
  <parameter>
    <name>T_CMB</name>
    <value>2.780</value>
  </parameter>
  <parameter>
    <name>Y_He</name>
    <value>0.24</value>
  </parameter>
  <extrapolation>
    <wavenumber>
      <limit>low</limit>
      <method>power law</method>
    </wavenumber>
    <wavenumber>
      <limit>high</limit>
      <method>power law</method>
    </wavenumber>
  </extrapolation>
</data>
\end{verbatim}
The {\tt datum} elements give wavenumber (in Mpc$^{-1}$) and transfer function pairs. The {\tt extrapolation} element defines how the tabulated function should be extrapolated to lower and higher wavenumbers. The two options for the {\tt method} are ``fixed'', in which case the transfer function is extrapolated assuming that it remains constant, and ``power law'' in which case the extrapolation is performed assuming a fixed power-law relation between transfer function and wavenumber. The {\tt column}, {\tt description} and {\tt parameter} elements are optional, but are encouraged to make the file easier to understand. Finally, the {\tt fileFormat} element should currently always contain the value $1$---this may change in future if the format of this file is modified.

\subsection{Events}

Events are triggered during merger tree evolution. Examples are when a node needs to be promoted to its parent node, or when a minor node merges with its parent.

\subsubsection{Node Merger Events}

Additional methods for the node merging (i.e. when a non-primary progenitor merges with its parent) can be added using the {\tt nodeMergersMethod} directive. The directive should contain a single argument, giving the name of a subroutine to be called to initialize the method. For example, the {\tt singleLevelHierarchy} method is described by a directive:
\begin{verbatim}
  !# <nodeMergersMethod>
  !#  <unitName>Events_Node_Merger_Initialize_SLH</unitName>
  !# </nodeMergersMethod>
\end{verbatim}
Here, {\tt Satellite\_Time\_Until\_Merging\_Lacey\_Cole\_Initialize} is the name of a subroutine which will be called to initialize the method. The initialization subroutine must have the following form:
\begin{verbatim}
   subroutine Events_Node_Merger_Initialize(nodeMergersMethod,Events_Node_Merger_Do)
    implicit none
    type(varying_string),          intent(in)    :: nodeMergersMethod
    procedure(),          pointer, intent(inout) :: Events_Node_Merger_Do

    if (nodeMergersMethod.eq.'myMethod') Events_Node_Merger_Do => My_Method_Procedure

    return
  end subroutine Events_Node_Merger_Initialize
\end{verbatim}
where {\tt myMethod} is the name of this method as will be specified by the {\tt nodeMergersMethod} input parameter. The procedure pointer {\tt Events\_Node\_Merger\_Do} must be set to point to a subroutine which handles the merging event as described below. The initialization subroutine should perform any other tasks required to initialize the module (such as reading parameters etc.).

The node merging subroutine must have the form:
\begin{verbatim}
 subroutine Events_Node_Merger_Do(thisNode)
    implicit none
    type(treeNode), intent(inout), pointer :: thisNode
    .
    .
    .
    return
  end subroutine Events_Node_Merger_Do_SLH
\end{verbatim}
The function must perform any processing required for the merger, and move {\tt thisNode} to the linked list of satellite nodes in {\tt thisNode\%parentNode}.

Currently defined node merger event methods are:
\begin{description}
 \item [\hyperlink{events.node_merger.single_level_hierarchy.F90:events_node_mergers_slh:events_node_merger_do_slh}{{\tt singleLevelHierarchy}}] The node merger is handled by placing the merging node into the linked list of satellites of the parent node. Any satellites in the merging node are also promoted to be satellites in the new node, thereby maintaining just a single hierarchy level of substructure.
\end{description}

\subsubsection{Node Promotion Events}

Additional methods for node promotion (i.e. when a primary progenitor reaches its parent halo) can be added using the {\tt nodePromotionTask} directive. The directive should contain a single argument, giving the name of a subroutine to be called to initialize the method. For example, the {\tt basic} tree node method uses this directive as follows:
\begin{verbatim}
  !# <nodePromotionTask>
  !#  <unitName>Tree_Node_Basic_Promote</unitName>
  !# </nodePromotionTask>
\end{verbatim}
Here, {\tt Tree\_Node\_Basic\_Promote} is the name of a subroutine which will be called to perform whatever tasks are required prior to the promotion. The subroutine must have the following form:
\begin{verbatim}
   subroutine Node_Promotion_Task(thisNode)
    implicit none
    type(treeNode), pointer, intent(inout) :: thisNode
    .
    .
    .
    return
  end subroutine Node_Promotion_Task
\end{verbatim}
where {\tt thisNode} is the node about to be promoted.

\subsection{Tasks}

Tasks are any processing which must be performed on a node as a result of some specific event (such as a merger).

\subsubsection{Calculation Reset Tasks}\label{sec:CalculationResetTask}

Additional methods for calculation reset tasks (i.e. flagging that the properties of a node may have changed so that any calculations must be performed anew) can be added using the {\tt calculationResetTask} directive. The directive should contain a single argument, giving the name of a subroutine to be called to perform the task. For example, the standard hot halo component adds a task as follows:
\begin{verbatim}
  !# <calculationResetTask>
  !#  <unitName>Tree_Node_Hot_Halo_Reset_Standard</unitName>
  !# </calculationResetTask>
\end{verbatim}
Here, {\tt Tree\_Node\_Hot\_Halo\_Reset\_Standard} is the name of a subroutine which will be called to perform whatever tasks are required. The subroutine must have the following form:
\begin{verbatim}
   subroutine Reset_Task(thisNode)
    implicit none
    type(treeNode), pointer, intent(inout) :: thisNode
    .
    .
    .
    return
  end subroutine Prederivative_Task
\end{verbatim}
where {\tt thisNode} is the node for which derivatives will be computed. Tasks typically involve precomputing quantities that will be used in finding the derivatives or resetting the state so that stored quantities will be recomputed as needed.

\subsubsection{Cooling Rate Modifiers}

Additional methods for modifying the cooling rate from the hot halo can be added using the {\tt coolingRateModifierMethod} directive. The directive should contain a single argument, giving the name of a subroutine to be called to modify the cooling rate. For example, the ``cut-off'' modifier is described by a directive:
\begin{verbatim}
 !# <coolingRateModifierMethod>
 !#  <unitName>Cooling_Rate_Modifier_Cut_Off</unitName>
 !# </coolingRateModifierMethod>
\end{verbatim}
Here, {\tt Cooling\_Rate\_Modifier\_Cut\_Off} is the name of the subroutine which will be called modify the cooling rate. The modification subroutine must have the following form:
\begin{verbatim}
  subroutine Modify_Rate(thisNode,coolingRate)
    implicit none
    type(treeNode)  , intent(inout), pointer :: thisNode
    double precision, intent(inout)          :: coolingRate    

    return
  end subroutine Modify_Rate
\end{verbatim}
The subroutine should modify the {\tt coolingRate} as necessary.

Currently defined cooling rate modifier tasks are:
\begin{description}
\item [\hyperlink{cooling.cooling_rate.modifier.cut_off.F90:cooling_rates_modifier_cut_off:cooling_rate_modifier_cut_off}{``cut-off''}] The cooling rate is set to zero in halos with virial velocities below {\tt [coolingCutOffVelocity]} at redshifts below/above {\tt [coolingCutOffRedshift]} for {\tt [coolingCutOffWhen]}$=${\tt after/before}. In other halos the cooling rate is not modified.
\end{description}

\subsubsection{Decode Property Identifier Tasks}\label{sec:DecodePropertyIndentifierTask}

Additional property identifier decoding tasks (i.e. determining the name of a property from a set of integer identifiers) can be added using the {\tt decodePropertyIdentifiersTask} directive. The directive should contain a single argument, giving the name of a subroutine to be called to perform the task. For example, the Hernquist spheroid component adds a task as follows:
\begin{verbatim}
  !# <decodePropertyIdentifiersTask>
  !#  <unitName>Hernquist_Spheroid_Property_Identifiers_Decode</unitName>
  !# </decodePropertyIdentifiersTask>
\end{verbatim}
Here, {\tt Hernquist\_Spheroid\_Property\_Identifiers\_Decode} is the name of a subroutine which will be called to perform the decoding task. The subroutine must have the following form:
\begin{verbatim}
   subroutine Property_Identifier_Decode_Task(propertyComponent,propertyObject,propertyIndex,matchedProperty,propertyName)
    implicit none
    integer,              intent(in)    :: propertyComponent,propertyObject,propertyIndex
    logical,              intent(inout) :: matchedProperty
    type(varying_string), intent(inout) :: propertyName
    .
    .
    .
    return
  end subroutine Property_Identifier_Decode_Task
\end{verbatim}
The task should check whether {\tt propertyComponent} matches its stored {\tt componentIndex} value. If it does, it should set {\tt propertyName} to a suitable name (e.g. {\tt hernquistSpheroid::stellarMass}) and set {\tt matchedProperty}$=${\tt true}. The value of {\tt propertyObject} will be either {\tt objectTypeProperty} indicating that the object in question is a standard property, or {\tt objectTypeHistory} indicating that it is a history. The value of {\tt propertyIndex} then gives the position of the object in question in the array of properties or histories.

\subsubsection{Evolution Timestep Tasks}

Merger tree nodes are evolved over some fixed timestep before evolution is stopped and other processing is allowed. The timestep is always sufficiently small such that the node does not evolve past the time of its parent node, nor does it evolve past the time of any of its satellite nodes. An arbitrary number of other criteria can be used to adjust the timestep. Such a criterion can be added using the {\tt timeStepsTask} directive. For example, the {\tt simple} timestep task adds itself using
\begin{verbatim}
 !# <timeStepsTask>
 !#  <unitName>Merger_Tree_Timestep_Simple</unitName>
 !# </timeStepsTask>
\end{verbatim}
Here, {\tt unitName} gives the name of the subroutine to be called to (possibly) adjust the timestep. It should have the following form:
\begin{verbatim}
  subroutine My_Timestep(thisNode,timeStep,End_Of_Timestep_Task,report,lockNode,lockType)
    implicit none
    type     (treeNode      ), intent(inout), pointer           :: thisNode
    procedure(              ), intent(inout), pointer           :: End_Of_Timestep_Task
    double precision         , intent(inout)                    :: timeStep
    logical                  , intent(in   )                    :: report
    type     (treeNode      ), intent(inout), pointer, optional :: lockNode
    type     (varying_string), intent(inout),          optional :: lockType
    .
    .
    .
    return
  end subroutine My_Timestep
\end{verbatim}
This subroutine should compute a suitable timestep for {\tt thisNode} and, if it is less than the currently defined value of {\tt timeStep} should set {\tt timeStep} to that value. Optionally, the procedure pointer {\tt End\_Of\_Timestep\_Task} can be set to point to a subroutine which will be called after the node is evolved to the end of the timestep. It is acceptable for this pointer to be null. Note that the {\tt End\_Of\_Timestep\_Task} will only be called for the task which provided the shortest timestep---other tasks can always request to be called again when the next timestep is determined. The subroutine to be called at the end of the timestep must have the form:
\begin{verbatim}
  subroutine My_End_Of_Timestep_Task(thisTree,thisNode,deadlockStatus)
    implicit none
    type(mergerTree), intent(in)             :: thisTree
    type(treeNode),   intent(inout), pointer :: thisNode
    integer,          intent(inout)          :: deadlockStatus
    .
    .
    .
    return
  end subroutine My_End_Of_Timestep_Task
\end{verbatim}
The {\tt deadlockStatus} argument should be set to {\tt isNotDeadlocked} (provided by the \hyperlink{merger_trees.evolve.deadlock_options.F90:merger_trees_evolve_deadlock_status}{\tt Merger\_Trees\_Evolve\_Deadlock\_Status} module) if, and only if, the end of timestep task makes some change to the state of the tree (e.g. merging a node), to indicate that the tree was not deadlocked in this pass (i.e. something actually changed in the tree).

If the {\tt report} argument is {\tt true} then the function should report the value of {\tt timestep} prior to exiting. (This is used in reporting on timestepping criteri in deadlocked trees.) It is recommended that the report be made using the \hyperlink{merger_trees.evolve.timesteps.report.F90:evolve_to_time_reports:evolve_to_time_report}{\tt Evolve\_To\_Time\_Report()} function. Additionally, if the optional {\tt lockNode} and {\tt lockType} arguments are present then additional information can be supplied to aid in diagnosing deadlock conditions. If the current task is limiting the timestep then the {\tt lockNode} pointer should be set to point to whichever node is causing the limit (which may be {\tt thisNode} or some other node, e.g. a satellite of {\tt thisNode}, etc.), and {\tt lockType} should be set to a short description label identifying the type of limit.

\subsubsection{Galactic Component Density}\label{sec:GalacticComponentDensity}

The function {\tt Galactic\_Structure\_Density()} computes the density of material at a given position within a node. To have their density counted, each component must register a task using:
\begin{verbatim}
 !# <densityTask>
 !#  <unitName>Density_Procedure</unitName>
 !# </densityTask>
\end{verbatim}
where {\tt Density\_Procedure} is the name of a function with the following template
\begin{verbatim}
 double precision function Density_Procedure(thisNode,position,coordinateSystem,componentType,massType,weightBy,weightIndex,haloLoaded)
    type(treeNode),   intent(inout), pointer  :: thisNode
    integer,          intent(in)              :: massType,coordinateSystem,componentType,weightBy,weightIndex
    double precision, intent(in)              :: radius
    logical         , intent(in),    optional :: haloLoaded
    .
    .
    .
    return
 end function Density_Procedure
\end{verbatim}
If {\tt componentType} is a match to the component then the function should return the density of the component matching type {\tt massType} at {\tt position} for {\tt thisNode}. {\tt componentType} and {\tt massType} can take one of the values described in \S\ref{sec:ComponentMassTypes}. In the above ``density'' can actually refer to different quantities depending on the values of {\tt weightBy} (and {\tt weightIndex}):
\begin{description}
\item [{\tt weightByMass}] The actual mass should be returned (the value of {\tt weightIndex} is irrelevant);
\item [{\tt weightByLuminosity}] The {\tt weightIndex}$^{\rm th}$ luminosity should be returned.
\end{description}
If {\tt haloLoaded}$=${\tt true} (which should be the default if this option is not present), then the effects of baryonic loading on the halo profile should be taken into account where necessary. Otherwise, the effects of baryonic loading should be ignored.

\subsubsection{Galactic Component Enclosed Mass}

The function {\tt Galactic\_Structure\_Enclosed\_Mass()} computes the mass within a specified radius in a node. To have their mass counted, each component must register a task using:
\begin{verbatim}
 !# <enclosedMassTask>
 !#  <unitName>Enclosed_Mass_Procedure</unitName>
 !# </enclosedMassTask>
\end{verbatim}
where {\tt Enclosed\_Mass\_Procedure} is the name of a function with the following template
\begin{verbatim}
 double precision function Enclosed_Mass_Procedure(thisNode,radius,componentType,massType,weightBy,weightIndex,haloLoaded)
    type(treeNode),   intent(inout), pointer  :: thisNode
    integer,          intent(in)              :: massType,componentType,weightBy,weightIndex
    double precision, intent(in)              :: radius
    logical         , intent(in),    optional :: haloLoaded
    .
    .
    .
    return
 end function Enclosed_Mass_Procedure
\end{verbatim}
If {\tt componentType} is a match to the component then the function should return the ``mass'' of the component matching type {\tt massType} within {\tt radius} for {\tt thisNode}.  {\tt componentType} and {\tt massType} can take one of the values described in \S\ref{sec:ComponentMassTypes}.
If {\tt radius} is equal to or greater than {\tt radiusLarge} the routine should the total ``mass'' (i.e. ``mass'' within infinite radius). In the above ``mass'' can actually refer to different quantities depending on the values of {\tt weightBy} (and {\tt weightIndex}):
\begin{description}
\item [{\tt weightByMass}] The actual mass should be returned (the value of {\tt weightIndex} is irrelevant);
\item [{\tt weightByLuminosity}] The {\tt weightIndex}$^{\rm th}$ luminosity should be returned.
\end{description}
If {\tt haloLoaded}$=${\tt true} (which should be the default if this option is not present), then the effects of baryonic loading on the halo profile should be taken into account where necessary. Otherwise, the effects of baryonic loading should be ignored.

\subsubsection{Galactic Component Rotation Curve}

The function {\tt Galactic\_Structure\_Rotation\_Curve()} computes the rotation curve at a specified radius in a node. To have their contribution counted, each component must register a task using:
\begin{verbatim}
 !# <rotationCurveTask>
 !#  <unitName>Rotation_Curve_Procedure</unitName>
 !# </rotationCurveTask>
\end{verbatim}
where {\tt Rotation\_Curve\_Procedure} is the name of a function with the following template
\begin{verbatim}
 double precision function Rotation_Curve_Procedure(thisNode,radius,componentType,massType,haloLoaded)
    type(treeNode),   intent(inout), pointer  :: thisNode
    integer,          intent(in)              :: massType,componentType
    double precision, intent(in)              :: radius
    logical         , intent(in),    optional :: haloLoaded
    .
    .
    .
    return
 end function Rotation_Curve_Procedure
\end{verbatim}
If {\tt componentType} is a match to the component then the procedure should return the contribution to the rotation curve due to the component matching type {\tt massType} within {\tt radius} for {\tt thisNode}. {\tt componentType} and {\tt massType} can take one of the values described in \S\ref{sec:ComponentMassTypes}. If {\tt haloLoaded}$=${\tt true} (which should be the default if this option is not present), then the effects of baryonic loading on the halo profile should be taken into account where necessary. Otherwise, the effects of baryonic loading should be ignored.

\subsubsection{Galactic Component Rotation Curve Gradient}

The function {\tt Galactic\_Structure\_Rotation\_Curve\_Gradient()} computes the gradient of the rotation curve at a specified radius in a node. To have their contribution counted, each component must register a task using:
\begin{verbatim}
 !# <rotationCurveGradientTask>
 !#  <unitName>Rotation_Curve_Gradient_Procedure</unitName>
 !# </rotationCurveGradientTask>
\end{verbatim}
where {\tt Rotation\_Curve\_Gradient\_Procedure} is the name of a function with the following template
\begin{verbatim}
 double precision function Rotation_Curve_Gradient_Procedure(thisNode,radius,componentType,massType,haloLoaded)
    type(treeNode),   intent(inout), pointer  :: thisNode
    integer,          intent(in)              :: massType,componentType
    double precision, intent(in)              :: radius
    logical         , intent(in),    optional :: haloLoaded
    .
    .
    .
    return
 end function Rotation_Curve_Gradient_Procedure
\end{verbatim}
If {\tt componentType} is a match to the component then the function should return the contribution to the gradient of $V_{\rm c}^2(r)$ due to the component matching type {\tt massType} within {\tt radius} for {\tt thisNode}. \emph{Note that this is the gradient of the square of the rotation curve to permit gradients to be directly summed.} {\tt componentType} and {\tt massType} can take one of the values described in \S\ref{sec:ComponentMassTypes}. If {\tt haloLoaded}$=${\tt true} (which should be the default if this option is not present), then the effects of baryonic loading on the halo profile should be taken into account where necessary. Otherwise, the effects of baryonic loading should be ignored.

\subsubsection{Galactic Component Potential}

The function {\tt Galactic\_Structure\_Potential()} computes the potential at a specified radius in a node. To have their contribution counted, each component must register a task using:
\begin{verbatim}
  !# <potentialTask>
  !#  <unitName>Potential_Task</unitName>
  !# </potentialTask>
\end{verbatim}
where {\tt Potential\_Task} is the name of a function with the following template
\begin{verbatim}
 double precision function Potential_Procedure(thisNode,radius,componentType,massType,haloLoaded)
    type(treeNode),   intent(inout), pointer   :: thisNode
    integer,          intent(in),    optional  :: componentType  
    double precision, intent(in)               :: radius
    logical         , intent(in),    optional :: haloLoaded
    .
    .
    return
 end function Potential_Procedure
\end{verbatim}
If {\tt componentType} is a match to the component then the procedure should return the contribution to the rotation curve due to the component matching type {\tt massType} within {\tt radius} for {\tt thisNode}. {\tt componentType} and {\tt massType} can take one of the values described in \S\ref{sec:ComponentMassTypes}. If {\tt haloLoaded}$=${\tt true} (which should be the default if this option is not present), then the effects of baryonic loading on the halo profile should be taken into account where necessary. Otherwise, the effects of baryonic loading should be ignored.

\subsubsection{Galactic Component Surface Density}

The function {\tt Galactic\_Structure\_Surface\_Density()} computes the surface density of material at a given position within a node. Note that while a 3-D position is specified the routine should return the surface density corresponding to integrating the component density through the minor axis (typically the $z$-axis). To have their surface density counted, each component must register a task using:
\begin{verbatim}
 !# <surfaceDensityTask>
 !#  <unitName>Surface_Density_Procedure</unitName>
 !# </surfaceDensityTask>
\end{verbatim}
where {\tt Surface\_Density\_Procedure} is the name of a function with the following template
\begin{verbatim}
 double precision function Surface_Density_Procedure(thisNode,position,coordinateSystem,componentType,massType,haloLoaded)
    type(treeNode),   intent(inout), pointer     :: thisNode
    integer,          intent(in)                 :: massType,coordinateSystem,componentType
    double precision, intent(in),    dimension(3):: position
    logical         , intent(in),    optional    :: haloLoaded
    .
    .
    .
    return
 end function Surface_Density_Procedure
\end{verbatim}
If {\tt componentType} is a match to the component then the function should return the surface density of the component matching type {\tt massType} at {\tt position} for {\tt thisNode}. {\tt componentType} and {\tt massType} can take one of the values described in \S\ref{sec:ComponentMassTypes}.
The coordinate system in which {\tt position} is specified is given by {\tt coordinateSystem} which can take on the following values:
\begin{description}
 \item [{\tt coordinateSystemCartesian}] Cartesian $(x,y,z)$;
 \item [{\tt coordinateSystemSpherical}] Spherical $(r,\theta,\phi)$;
 \item [{\tt coordinateSystemCylindrical}] Cylindrical $(R,\phi,z)$.
\end{description}
If {\tt haloLoaded}$=${\tt true} (which should be the default if this option is not present), then the effects of baryonic loading on the halo profile should be taken into account where necessary. Otherwise, the effects of baryonic loading should be ignored.

\subsubsection{Halo Formation Events}\label{sec:HaloFormationEvents}

Tasks to be performed when a halo is deemed to have ``formed'' (or reformed) can be registered using the {\tt haloFormationTask} directive. For example, the {\tt Tree\_Node\_Methods\_Hot\_Halo} module registers a task using
\begin{verbatim}
 !# <haloFormationTask>
 !#  <unitName>Hot_Halo_Formation_Task</unitName>
 !# </haloFormationTask>
\end{verbatim}
The contents of {\tt <unitName>} should give the name of the subroutine to be called on halo formation. The subroutine should have a single argument, {\tt thisNode}, which is the node that has (re)formed.

\subsubsection{HDF5 File Close}

Tasks to be performed just prior to closing the \glc\ output HDF5 file (typically involving writing accumulated data to that file) can be registered using the {\tt hdfPreCloseTask} directive. For example, the {\tt Merger\_Tree\_Timesteps\_History} module registers a task using
\begin{verbatim}
 !# <hdfPreCloseTask>
 !#  <unitName>Merger_Tree_History_Write</unitName>
 !# </hdfPreCloseTask>
\end{verbatim}
The contents of {\tt <unitName>} should give the name of the subroutine to be called prior to HDF5 file closure. The subroutine should have no arguments.

\subsubsection{Initial Mass Functions}\label{sec:imfTasks}

New \gls{imf} s can be added using the {\tt imfRegister} and {\tt imfRegisterName} task directives. For example, the {\tt Salpeter} \gls{imf} is registered using the directives:
\begin{verbatim}
 !# <imfRegister>
 !#  <unitName>Star_Formation_IMF_Register_Salpeter</unitName>
 !# </imfRegister>
\end{verbatim}
and
\begin{verbatim}
 !# <imfRegisterName>
 !#  <unitName>Star_Formation_IMF_Register_Name_Salpeter</unitName>
 !# </imfRegisterName>
\end{verbatim}
The {\tt unitName} tags specify subroutines that are called to register the \gls{imf}. These subroutines should have the following forms:
\begin{verbatim}
 subroutine Star_Formation_IMF_Register_My_IMF(imfAvailableCount)
   integer, intent(inout) :: imfAvailableCount

   imfAvailableCount=imfAvailableCount+1
   myImfIndex       =imfAvailableCount
   return
 end subroutine Star_Formation_IMF_Register_My_IMF

 subroutine Star_Formation_IMF_Register_Name_My_IMF(imfNames,imfDescriptors)
   type(varying_string), intent(inout), dimension(:) :: imfNames,imfDescriptors

   imfNames      (myImfIndex)="Salpeter"
   imfDescriptors(myImfIndex)="Salpeter"
   return
 end subroutine Star_Formation_IMF_Register_Name_My_IMF
\end{verbatim}
The first routine should increment the {\tt imfAvailableCount} counter by 1 and keep a record of the resulting index---this will be the index by which the \gls{imf} is referred to. The second routine should store the name and descriptor of the \gls{imf} in the appropriate position in the supplied {\tt imfNames()} and {\tt imfDescriptors()} arrays. The ``name'' is the label used to identify the \gls{imf} in input parameters for example. The ``descriptor'' should be a label sufficient to uniquely identify the \gls{imf}, and is used, for example, in constructing file names when storing \gls{imf} related data. Often, the name and descriptor are identical. However, if the \gls{imf} has user-specififable parameters then those parameters should be encoded into the descriptor.

Each registered \gls{imf} should supply a set of functions as described in \S\ref{sec:IMF_functions}.

\subsubsection{Merger Tree Extra Output Tasks}

Extra outputs for merger trees (i.e. those which do not involve output of a fixed number of properties for every node---examples might be star formation histories for a subset of galaxies) can be added using the directive: {\tt mergerTreeExtraOutputTask}. The directive should give the name of the subroutine to be called to perform the task. A template for this task is:
\begin{verbatim}
  !# <mergerTreeExtraOutputTask>
  !#  <unitName>Galacticus_Extra_Output_Example</unitName>
  !# </mergerTreeExtraOutputTask>
  subroutine Galacticus_Extra_Output_Example(thisNode,iOutput,treeIndex,nodePassesFilter)
    implicit none
    type(treeNode),          intent(inout), pointer :: thisNode
    integer,                 intent(in)             :: iOutput
    integer(kind=kind_int8), intent(in)             :: treeIndex
    logical,                 intent(in)             :: nodePassesFilter
    .
    .
    .
    return
  end subroutine Galacticus_Extra_Output_Example
\end{verbatim}
The subroutine will be called for each node in each merger tree at each output, and should perform whatever extra output related to {\tt thisNode}. The index of the output and tree are provided as {\tt iOutput} and {\tt treeIndex} for reference, and may be used in organizing output. The {\tt nodePassesFilter} flag will be set to {\tt true} if {\tt thisNode} passed all active output filters (see \S\ref{sec:OutputFilters}). If it is {\tt false} then typically no output should occur (although other tasks may still be undertaken).

\subsubsection{Merger Tree Output Tasks}

Additional outputs for merger trees can be added using three directives: {\tt mergerTreeOutputPropertyCount}, {\tt mergerTreeOutputNames} and {\tt mergerTreeOutputTask}. Each directive should give the name of the subroutine to be called to perform the task and, additionally, a name for sorting (this should be the same for all three directives and ensures that output tasks are always called in the correct order). Templates for these tasks are:
\begin{verbatim}
  !# <mergerTreeOutputNames>
  !#  <unitName>Galacticus_Output_Tree_Example_Names</unitName>
  !#  <sortName>Galacticus_Output_Tree_Example</sortName>
  !# </mergerTreeOutputNames>
  subroutine Galacticus_Output_Tree_Example_Names(integerProperty,integerPropertyNames,integerPropertyComments,integerPropertyUnitsSI &
       &,doubleProperty,doublePropertyNames,doublePropertyComments,doublePropertyUnitsSI,time)
    implicit none
    double precision, intent(in)                  :: time
    integer,          intent(inout)               :: integerProperty,doubleProperty
    character(len=*), intent(inout), dimension(:) :: integerPropertyNames,integerPropertyComments,doublePropertyNames &
         &,doublePropertyComments
    double precision, intent(inout), dimension(:) :: integerPropertyUnitsSI,doublePropertyUnitsSI
    .
    .
    .
    return
  end subroutine Galacticus_Output_Tree_Example_Names

  !# <mergerTreeOutputPropertyCount>
  !#  <unitName>Galacticus_Output_Tree_Example_Property_Count</unitName>
  !#  <sortName>Galacticus_Output_Tree_Example</sortName>
  !# </mergerTreeOutputPropertyCount>
  subroutine Galacticus_Output_Tree_Example_Property_Count(integerPropertyCount,doublePropertyCount)
    implicit none
    integer, intent(inout) :: integerPropertyCount,doublePropertyCount
    .
    .
    .
    return
  end subroutine Galacticus_Output_Tree_Example_Property_Count

  !# <mergerTreeOutputTask>
  !#  <unitName>Galacticus_Output_Tree_Example</unitName>
  !#  <sortName>Galacticus_Output_Tree_Example</sortName>
  !# </mergerTreeOutputTask>
  subroutine Galacticus_Output_Tree_Example(thisNode,integerProperty,integerBufferCount,integerBuffer,doubleProperty&
       &,doubleBufferCount,doubleBuffer)
    implicit none
    type(treeNode),          intent(inout), pointer :: thisNode
    integer,                 intent(inout)          :: integerProperty,integerBufferCount,doubleProperty,doubleBufferCount
    integer(kind=kind_int8), intent(inout)          :: integerBuffer(:,:)
    double precision,        intent(inout)          :: doubleBuffer(:,:)
    .
    .
    .
    return
  end subroutine Galacticus_Output_Tree_Example
\end{verbatim}
The {\tt mergerTreeOutputPropertyCount} subroutine must simply increment {\tt integerPropertyCount} and {\tt doublePropertyCount} by the number of integer and double precision properties that will be output respectively. The {\tt mergerTreeOutputNames} subroutine must store the dataset names, comments and units in the SI system\footnote{For dimensionless quantities, the units may be set to zero. In such cases, the {\tt unitsInSI} attribute for the dataset will not be written to the \protect\glc\ output file.} for each integer and double precision property in the supplied arrays. The value of {\tt integerProperty} and {\tt doubleProperty} should be incremented by 1 before each property name/comment is set---these then supply the position within the input arrays in which to store the name. The {\tt mergerTreeOutputTask} subroutine must similarly place the desired property values for {\tt thisNode} into the supplied arrays. The value of {\tt integerProperty} and {\tt doubleProperty} should be incremented by 1 
before each property value is set. The value can then be stored in, for example, {\tt integerBuffer(integerBufferCount,integerProperty)}.

\subsubsection{Merger Tree Pre-Construction Tasks}\label{sec:MergerTreePreConstructionTask}

Additional tasks to be performed prior to the construction of each merger tree can be added using the {\tt mergerTreePreTreeConstructionTask} directive. For example, the tree timing task uses this directive as follows:
\begin{verbatim}
  !# <mergerTreePreTreeConstructionTask>
  !#   <unitName>Meta_Tree_Timing_Pre_Construction</unitName>
  !# </mergerTreePreTreeConstructionTask>
\end{verbatim}
Here, {\tt Meta\_Tree\_Timing\_Pre\_Construction} is the name of a subroutine which will be called to perform whatever tasks are required. The subroutine must have the following form:
\begin{verbatim}
   subroutine Merger_Tree_PreConstruction_Task()
    implicit none
    .
    .
    .
    return
  end subroutine Merger_Tree_PreConstruction_Task
\end{verbatim}
The subroutine will be called once for each tree, before the tree has been constructed.

\subsubsection{Merger Tree Post-Evolution Tasks}\label{sec:MergerTreePostEvolveTask}

Additional tasks to be performed after the evolution (and subsequent destruction) of each merger tree can be added using the {\tt mergerTreePostEvolveTasker} directive. For example, the tree timing task uses this directive as follows:
\begin{verbatim}
  !# <mergerTreePostEvolveTask>
  !#   <unitName>Meta_Tree_Timing_Post_Evolve</unitName>
  !# </mergerTreePostEvolveTask>
\end{verbatim}
Here, {\tt Meta\_Tree\_Timing\_Post\_Evolve} is the name of a subroutine which will be called to perform whatever tasks are required. The subroutine must have the following form:
\begin{verbatim}
   subroutine Merger_Tree_PostEvolution_Task()
    implicit none
    .
    .
    .
    return
  end subroutine Merger_Tree_PostEvolution_Task
\end{verbatim}
The subroutine will be called once for each tree, after the tree has been evolved and destroyed.

\subsubsection{Merger Tree Pre-Evolution Tasks}\label{sec:MergerTreePreEvolveTask}

Additional tasks to be performed on merger trees prior to their evolution can be added using the {\tt mergerTreePreEvolveTask} directive. For example, the mass accretion history task uses this directive as follows:
\begin{verbatim}
  !# <mergerTreePreEvolveTask>
  !#   <unitName>Merger_Tree_Mass_Accretion_History_Output</unitName>
  !# </mergerTreePreEvolveTask>
\end{verbatim}
Here, {\tt Merger\_Tree\_Mass\_Accretion\_History\_Output} is the name of a subroutine which will be called to perform whatever tasks are required. The subroutine must have the following form:
\begin{verbatim}
   subroutine Merger_Tree_PreEvolution_Task(thisTree)
    implicit none
    type(mergerTree), intent(in) :: thisTree
    .
    .
    .
    return
  end subroutine Merger_Tree_PreEvolution_Task
\end{verbatim}
where {\tt thisTree} is the tree to be processed. The function will be called once for each tree, prior to the tree being evolved. Note that {\tt thisTree} may link to other trees via its {\tt nextTree} pointer. The function may want to process each tree in this linked list.

\subsubsection{Merger Tree Initialization Tasks}

Additional tasks to be performed during merger tree initialization can be added using the {\tt mergerTreeInitializeTask} directive. The directive should contain a single argument, giving the name of a subroutine to be called to initialize the method. For example, the {\tt standard} basic component method uses this directive as follows:
\begin{verbatim}
  !# <satelliteMergerTask>
  !#  <unitName>Halo_Mass_Accretion_Rate</unitName>
  !# </satelliteMergerTask>
\end{verbatim}
Here, {\tt Halo\_Mass\_Accretion\_Rate} is the name of a subroutine which will be called to perform whatever tasks are required as a result of the merger. The subroutine must have the following form:
\begin{verbatim}
   subroutine Merger_Tree_Initialize_Task(thisNode)
    implicit none
    type(treeNode), pointer, intent(inout) :: thisNode
    .
    .
    .
    return
  end subroutine Merger_Tree_Initialize_Task
\end{verbatim}
where {\tt thisNode} is the node to be initialized. The subroutine will be called once for each node in the tree.

\subsubsection{Merger Tree Structure Output Tasks}

Additional outputs for merger tree structure output can be added using the {\tt mergerTreeStructureOutputTask}. The directive should give the name of the subroutine to be called to perform the task. The templates for this tasks is:
\begin{verbatim}
  !# <mergerTreeStructureOutputTask>
  !#  <unitName>Structure_Output_Task</unitName>
  !# </mergerTreeStructureOutputTask>
  subroutine Structure_Output_Task(baseNode,nodeProperty,treeGroup)
    implicit none
    type(treeNode),   intent(in),    pointer      :: baseNode
    double precision, intent(inout), dimension(:) :: nodeProperty
    type(hdf5Object), intent(inout)               :: treeGroup
    .
    .
    .
    return
  end subroutine Structure_Output_Task
\end{verbatim}
The subroutine must walk the merger tree beginning from the given {\tt baseNode} and store each property to output in the given {\tt nodeProperty} array. Once populated, this array can be written to the appropriate HDF5 group, given by {\tt treeGroup}, in the \glc\ output file.

\subsubsection{Node Dump}

The function {\tt Node\_Dump(thisNode)} writes out all properties of a node to the display. To have their properties listed, each component must register a task using:
\begin{verbatim}
 !# <nodeDumpTask>
 !#  <unitName>Node_Dump_Procedure</unitName>
 !# </nodeDumpTask>
\end{verbatim}
where {\tt Node\_Dump\_Procedure} is the name of a subroutine with the following template
\begin{verbatim}
 subroutine Node_Dump_Procedure(thisNode)
    type(treeNode),   intent(inout), pointer :: thisNode
    .
    .
    .
    return
 end subroutine Node_Dump_Procedure
\end{verbatim}
If the node contains an active component, this subroutine should display all relevant properties of the component. If not, it can display a short message indicating that fact.

\subsubsection{Output Filter Tasks}\label{sec:OutputFilters}\index{output!filtering}\index{filtering!output}

Extra filters for controlling which galaxies are output can be added using the directives {\tt mergerTreeOutputFilterInitialize} and {\tt mergerTreeOutputFilter}. Each directive should give the name of the function to be called to initialize or apply the filter respectively. A template for these tasks is:
\begin{verbatim}
  !# <mergerTreeOutputFilterInitialize>
  !#  <unitName>Galacticus_Merger_Tree_Output_Filter_Initialzie_Example</unitName>
  !# </mergerTreeOutputFilterInitialize>
  subroutine Galacticus_Merger_Tree_Output_Filter_Initialize_Example(filterNames)
    implicit none
    type(varying_string), intent(in), dimension(:) :: filterNames
    .
    .
    .
    return
  end subroutine Galacticus_Merger_Tree_Output_Filter_Initialize_Example

  !# <mergerTreeOutputFilter>
  !#  <unitName>Galacticus_Merger_Tree_Output_Filter_Example</unitName>
  !# </mergerTreeOutputFilter>
  logical function Galacticus_Merger_Tree_Output_Filter_Example(thisNode,doOutput)
    implicit none
    type(treeNode),       intent(inout), pointer :: thisNode
    logical,              intent(inout)          :: doOutput)
    .
    .
    .
    return
  end function Galacticus_Merger_Tree_Output_Filter_Example
\end{verbatim}
The initialization subroutine will be called prior to any use of the filter function. The {\tt filterNames} arrays contains a list of all filters which were requested to be applied. The function should check if its filter is listed in this array and activate itself if necessary. The filter function will be called for each node, {\tt thisNode}, which is being considered for output. If the filter is activate, it should determine whether {\tt thisNode} passes its criteria for output. If it does not, {\tt doOutput} should be set to false. If the output criteria are met, then {\tt doOutput} \emph{should not be changed} (as it may already have been set false by some other filter).

Currently available filters, selected using the {\tt [mergerTreeOutputFilters]} input parameter, are:
\begin{itemize}
 \item [{\tt lightcone}] Restricts output to those galaxies which fall within a specified lightcone geometry. See \S\ref{sec:OutputLightcone} for further details;
 \item [{\tt stellarMass}] Restricts output to those galaxies with a total stellar mass equal to or greater than {\tt [stellarMassFilterThreshold]};
 \item [{\tt luminosity}] Restricts output to those galaxies with a total absolute AB magnitude less than or equal to\footnote{That is, galaxies which are at least as luminous as the specified threshold.} {\tt [luminosityFilterAbsoluteMagnitudeThresholds]}. This input parameter should be an array, with one entry for each luminosity being computed. The filter will be applied only for those luminosities that are being output at the current time.
\end{itemize}

\subsubsection{Output Group Output Tasks}\index{output groups}\index{tasks!output}

Extra outputs for output groups (i.e. the groups which hold all merger tree data for a given output time) can be added using the directive: {\tt outputGroupOutputTask}. The directive should give the name of the subroutine to be called to perform the task. A template for this task is:
\begin{verbatim}
  !# <outputGroupOutputTask>
  !#  <unitName>Galacticus_Output_Group_Output_Example</unitName>
  !# </outputGroupOutputTask>
  subroutine Galacticus_Output_Group_Output_Example(outputGroup,time)
    implicit none
    type(hdf5Object), intent(inout) :: outputGroup
    double precision, intent(in)    :: time
    .
    .
    .
    return
  end subroutine Galacticus_Output_Group_Output_Example
\end{verbatim}
The subroutine will be called for each output group created, and should perform whatever extra output it requires. The {\tt outputGroup} object and the corresponding output {\tt time} are provided as input parameters.

\subsubsection{Pre-derivative Tasks}\index{pre-derivative task}\index{task, pre-derivative}

Additional methods for pre-derivative tasks (i.e. things that should be done just prior to the computation of derivatives or properties for a node) can be added using the {\tt preDerivativeTask} directive. The directive should contain a single argument, giving the name of a subroutine to be called to perform the task. For example, the standard hot halo component adds a task as follows:
\begin{verbatim}
  !# <preDerivativeTask>
  !#  <unitName>Tree_Node_Hot_Halo_Prederivative_Standard</unitName>
  !# </preDerivativeTask>
\end{verbatim}
Here, {\tt Tree\_Node\_Hot\_Halo\_Prederivative\_Standard} is the name of a subroutine which will be called to perform whatever tasks are required. The subroutine must have the following form:
\begin{verbatim}
   subroutine Prederivative_Task(thisNode)
    implicit none
    type(treeNode), pointer, intent(inout) :: thisNode
    .
    .
    .
    return
  end subroutine Prederivative_Task
\end{verbatim}
where {\tt thisNode} is the node for which derivatives will be computed. Tasks typically involve precomputing quantities that will be used in finding the derivatives.

\subsubsection{Radius Solver Tasks}\label{sec:radius_solver}

Galactic radii solver functions (see \S\ref{sec:galactic_radii_solvers}) need to be able to interact with the components of a tree node to
\begin{enumerate}
 \item Determine which components want a radius to be solved for;
 \item Get and set the properties of those components.
\end{enumerate}
The  {\tt radiusSolverPlausibility} and {\tt radiusSolverTask} directives facilitate this. A component which has a radius to be solved for should include directives of the form:
\begin{verbatim}
 !# <radiusSolverTask>
 !#  <unitName>Component_Radius_Solver_Plausibility</unitName>
 !# </radiusSolverTask>
\end{verbatim}
and
\begin{verbatim}
 !# <radiusSolverTask>
 !#  <unitName>Component_Radius_Solver</unitName>
 !# </radiusSolverTask>
\end{verbatim}
where {\tt Component\_Radius\_Solver\_Plausibility} is the name of a subroutine which will specify whether or not the component is physically plausible for radius solving (e.g. has non-negative mass) and should have the following form:
\begin{verbatim}
 subroutine Component_Radius_Solver_Plausibility(thisNode,galaxyIsPhysicallyPlausible)
    implicit none
    type(treeNode), pointer, intent(inout) :: thisNode
    logical,                 intent(inout) :: galaxyIsPhysicallyPlausible
    .
    .
    .
    return
 end subroutine Component_Radius_Solver_Plausibility
\end{verbatim}
which should set {\tt galaxyIsPhysicallyPlausible} to false if the component is not physically plausible, but should otherwise leave {\tt galaxyIsPhysicallyPlausible} unchanged. Additionally, {\tt Component\_Radius\_Solver} is the name of a subroutine which will supply the necessary information about the node, and which should have the following form:
\begin{verbatim}
 subroutine Component_Radius_Solver(thisNode,componentActive,specificAngularMomentum,Radius_Get,Radius_Set,Velocity_Get,Velocity_Set)
    implicit none
    type(treeNode),   pointer, intent(inout) :: thisNode
    logical,                   intent(out)   :: componentActive
    double precision,          intent(out)   :: specificAngularMomentum
    procedure(),      pointer, intent(out)   :: Radius_Get,Velocity_Get
    procedure(),      pointer, intent(out)   :: Radius_Set,Velocity_Set
    .
    .
    .
    return
 end subroutine Component_Radius_Solver
\end{verbatim}
When called, the subroutine should set {\tt componentActive} to indicate whether or not this nod contains an active component of the type. If it does, it should also set {\tt specificAngularMomentum} to reflect the specific angular momentum (in km s$^{-1}$ Mpc) of the component (at whatever point in its profile the radius is required) and should point the four procedure pointers to routines which get and set the radius and circular velocity properties of the component (which should have the standard form for component get and set methods). It is acceptable for the set procedures to point to dummy routines.

The galactic structure radii solver routines will use this information to determine (and set) the radius and circular velocity of the component. An advantage of this approach is that different radii solver methods can all use this same system, ensuring that just a single interface is needed in each component.

\subsubsection{Satellite Host Change Tasks}

Additional methods for satellite host change events (i.e. when a satellite node moves to a new host) can be added using the {\tt satelliteHostChangeTask} directive. The directive should contain a single argument, giving the name of a subroutine to be called to initialize the method. For example, the {\tt simple} satellite orbits components uses this directive as follows:
\begin{verbatim}
  !# <satelliteHostChangeTask>
  !#  <unitName>Satellite_Orbit_New_Host</unitName>
  !# </satelliteHostChangeTask>
\end{verbatim}
Here, {\tt Satellite\_Orbit\_New\_Host} is the name of a subroutine which will be called to perform whatever tasks are required as a result of the host change. The subroutine must have the following form:
\begin{verbatim}
   subroutine New_Host_Task(thisNode)
    implicit none
    type(treeNode), pointer, intent(inout) :: thisNode
    .
    .
    .
    return
  end subroutine New_Host_Task
\end{verbatim}
where {\tt thisNode} is the node which has changed host (the new host halo is {\tt thisNode\%parentNode}).

\subsubsection{Satellite Merger Tasks}

Additional methods for satellite merger tasks can be added using the {\tt satelliteMergerTask} directive. The directive should contain a single argument, giving the name of a subroutine to be called to initialize the method. For example, the {\tt simple} satellite orbits components uses this directive as follows:
\begin{verbatim}
  !# <satelliteMergerTask>
  !#  <unitName>Satellite_Merger_Task</unitName>
  !# </satelliteMergerTask>
\end{verbatim}
Here, {\tt Satellite\_Merger\_Task} is the name of a subroutine which will be called to perform whatever tasks are required as a result of the merger. The subroutine must have the following form:
\begin{verbatim}
   subroutine Satellite_Merger_Task(thisNode)
    implicit none
    type(treeNode), pointer, intent(inout) :: thisNode
    .
    .
    .
    return
  end subroutine Satellite_Merger_Task
\end{verbatim}
where {\tt thisNode} is the node about to merge with {\tt thisNode\%parentNode}.

\subsubsection{Star Formation History Tasks}\index{star formation history}\label{sec:StarFormationHistoryTasks}

Additional methods for star formation history tracking  can be added using the {\tt starFormationHistoriesMethod} directive. The directive should contain a single argument, giving the name of a subroutine to be called to initialize the method. For example, the {\tt metallicitySplit} method uses this directive as follows:
\begin{verbatim}
  !# <starFormationHistoriesMethod>
  !#  <unitName>Star_Formation_Histories_Metallicity_Split_Initialize</unitName>
  !# </starFormationHistoriesMethod>
\end{verbatim}
Here, {\tt Star\_Formation\_Histories\_Metallicity\_Split\_Initialize} is the name of a subroutine which will be called to initialize the method. The subroutine must have the following form:
\begin{verbatim}
  subroutine Method_Initialize( starFormationHistoriesMethod     &
       &                       ,Star_Formation_History_Create_Do &
       &                       ,Star_Formation_History_Scales_Do &
       &                       ,Star_Formation_History_Record_Do &
       &                       ,Star_Formation_History_Output_Do &
       &                      )
    implicit none
    type(varying_string),          intent(in)    ::  starFormationHistoriesMethod
    procedure(),          pointer, intent(inout) ::  Star_Formation_History_Create_Do &
       &                                            ,Star_Formation_History_Scales_Do &
       &                                            ,Star_Formation_History_Record_Do &
       &                                            ,Star_Formation_History_Output_Do
    
    if (starFormationHistoriesMethod == 'myMethod') then
       Star_Formation_History_Create_Do => My_Create
       Star_Formation_History_Scales_Do => My_Scales
       Star_Formation_History_Record_Do => My_Record
       Star_Formation_History_Output_Do => My_Output
    end if
    return
  end subroutine Method_Initialize
\end{verbatim}
where {\tt myMethod} is the name of this method as will be specified by the {\tt starFormationHistoriesMethod} input parameter. The procedure pointers must be set to point to subroutines which perform the functions described below. The initialization subroutine should perform any other tasks required to initialize the module (such as reading parameters etc.).

The {\tt Star\_Formation\_History\_Create\_Do} subroutine must have the form:
\begin{verbatim}
  subroutine My_Create(thisNode,thisHistory)
    implicit none
    type(treeNode), intent(inout), pointer :: thisNode
    type(history),   ntent(inout)          :: thisHistory
    return
  end subroutine My_Create
\end{verbatim}
and should return a history object in {\tt thisHistory} suitable for holding a star formation history for {\tt thisNode}.

The {\tt Star\_Formation\_History\_Scales\_Do} subroutine must have the form:
\begin{verbatim}
  subroutine My_Scales(thisHistory,stellarMass,stellarAbundances)
    implicit none
    double precision,          intent(in)    :: stellarMass
    type(abundancesStructure), intent(in)    :: stellarAbundances
    type(history),             intent(inout) :: thisHistory
    return
  end subroutine My_Scales
\end{verbatim}
and should set the ODE solver error tolerance scales in {\tt thisHistory}, using the provided information on {\tt stellarMass} and {\tt stellarAbundances} if required.

The {\tt Star\_Formation\_History\_Record\_Do}\index{star formation history!recording} subroutine must have the form:
\begin{verbatim}
  subroutine My_Record(thisNode,thisHistory,fuelAbundances,starFormationRate)
    implicit none
    type(treeNode),            intent(inout), pointer :: thisNode
    type(history),             intent(inout)          :: thisHistory
    type(abundancesStructure), intent(in)             :: fuelAbundances
    double precision,          intent(in)             :: starFormationRate
    return
  end subroutine My_Record
\end{verbatim}
and should record the contribution to the star formation history in {\tt thisHistory} for {\tt thisNode} given the current {\tt starFormationRate} and star formation {\tt fuelAbundances}. That is, the subroutine should adjust the rates in {\tt thisHistory} appropriately.

The {\tt Star\_Formation\_History\_Output\_Do}\index{star formation history!outputting} subroutine must have the form:
\begin{verbatim}
  subroutine My_Output(thisNode,thisHistory,iOutput,treeIndex,componentLabel)
    implicit none
    type(treeNode),          intent(inout), pointer :: thisNode
    type(history),           intent(inout)          :: thisHistory
    integer,                 intent(in)             :: iOutput
    integer(kind=kind_int8), intent(in)             :: treeIndex
    character(len=*),        intent(in)             :: componentLabel
    return
  end subroutine My_Output
\end{verbatim}
and should write the star formation history, {\tt thisHistory}, for {\tt thisNode} to the output file. The output number and tree index are provided as {\tt iOutput} and {\tt treeIndex} for reference, and {\tt componentLabel} provides a suitable label for the component to which the history belongs (and so should be used in the name of the datasets to which the history is written for example).

Conventionally, star formation histories are output as follows:
\begin{verbatim}
HDF5 "galacticus.hdf5" {
GROUP "starFormationHistories" {
   COMMENT "Star formation history data."
   GROUP "Output1" {
      COMMENT "Star formation histories for all trees at each out"
      GROUP "mergerTree1" {
         COMMENT "Star formation histories for each tree."
         DATASET "diskSFH<nodeID>" {
         COMMENT "Star formation history stellar masses of the disk "
            DATATYPE  H5T_IEEE_F64LE
            DATASPACE  SIMPLE { }
         }
         DATASET "diskTime<nodeID>" {
         COMMENT "Star formation history times of the disk component"
            DATATYPE  H5T_IEEE_F64LE
            DATASPACE  SIMPLE { }
         }
         DATASET "spheroidSFH<nodeID>" {
         COMMENT "Star formation history stellar masses of the spher"
            DATATYPE  H5T_IEEE_F64LE
            DATASPACE  SIMPLE { }
         }
         DATASET "spheroidTime<nodeID>" {
         COMMENT "Star formation history times of the spheroid compo"
            DATATYPE  H5T_IEEE_F64LE
            DATASPACE  SIMPLE { }
         }
      }
      GROUP "mergerTree2" {
      .
      .
      .
      }
   }
   GROUP "Output1" {
   .
   .
   .
   }
}
}
\end{verbatim}
where {\tt nodeID} is the index of the relevant node. The specifics of each dataset will depend on the selected star formation history method.

Currently defined star formation history methods are:
\begin{description}
 \item [\hyperlink{galacticus.output.merger_tree.star_formation.metallicity_split.F90:star_formation_histories_metallicity_split}{{\tt metallicitySplit}}] The star formation history is tabulated on a grid of time and metallicity. The binning in time is chosen such that bins are at most of size {\tt [starFormationHistoryTimeStep]} between the time at which each galaxy formed and the final output time, and at most of size {\tt [starFormationHistoryFineTimeStep]} in the period {\tt [starFormationHistoryFineTime]} prior to each output time (all times specified in Gyr). The allows fine binning of recent star formation just prior to each output. The metallicity binning is arranged logarithmically in metallicity with {\tt [starFormationHistoryMetallicityCount]} bins between {\tt [starFormationHistoryMetallicityMinimum]} and {\tt [starFormationHistoryMetallicityMaximum]} (specified in Solar units). Note that the metallicity associated with each bin is the minimum metallicity for that bin (the maximum being the 
metallicity value associated with the next bin, except for the final bin which extends to infinite metallicity). If {\tt [starFormationHistoryMetallicityCount]}$=0$ is set, then the star formation history is not split by metallicity (i.e. a single metallicity bin encompassing all metallicities from zero to infinity is used).  Output follows the conventional format, with 2D star formation history datasets to represent the history as a function of time and metallicity. An additional {\tt metallicities} dataset is added to the {\tt starFormationHistories} output group to record the metallicity binning as follows:
\begin{verbatim}
DATASET "metallicities" {
 COMMENT "Metallicities at which star formation histories are tabulated"
   DATATYPE  H5T_IEEE_F64LE
   DATASPACE  SIMPLE { ( [starFormationHistoryMetallicityCount] ) / ( [starFormationHistoryMetallicityCount] ) }
}
\end{verbatim}
\end{description}

\section{Subsystems}

This section describes some of the subsystems within \glc\ that support various physical entities or processes.

\subsection{Kepler Orbits}\label{sec:KeplerOrbits}

The {\tt keplerOrbit} object (provided by the \href{objects.kepler_orbits.F90:kepler_orbits_structure}{{\tt Kepler\_Orbits\_Structure}} module) stores the parameters of a single Keplerian orbit. It internally handles computation of additional/alternate orbital parameters once an orbit has been fully defined. Currently, the orientation of orbits (i.e. the unit vector normal to the orbital plane and the argument of periapsis) is not tracked. As such, orbits are fully defined by three parameters (in addition to the masses of the orbitting bodies). The following limitations presently apply to the {\tt keplerOrbit} object:
\begin{itemize}
 \item If an orbit is overdefined (i.e. if more than three parameters are set manually) no checking is performed to ensure that the parameters are consistent with a Keplerian orbit;
 \item Not all interconversions between parameters are supported\footnote{The {\tt keplerOrbit} object works by trying to convert to the combination radius, radial and tengential velocities. Once these are defined, all other parameters can be computed. However, for orbits defined in terms of other parameters, the {\tt keplerOrbit} object does not know how to convert from every such combination of parameters.}. If a conversion cannot be performed, an error message will be given. 
\end{itemize}
A {\tt keplerOrbit} object can be reset by calling the {\tt reset()} method, and its defined/undefined status can be tested with the {\tt isDefined()} method or asserted with the {\tt assertIsDefined()} method. The following orbital parameters are supported, each method returning the value of the parameter and a corresponding method suffixed with {\tt Set} can be used to set the parameter: {\tt radius}, {\tt velocityRadial}, {\tt velocityTengentail}, {\tt energy}, {\tt angularMomentum}, {\tt eccentricity}, {\tt semiMajorAxis}, {\tt radiusPericenter}, {\tt radiusApocenter}. Additionally, the masses of the orbitting bodies are provided by the {\tt hostMass()} and {\tt reducedMassSpecific()}$=M_{\rm host}/(M_{\rm host}+M_{\rm satellite})$ methods. Finally, the {\tt velocityScale()} method returns ${\rm G}M_{\rm host}/r$ where $r$ is the radius of the orbit.

\subsection{Chemicals}\label{sec:ChemicalSubsystem}

The chemicals subsystem provides both a interface to a database of known chemicals (allowing their physical properties to be queried) and a structure to store abundances/masses/etc. of the set of chemicals being tracked in \glc. The name ``chemicals'' is used to denote any chemical species that might be involved in reactions, including molecules, atoms, atomic and molecular ions and electrons.

\subsubsection{Chemical Database}

The file {\tt data/Chemical\_Database.cml} contains a database of chemicals that can currently be used by \glc. It uses a simplified version of the \href{http://www.xml-cml.org}{Chemical Markup Language} to describe chemicals. An excerpt from the database is shown below:
\begin{verbatim}
 <list>
  <chemical>
   <id>MolecularHydrogenAnion</id>
   <formalCharge>-1</formalCharge>
   <atomArray>
     <atom>
      <id>1</id>
      <elementType>H</elementType>
     </atom>
     <atom>
      <id>2</id>
      <elementType>H</elementType>
     </atom>
   </atomArray>
   <bondArray>
     <bond>
      <atomRefs2>1 2</atomRefs2>
      <order>1</order>
     </bond>
   </bondArray>
  </chemical>
  .
  .
  .
 </list>
\end{verbatim}
The database contains a {\tt list} of chemicals, each contained within a {\tt chemical} element. The {\tt id} element provides a label for the chemical (usually a descriptive name with no white space). The {\tt formalCharge} element gives the charge of the chemical in units of the elementary charge. The chemical is then describe by a list of atoms and bonds inside {\tt atomArray} and {\tt bondArray} elements respectively. The {\tt atomArray} can contain any number of {\tt atom} elements, which should describe each atom in the chemical giving it a unique {\tt id} number and an {\tt elementType}, which is the short one or two letter label for the element (e.g. H, Ni, etc.). The {\tt bondArray} should contain a {\tt bond} entry for each atomic bond, which itself contains a {\tt atomRefs2} element giving the IDs of the two atoms participating in the bond and an {\tt order} element which gives the order of the bond (e.g. ``1'' for a single bond).

\subsubsection{Chemical Structure}

Within \glc\ a chemical is represeted using the {\tt chemicalStructure} type which is provided by the {\tt Chemical\_Structures} module. A {\tt chemicalStructure} object can be assigned a particular chemical by retrieving that chemical from the database using:
\begin{verbatim}
 call myChemical%retrieve("chemicalID")
\end{verbatim}
where {\tt chemicalID} is the ID of the chemical in the databse. Any chemical can be exported to a CML file using
\begin{verbatim}
 call myChemical%export(fileName)
\end{verbatim}
where {\tt fileName} gives the name of the file to which to export.

Once assigned a chemical, basic properties such as mass and charge (in atomic units) can be accessed using {\tt myChemical\%mass} and {\tt myChemical\%charge} respectively. The mass is computed from the known atomic masses of the constituent atoms of the chemical.

\subsubsection{Chemical Abundances}

Within \glc\ a set of abundances (or masses, or densities\ldots) for all chemicals being tracked, as specified by the {\tt [chemicalsToTrack]} input parameter, is stored within a {\tt chemicalAbundancesStructure} type, as provided by the {\tt Chemical\_Abundances\_Structure} module. The structure provides interfaces for setting and retrieving the abundance of a given chemical species, to pack/unpack all chemicals to/from an array, to convert from mass-weighted to number-weigted quantities and to multiplty and divide the chemicals abundances by a given amount. Additionally, the {\tt Chemical\_Abundances\_Structure} module provides functions which provide a count of the number of chemicals tracked, to look up the index of a chemical array represetation from its name, and to retrieve the name of a given chemical.

\subsection{Radiation}\label{sec:RadiationSubsystem}

This subsystem handles radiation fields, providing convenient means to communicate radiation fields from one part of the \glc\ code to another. A radiation object can hold multiple different types of radiation field (e.g. it could contain both the cosmic microwave background and an interstellar radiation field localized to a specific galaxy).

\subsubsection{Radiation Structure}

Within \glc\ radiation fields are represented by the {\tt radiationStructure} type which is provided by the {\tt Radiation\_Structures} module. A {\tt radiationStructure} object must first be defined using:
\begin{verbatim}
 call myRadiation%define([radiationType1,radiationType2])
\end{verbatim}
where the list of {\tt radiationType}s specifies what radiation components will be present in this radiation object. Currently defined radiation types are:
\begin{description}
 \item[{\tt CMB}] The cosmic microwave background;
 \item[{\tt Null}] A null (zero radiation) component.
\end{description}
For example,
\begin{verbatim}
 call myRadiation%define([radiationTypeCMB])
\end{verbatim}
will define the {\tt myRadiation} object to contain just the cosmic microwave background.

Once defined, the specific radiation field can be set using:
\begin{verbatim}
 call myRadiation%set(thisNode)
\end{verbatim}
This will cause all components to set their radiation fields using (if necessary) the properties of {\tt thisNode}. Radiation objects can be queried using the following methods:
\begin{description}
 \item[{\tt temperature(radiationTypes)}] Returns the temperature (in Kelvin) of the radiation object. The optional {\tt radiationTypes} array specifies which radiation types are to be queried.
 \item[{\tt flux(wavelength,radiationTypes)}] Returns the flux (in ergs cm$^2$ s$^{-1}$ Hz$^{-1}$ ster$^{-1}$) of the radiation object at the given {\tt wavelength} (specified in units of \AA). The optional {\tt radiationTypes} array specifies which radiation types are to be queried.
\end{description}

\subsection{Mass Distributions}\label{sec:MassDistributions}

The {\tt massDistribution} class, provided by the \hyperlink{objects.mass_distributions.F90:mass_distributions}{\tt Mass\_Distributions} module provides an object describing a distribution of mass in space, together with methods to query for the density, enclosed mass etc. of that mass distribution. Mass distribution objects make use of the {\tt coordinates} subsystem (see \S\ref{sec:Coordinates}) for specifying positions within mass distributions.

The base class provides the following methods:
\begin{description}
\item [{\tt symmetry}] Returns one of the following labels to indicate the symmetry of the mass distribution:
 \begin{description}
  \item [{\tt massDistributionSymmetryNone}] Indicates that the mass distribution has no particular symmetry;
  \item [{\tt massDistributionSymmetryCylindrical}] Indicates that the mass distribution has cylindrical symmetry (conventionally around the $z$-axis;
  \item [{\tt massDistributionSymmetrySpherical}] Indicates that the mass distribution has spherical symmetry.
 \end{description}
\item [{\tt isDimensionless}] Returns a Boolean indicating whether this is a dimensionless or dimensionful mass distribution;
\item [{\tt density}] Returns the density of the mass distribution at the supplied {\tt coordinates};
\item [{\tt densityRadialMoment}] Returns the $n^{\rm th}$ moment of the integral of the density over radius, $\int_0^\infty \rho({\bf x}) |x|^n {\rm d} {\bf x}$; 
\item [{\tt massEnclosedBySphere}] Returns the mass enclosed by a sphere of given {\tt radius} (centered on the origin);
\item [{\tt potential}] Returns the gravitational potential at the specified {\tt coordinates}.
\end{description}

The {\tt massDistributionSpherical} class extends the {\tt massDistribution} base class with an additional method:
\begin{description}
 \item [{\tt halfMassRadius}] Returns the radius enclosing half of the mass of the density distribution.
\end{description}

Mass distributions are created using:
\begin{verbatim}
 myMassDistribution => Mass_Distribution_Create(type)
\end{verbatim}
where {\tt type} is the name of the mass distribution (see below). After creation, the parameters of the profile must usually be initialized using:
\begin{verbatim}
 call myMassDistribution%initialize(....)
\end{verbatim}
Arguments for initialization vary for each mass distribution (see below).

Currently implemented mass distributions include:
\begin{description}
 \item [{\tt nfw}] An \gls{nfw} \citep{navarro_universal_1997} density profile. Initialization is by
\begin{verbatim}
 call myNfwProfile%initialize(scaleLength,concentration,densityNormalization,mass, &
     &                         virialRadius,isDimensionless)
\end{verbatim}
All arguments are optional, but the combination given must be sufficient to allow the scale length and density normalization to be determined. The profile will be assumed to be dimensionful unless the {\tt isDimensionless} argument specifies otherwise.
 \item [{\tt betaProfile}] A $\beta$-profile, $\rho(r)=\rho_0/[1+(r/r_{\rm core})^2]^{3\beta/2}$. Initialization is by
\begin{verbatim}
 call myBetaProfile%initialize(beta,coreRadius,densityNormalization,mass, &
     &                         outerRadius,isDimensionless)
\end{verbatim}
 {\tt beta}$=\beta$ and {\tt coreRadius}$=r_{\rm core}$ must always be specified. The density normalization must be specified either by the {\tt densityNormalization} argument, or by supplying both {\tt mass} and {\tt outerRadius}. The profile will be assumed to be dimensionful unless the {\tt isDimensionless} argument specifies otherwise.
 \item [{\tt hernquist}] The Hernquist \citep{hernquist_analytical_1990} profile. Initialization is by
\begin{verbatim}
 call myHernquistProfile%initialize(scaleLength,densityNormalization,mass, &
     &                               isDimensionless)
\end{verbatim}
All arguments are optional, but the combination given must be sufficient to allow the scale length and density normalization to be determined unless the profile is dimensionless (in which case scale length and total mass are set to unity). The profile will be assumed to be dimensionful unless the {\tt isDimensionless} argument specifies otherwise.
 \item [{\tt sersic}] The S\'ersic \citep{sersic_influence_1963} profile. Initialization is by
\begin{verbatim}
 call mySersicProfile%initialize(index,halfMassRadius,densityNormalization,mass, &
     &                             isDimensionless)
\end{verbatim}
The S\'ersic {\tt index} must be specified. All other arguments are optional, but the combination given must be sufficient to allow the scale length and density normalization to be determined unless the profile is dimensionless (in which case scale length and total mass are set to unity). The profile will be assumed to be dimensionful unless the {\tt isDimensionless} argument specifies otherwise.
\end{description}

\subsection{Coordinates}\label{sec:Coordinates}

The {\tt coordinate} class, provided by the \hyperlink{objects.coordinates.F90:coordinates}{\tt Coordinates} module provides an object describing a position in three-dimensional space. Each extension of this class (currently, {\tt coordinateCartesian}, {\tt coordinateCylindrical}, and {\tt coordinateSpherical}) supply methods to convert to and from Cartesian coordinates. The assignment operator ({\tt =}) is overloaded such that coordinate objects of any class can be assigned to any other class and conversion to the appropriate coordinate system will happen automatically. A function accepting a {\tt class(coordinate)} object can therefore convert it to, for example, spherical coordinates simply using
\begin{verbatim}
 class(coordinate         ), intent(in) :: coordinates
 type (coordinateSpherical)             :: coordinatesSpherical
 coordinatesSpherical=coordinates
\end{verbatim}
and thereby allow a position to be passed to it in any coordinate system.

Each extension of the base class also provides methods to get and set the values of each component of the relevant coordinate system (see \S\ref{sec:AutoMethodsCoordinate} for complete details).


\chapter{Auxilliary Methods}

\section{Conditional Stellar Mass Function}

Empirical conditional stellar mass functions are used by \glc\ in calculations of halo mass function sampling. \glc\ implements the following calculations of tree processing times, which can be selected via the {\normalfont \ttfamily [conditionalStellarMassFunctionMethod]} input parameter.

\subsection{Behroozi (2010) Method}

Currently the only option, and selected using {\normalfont \ttfamily [conditionalStellarMassFunctionMethod]}$=${\normalfont \ttfamily Behroozi2010}, this method adopts the fitting function of \cite{behroozi_comprehensive_2010}:
\begin{equation}
 \langle N_{\mathrm c}(M_\star|M)\rangle \equiv \int_{M_\star}^\infty \phi_{\mathrm c}(M_\star^\prime) \d \ln M_\star^\prime = {1 \over 2} \left[ 1 - \hbox{erf}\left( {\log_{10}M_\star - \log_{10} f_{\mathrm SHMR}(M) \over \sqrt{2}\sigma_{\log M_\star}} \right) \right].
\end{equation}
Here, the function $f_{\mathrm SHMR}(M)$ is the solution of
\begin{equation}
 \log_{10}M = \log_{10}M_1 + \beta \log_{10}\left({M_\star \over M_{\star,0}}\right) + {(M_\star/M_{\star,0})^\delta \over 1 + (M_\star/M_{\star,0})^{-\gamma}} - {1/2}.
\end{equation}
For satellites,
\begin{equation}
 \langle N_{\mathrm s}(M_\star|M)\rangle \equiv \int_{M_\star}^\infty \phi_{\mathrm s}(M_\star^\prime) \d \ln M_\star^\prime =  \langle N_{\mathrm c}(M_\star|M)\rangle \left({f^{-1}_{\mathrm SHMR}(M_\star) \over M_{\mathrm sat}}\right)^{\alpha_{\mathrm sat}} \exp\left(- {M_{\mathrm cut} \over f^{-1}_{\mathrm SHMR}(M_\star)} \right),
\end{equation}
where
\begin{equation}
 {M_{\mathrm sat} \over 10^{12} M_\odot} = B_{\mathrm sat} \left({f^{-1}_{\mathrm SHMR}(M_\star) \over 10^{12} M_\odot}\right)^{\beta_{\mathrm sat}},
\end{equation}
and
\begin{equation}
 {M_{\mathrm cut} \over 10^{12} M_\odot} = B_{\mathrm cut} \left({f^{-1}_{\mathrm SHMR}(M_\star) \over 10^{12} M_\odot}\right)^{\beta_{\mathrm cut}}.
\end{equation}
By default, parameter values are taken from the fit of \cite{leauthaud_new_2011}, specifically their {\normalfont \ttfamily SIG\_MOD1} method for their $z_1$ sample. These default values, and the \glc\ input parameters which can be used to adjust them are shown in Table~\ref{table:Behroozi2010FitParameters}. This method assumes that $P_{\mathrm s}(N|M_\star,M;\delta \ln M_\star)$ is a Poisson distribution while $P_{\mathrm c}(N|M_\star,M;\delta \ln M_\star)$ has a Bernoulli distribution, with each distribution's free parameter fixed by requiring
\begin{equation}
 \phi(M_\star;M) \delta \ln M_\star = \sum_{N=0}^\infty N P(N|M_\star,M;\delta \ln M_\star)
\end{equation}

\begin{table}
\caption{Parameters of the \cite{behroozi_comprehensive_2010} conditional stellar mass function model, along with their default values and the corresponding \glc\ input parameters.}
\label{table:Behroozi2010FitParameters}
\begin{center}
\begin{tabular}{lr@{.}ll}
\hline
{\normalfont \bfseries Parameter} & \multicolumn{2}{c}{{\normalfont \bfseries Default}} & {\normalfont \bfseries \glc\ name} \\
\hline
$\alpha_{\mathrm sat}$& 1&0& {\normalfont \ttfamily [conditionalStellarMassFunctionBehrooziAlphaSatellite]} \\
$\log_{10} M_1$& 12&520& {\normalfont \ttfamily [conditionalStellarMassFunctionBehrooziLog10M1]} \\
$\log_{10} M_{\star,0}$& 10&916& {\normalfont \ttfamily [conditionalStellarMassFunctionBehrooziLog10Mstar0]} \\
$\beta$& 0&457& {\normalfont \ttfamily [conditionalStellarMassFunctionBehrooziBeta]} \\
$\delta$& 0&5666& {\normalfont \ttfamily [conditionalStellarMassFunctionBehrooziDelta]} \\
$\gamma$& 1&53& {\normalfont \ttfamily [conditionalStellarMassFunctionBehrooziGamma]} \\
$\sigma_{\log M_\star}$& 0&206& {\normalfont \ttfamily [conditionalStellarMassFunctionBehrooziSigmaLogMstar]} \\
$B_{\mathrm cut}$& 1&47& {\normalfont \ttfamily [conditionalStellarMassFunctionBehrooziBCut]} \\
$B_{\mathrm sat}$& 10&62& {\normalfont \ttfamily [conditionalStellarMassFunctionBehrooziBSatellite]} \\
$\beta_{\mathrm cut}$& $-$0&13& {\normalfont \ttfamily [conditionalStellarMassFunctionBehrooziBetaCut]} \\
$\beta_{\mathrm sat}$& 0&859& {\normalfont \ttfamily [conditionalStellarMassFunctionBehrooziBetaCut]} \\
\hline
\end{tabular}
\end{center}
\end{table}

\section{Tree Timing}\label{sec:TreeTimingFile}

Estimates of the time taken to process a merger tree are used in some halo sampling rate functions and may in future be used in load balancing algorithms. \glc\ implements the following calculations of tree processing times, which can be selected via the {\normalfont \ttfamily [timePerTreeMethod]} input parameter.

\subsection{File Method}

Currently the only option, and selected using {\normalfont \ttfamily [timePerTreeMethod]}$=${\normalfont \ttfamily file}, this method assumes that the time taken to process a tree is given by
\begin{equation}
 \log_{10} [ \tau_{\mathrm tree}(M)] = \sum_{i=0}^2 C_i (\log_{10} M)^i,
\end{equation}
where $M$ is the root mass of the tree and the coefficients $C_i$ are read from a file, the name of which is specified via the {\normalfont \ttfamily [timePerTreeFitFileName]} parameter. This file should be an XML document with the structure:
\begin{verbatim}
<timing>
 <fit>
   <coefficient>-0.73</coefficient>
   <coefficient>-0.20</coefficient>
   <coefficient>0.03</coefficient>
 </fit>
</timing>
\end{verbatim}
where the array of coefficients give the values $C_0$, $C_1$ and $C_2$.

Note that, if \glc\ is run with {\normalfont \ttfamily [metaCollectTimingData]}$=${\normalfont \ttfamily true}, then it will output measures of tree processing time to the output file (see \S\ref{sec:MetaTreeTimingProfiler}). The analysis script {\normalfont \ttfamily scripts/analysis/treeTiming.pl} can be used to extract tree timing data from such an output file and output fitting coefficients in the above format. It is used as follows:
\begin{verbatim}
 treeTiming.pl <modelFile> [options.....]
\end{verbatim}
where {\normalfont \ttfamily <modelFile>} is the name of the \glc\ output file to analyze. The following options can be specified:
\begin{description}
 \item [{\normalfont \ttfamily accumulate}] If present, this argument will cause new timing data from the {\normalfont \ttfamily <modelFile>} is be accumulated with any timing data already present in the output file (which must be specified in this case). The fit is recomputed from the totallity of the accumulated data;
 \item [{\normalfont \ttfamily outputFile}] If present, the timing data for individual trees together with the fitting coefficiencts will be output to the specified file;
 \item [{\normalfont \ttfamily maxPoints}] When accumulating trees to the output file, this paramter, if present, will limit the number of trees stored in the file to the given value. The oldest trees added to the file will be dropped first;
 \item [{\normalfont \ttfamily plotFile}] If present, a plot of the tree timing as a function of halo mass, together with the fitting function, will be output to the specified file.
\end{description}
Note that the output file will contain both the fitting coefficients in the format described above and, additionally, a list of tree root masses and processing times (necessary if you later want to append trees from another run to this file).


\chapter{Source Code Documentation}

\noindent{\normalfont \bfseries file:} \hypertarget{work/build/objects.nodes.components.Inc}{{\normalfont \ttfamily work/build/objects.nodes.components.Inc}}

\begin{supertabular}{lp{70mm}p{70mm}}
\emph{Description:} & \multicolumn{2}{l}{
\begin{minipage}[t]{140mm}
 Auto-generated file describing the hierarchy of node and component objects.
\end{minipage}
}\\
\emph{Code lines:} & \multicolumn{2}{l}{N/A} \\
\emph{Contained by:} & \multicolumn{2}{l}{file \hyperlink{accretion.Bondi_Hoyle_Lyttleton.F90}{{\normalfont \ttfamily accretion.Bondi\_Hoyle\_Lyttleton.F90}}} \\ 
\emph{Used by:}  & \RaggedRight file \hyperlink{objects.nodes.F90}{{\normalfont \ttfamily objects.nodes.F90}} & \\
 & & \\
\end{supertabular}
\\


\input{source_documentation}



\part{Contributions and Acknowledgements}

\include{contributions}

\chapter{Acknowledgements}

In addition to the tools and libraries required to compile and run \glc, development of \glc\ has benefitted from extensive use of the following: \href{http://www.gnu.org/software/octave/}{{\sc GNU Octave}}, \href{http://maxima.sourceforge.net/}{{\sc Maxima}}, \href{http://edu.kde.org/cantor/}{{\sc Cantor}}, \href{http://kile.sourceforge.net/}{{\sc Kile}}, \href{http://www.gnu.org/software/emacs/}{{\sc Emacs}} and \href{http://valgrind.org/}{{\sc Valgrind}}. We are grateful to the members of the {\sc GNU Fortran} mailing list for invaluable discussions and fixes for compiler problems. We thank John Burkardt for making available the {\sc Bivar} algorithm for performing interpolation on data irregularly spaced on a 2D plane, Dima Verner for making available codes to compute various atomic data for astrophysics, and Warren Perger for making available his {\sc pFq} code for computing generalized hypergeometric functions. Chris Power provided instructions for installing \glc\ under Mac OS X. The community of \glc\ users\footnote{In particular, Christoph Behrens, Jianling Gan, Markus Haider, Harald H\"oller, Eve Kovacs, Ting-Wen Lan, Adrian Pope, Luiz Felippe Rodrigues, Sergio Sanes, Martin White and Liyan Xu.} have provided invaluable feedback and bug reports. Gian Luigi Granato and Laura Silva kindly provided modifications to their \href{http://adlibitum.oat.ts.astro.it/silva/grasil/grasil.html}{\sc Grasil} code to allow it to read \glc\ outputs.


\appendix

\part{Appendices}

\chapter{Merger Tree Files}

\glc\ uses a standardized HDF5 file structure for merger tree data which allows for portability, and extensibility. The definition of the file format can be found \href{https://github.com/galacticusorg/galacticus/wiki/Merger-Tree-File-Format}{here}.

\section{Merger Tree Builder}\index{merger trees!building}

\glc\ contains software which builds merger tree files in the format described in \S\ref{sec:MergerTreeFormatDescription} from merger tree descriptions in other formats, such as ASCII output from an SQL database. The merger tree building engine can be found in \hyperlink{objects.merger_tree_data.F90}{{\tt source/objects.merger\_tree\_data.F90}}. Examples of how this engine is used can be found in \hyperlink{Millennium_Merger_Tree_File_Maker.F90}{{\tt source/Millennium\_Merger\_Tree\_File\_Maker.F90}} and \hyperlink{merger_trees.file_maker.Millennium.F90}{{\tt source/merger\_trees.file\_maker.Millennium.F90}} which are designed to work with the {\tt scripts/aux/Millennium\_Trees\_Grab.pl} script to convert data extracted from the \href{http://www.g-vo.org/MyMillennium3/}{Millennium Simulation database} into a format that \glc\ can read, and in \hyperlink{Simple_Merger_Tree_File_Maker.F90}{{\tt source/Simple\_Merger\_Tree\_File\_Maker.F90}} and \hyperlink{merger_trees.file_maker.simple.F90}{{\tt source/merger\_trees.file\_maker.simple.F90}} which are designed to work with ASCII file representations of mergers trees that contain just the mass, redshift and descendent of each node.

The basic process for building a merger tree file is to inform the engine of the data file to read and where specific information is located within that file. The data can then be processed and, finally, output in the required format. Specific interfaces that can be used are described below. Many of these interfaces work on an object {\tt mergerTrees} of {\tt mergerTreeData} type. This object stores all information on the merger trees while they are being internally processed.

\noindent \emph{Setting property locations:} Before reading data from a file it is necessary to inform the tree builder engine of which column in the file corresponds to which property. This is done with repeated calls to {\tt setProperty}, one for each column to read, as follows:
\begin{verbatim}
call mergerTrees%setProperty(propertyType,columnIndex)
\end{verbatim}
where {\tt columnIndex} is the number of the column (counting from 1) which contains the property specified by {\tt propertyType}. {\tt propertyType} can take one of the following values:
\begin{description}
 \item [{\tt propertyTypeTreeIndex}] A unique ID number for the tree to which this node belongs;
 \item [{\tt propertyTypeNodeIndex}] An ID (unique within the tree) for this node;
 \item [{\tt propertyTypeDescendentIndex}] The ID of the node's descendent node;
 \item [{\tt propertyTypeHostIndex}] The ID of the larger halo in which this node is hosted (equal to the node's own ID if the node is self-hosting);
 \item [{\tt propertyTypeRedshift}] The redshift of the node;
 \item [{\tt propertyTypeNodeMass}] The mass of the node;
 \item [{\tt propertyTypeParticleCount}] The number of particles in the node;
 \item [{\tt propertyTypePositionX}] The $x$-position of the node (if present, both $y$ and $z$ components must also be present);
 \item [{\tt propertyTypePositionY}] The $y$-position of the node (if present, both $x$ and $z$ components must also be present);
 \item [{\tt propertyTypePositionZ}] The $z$-position of the node (if present, both $x$ and $y$ components must also be present);
 \item [{\tt propertyTypeVelocityX}] The $x$-velocity of the node (if present, both $y$ and $z$ components must also be present);
 \item [{\tt propertyTypeVelocityY}] The $y$-velocity of the node (if present, both $x$ and $z$ components must also be present);
 \item [{\tt propertyTypeVelocityZ}] The $z$-velocity of the node (if present, both $x$ and $y$ components must also be present);
 \item [{\tt propertyTypeSpinX}] The $x$ component of the node's spin parameter (if present, both $y$ and $z$ components must also be present; cannot be present if spin magnitude is given);
 \item [{\tt propertyTypeSpinY}] The $y$ component of the node's spin parameter (if present, both $x$ and $z$ components must also be present; cannot be present if spin magnitude is given);
 \item [{\tt propertyTypeSpinZ}] The $z$ component of the node's spin parameter (if present, both $x$ and $y$ components must also be present; cannot be present if spin magnitude is given);
 \item [{\tt propertyTypeSpin}] The magnitude of the node's spin parameter (cannot be present if spin vector components are given);          
 \item [{\tt propertyTypeAngularMomentumX}] The $x$-component of the node's angular momentum (if present, both $y$ and $z$ components must also be present; cannot be present if angular momentum magnitude is given);
 \item [{\tt propertyTypeAngularMomentumY}] The $y$-component of the node's angular momentum (if present, both $x$ and $z$ components must also be present; cannot be present if angular momentum magnitude is given);
 \item [{\tt propertyTypeAngularMomentumZ}] The $z$-component of the node's angular momentum (if present, both $x$ and $y$ components must also be present; cannot be present if angular momentum magnitude is given);
 \item [{\tt propertyTypeAngularMomentum}] The magnitude of the node's angular momentum (cannot be present if angular momentum vector components are given);
 \item [{\tt propertyTypeHalfMassRadius}] The half-mass radius of the node;
 \item [{\tt propertyTypeMostBoundParticleIndex}] The index of the most bound particle in this node.
\end{description}
Not all properties must be specified---any required properties that are not specified will result in an error. Likewise, some properties, if present, require that other properties also be present. For example, if any of the position properties is given then all three positions are required.\\

\noindent \emph{Reading ASCII data:} Once property columns have been specified, data from an ASCII file with one node per line can be read as follows:
\begin{verbatim}
call mergerTrees%readASCII(nodesFile,lineNumberStart,lineNumberStop,separator=",")
\end{verbatim}
where {\tt nodesFile} is the name of the file to read. The optional {\tt lineNumberStart} and {\tt lineNumberEnd} arguments give the first and last lines of the file to read, while the optional {\tt separator} argument specifies the character used to separate columns (white space is assumed by default).\\

\noindent \emph{Setting particle property locations:} If particle information is to be stored in the file, the locations of particle properties within the input file must be specified with repeated calls to {\tt setParticleProperty} as follows:
\begin{verbatim}
call mergerTrees%setParticleProperty(propertyType,columnIndex)
\end{verbatim}
where {\tt columnIndex} is the number of the column (counting from 1) which contains the property specified by {\tt propertyType}. {\tt propertyType} can take one of the following values:
\begin{description}
 \item [{\tt propertyTypeParticleIndex}] A unique ID for the particle;
 \item [{\tt propertyTypeRedshift}] The redshift of the particle;
 \item [{\tt propertyTypeNodeMass}] The mass of the particle;
 \item [{\tt propertyTypeParticleCount}] The number of particles in the particle;
 \item [{\tt propertyTypePositionX}] The $x$-position of the particle (if present, both $y$ and $z$ components must also be present);
 \item [{\tt propertyTypePositionY}] The $y$-position of the particle (if present, both $x$ and $z$ components must also be present);
 \item [{\tt propertyTypePositionZ}] The $z$-position of the particle (if present, both $x$ and $y$ components must also be present);
 \item [{\tt propertyTypeVelocityX}] The $x$-velocity of the particle (if present, both $y$ and $z$ components must also be present);
 \item [{\tt propertyTypeVelocityY}] The $y$-velocity of the particle (if present, both $x$ and $z$ components must also be present);
 \item [{\tt propertyTypeVelocityZ}] The $z$-velocity of the particle (if present, both $x$ and $y$ components must also be present).
\end{description}

\noindent \emph{Reading ASCII particle data:} Once property columns have been specified, particle data from an ASCII file with one particle per line can be read as follows:
\begin{verbatim}
call mergerTrees%readParticlesASCII(particlesFile,lineNumberStart,lineNumberStop,separator=",")
\end{verbatim}
where {\tt particlesFile} is the name of the file to read. The optional {\tt lineNumberStart} and {\tt lineNumberEnd} arguments give the first and last lines of the file to read, while the optional {\tt separator} argument specifies the character used to separate columns (white space is assumed by default).\\

\noindent \emph{Setting particle mass:} The particle mass, {\tt particleMass}, can be specified using:
\begin{verbatim}
call mergerTrees%setParticleMass(particleMass)
\end{verbatim}

\noindent \emph{Specifying tree self-containment:} Whether or not trees are self-contained can be specified using:
\begin{verbatim}
call mergerTrees%setSelfContained([.true.|.false.])
\end{verbatim}

\noindent \emph{Specifying Hubble flow inclusion:} Whether or not velocities include the Hubble flow can be speified using:
\begin{verbatim}
call mergerTrees%setIncludesHubbleFlow([.true.|.false.])
\end{verbatim}

\noindent \emph{Specifying subhalo mass inclusion:} Whether or not halo masses include the masses of any subhalos can be specified using:
\begin{verbatim}
call mergerTrees%setIncludesSubhaloMasses([.true.|.false.])
\end{verbatim}

\noindent \emph{Specifying reference creation:} Whether or not HDF5 reference to individual merger trees within the {\tt haloTrees} datasets should be made can be specified using:
\begin{verbatim}
call mergerTrees%makeReferences([.true.|.false.])
\end{verbatim}

\noindent \emph{Specifying units:} The units used in the files can be specified with repeated calls to {\tt setUnits} as follows:
\begin{verbatim}
call mergerTrees%setUnits(unitsType,unitsInSI,hubbleExponent,scaleFactorExponent)
\end{verbatim}
where {\tt unitsType} is one of:
\begin{description}
 \item [{\tt unitsMass}] Units of mass;
 \item [{\tt unitsLength}] Units of length;
 \item [{\tt unitsTime}] Units of time;
 \item [{\tt unitsVelocity}] Units of velocity;
\end{description}
{\tt unitsInSI} gives the units in the SI system, {\tt hubbleExponent} specifies the power to which $h$ appears in the units and {\tt scaleFactorExponent} specifies the number of powers of the expansion factor by which the quantity should be multiplied to place it into physical units.\\

\noindent \emph{Adding metadata:} Meta-data can be added to the file by making repeated calls to {\tt addMetadata} as follows:
\begin{verbatim}
call mergerTrees%addMetadata(metaDataType,label,value)
\end{verbatim}
where {\tt metaDataType} is one of:
\begin{description}
 \item [{\tt metaDataGeneric}] Add to the generic {\tt metaData} group;
 \item [{\tt metaDataCosmology}] Add to the {\tt cosmology} group;
 \item [{\tt metaDataSimulation}] Add to the {\tt simulation} group;
 \item [{\tt metaDataGroupFinder}] Add to the {\tt groupFinder} group;
 \item [{\tt metaDataTreeBuilder}] Add to the {\tt treeBuilder} group;
 \item [{\tt metaDataProvenance}] Add to the {\tt provenance} group,
\end{description}
{\tt label} is a label for this metadatum and {\tt value} is the value to store. Currently integer, double precision and character data types are supported for metadata.\\

\noindent \emph{Exporting the data:} Once the data has been read, units and properties specified and any metadata added, the trees can be exported to an HDF5 file using:
\begin{verbatim}
  call mergerTrees%export(outputFile,hdfChunkSize,hdfCompressionLevel)
\end{verbatim}
where {\tt outputFile} is the name of the file to which the trees should be exported, and {\tt hdfChunkSize} and {\tt hdfCompressionLevel} respectively give the chunk size and compression level to use when writing the file.


\chapter{Plotting Support}\index{plotting}

\section{Plotting with {\normalfont \scshape Gnuplot}}

While \glc\ data can, of course, be plotted using whatever method you choose, two Perl modules are provided that we find useful for plotting \glc\ data. These are intended for use with \href{http://www.gnuplot.info/}{\normalfont \scshape GnuPlot} and with datasets stored as \href{http://pdl.perl.org/}{\normalfont \ttfamily PDL} variables. The first module, {\normalfont \ttfamily GnuPlot::PrettyPlots} plots lines and points with two color style (typically a lighter interior color and a darker border) with support for errorbars and limits (show as arrows) on points. The second, {\normalfont \ttfamily GnuPlot::LaTeX} provides a convenient way to process output from {\normalfont \scshape GnuPlot}'s {\normalfont \ttfamily epslatex} terminal into PDF files (suitable for inclusion in documents), PNG images with transparent backgrounds or \href{http://www.openoffice.org/}{OpenOffice} \href{http://www.wikimedia.org/wikipedia/en/wiki/OpenDocument}{ODG} files (suitable for inclusion into presentations\footnote{If you create an OpenOffice ODG file it's recommended that you covert it to a Metafile within OpenOffice before putting it into a presentation---this seems to prevent a bug which occasionally causes an element of the plot to be lost during saving\ldots}).

A typical use of these packages would look as follows:
\begin{verbatim}
use lib "./perl";
use PDL;
use GnuPlot::LaTeX;
use GnuPlot::PrettyPlots;

$outputFile = "myImage";
open($gnuPlot,"|gnuplot");
print $gnuPlot "set terminal epslatex color colortext lw 2 solid 7\n";
print $gnuPlot "set output '".$outputFile.".eps'\n";
print $gnuPlot "set xlabel '\$x-axis label\$'\n";
print $gnuPlot "set ylabel '\$y-axis label\$'\n";
print $gnuPlot "set lmargin screen 0.15\n";
print $gnuPlot "set rmargin screen 0.95\n";
print $gnuPlot "set bmargin screen 0.15\n";
print $gnuPlot "set tmargin screen 0.95\n";
print $gnuPlot "set key spacing 1.2\n";
print $gnuPlot "set key at screen 0.4,0.8\n";
print $gnuPlot "set key left bottom\n";
print $gnuPlot "set xrange [0.0:6.0]\n";
print $gnuPlot "set yrange [0.0:1.0]\n";
print $gnuPlot "set pointsize 2.0\n";
&PrettyPlots::Prepare_Dataset(\$plot,
			      $x1Data, $y1Data,
                              title => "First dataset",
                              style => line,
                              linePattern => 0,
                              weight => [7,3],
			      color => $PrettyPlots::colorPairs{'lightGoldenrod'}
                             );
&PrettyPlots::Prepare_Dataset(\$plot,
			      $x2Data, $y2Data,
			      errorDown => $errorDown,
			      errorUp   => $errorUp,
                              title => "Galacticus",
			      style => point,
                              symbol => [6,7],
                              weight => [5,3],
			      color => $PrettyPlots::colorPairs{'redYellow'}
                             );
&PrettyPlots::Plot_Datasets($gnuPlot,\$plot);
close($gnuPlot);
&LaTeX::GnuPlot2PNG($outputFile.".eps", backgroundColor => "#000080", margin => 1);
\end{verbatim}

The process begins by opening a pipe to {\normalfont \scshape GnuPlot} and specifying the {\normalfont \ttfamily epslatex} terminal along with {\normalfont \ttfamily color} and {\normalfont \ttfamily colortext} options, any line weight preferences and the output EPS file. This is followed by commands to set up the plot, including labels, ranges etc. Note that you \emph{must} specify margins manually\footnote{The {\normalfont \ttfamily GnuPlot::PrettyPlots} module works by generating multiple layers of plotting which are overlaid. Axes are only drawn for the first layer. If you do not specify margins manually, they will be computed automatically for each layer and so will not match up between all layers. This will result in data being plotted incorrectly.}. Following this are calls to {\normalfont \ttfamily \&PrettyPlots::Prepare\_Dataset} which prepares instructions for plotting of a single dataset. The first argument is a reference to a structure which will store the instructions, while the second and third arguments are PDLs containing the $x$ and $y$ data to be plotted. Following this are multiple options as follows:
\begin{description}
\item[{\normalfont \ttfamily title}] Gives the title of the dataset for inclusion in the plot key;
\item[{\normalfont \ttfamily style}] Specifies how the dataset should be drawn: either {\normalfont \ttfamily line}, {\normalfont \ttfamily point}, {\normalfont \ttfamily boxes}, or {\normalfont \ttfamily filledCurve};
\item{{\normalfont \ttfamily linePattern}} Specifies the line pattern (as defined for {\normalfont \scshape GnuPlot}'s {\normalfont \ttfamily lt} option) to use;
\item[{\normalfont \ttfamily symbol}] A two element list giving the symbol indices that should be used to plot the border and inner parts of each point respectively;
\item[{\normalfont \ttfamily weight}] A two element list giving the line weights to be used for border and inner parts of each point/line respectively;
\item[{\normalfont \ttfamily color}] A two element list giving the color of border and inner parts of each point/line respectively. Colors should be specified as {\normalfont \ttfamily \#RRGGBB} in hexadecimal. Several suitable color pairs and sequences of pairs are defined in the {\normalfont \ttfamily GnuPlot::PrettyPlots} module;
\item[{\normalfont \ttfamily pointSize}] Specifies the size of the points to be used;
\item[{\normalfont \ttfamily errorNNN}] Gives a PDL containing sizes of errors to be plotted on points in the up, down, left and right directions. A zero value will cause the error bar to be omitted, while a negative value will cause an arrow to be drawn with a length equal to the absolute value of the specified value;
\item[{\normalfont \ttfamily filledCurve}] If the {\normalfont \ttfamily filledCurve} style is used, this option specifies the type of filled curve ({\normalfont \ttfamily closed}, {\normalfont \ttfamily x1}, {\normalfont \ttfamily x2}, etc.---see the {\normalfont \scshape GnuPlot} {\normalfont \ttfamily help filledcurve} text for complete options). The default is {\normalfont \ttfamily closed};
\item[{\normalfont \ttfamily y2}] If the {\normalfont \ttfamily filledCurve} style is used along with the {\normalfont \ttfamily filledCurve}$=${\normalfont \ttfamily closed} option, this option is used to specify a second PDL of $y$-axis values. The region between this curve and the usual $y$-axis curve will be filled.
\end{description}
Once all datasets have been prepared, the call to {\normalfont \ttfamily \&PrettyPlots::Plot\_Datasets} will generate the EPS and \LaTeX\ files necessary to make the plot. This resulting plot can be converted to PDF, PNG or ODG form by calling {\normalfont \ttfamily \&LaTeX::GnuPlot2PDF}, {\normalfont \ttfamily \&LaTeX::GnuPlot2PNG} or {\normalfont \ttfamily \&LaTeX::GnuPlot2ODG} respectively. The EPS file will be replaced with the appropriate file. The {\normalfont \ttfamily \&LaTeX::GnuPlot2PNG} routine accepts an optional {\normalfont \ttfamily backgroundColor} argument in {\normalfont \ttfamily \#RRGGBB} format. If present, this color will be used to set the background color of the plot (otherwise white is assumed). Although the background is made transparent in the PNG, setting the background color is important as antialiasing will make use of this background. Note that both PNG and ODG options will switch black axes and labels to white\footnote{This is just a presonal preference for plots displayed in presentations---other options could be added}. Finally, the {\normalfont \ttfamily \&LaTeX::GnuPlot2PNG} routine accepts an optional {\normalfont \ttfamily margin} argument which specifies the size of the margin (in pixels) to be left around the plot when cropping.

The ODG option requires that both \href{http://www.cityinthesky.co.uk/opensource/pdf2svg}{{\normalfont \ttfamily pdf2svg}} and \href{http://www.haumacher.de/svg-import/}{{svg2office}} be installed on your system ({\normalfont \ttfamily svg2office} should be located in {\normalfont \ttfamily /usr/local/bin}).

\section{Merger Tree Diagrams with {\normalfont \scshape dot}}\index{merger trees!graphing}

The {\normalfont \scshape dot} command, which is a part of \href{http://www.graphviz.org/}{{\normalfont \scshape GraphViz}}\index{graphviz@{\normalfont \scshape GraphViz}} is useful for creating diagrams of merger trees. \glc\ provides a function to output the structure of any merger tree in {\normalfont \scshape GraphViz} format. This function, \hyperlink{objects.merger_trees.dump.F90:merger_trees_dump:merger_tree_dump}{{\normalfont \ttfamily Merger\_Tree\_Dump}}, is provided by the \hyperlink{objects.merger_trees.dump.F90:merger_trees_dump:merger_tree_dump}{{\normalfont \ttfamily Merger\_Trees\_Dump}} module. Usage is as follows:
\begin{lstlisting}[escapechar=@,breaklines,prebreak=\&,postbreak=\&]
call Merger_Tree_Dump(                                    &
     &                index                             , &
     &                baseNode                          , &
     &                highlightNodes     =highlightNodes, &
     &                backgroundColor    ='white'       , &
     &                nodeColor          ='black'       , &
     &                highlightColor     ='black'       , &
     &                edgeColor          ='#DDDDDD'     , &
     &                nodeStyle          ='solid'       , &
     &                highlightStyle     ='filled'      , &
     &                edgeStyle          ='solid'       , &
     &                labelNodes         =.false.       , &
     &                scaleNodesByLogMass=.true.        , &
     &                edgeLengthsToTimes =.true.        , &
     &                path               ='/my/path'      &
     &               )
\end{lstlisting}
Here {\normalfont \ttfamily index} is the tree index (successive calls to \hyperlink{objects.merger_trees.dump.F90:merger_trees_dump:merger_tree_dump}{{\normalfont \ttfamily Merger\_Tree\_Dump}} with the same index will result in a sequence of output files---see below), and {\normalfont \ttfamily baseNode} is a pointer to the base node of the tree to be dumped. All other arguments are optional:
\begin{description}
 \item [{\normalfont \ttfamily highlightNodes}] A list of node IDs. All nodes listed will be highlighted in the diagram;
 \item [{\normalfont \ttfamily backgroundColor}] The color for the background of the diagram;
 \item [{\normalfont \ttfamily nodeColor}] The color used to draw nodes;
 \item [{\normalfont \ttfamily highlightColor}] The color used for highlighted nodes;
 \item [{\normalfont \ttfamily edgeColor}] The color of edges (lines joining nodes);
 \item [{\normalfont \ttfamily nodeStyle}] The style to use when drawing nodes;
 \item [{\normalfont \ttfamily highlightStyle}] The style to use when drawing highlighted nodes;
 \item [{\normalfont \ttfamily edgeStyle}] The style to use when drawing edges;
 \item [{\normalfont \ttfamily labelNodes}] Specifies whether or not nodes should be labelled (labels consist of the node ID followed by the redshift);
 \item [{\normalfont \ttfamily scaleNodesByLogMass}] If true, the size of nodes will be set to be proportional to the logarithm of the node mass;
 \item [{\normalfont \ttfamily edgeLengthsToTimes}] If true, the spacing between parent and child nodes will be proportional to the logarithmic time interval between them.
 \item [{\normalfont \ttfamily path}] If present, write tree dumps into this directory. Otherwise, the current directory will be used.
\end{description}
All colors and styles are character strings and can be in any format understood by {\normalfont \scshape dot}. The tree structure will be dumped to file named {\normalfont \ttfamily mergerTreeDump:$\langle$ID$\rangle$:$\langle$N$\rangle$.gv} where {\normalfont \ttfamily $\langle$ID$\rangle$} is the index of the tree and {\normalfont \ttfamily $\langle$N$\rangle$} increasing incrementally from $1$ each time the same tree is consecutively dumped. These files can be processed using {\normalfont \scshape dot}. For example
\begin{verbatim}
 dot -Tps mergerTreeDump:1:1.gv > tree.ps
\end{verbatim}
will create a tree diagram as the PostScript file {\normalfont \ttfamily tree.ps}.


\backmatter

\bibliographystyle{plainnat}
\bibliography{GalacticusAccented}

\printglossaries

\citeindextrue
\printindex
\printindex[code]

\end{document}
